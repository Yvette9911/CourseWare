%\chapter{Introduction to Linear Programming}
\chapter{The Basic Algorithm for Nonlinear Programming}

\section{Gradient Algorithms}
\subsection{Preliminaries: convergence analysis}
Consider an iterative algorithm for solving the optimization problem $\min f(x)$, producing iterates $\{x^0,x^1,\dots\}$.
\begin{enumerate}
\item
The possible error measurements are as follows. The stopping criteria depends on these error measurements.
\begin{itemize}
\item
$e(x^k):=\|x^k-x^*\|$;
\item
$e(x^k) = f(x^k) - f(x^*)$;
\end{itemize}
where $x^*$ denotes the underlying optimal solution.
\item
We say the algorithm converges if $\lim_{k\to\infty}e(x^k) = 0$
\item
There are different types of convergence rate:
\begin{enumerate}
\item
R-linear convergence: there exists $a\in(0,1)$ such that $e(x^k)\le Ca^k$;
\item
Q-linear convergence: there exists $a\in(0,1)$ such that $\frac{e(x^{k+1})}{e(x^k)}\le a$;
\item
Sub-linear convergence: $e(x^k)\le C/k^p$ for some $p>0$.
\end{enumerate}
\end{enumerate}
question: when say about convergence rate, do we need to specify which error measurements we use?
\subsection{The (Sub)gradient algorithm for Unconstrained Optimization}
Consider an unconstrained optimization problem $\min f(x)$, where $f$ may not necessarily be smooth. Let $\{t_k>0\mid k=0,1,\dots\}$ be a sequence of step-sizes. Let's study the simpleest first order optimization algorithm.
\begin{algorithm}[htb] 
\caption{The (Sub)gradient Algorithm} 
\label{alg:GA} 
\begin{algorithmic}[0] %show number in each rows
\REQUIRE ~ %算法的输入参数:Input
Initial guess $x^0\in\mathcal{X}$
\ENSURE ~ %算法的输出:Output
Optimal solution $\hat x$
\STATE \textbf{For $k=0,1,\dots,$ do}
\begin{itemize}
\item
Take $d^k\in\partial f(x^k)$;
\item
$x^{k+1} \leftarrow x^k - t_kd^k $
\end{itemize}
\textbf{end for.}
\label{code:GA}
\end{algorithmic}
\end{algorithm}
\paragraph{Worst Case Bounds}
Consier a \emph{convex optimization model} where $f$ is a completely unknown function. 
The first order type algorithm esentially produces a sequence of iterates $\{x^k\mid k=0,1,2,\dots\}$ in such a way that $x^k$ is in the \emph{affine space} spanned by
\[
x^0,g(x^0),\dots,g(x^{k-1}),\ \text{where }g(\cdot) = \partial f(\cdot).
\]
\begin{itemize}
\item
Suppose $f$ is \emph{Lipschitz continuous} and no other information is known, we can construct an example such that
\[
\min_{x\in\text{Span}\{x^0,g(x^0),\dots,g(x^{k-1})\}}f(x) - f(x^*)\ge \mathcal{O}(\frac{1}{\sqrt{k}}),\ \forall k=1,2,\dots,\lfloor\frac{n}{2}\rfloor
\]
Therefore, the first order type algorithm can never reach the convergence rate faster than $\mathcal{O}(\frac{1}{\sqrt{k}})$.
\item
Additionally, if we know $f$ is \emph{differentiable} and $\nabla f$ is \emph{Lipschitz continuous}, then we can construct an example such that
\[
\min_{x\in\text{Span}\{x^0,g(x^0),\dots,g(x^{k-1})\}}f(x) - f(x^*)\ge \mathcal{O}(\frac{1}{k^2}),\ \forall k=1,2,\dots,\lfloor\frac{n}{2}\rfloor
\]
Therefore, the first order type algorithm can never reach the convergence rate faster than $\mathcal{O}(\frac{1}{k^2})$ for optimizing this class of function.
\end{itemize}

\subsection{Gradient Algorithm with Exact Line-Search}
First we discuss the optimization with a uniform convex function. A nice convergence result is obtained:
\begin{theorem}
Suppose there exists $0< m\le M$ such that
$0\succ mI\succeq \nabla^2f(x)\succeq MI$ (i.e., $f$ is \emph{uniformly convex}), and an exact line search is performed per iteration:
\[
t_k:=\arg\min_tf(x^k - t\nabla f(x^k)),
\]
then
\begin{equation}\label{Eq:5:1}
f(x^{k+1}) - f(x^*)\le \left(1-\frac{m}{M}\right)[f(x^k) - f(x^*)]
\end{equation}
\end{theorem}
\begin{proof}
\begin{itemize}
\item
\textbf{(Uniform Convexity implies Strongly Convexity)}
For $\forall \bm x_1,\bm x_2\in\text{dom}(f)$, by mean-value theorem,
\[
f(\bm x_2) = f(\bm x_1) + \inp{\nabla f(\bm x_1)}{\bm x_2-\bm x_1} + \frac{1}{2}(\bm x_2-\bm x_1)\trans\nabla^2f(\bm\xi)(\bm x_2-\bm x_1),
\]
where $\bm\xi$ is some number between $\bm x_2$ and $\bm x_1$. Applying the uniform convexity of $f$, we derive the strongly convexity property:
\begin{equation}\label{Eq:5:2}
\frac{m}{2}\|\bm x_1-\bm x_2\|_2^2\le f(\bm x_2)-f(\bm x_1) - \inp{\nabla f(\bm x_1)}{\bm x_2-\bm x_1}\le\frac{M}{2}\|\bm x_1-\bm x_2\|_2^2
\end{equation}
\item
\textbf{(Applying Strongly Convexity Property)}
On the one hand, by setting $\bm x_1 = \bm x^*$ and $\bm x_2 = \bm x$ in (\ref{Eq:5:2}), we obtain:
\begin{equation}
\frac{m}{2}\|\bm x-\bm x^*\|_2^2\le f(\bm x)-f(\bm x^*) \le\frac{M}{2}\|\bm x-\bm x^*\|_2^2
\end{equation}
On the other hand, by setting $\bm x_1 = \bm x$ and $\bm x_2 = \bm x^*$ in (\ref{Eq:5:2}), we obtain
\begin{align*}
\frac{m}{2}\|\bm x-\bm x^*\|_2^2&\le f(\bm x^*)-f(\bm x) - \inp{\nabla f(\bm x)}{\bm x^*-\bm x}\\
&\le f(\bm x^*)-f(\bm x) + \|\nabla f(\bm x)\|\cdot\|\bm x^*-\bm x\|\\
&\le \|\nabla f(\bm x)\|\cdot\|\bm x^*-\bm x\|
\end{align*}
which implies $m\|\bm x-\bm x^*\|\le\|\nabla f(\bm x)\|$. Similarly, we get
\begin{equation}
m\|\bm x-\bm x^*\|\le\|\nabla f(\bm x)\|\le M\|\bm x-\bm x^*\|
\end{equation}
\item
\textbf{(Upper Bounding left and right side of (\ref{Eq:5:1}))}
Moreover, we upper bounding the left side of (\ref{Eq:5:1}) by setting $\bm x_2 = \bm x^{k+1}$ and $\bm x_1 = \bm x^k$ in (\ref{Eq:5:2}):
\begin{equation}\label{Eq:5:5}
\begin{aligned}
f(\bm x^{k+1}) - f(\bm x^k)&\le \inp{\nabla f(\bm x^k)}{\bm x^{k+1} - \bm x^k}+\frac{M}{2}\|\bm x^{k+1}-\bm x^k\|^2\\
&\le-\frac{1}{2M}\|\nabla f(x^k)\|^2
\end{aligned}
\end{equation}
On the other hand, by setting $\bm x_2 = \bm x^*$ and $\bm x_1 = \bm x^k$ in (\ref{Eq:5:2}), we obtain
\begin{equation}\label{Eq:5:6}
\begin{aligned}
f(\bm x^k) - f(\bm x^*)&\le \inp{\nabla f(x^k)}{\bm x^k - \bm x^*}-\frac{m}{2}\|\bm x^k-\bm x^*\|_2^2\\
&\le \|\nabla f(x^k)\|\|\bm x^k - \bm x^*\| - \frac{m}{2}\|\bm x^k-\bm x^*\|_2^2\\
&\le \frac{1}{2m}\|\nabla f(\bm x^k)\|^2
\end{aligned}
\end{equation}
Therefore, substituting (\ref{Eq:5:6}) into (\ref{Eq:5:5}), we obtain
\[
f(\bm x^{k+1}) - f(\bm x^k)\le -\frac{m}{M}[f(\bm x^k) - f(\bm x^*)]
\]
Or equivalently,
\[
f(\bm x^{k+1}) - f(\bm x^*)\le\left(1-\frac{m}{M}\right)[f(\bm x^k) - f(\bm x^*)]
\]



\end{itemize}




\end{proof}














