\subsection{Final Exam Solution}
\begin{enumerate}
%q1
\item
\begin{enumerate}
%a
\item
For $\forall f,g\in\{\textbf{polynomials of degree $\le4$}\}$, we obtain:
\begin{itemize}
\item
\[
T(f+g)=(x-2)\frac{\diff}{\diff x}(f+g)=(x-2)\frac{\diff}{\diff f}+(x-2)\frac{\diff}{\diff g}
=T(f)+T(g)
\]
\item
\[
T(cf)=(x-2)\frac{\diff}{\diff x}(cf)=c(x-2)\frac{\diff}{\diff f}
=cT(f).
\]
where $c$ is a scalar.
\end{itemize}
Since $T$ satisfies the vector addition and scalar multiplication rule, it is a linear transformation.\\
Moreover, we obtain:
\begin{align*}
T(1)&=(x-2)\frac{\diff 1}{\diff x}=0\\
T(x)&=(x-2)\frac{\diff x}{\diff x}=x-2\\
T(x^2)&=(x-2)\frac{\diff x^2}{\diff x}=2x(x-2)=2x^2-4x\\
T(x^3)&=(x-2)\frac{\diff x^3}{\diff x}=3x^2(x-2)=3x^3-6x^2\\
T(x^4)&=(x-2)\frac{\diff x^4}{\diff x}=4x^3(x-2)=4x^4-8x^3.
\end{align*}
Hence the matrix representation is given by:
\[
\begin{bmatrix}
0&0&0&0&0\\
-2&1&0&0&0\\
0&-4&2&0&0\\
0&0&-6&3&0\\
0&0&0&-8&4
\end{bmatrix}
\]
%b
\item
\begin{itemize}
\item
For $f=1$, $T(f)=(x-2)\frac{\diff f}{\diff x}=0=0f$.\\
Hence $f=1$ is an eigenvector of $T$ associated with eigenvalue $\lambda=0.$
\item
For $f=x-2$, $T(f)=(x-2)\frac{\diff f}{\diff x}=x-2=f$.\\
Hence $f=x-2$ is an eigenvector of $T$ associated with eigenvalue $\lambda=1.$
\end{itemize}
Moreover, we have $\alpha_1\x(1)+\alpha_2\x(x-2)=0$, where $\alpha_1,\alpha_2$ are scalars, then we derive
\[
x(\alpha_1+\alpha_2)-2\alpha_2=0.\implies
\alpha_1=\alpha_2=0.
\]
Hence $(x-2)$ and $1$ are independent.\\
Hence two independent eigenvectors of $T$ are $1$ and $(x-2)$.
\end{enumerate}
\item
\begin{enumerate}
\item
Firstly, we set $\bm x=\begin{bmatrix}
1\\1\\-2
\end{bmatrix},\bm y=\begin{bmatrix}
-1\\-1\\4
\end{bmatrix}.$ 
Obviously, they are independent. \\Hence $\{\bm x,\bm y\}$ is the basis for column space of matrix $\begin{bmatrix}
1&-1\\1&-1\\-2&4
\end{bmatrix}.$\\
Then we convert $\{\bm x,\bm y\}$ into orthogonal basis $\{\bm q_1,\bm q_2\}$:
\begin{itemize}
\item
\[
\bm q_1=\bm x
\]
\item
\[
\bm q_2=\bm y-\Proj_{\bm y}(\bm q_1)=\bm y-\frac{\inp{\bm y}{\bm x}}{\inp{\bm x}{\bm x}}\bm x
=\begin{bmatrix}
\frac{2}{3}\\\frac{2}{3}\\\frac{2}{3}
\end{bmatrix}.
\]
\end{itemize}
\begin{itemize}
\item
The projection of $\bm z$ onto the vector $\bm q_1$ is 
\[
\Proj_{\bm q_1}(\bm z)=\frac{\inp{\bm x}{\bm z}}{\inp{\bm x}{\bm x}}\bm x=\begin{bmatrix}
-\frac{1}{6}\\-\frac{1}{6}\\\frac{1}{3}
\end{bmatrix}.
\]
\item
The projection of $\bm z$ onto the vector $\bm q_2$ is 
\[
\Proj_{\bm q_2}(\bm z)=\frac{\inp{\bm q_2}{\bm z}}{\inp{\bm q_2}{\bm q_2}}\bm q_2=\begin{bmatrix}
\frac{2}{3}\\\frac{2}{3}\\\frac{2}{3}
\end{bmatrix}.
\]
\end{itemize}
Hence the projection of $\bm z$ onto $\Span\{\bm x,\bm z\}$ is given by:
\[
\Proj_{\Span\{\bm q_1,\bm q_2\}}(\bm z)=\Proj_{\bm q_1}(\bm z)+\Proj_{\bm q_2}(\bm z)=\begin{bmatrix}
-\frac{1}{6}\\-\frac{1}{6}\\\frac{1}{3}
\end{bmatrix}+\begin{bmatrix}
\frac{2}{3}\\\frac{2}{3}\\\frac{2}{3}
\end{bmatrix}=\begin{bmatrix}
\frac{1}{2}\\\frac{1}{2}\\1
\end{bmatrix}
\]
Hence the projection onto the column space of $\begin{bmatrix}
1&-1\\1&-1\\-2&4
\end{bmatrix}$ is $\begin{bmatrix}
\frac{1}{2}\\\frac{1}{2}\\1
\end{bmatrix}.$
\item
We construct an isomorphism from $\mathbb{R}^{2\x 2}$ to $\mathbb{R}^{4\x 1}$:
\[
\begin{bmatrix}
a&b\\c&d
\end{bmatrix}\mapsto\begin{bmatrix}
a&b&c&d
\end{bmatrix}\trans.
\]
The matrix representation $\bm A$ for the mapping 
\[
\begin{bmatrix}
a\\b
\end{bmatrix}\mapsto\begin{bmatrix}
a+b&a-b&-2a+4b&0
\end{bmatrix}\trans
\] is given by:
\[
\bm A=\begin{bmatrix}
1&1\\1&-1\\-2&4\\0&0
\end{bmatrix}.
\]
We define $K=\{\bm A\bm x|\bm x\in\mathbb{R}^{2\x 1}\}.$ \\
Hence we only need to find the best approximation of $\bm b:=\begin{bmatrix}
1\\2\\7\\1
\end{bmatrix}$ in the space $K$.\\
We define $\bm x:=\begin{bmatrix}
1\\1\\-2\\0
\end{bmatrix},\bm y:=\begin{bmatrix}
1\\-1\\4\\0
\end{bmatrix}.$ Then we convert $\{\bm x,\bm y\}$ into orthogonal vectors:
\begin{itemize}
\item
We set $\bm q_1=\bm x$.
\item
We set $\bm q_2=\bm y-\Proj_{\bm q_1}(\bm y)$. Hence
\begin{align*}
\bm q_2&=\bm y-\Proj_{\bm q_1}(\bm y)\\
&=\bm y-\frac{\inp{\bm q_1}{\bm y}}{\inp{\bm q_1}{\bm q_1}}\bm q_1\\
&=\begin{bmatrix}
\frac{7}{3}&\frac{1}{3}&\frac{4}{3}&0
\end{bmatrix}\trans.
\end{align*}
Hence the projection of $\bm b$ onto the space $K$ is:
\begin{align*}
\Proj_{\Span\{\bm x,\bm y\}}(\bm b)&
=\Proj_{\Span\{\bm q_1,\bm q_2\}}(\bm b)\\
&=\Proj_{\bm q_1}(\bm b)+\Proj_{\bm q_2}(\bm b)\\
&=\frac{\inp{\bm q_1}{\bm b}}{\inp{\bm q_1}{\bm q_1}}\bm q_1+\frac{\inp{\bm q_2}{\bm b}}{\inp{\bm q_2}{\bm q_2}}\bm q_2\\
&=-\frac{6}{11}\begin{bmatrix}
1\\1\\-2\\0
\end{bmatrix}+\frac{37}{66}\begin{bmatrix}
7\\1\\4\\0
\end{bmatrix}\\
&=\frac{1}{11}\begin{bmatrix}
23\\-14\\65\\0
\end{bmatrix}.
\end{align*}
Hence the best approximation for $\bm b=\begin{bmatrix}
1\\2\\7\\1
\end{bmatrix}$ is $\frac{1}{11}\begin{bmatrix}
23\\-14\\65\\0
\end{bmatrix}.$\\
Correspondingly, the best approximation for $\bm B=\begin{bmatrix}
1&2\\7&1
\end{bmatrix}$ is $\frac{1}{11}\begin{bmatrix}
23&-14\\65&0
\end{bmatrix}.$ 
\end{itemize}
\end{enumerate}
\item
\begin{enumerate}
\item
False.\\
\textbf{Reason: }For example, if $\bm A=\begin{bmatrix}
1&1\\1&1
\end{bmatrix}$, $\bm A^{-1}$ doesn't exist.
\item
True.\\
\textbf{Reason: }For orthogonal matrix $\bm Q$, we obtain $\bm Q\trans\bm Q=\bm I.$ Thus
\[
\det(\bm Q\trans\bm Q)=\det(\bm I)
\implies
\det(\bm Q\trans)\det(\bm Q)=\det(\bm I)
\implies
[\det(\bm Q)]^2=1
\]
Hence $\det(\bm Q)=\pm1.$
\item
True.\\
\textbf{Reason: }For real symmetric $\bm A$, $-\bm A$ is PSD. $-\bm A$ could be diagonalized by orthogona matrix $\bm P$:
\[
\bm P\trans(-\bm A)\bm P=\bm D\Longleftrightarrow
\bm P\bm D\bm P\trans=-\bm A
\]
where $\bm D=\diag(\lambda_1,\dots,\lambda_n)$, where $\lambda_i$'s are eigenvalues for $-\bm A$.\\
Since $-\bm A=-\bm A\trans$, we obtain
\[
(-\bm A)(-\bm A)\trans=\bm P\bm D\bm P\trans\bm P\bm D\bm P\trans=\bm P\bm D^2\bm P\trans.
\]
Or equivalently, $\bm D^2=\bm P\trans(-\bm A)(-\bm A)\trans\bm P.$
where the eigenvalues for $(-\bm A)(-\bm A)\trans$ are on the diagonal of $\bm D^2$.\\
This shows that if $\lambda$ is the eigenvalue for $-\bm A$, then $\lambda^2$ is the eigenvalue for $(-\bm A)(-\bm A)\trans=\bm A\bm A\trans.$\\
Since $-\bm A$ is PSD, all eigenvalues of $-\bm A$ are positive. Hence $\lambda=\sqrt{\lambda^2}.$\\
If $\lambda$ is the eigenvalue for $-\bm A$, then $-\lambda$ is the eigenvalue for $\bm A$. Hence the absolute value of eigenvalues for $\bm A$ are the same as the singular values for $\bm A.$
\item
False.\\
\textbf{Reason: }For example, $\bm A=\begin{bmatrix}
0&1\\0&0
\end{bmatrix}$, then $P_{\bm A}(t)=\begin{vmatrix}
t&-1\\0&t
\end{vmatrix}
=t^2.$
\item
True.\\
\textbf{Reason: }$\rank(\bm A)=$the smallest number of rank 1 matrices with sum $\bm A$. Hence $\rank(\bm A)\le 5.$
\end{enumerate}
\item
\begin{enumerate}
\item
\[
|\lambda\bm I-\bm A|=0\implies
\begin{vmatrix}
\lambda&1\\-4&\lambda
\end{vmatrix}=0
\implies
\lambda^2+4=0.
\]
Hence the eigenvalues for $\bm A$ are $\lambda_1=2i,\lambda_2=-2i.$\\
\begin{itemize}
\item
When $\lambda=2i,$ $(\lambda\bm I-\bm A)\bm x=\bm 0\implies\bm x=\alpha\begin{pmatrix}
1\\-2i
\end{pmatrix},$ where $\alpha$ is a scalar.
\item
When $\lambda=-2i,$ $(\lambda\bm I-\bm A)\bm x=\bm 0\implies\bm x=\beta\begin{pmatrix}
1\\2i
\end{pmatrix},$ where $\beta$ is a scalar.
\end{itemize}
Hence $\alpha\begin{pmatrix}
1\\-2i
\end{pmatrix}$ are eigenvectors of $\bm A$ associated with eigenvalue $\lambda=2i$; $\beta\begin{pmatrix}
1\\2i
\end{pmatrix}$ are eigenvectors of $\bm A$ associated with eigenvalue $\lambda=-2i$.\\
Moreover, $\bm u=\begin{pmatrix}
1\\2i
\end{pmatrix}+\begin{pmatrix}
1\\-2i
\end{pmatrix}.$
\item
\begin{itemize}
\item
Firstly, $\bm A\trans\bm A=\begin{bmatrix}
16&0\\0&1
\end{bmatrix}.$ And we have 
\[
|\lambda\bm I-\bm A\trans\bm A|=\begin{vmatrix}
\lambda-16&0\\0&\lambda-1
\end{vmatrix}=(\lambda-16)(\lambda-1)=0
\implies\lambda_1=16,\lambda_2=1.
\]
\begin{itemize}
\item
When $\lambda=16,$ $(\lambda\bm I-\bm A\trans\bm A)\bm x=\bm 0\implies
\bm x=\alpha\begin{pmatrix}
1\\0
\end{pmatrix},
$ where $\alpha$ is a scalar.
\item
When $\lambda=1,$ $(\lambda\bm I-\bm A\trans\bm A)\bm x=\bm 0\implies
\bm x=\beta\begin{pmatrix}
0\\1
\end{pmatrix},
$ where $\beta$ is a scalar.
\end{itemize}
Hence $\bm x_1=\alpha\begin{pmatrix}
1\\0
\end{pmatrix}$ are eigenvectors of $\bm A\trans\bm A$ associated with $\lambda_1=16$; $\bm x_2=\beta\begin{pmatrix}
0\\1
\end{pmatrix}$ are eigenvectors of $\bm A\trans\bm A$ associated with $\lambda_2=1.$\\
Hence $\Sigma=\diag(\sqrt{\lambda_1},\sqrt{\lambda_2})=\diag(4,1)$. \\
If we set $\alpha=1,\beta=1$, then $\bm V=\begin{bmatrix}
1&0\\0&1
\end{bmatrix}.$
\item
Secondly, Since we have known $\bm A=\bm U\bm\Sigma\bm V\trans$, we derive
\[
\bm U=\bm A\bm V\bm\Sigma^{-1}=\begin{bmatrix}
0&-1\\4&0
\end{bmatrix}\begin{bmatrix}
1&0\\0&1
\end{bmatrix}\begin{bmatrix}
\frac{1}{4}&0\\0&1
\end{bmatrix}=\begin{bmatrix}
0&-1\\1&0
\end{bmatrix}.
\]
\end{itemize}
In conclusion, our SVD decomposition is given by:
\[
\bm A=\begin{bmatrix}
0&-1\\1&0
\end{bmatrix}\begin{bmatrix}
4&0\\0&1
\end{bmatrix}\begin{bmatrix}
1&0\\0&1
\end{bmatrix}\trans
\]
\end{enumerate}
\item
\begin{enumerate}
\item
Suppose $\bm S^{-1}\bm A\bm S=\bm D_1$, $\bm S^{-1}\bm B\bm S=\bm D_2$, where $\bm D_1,\bm D_2$ are diagonal matrices.\\
Then equivalently,
\[
\bm A=\bm S\bm D_1\bm S^{-1}\quad
\bm B=\bm S\bm D_2\bm S^{-1}
\]
Hence the product $\bm{AB}$ is given by:
\begin{align*}
\bm{AB}&=(\bm S\bm D_1\bm S^{-1})(\bm S\bm D_2\bm S^{-1})\\
&=\bm S\bm D_1\bm D_2\bm S^{-1}\\
&=\bm S\bm D_2\bm D_1\bm S^{-1}\\
&=\bm S\bm D_2\bm S^{-1})(\bm S\bm D_1\bm S^{-1})\\
&=\bm{BA}.
\end{align*}
\item
We let $\bm v_1,\dots,\bm v_n$ be linearly independent eigenvectors of $\bm A$ associated with $n$ distinct eigenvalues $\lambda_1,\dots,\lambda_n$. \\
Thus $\bm A\bm v_i=\lambda_i\bm v_i$. By postmultiplying $\bm B$ we find that
\begin{equation}
\label{eq_5.1}
\bm{BA}\bm v_i=\lambda_i\bm B\bm v_i\text{ for }i=1,2,\dots,n.
\end{equation}
Notice that $\{\bm x_1,\dots,\bm x_n\}$ spans the whole $\mathbb{R}^n$, thus any vector in $\mathbb{R}^n$ could be expressed as the linear combination of $\{\bm x_1,\dots,\bm x_n\}$. Hence for $\bm B\bm v_i\in\mathbb{R}^n$, we set
\begin{equation}\label{eq_5.2}
\bm B\bm v_i=\beta_1\bm v_1+\beta_2\bm v_2+\dots+\beta_n\bm v_n.
\end{equation}
By postmultiplying $\bm A$ for equation (\ref{eq_5.2}) we find that
\begin{equation}
\begin{aligned}\label{eq_5.3}
\bm{AB}\bm v_i&=\beta_1\bm A\bm v_1+\beta_2\bm A\bm v_2+\dots+\beta_n\bm A\bm v_n\\
&=\beta_1\lambda_1\bm v_1+\beta_2\lambda_2\bm v_2+\dots+\beta_n\lambda_n\bm v_n
\end{aligned}
\end{equation}
Also, by applying equation (\ref{eq_5.2}) into equation (\ref{eq_5.1}) we derive:
\begin{equation}\label{eq_5.4}
\begin{aligned}
\bm{BA}\bm v_i&=\lambda_i(\beta_1\bm v_1+\beta_2\bm v_2+\dots+\beta_n\bm v_n)\\
&=\beta_1\lambda_i\bm v_1+\beta_2\lambda_i\bm v_2+\dots+\beta_n\lambda_i\bm v_n
\end{aligned}
\end{equation}
Since $\bm{AB}=\bm{BA}$, we derive $\bm{AB}\bm v_i=\bm{BA}\bm v_i$. Combining equation (\ref{eq_5.3}) and (\ref{eq_5.4}) we obtain:
\[
\bm0=\bm{AB}\bm v_i-\bm{BA}\bm v_i
=\beta_1(\lambda_1-\lambda_i)\bm v_1+\beta_2(\lambda_2-\lambda_i)\bm v_2+\dots+\beta_n(\lambda_n-\lambda_i)\bm v_n
\]
Due to the independence of $\bm v_i$, we derive 
\[
\beta_1(\lambda_1-\lambda_i)=\beta_2(\lambda_2-\lambda_i)=\cdots=\beta_n(\lambda_n-\lambda_i)=0.
\]
Since eigenvalues of $\bm A$ are distinct, we get $\lambda_j-\lambda_i\ne0$ for $j\ne i.$ Hence $\beta_j=0$ for $j\ne i.$\\
Considering equation (\ref{eq_5.2}), we derive $\bm B\bm v_i=\beta_i\bm v_i$, which means $\bm v_i$ is also the eigenvector of $\bm B$.\\
Hence $\bm A$ and $\bm B$ has the same eigenvectors $\bm v_1,\dots,\bm v_n$. Since $\bm A$ can be diagonalized by matrix $\bm S=\begin{bmatrix}
\bm v_1&\bm v_2&\dots&\bm v_n
\end{bmatrix}$, $\bm B$ could be also diagonalized by matrix $\bm S$.
\item
We need to show that there exists $\bm S$ that can diagonalize $\bm A$ and $\bm B$:\\
Suppose $\lambda_1,\lambda_2,\dots,\lambda_h$ be the distinct eigenvalues of $\bm A$ with multiplicities $r_1,r_2,\dots,r_h$ respectively. Since $\bm A$ is diagonalizable, there exists $\bm Q$ satisfying
\[
\bm Q^{-1}\bm A\bm Q:=\bm D=\diag(\lambda_1\bm I_{r_1},\lambda_2\bm I_{r_2},\dots,\lambda_h\bm I_{r_h})
=\begin{pmatrix}
\lambda_1\bm I_{r_1}&  &  &\\
&\lambda_2\bm I_{r_2}&  &\\
&&\ddots&\\
&&&\lambda_h\bm I_{r_h}
\end{pmatrix}.
\]
Also, we can obtain the product $\bm Q^{-1}\bm B\bm Q$ and partition it into block matrix (We partition it in the same way that $\bm D$ has been partitioned):
\[
\bm Q^{-1}\bm B\bm Q:=\bm C=\begin{bmatrix}
\bm C_{11}&\bm C_{12}&\cdots&\bm C_{1h}\\
\bm C_{21}&\bm C_{22}&\cdots&\bm C_{2h}\\
\vdots&\vdots&\ddots&\vdots\\
\bm C_{h1}&\bm C_{h2}&\cdots&\bm C_{hh}
\end{bmatrix}
\]
where $\bm C_{ij}$ is $r_{i}\x r_{j}$ matrix.
\begin{itemize}
\item
Firstly, we show $\bm C$ is \textit{block diagonal:}\\
Note that $\bm{AB}=\bm{BA}$, thus we have
\[
\begin{aligned}
\bm D\bm C&=(\bm Q^{-1}\bm A\bm Q)(\bm Q^{-1}\bm B\bm Q)\\
&=\bm Q^{-1}\bm A\bm B\bm Q=\bm Q^{-1}\bm B\bm A\bm Q\\
&=(\bm Q^{-1}\bm B\bm Q)(\bm Q^{-1}\bm A\bm Q)\\
&=\bm C\bm D.
\end{aligned}
\]
Notice that the $(i,j)$th submatrix of $\bm{DC}$ is equal to the $(i,j)$th submatrix of $\bm{CD}$, which yields $\lambda_i\bm I_{r_i}\bm C_{ij}=\bm C_{ij}\lambda_j\bm I_{r_j}\implies\lambda_i\bm C_{ij}=\lambda_j\bm C_{ij}$.\\
Since $\lambda_i\ne\lambda_j$ for $i\ne j$, we derive $\bm C_{ij}=\bm0$ for $i\ne j$; thus 
\[
\bm C=\diag(\bm C_{11},\bm C_{22},\dots,\bm C_{hh})
=
\begin{pmatrix}
\bm C_{11}&  &  &\\
&\bm C_{22}&  &\\
&&\ddots&\\
&&&\bm C_{hh}
\end{pmatrix}.
\]
is \textit{block diagonal}.
\item
Then we show $\bm C$ is diagonalizable:\\
Since $\bm B$ is diagonalizable, there exists $\bm M$ satisfying
\[
\bm M^{-1}\bm B\bm M=\bm N\implies
\bm B=\bm M\bm N\bm M^{-1}
\]
where $\bm N$ is diagonal. And since $\bm Q^{-1}\bm B\bm Q=\bm C$, we derive
\[
\bm Q^{-1}\bm M\bm N\bm M^{-1}\bm Q=\bm C
\implies(\bm Q^{-1}\bm M)^{-1}\bm C(\bm Q^{-1}\bm M)=\bm N
\]
If we define $\bm T:=\bm Q^{-1}\bm M$, then $\bm T^{-1}\bm C\bm T=\bm N$. So $\bm C$ is also diagonalizable.
\item
Then we show each $\bm C_{ii}$ is diagonalizable:\\
Moreover, if we partition $\bm T$ as:
\[
\bm T=\begin{bmatrix}
\bm T_{11}&\bm T_{12}&\cdots&\bm T_{1h}\\
\bm T_{21}&\bm T_{22}&\cdots&\bm T_{2h}\\
\vdots&\vdots&\ddots&\vdots\\
\bm T_{h1}&\bm T_{h2}&\cdots&\bm T_{hh}
\end{bmatrix},
\]
where $\bm T_{ij}$ is $r_i\x r_j$ matrix, then we find the product $\bm{CT}$ is always block diagonal matrix.\\
Similarly, the product $\bm T^{-1}\x(\bm{CT})$ is also block diagonal matrix.\\
Hence without loss of generailty, we can say there must exist block diagonal matrix $\bm T_{*}=\diag(\bm T_{11},\bm T_{22},\dots,\bm T_{hh})$ such that
\begin{equation}\label{eq_5.7}
\begin{aligned}
\bm T_*^{-1}\bm C\bm T_*&=\begin{pmatrix}
\bm T_{11}^{-1}&  &  &\\
&\bm T_{22}^{-1}&  &\\
&&\ddots&\\
&&&\bm T_{hh}^{-1}
\end{pmatrix}\begin{pmatrix}
\bm C_{11}&  &  &\\
&\bm C_{22}&  &\\
&&\ddots&\\
&&&\bm C_{nn}
\end{pmatrix}\begin{pmatrix}
\bm T_{11}&  &  &\\
&\bm T_{22}&  &\\
&&\ddots&\\
&&&\bm T_{hh}
\end{pmatrix}\\&=
\begin{pmatrix}
\bm T_{11}^{-1}\bm C_{11}\bm T_{11}&  &  &\\
&\bm T_{22}^{-1}\bm C_{22}\bm T_{22}&  &\\
&&\ddots&\\
&&&T_{hh}^{-1}\bm C_{hh}\bm T_{hh}
\end{pmatrix}=\bm N.
\end{aligned}
\end{equation}
Hence each $\bm C_{ii}$ is also diagonalizable.
\item
Finally, we set $\bm P=\bm Q\bm T_{*}$, we show that both $\bm P^{-1}\bm A\bm P$ and $\bm P^{-1}\bm B\bm P$ are diagonal:
\begin{align*}
\bm P^{-1}\bm A\bm P&=\bm T_{*}^{-1}\bm Q^{-1}\bm A\bm Q\bm T_{*}=\bm T_{*}^{-1}\bm D\bm T_{*}\\
&=\diag(\bm T_{11}^{-1},\bm T_{22}^{-1},\dots,\bm T_{hh}^{-1})\diag(\lambda_1\bm I_{r_1},\lambda_2\bm I_{r_2},\dots,\lambda_h\bm I_{r_h})\diag(\bm T_{11},\bm T_{22},\dots,\bm T_{hh})\\
&=\diag(\lambda_1\bm T_{11}^{-1}\bm T_{11},\lambda_2\bm T_{22}^{-1}\bm T_{22},\dots,\lambda_hT_{hh}^{-1}T_{hh})\\
&=\diag(\lambda_1\bm I_{r_1},\lambda_2\bm I_{r_2},\dots,\lambda_h\bm I_{r_h})=\bm D
\end{align*}
and
\begin{align*}
\bm P^{-1}\bm B\bm P&=\bm T_{*}^{-1}\bm Q^{-1}\bm B\bm Q\bm T_{*}=\bm T_{*}^{-1}\bm C\bm T_{*}\\
&=\bm N\qquad\text{(You may check equation $(\ref{eq_5.7})$ to see why.)}
\end{align*}
Hence both $\bm P^{-1}\bm A\bm P$ and $\bm P^{-1}\bm B\bm P$ are diagonal. The proof is complete.
\end{itemize}
\end{enumerate}
\item
\begin{enumerate}
\item
Firstly, we extend the \emph{Hadamard Product} into vectors:\\
For $\bm u,\bm v\in\mathbb{R}^{n\x 1}$, we obtain:
\[
\begin{bmatrix}
\bm u\circ\bm v
\end{bmatrix}=\begin{bmatrix}
u_1v_1&u_2v_2&\dots&u_nv_n
\end{bmatrix}\trans.
\]
Secondly, it's easy for you to verify the two properties:
\begin{proposition}\label{prposition_q6.1}
For matrices $\bm A,\bm B,\bm C\in\mathbb{R}^{n\x n}$, we have
\[
(\bm A+\bm B)\circ\bm C=\bm A\circ\bm C+\bm B\circ\bm C
\]
\end{proposition}
\begin{proposition}\label{prposition_q6.2}
For vectors $\bm u_1,\bm v_1,\bm u_2,\bm v_2\in\mathbb{R}^{n\x 1}$, we have
\[
(\bm u_1\bm v_1\trans)\circ(\bm u_2\bm v_2\trans)=(\bm u_1\circ\bm u_2)\times(\bm v_1\circ\bm v_2)\trans.
\]
\end{proposition}
So we begin to show $\rank(\bm A\circ\bm B)\le\rank(\bm A)\rank(\bm B):$\\
We let $r_1=\rank(\bm A),r_2=\rank(\bm B)$. Due to the theorem $(\ref{proposition_8.4})$, we can decompose $\bm A$ and $\bm B$ as:
\begin{align*}
\bm A&=\sigma_1\bm u_1\bm v_1\trans+\sigma_2\bm u_2\bm v_2\trans+\dots+\sigma_{r_1}\bm u_{r_1}\bm v_{r_1}\trans\\
\bm B&=\eta_1\bm w_1\bm x_1\trans+\eta_2\bm w_2\bm x_2\trans+\dots+\eta_{r_2}\bm w_{r_2}\bm x_{r_2}\trans
\end{align*}
where $\bm u_i,\bm v_i,\bm w_i,\bm x_i$'s are all $\mathbb{R}^{n\x 1}$ vectors.\\
Hence the Hadamard product $\bm A\circ\bm B$ is given by:
\begin{align*}
\bm A\circ\bm B&=\left(
\sum_{i=1}^{r_1}\sigma_i\bm u_i\bm v_i\trans
\right)\circ\left(
\sum_{j=1}^{r_2}\eta_j\bm w_j\bm x_j\trans
\right)\\
&=\sum_{i=1}^{r_1}\sum_{j=1}^{r_2}\sigma_i\eta_j(\bm u_i\bm v_i\trans\circ\bm w_j\bm x_j\trans)
\qquad&\text{Due to the proposition (\ref{prposition_q6.1})}
\\
&=\sum_{i=1}^{r_1}\sum_{j=1}^{r_2}\sigma_i\eta_j(\bm u_i\circ\bm w_j)(\bm v_i\circ\bm x_j)\trans\qquad&\text{Due to the proposition (\ref{prposition_q6.2})}
\end{align*}
Notice that $(\bm u_i\circ\bm w_j)$ and $(\bm v_i\circ\bm x_j)$ are all $\mathbb{R}^{n\x 1}$ vectors, so $(\bm u_i\circ\bm w_j)(\bm v_i\circ\bm x_j)$ are rank 1 matrix. 
\\Hence we express $\bm A\circ\bm B$ as the sum of $r_1r_2$ matrices with rank 1.\\Thus $\rank(\bm A\circ\bm B)\le r_1r_2=\rank(\bm A)\rank(\bm B).$
\item
Since $\bm A\succeq$, we decompose $\bm A$ as:
\[
\bm A=\bm U\trans\bm U\text{ where $\bm U$ is square.}
\]
If we set $\bm U:=\begin{bmatrix}
\bm u_1\trans\\\bm u_2\trans\\\vdots\\\bm u_n\trans
\end{bmatrix}$, we can write $\bm A$ as:
\[
\bm A=\bm U\trans\bm U=\begin{bmatrix}
\bm u_1&\bm u_2&\cdots&\bm u_n
\end{bmatrix}\begin{bmatrix}
\bm u_1\trans\\\bm u_2\trans\\\vdots\\\bm u_n\trans
\end{bmatrix}=\bm u_1\bm u_1\trans+\bm u_2\bm u_2\trans+\cdots+\bm u_n\bm u_n\trans
\]
Similarly, we can write $\bm B$ as:
\[
\bm B=\bm v_1\bm v_1\trans+\bm v_2\bm v_2\trans+\cdots+\bm v_n\bm v_n\trans.
\]
Hence $\bm A\circ\bm B$ can be written as
\begin{align*}
\bm A\circ\bm B&=\left(
\sum_{i=1}^{n}\bm u_i\bm u_i\trans
\right)\circ\left(
\sum_{j=1}^{n}\bm v_j\bm v_j\trans
\right)\\
&=\sum_{i=1}^{n}\sum_{j=1}^{n}(\bm u_i\bm u_i\trans\circ\bm v_j\bm v_j\trans)\\
&=\sum_{i=1}^{n}\sum_{j=1}^{n}(\bm u_i\circ\bm v_j)(\bm u_i\circ\bm v_j)\trans
\end{align*}
If we set $\bm w_{ij}=\bm u_i\circ\bm v_j$, then we obtain:
\[
\bm A\circ\bm B=\sum_{i=1}^{n}\sum_{j=1}^{n}\bm w_{ij}\bm w_{ij}\trans
\]
Hence for $\forall\bm x\in\mathbb{R}^{n}$, we derive
\begin{align*}
\bm x\trans(\bm A\circ\bm B)\bm x&=\sum_{i=1}^{n}\sum_{j=1}^{n}\bm x\trans\bm w_{ij}\bm w_{ij}\trans\bm x\\
&=\sum_{i=1}^{n}\sum_{j=1}^{n}\inp{\bm x\bm w_{ij}}{\bm x\bm w_{ij}}\\
&=\sum_{i=1}^{n}\sum_{j=1}^{n}\|\bm x\bm w_{ij}\|^2\ge0.
\end{align*}
By definition, $\bm A\circ\bm B\succeq0.$
\end{enumerate}
\end{enumerate}