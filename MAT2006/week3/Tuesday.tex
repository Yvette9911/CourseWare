
\chapter{Week3}

\section{Tuesday}\index{week3_Tuesday_lecture}
\subsection{Application of Heine-Borel Theorem}
\begin{theorem}
Let $f(x)=\sum_{k=0}^\infty a_kx^k$ which converges in $|x|<1$. If for every $x\in[0,1)$, there exists $n (=n(x))$ such that $\sum_{n+1}^\infty a_kx^k=0$, then $f$ is a polynomial, i.e., $n$ does not depend on $x$.
\end{theorem}
\begin{proof}
Let $E_N:=\{x\in[0,\frac{1}{2}]\mid \sum_{k=N+1}^\infty a_kx^k=0\}$. It follows that
\[
[0,\frac{1}{2}] = \bigcup_{N=1}^\infty E_N,
\]
which implies that at least one $E_N$ is uncountable, say, $E_m$ is uncountable. (In particular, $E_m$ is infinite)

Therefore, (B-W) there $\exists $ a sequence $x_1,x_2,\dots,x_k,\dots\to x_0\in E_m$ as $E_m$ is closed.

Hence, $f(x)=a_0+a_1x+\cdots+a_mx^m$ holds for the sequence $\{x_1,x_2,\dots\}$. Hence we conclude the power series and the analytics function coincide each other:
\[
f(x)\equiv a_0+a_1x+\cdots+a_mx^m
\]

\end{proof}

\begin{proposition}
Let $g$ be analytic, i.e., $g(x)= b_0+b_1x+\dots+b_nx^n+\cdots$ on $(-1,1)$; and $g(x_k)=0$ for all $k\ge1$, where $\{x_k\}\to x_0$ (change 0 for simplicity). Then $g\equiv0$ on $(-1,1)$ (i.e., $b_0=b_1=\cdots=0$)
\end{proposition}
First, observe that $g(0)=0$ due to continuity property.

At the same time, $g(0)=b_0=0$. It follows that
\[
g(x) = x(b_1+b_2x+\cdots+b_nx^{n-1}+)
\]
Note that
\[
0=g(x_k) = x_k(b_1+b_2x_k+\cdots+b_nx^{n-1}_k+)
\]
Taking limit both sides, we derive $b_1=0$.

Hence, $g(x) = x^2(b_2+b_3x+\cdots)$ Note $0=g(x_k) = x_k^2(b_2+b_3x+\cdots)\implies b_2=0$.

We can show $b_k=0$ for $\forall k$. The remaining proof requires induction.

Now we talk about something mature for understanding.

\subsection{Set Structure Analysis}
\begin{definition}[Nowhere Dense]
A set $\bm B$ is said to be \emph{nowhere dense} if its closure $\overline{B}$ contain no non-empty open set.
\end{definition}
For example,
\[
B = \{1,\frac{1}{2},\frac{1}{3},\dots,\frac{1}{n},..\}\implies
\overline{B} = B\bigcup\{0\},
\]
which contains no open set.

\begin{definition}[$1$st category]
A set of $\bm B$ is said to be of $1$st category if it can be written as the \emph{union} of \emph{finitely} many or \emph{countably} many \emph{nowhere} dense sets.
\end{definition}
\begin{definition}[$2$rd category]
A set is said to be of $2$rd category if it is \emph{not} of $1$st category
\end{definition}
\begin{theorem}[Baire-Category Theorem]
$\mathbb{R}$ is of $2$rd category, i.e., $\mathbb{R}$ cannot be written as the union of countably many nowhere dense sets; or equivalently, if $\mathbb{R}=\bigcup_{n=1}^\infty A_n$, then at least one $A_n$ whose closure contains a non-empty set.
\end{theorem}

\begin{proof}
Assume $\mathbb{R} = \bigcup_{n=1}^\infty A_n$ such that all $A_n$'s are nowhere dense. It follows that
\[
\mathbb{R}\setminus \overline{A_1}\mbox{ is open}.
\]
We choose an open set $N_1$ such that $\overline{N_1}\subseteq\mathbb{R}\setminus \overline{A_1}$. Since $A_2$ is nowhere dense, we imply $\overline{A_2}$ does not contain $N_1$, i.e., $N_1\setminus \overline{A_2}$ is open; choose an open set $N_2$ such that $\overline{N_2}\subseteq N_1\setminus\overline{A_2}$.

$A_3$ is nowhere dense, i.e., $\overline{A_3}$ contains no open set. Thus $N_2\setminus\overline{A_3}$ is non-empty open set; choose open set ...

Repeating this process, weobtain a sequence of nested  sets $\overline{N_1}\supseteq N_1\supset\overline{N_2}\supset N_2\cdots$. The cantor's theorem implies that $\bigcap_{k=1}^\infty\overline{N_k}\ne\emptyset$.

On the other hand, $\bigcap_{k=1}^\infty\overline{N_k}\subseteq\mathbb{R}\setminus \bigcup_{n=1}^mA_n$ for any finite $m$.

Therefore, $\emptyset\ne\bigcup_{k=1}^\infty\overline{N_k}\subseteq\mathbb{R}\setminus \bigcup_{n=1}^\infty A_n=\emptyset$
\end{proof}

most continuous function is nowhere differentiable.
converge pointwise

review: sequence and series.

\begin{remark}
$\mathbb{R}$ is of 2nd category, i.e., if $\mathbb{R} = \bigcup_{n=1}^\infty A_m$, then at least $A_n$ whose closure contains a \emph{non-empty} open sets; The theorem also holds if we replace $\mathbb{R}$ by a \emph{complete} metric space (essentially the same proof).
\end{remark}

For $\mathbb{R}$, $d(x,y) = |x-y|$, so it is a metric space; $\mathbb{Q}$ is also metric space.

The second example is $\mathbb{R}^n$, with $d(\bm x,\bm y) = |\bm x - \bm y|$.

The set of all bounded sequences, with $d(\{x_n\},\{y_n\}) = \sup\{|x_i - y_i|\mid i=1,2,\dots\}$

The set of all bounded continuous functions on $\mathbb{R}$ (different domains), with $d_1(f,g) = \sup\{|f(x) - g(x)|\mid x\in\mathbb{R}\}$, or $d_2(f,g) =( \int_0^1|f(x) - g(x)|^2\diff x)^{1.2}$ Note that $(\xi[0,1],d_1)$ is complete, and $(\xi[0,1],d_2)$ is not complete. (exercise)

Different distance definition corresponds to different metric spaces.

Complete: all Cauchy sequence converge.
\begin{definition}[Metric Space]
\begin{itemize}
\item

\item

\end{itemize}•

\end{definition}


To show that most continuous function is nowhere differentiable, we will apply the Baire Category Theorem on $(\xi[0,1],d_1)$

\subsection{Reviewing}
\begin{definition}[Sequence]
$f:\mathbb{N}\to\mathbb{R}$, denoted as $\{f(0), f(1),\dots\}$l conventionally we denote it as $x_1,x_2,\dots$
\end{definition}
\begin{definition}
A number $\alpha$ is the limit of $\{x_1,x_2,\dots\}$ if $\forall\epsilon>0$, there $\exists N = N(\epsilon)$ such that $|x_k-\alpha|<\epsilon$ for $\forall k\ge N$, denoted by $\alpha_n\to\alpha$
\end{definition}
\begin{definition}
\[
\lim\inf_{k\to\infty}x_k :=\lim_{n\to\infty}\inf_{k\ge n}x_k
\]
which is the smallest limit point of the sequence
\[
\lim\sup_{k\to\infty}x_k :=\lim_{n\to\infty}\sup_{k\ge n}x_k
\]
which is the largest limit point of the sequence.
\end{definition}
A sequence always have liminf and limsup.
\begin{definition}[Partial Sum]
The series $\sum_ia_i$, the partial sum are defined as:
\[
s_n = a_1+\cdots+a_n
\]
the sum is defined as the limit of the partial sum,
\end{definition}













