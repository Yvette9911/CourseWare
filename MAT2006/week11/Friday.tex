
\section{Friday}\index{week8_Thursday_lecture}
\subsection{Analysis on IFT}
This lecture will talk about the full verison of IFT.
\paragraph{Elementrary Version}$F: U(x_0,y_0)\to\mathbb{R}$, which is $\mathcal{C}^p, p\ge1$.
\[
\begin{array}{ll}
F(x_0,y_0)=0,
&
F_y(x_0,y_0)\ne0
\end{array}
\]
which implies that there exists $I_x=\{x\mid |x-x_0|<\alpha\}$ and $I_y=\{y\mid |y-y_0|<\beta\}$ and $f\in\mathcal{C}^p(I_x;I_y)$ such that
\[
F(x,y)=0\quad\forall (x,y)\in I_x\times I_y\Longleftrightarrow
y = f(x)
\]
and
\[
f'(x)=-\frac{F_x(x,f(x))}{F_y(x,f(x))}
\]


\paragraph{Generalized version}
$F: U(\bm x_0,y_0)(\subseteq\mathbb{R}^m\times\mathbb{R})\to\mathbb{R}$, which is $\mathcal{C}^p,p\ge1$.
\[
\begin{array}{ll}
F(\bm x_0,y_0)=0,
&
F_y(\bm x_0,y_0)\ne0
\end{array}
\]
which implies that there exists $I_{\bm x}=\{\bm x\in\mathbb{R}^m\mid |\bm x-\bm x_0|<\alpha\}$ and $I_y=\{y\in\mathbb{R}\mid |y-y_0|<\beta\}$ and $f\in\mathcal{C}^p(I_{\bm x};I_y)$ such that
\[
F(\bm x,y)=0\quad\forall (\bm x,y)\in I_{\bm x}\times I_y\Longleftrightarrow
y = f(\bm x)
\]
and
\[
Df(\bm x)=-\frac{1}{F_y(\bm x,f(\bm x))}D_{\bm x}F(\bm x,f(\bm x))
\]
where $Df(\bm x)=\nabla\trans f(\bm x)$ and $D_{\bm x}F(\bm x,f(\bm x))=\nabla_{\bm x}\trans F(\bm x,f(\bm x))$.

\paragraph{Full Version}
$F: U(\bm x_0,\bm y_0)(\subseteq\mathbb{R}^m\times\mathbb{R}^n)\to\mathbb{R}^n$, which is $\mathcal{C}^p,p\ge1$.
\[
\begin{array}{ll}
F(\bm x_0,\bm y_0)=\bm 0,
&
D_yF(\bm x_0,\bm y_0)\mbox{ is invertible}
\end{array}
\]
which implies that there exists $I_{\bm x}=\{\bm x\in\mathbb{R}^m\mid |\bm x-\bm x_0|<\alpha\}$ and $I_{\bm y}=\{\bm y\in\mathbb{R}^n\mid |\bm y-\bm y_0|<\beta\}$ and $f\in\mathcal{C}^p(I_{\bm x};I_{\bm y})$ such that
\[
F(\bm x,\bm y)=0\quad\forall (\bm x,\bm y)\in I_{\bm x}\times I_{\bm y}\Longleftrightarrow
\bm y = f(\bm x)
\]
and
\[
Df(\bm x) = -[D_{\bm y}F(\bm x,f(\bm x))]^{-1}D_{\bm x}F(\bm x,f(\bm x))
\]
where $Df(\bm x)\in\mathbb{R}^{n\times m}$; $D_{\bm y}F(\bm x,f(\bm x))\in\mathbb{R}^{n\times n}$; and $D_{\bm x}F(\bm x,f(\bm x))\in\mathbb{R}^{n\times m}$.
\begin{proof}
Fix $m$, induction on $n$.
\begin{enumerate}
\item
As $n=1$, it is done.
\item
The rest are similar to the proof in elementary version.
\end{enumerate}
\end{proof}

In this lecture and next upcoming Wednesday, we will talk about the application of IFT. For example, how to apply Chain rule to differentiate; how to compute Jacobian matrix, and the inverse. Pay attention to computational aspect. This will show up in second quiz, as well as the final, too.

\subsection{Applications on IFT}
\paragraph{Inverse Function Theorem}
\begin{theorem}
Given a function $f:E\to\mathbb{R}^m$, where $E$ is a \emph{domain} (pre-assume it is connected) in $\mathbb{R}^m$ with the property that:
\begin{enumerate}
\item
$f\in\mathcal{C}^p(E;\mathbb{R}^m), p\ge1$
\item
$Df(\bm x_0)$ is invertible, where $\bm x_0\in E$
\end{enumerate}
which implies that
\begin{enumerate}
\item
$g$ is invertible near $f(\bm x_0)(:=\bm y_0)$, i.e., there exists $U(\bm x_0)\subseteq E$ and $V(\bm y_0)\in\mathbb{R}^m$ such that $g$ is a $\mathcal{C}^p$-diffeomorphism from $U(\bm x_0)$ to $V(\bm x_0)$; and
\[
Dg(\bm y)=[Df(g(\bm y))]^{-1}
\]

Note that $\mathcal{C}^p$diffeomorphism means $g$ is one-to-one onto mapping and $f\in\mathcal{C}^p$.
\end{enumerate}
\end{theorem}
\begin{proof}
Define $F(\bm x,\bm y)=f(\bm x)-\bm y$, $F:E\times\mathbb{R}^m\to\mathbb{R}^m$ is $\mathcal{C}^p$.

$D_{\bm x}F(\bm x_0;\bm y_0) = Df(\bm x_0)$ is invertible.

$F(\bm x_0,\bm y_0)=0$.

Applying IFT, we imply that there exists a neighborhood $I_{\bm x}=\{\bm x\in E\mid |\bm x-\bm x_0|<\alpha\}$ and $I_{\bm y}=\{\bm y\in\mathbb{R}^m\mid |\bm y-\bm y_0|<\beta\}$ and $g\in\mathcal{C}^p(I_{\bm y};I_{\bm x})$ such that
\[
F(\bm x,\bm y)=0\quad\forall (\bm x,y)\in I_{\bm x}\times I_y\Longleftrightarrow
\bm x=g(\bm y),
\]
i.e., $f(g(\bm y))=\bm y$ iff $\bm x=g(\bm y)$; and
\[
Dg(\bm y)=-[D_{\bm x}F(g(\bm y), \bm y)]^{-1}D_{\bm y}F(g(\bm y),\bm y).
\]
Notet that $D_{\bm y}F(g(\bm y),\bm y)=-\bm I$, and therefore
\[
Dg(\bm y)=[Df(\bm x)]^{-1}
\]
\end{proof}
Pay attention to the order of derivative when applying full version IFT.

\paragraph{Rank Theorem}
\begin{definition}[Rank]
The \emph{rank} of a vector function $f:U(\subseteq\mathbb{R}^m)\to\mathbb{R}^n$ at a point $\bm x\in U$ is defined to be the \textit{rank} of $Df(\bm x)$.
\end{definition}
%
\begin{theorem}[Rank Theorem]
Suppose $f\in\mathcal{C}^p(U(\bm x_0;\mathbb{R}^n))$, where $U(\bm x_0)$ is a neighborhood of $\bm x_0\in\mathbb{R}^m$. If $f$ has the same constant rank $k$ at every point $\bm x\in U(\bm x_0)$, then there exists a neighborhood $N(\bm x_0)$ of $\bm x_0$ and a neighborhood $N(\bm y_0)$ of $\bm y_0:=f(\bm x_0)$ and two $\mathcal{C}^p$-diffeomorphism,
\[
\begin{array}{ll}
\mbox{$u=\phi(\bm x)$ in $N(\bm x_0)$}
&
\mbox{$v=\psi(\bm y)$ in $N(\bm y_0)$}
\end{array}
\]
such that $v = \psi\circ f\circ\phi^{-1}(u)$ takes the form
\[
\bm u:=(u_1,\dots,u_k,u_{k+1},\dots,u_m)\to
v=(v_1,\dots,v_n):=v(u_1,\dots,u_k,0,\dots,0)
\]

\end{theorem}
\begin{remark}
Given $f:\bm x\in U(\bm x_0)\to \bm y\in V(\bm y_0)$, if $f$ has constant rank $k$, then we have
\[
\phi^{-1}:u\in \phi(N(\bm x_0))\to U(\bm x_0)
\]
and
\[
\psi: y\in V(\bm y_0)\to v\in \psi^{-1}(N(\bm y_0))
\]
and
\[
u=(u_1,\dots,u_k,u_{k+1},\dots,u_m)\to
(u_1,\dots,u_k,0,0,\dots,0).
\]
\end{remark}

\begin{proof}
(I) For $f:=(f_1,\dots,f_n)$, we have
\[
Df(\bm x_0)=\begin{pmatrix}
\frac{\partial f_1}{\partial x_1}&\cdots&\frac{\partial f_1}{\partial x_m}\\
\vdots&\ddots&\vdots\\
\frac{\partial f_n}{\partial x_1}&\cdots&\frac{\partial f_n}{\partial x_m}
\end{pmatrix}(\bm x_0)
\]
w.l.o.g., the first $k$ pricipal minor 
\[
\begin{pmatrix}
\frac{\partial f_1}{\partial x_1}&\cdots&\frac{\partial f_1}{\partial x_k}\\
\vdots&\ddots&\vdots\\
\frac{\partial f_k}{\partial x_1}&\cdots&\frac{\partial f_k}{\partial x_k}
\end{pmatrix}(\bm x_0)
\]
is non-singular, which implies that there exists $N(\bm x_0)$ such that for $\forall x\in N(\bm x_0)$,
\[
\begin{pmatrix}
\frac{\partial f_1}{\partial x_1}&\cdots&\frac{\partial f_1}{\partial x_k}\\
\vdots&\ddots&\vdots\\
\frac{\partial f_k}{\partial x_1}&\cdots&\frac{\partial f_k}{\partial x_k}
\end{pmatrix}(\bm x)
\]
is non-singular.

(II) Define
\[
\phi(\bm x)=\begin{pmatrix}
f_1(\bm x)\\\vdots\\f_k(\bm x)\\ x_{k+1}
\end{pmatrix}
\]
then
\[
D\phi(\bm x)=\begin{pmatrix}
\frac{\partial f_1}{\partial x_1}&\cdots\frac{\partial f_1}{\partial x_k}\mid &\\
\vdots&\ddots&\vdots\\
\frac{\partial f_k}{\partial x_1}&\cdots\frac{\partial f_k}{\partial x_k}\mid &\\
&&&\bm I
\end{pmatrix}
\]
which is invertible as well, which implies that $\phi^{-1}$ exists, and $\phi$ is a $\mathcal{C}^p$-differentiaerom.

(III) Let $g=f\circ\phi^{-1}(\bm u)$, which follows that
\begin{align*}
y_1&=f_1\circ\phi^{-1}(u_1,\dots,u_m)=u_1\\
\vdots\\
y_k&=f_k\circ\phi^{-1}(u_1,\dots,u_m)=u_k\\
y_{k+1}&=f_{k+1}\circ\phi^{-1}(u_1,\dots,u_m)=g_{k+1}(u_1,\dots,u_m)\\
\vdots\\
y_n&=f_{n}\circ\phi^{-1}(u_1,\dots,u_m)=g_{n}(u_1,\dots,u_m)
\end{align*}
Note that $Dg(u)=Df(\phi^{-1}(u))\circ D\phi^{-1}(u)$ has rank $k$ at every point, but note that
\[
Dg(u)=\begin{pmatrix}
\bm I_{k\times k}&\bm0\\
***&\frac{\partial g[k+1:n]}{\partial u[k+1:m]}
\end{pmatrix}
\]
which implies $\frac{\partial g[k+1:n]}{\partial u[k+1:m]}$ is a zero matrix, which implies $g[k+1:n]$ is independent of $u[k+1:m]$, i.e., 
\[
g_{k+1}(u_1,\dots,u_m)=g_{k+1}(u_1,\dots,u_k). 
\]

(IV) Define
\[
\psi(y)=v=(v_1,\dots,v_n),
\]
where
\begin{align*}
\psi_1(y)&=y_1\\
&\vdots\\
\psi_k(y)&=y_k\\
\psi_{k+1}(y)&=y_{k+1} - g_{k+1}(y_1,\dots,y_k)\\
&\vdots\\
\psi_{n}(y)&=y_n - g_n(y_1,\dots,y_k)
\end{align*}
and 
\[
D\psi(\bm y_0)=\begin{pmatrix}
\bm I&\bm0\\
***&\bm I
\end{pmatrix},
\]
which is invertible. 

(V)  It follows that $\psi\circ f\circ\phi^{-1} = \psi\circ g$, which maps $(u_1,\dots,u_k,u_{k+1},u_m)$ to 
\[
(u_1,\dots,u_k, 0,0,\dots,0).
\]



\end{proof}



















