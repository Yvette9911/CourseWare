
\section{Friday}\index{week7_Friday_lecture}
\subsection{Recap}
\paragraph{Announcement}
As promised, the taylor series and uniform continuity will definitely show up in the mid-term. (remember the Theorem(\ref{The:4:2})?)
\paragraph{Summary about Taylor series}
This lecture will mainly discuss the integration, but first let's review what we have learnt from last lecture.

The Taylor series has its connection with complex numbers:
\begin{example}
Given a function $f(x)=\frac{1}{1+x^2}$, which is infinitely differentiable. However, when expanding its Taylor series at point $x=0$, we have
\[
f(x)=\frac{1}{1-(-x^2)}=1-x^2+x^4-x^6+\cdots,\quad\mbox{holds for $x^2<1$}
\]
\begin{quotation}
Why would the function $f$ have taylor series convergent only hold for $x^2<1$, but it is infinitely differentiable on the whole real line? 
\end{quotation}
\begin{proposition}
The function $f(z)$ for $z\in\mathbb{C}$ is infinitely differentiable on domain $D$ iff it has Taylor series convergent on $D$. Note that real function does not necessarily have such a property.
\end{proposition}

If extending the domain into complex plane, the function $\frac{1}{1+z^2}$ have poles $\pm i$, and thus have no chance to have taylor expansion beyond $|z|<1$. Then
 projecting the domain $\{z:|z|<1\}$ into the real line, we derive the function $f$ have Taylor series convergent for $|x|<1$.
\end{example}
\paragraph{Exercise}
Exercise: find the taylor series of $\frac{1}{1+x^2}$ at $x=1$ and determine its radius of convergence. Answer: $R=\sqrt{2}$
\subsection{Riemann Integration}
\paragraph{Set Up}
Given a \emph{bounded} function $f$ on the closed (finite) interval $[a,b]$. A partition $\mathcal{P}$ is a set of points $\{x_i\}_{i=0}^n$:
\[
a_1=x_0\le x_1\le x_2\le\cdots\le x_n=b
\]
where the \emph{mesh} of $\mathcal{P}$ is defined to be $\lambda(\mathcal{P})=\max_{1\le i\le n}|\Delta x_i|$.

On each interval $[x_{i-1},x_i]$, define 
\[
\begin{array}{ll}
m_i=\inf_{x_{i-1}\le x\le x_i}f(x),
&
M_i=\sup_{x_{i-1}\le x\le x_i}f(x)
\end{array}
\]
The lower sum and upper low sum associated with partition $\mathcal{P}$ is defined as:
\begin{align*}
L(\mathcal{P},f)&=\sum_{i=1}^{n} m_i(x_i - x_{i-1})
=\sum_{i=1}^n m_i\Delta x_i\\
U(\mathcal{P},f)&=\sum_{i=1}^{n} M_i(x_i - x_{i-1})
=\sum_{i=1}^n M_i\Delta x_i
\end{align*}
Now we define the lower and upper Riemann intergral as:
\begin{align*}
\loRiemannint{a}{b}f(x)\diff x&=\sup_{\mathcal{P}}L(\mathcal{P},f)\\
\upRiemannint{a}{b}f(x)&=\inf_{\mathcal{P}}U(\mathcal{P},f)
\end{align*}
These definitions are well-defined.
\begin{definition}[integrable]
We say that $f$ is (Riemann) integrable if \[\loRiemannint{a}{b}f(x)\diff x=\upRiemannint{a}{b}f(x)\diff x,\] and denote the common values of which as $\int_a^bf(x)\diff x$. The set of all \emph{Riemann integrable functions} on $[a,b]$ is denoted as $\mathcal{R}[a,b]$.
\end{definition}
\begin{example}
Check if these functions are integrable or not:
\begin{enumerate}
\item
$f(x)\equiv 1$ on $[0,1]$; then $\loRiemannint{a}{b}f(x)\diff x=\upRiemannint{a}{b}f(x)=1$.
\item
Dirichlet function:
\[
D(x)=\left\{
\begin{aligned}
0,&\quad x\notin\mathbb{Q}\\
1,&\quad x\in\mathbb{Q}
\end{aligned}
\right.
\]
Verify that this function always has lower sum $0$ and upper sum $1$, and therefore not integrable.
\item
Riemann function on $[0,1]$:
\[
R(x)=\left\{
\begin{aligned}
0,&\quad x\notin\mathbb{Q}\\
\frac{1}{q},&\quad x=\frac{p}{q}, q>0, (p,q)=1
\end{aligned}
\right.
\]
We will show that it is integrable later. First keep in mind that the function with countably many discontinuites on $[a,b]$ is integrable.
\item
The function defined on $[0,1]$:
\[
f(x)=\left\{
\begin{aligned}
0,&\quad x=0\\
\sin\frac{1}{x},&\quad x\ne0
\end{aligned}
\right.
\]
It is integrable.
\item
Evaluate the limit from definition
\[
\lim_{n\to\infty}\left[\frac{1}{n+1}+\frac{1}{n+2}+\cdots+\frac{1}{2n}\right].
\]
Construct
\[
x_n=\sum_{k=1}^n\frac{1}{n+k}=\frac{1}{n}\sum_{k=1}^n\frac{1}{1+k/n}
\]
Note that $x_n$ is just the lower sum associated with the partitition
\[
\mathcal{P}=\{x_0:=0,x_1=\frac{1}{n},x_2=\frac{2}{n},\dots,x_n:=1\}.
\]
As $n\to\infty$, inituitively $\mathcal{L}(\mathcal{P},f)\to\mathcal{U}(\mathcal{P},f)$. Therefore
\[
x_n\to\int_0^1\frac{\diff x}{1+x}=\log2
\]
\end{enumerate}
\end{example}
\begin{definition}[Refinement]
Given a partition $\mathcal{P}$, we say $\mathcal{P}^*$ is a refinement of $\mathcal{P}$ if $\mathcal{P}^*$ contains all the sub-division points of $\mathcal{P}$
\end{definition}
\begin{proposition}
Let $f: [a,b]\mapsto\mathbb{R}$ with $m\le f(x)\le M$ on $[a,b]$, then
\begin{enumerate}
\item
$L(\mathcal{P},f)\le L(\mathcal{P}^*,f)$ and $U(\mathcal{P}^*,f)\le U(\mathcal{P},f)$ holds for any refinement $\mathcal{P}^*$ of $\mathcal{P}$
\item
$L(\mathcal{P}_1,f)\le U(\mathcal{P}_2,f)$ for any refinements $\mathcal{P}_1,\mathcal{P}_2$.
\item
\[
m(b-a)\le \loRiemannint{a}{b}f(x)\diff x\le
\upRiemannint{a}{b}f(x)\diff x\le M(b-a)
\]
\item
$f$ is \emph{Riemann integrable} iff $\forall\varepsilon$, there exists $\mathcal{P}$ s.t. $U(\mathcal{P},f) - L(\mathcal{P},f)\le\varepsilon$.
\end{enumerate}
\end{proposition}
\begin{proof}
\begin{enumerate}
\item
Check Theorem(6.4) in Rudin's book for detail
\item
Take the $\mathcal{P}^*$ as common refinement for $\mathcal{P}_1,\mathcal{P}_2$, and show that
\[
L(\mathcal{P}_1,f)\le 
L(\mathcal{P}^*,f)\le U(\mathcal{P}^*,f)\le
U(\mathcal{P}_2,f)
\]
\item
For every $\mathcal{P}$,
\[
m(b-a)\le L(\mathcal{P},f)\le U(\mathcal{P},f)\le M(b-a).
\]
\item
Check Theorem(6.6) in Rudin's book for detail.
\end{enumerate}
\end{proof}
\begin{theorem}
If $f$ is continuous on $[a,b]$, then $f$ is Riemann integrable on $[a,b]$.
\end{theorem}

\begin{proof}
The function $f$ is continuous on $[a,b]$ implies $f$ is uniform continuous, i.e.,$\forall \varepsilon>0,\exists\delta>0$ s.t. for $|x-y|<\delta$,
\[
|f(x)-f(y)|<\varepsilon.
\]
Pick a partition $\mathcal{P}=\{x_0:=a,x_1:=a+h,x_2:=a+2h,\dots,x_n:=a+nh:=b\}$ with $h=\frac{b-a}{n}<\delta$. It follows that on interval $[x_{i-1},x_i]$, we have
\[
M_i-m_i<\varepsilon\implies
U(\mathcal{P},f) - L(\mathcal{P},f) = \sum_{i=1}^n(M_i-m_i)\Delta x_i\le\varepsilon\sum_{i=1}^n\Delta x_i=\varepsilon(b-a)
\]
\end{proof}
We aeert a proposition but without proof first:
\begin{corollary}
If $f$ is continuous expect for finitely many points on $[a,b]$, then $f$ is Riemann integrable.
\end{corollary}
We can apply this theorem to show that $f(x)=\sin\frac{1}{x}$ is integrable, since $f$ has only one discontinuity on $[0,1]$. 

Alternatively, we can separate the interval $[0,1]$ as $[\epsilon,1]$ and $[0,\epsilon]$. Note that on $[\epsilon,1]$ $f$ is continuous, thus Riemann integrable, and then we can bound the difference of lower and upper sum by taking $\epsilon\to0$.

We can also apply this corollary to show the Riemann function is integrable, but we need another useful tool: uniform convergence.
\begin{theorem}
Let $\{f_n\}$ be a sequence of \emph{Riemann integrable} functions on $[a,b]$, and $f_n$ converges uniformly to $f$. Then $f$ is Riemann integrable and 
\[
\int_a^bf(x)=\lim_{n\to\infty}\int_a^bf_n
\]
\end{theorem}
\begin{remark}
Note that this theorem essentially says that a sequence of uniformly convergent function $\{f_n\}$ has the property
\[
\int_a^b\lim_{n\to\infty}f_n(x)\diff x=\lim_{n\to\infty}\int_a^bf_n(x)\diff x
\]
which may not necessarily true if we ignore the uniform convergence condition. 
\end{remark}
\begin{example}[counter-example for pointwise convergent function sequence]
For the function sequence $\{f_n\}$ with
 \[
f_n(x)=\left\{
\begin{aligned}
n,&\quad x\in[0,\frac{1}{n})\\
0,&\quad x\notin(0,\frac{1}{n})
\end{aligned}
\right.
\]
we find that 
\[
f=\lim_{n\to\infty}f_n=0\implies
\int_a^b f\diff x=0,
\]
while $\int_0^1f_n(x)\diff x=1$.
\end{example}
Let's review the definition for uniform convergence:
\begin{definition}[Uniform Convergence]
Let $f$ be the pointwise limit of $f_n$, then $f_n$ is said to converge uniformly to $f$ if 
\[
\sup_{a\le x\le b}|f_n(x) - f(x)|\to0,\mbox{ as }n\to\infty.
\]
\end{definition}
\begin{remark}
The value $\sup_{a\le x\le b}|f_n(x) - f(x)|$ tends to zero for pointwise convergent function $f_n$, e.g., $f_n(x) = x^n$ for $x\in[0,1]$.
\end{remark}

\begin{corollary}
Riemann function is Riemann integrable.
\end{corollary}
\begin{proof}
Construct the function
\[
f_N(x)=\left\{
\begin{aligned}
\frac{1}{q},&\quad x=\frac{p}{q}, 0<q<N, (p,q)=1,\\
0,&\quad \mbox{otherwise}\\
\end{aligned}
\right.
\]
Thus $f_N$ is Riemann integrable since it contains finitely many discontinuities. We can also show it is uniformly convergent with limit $R(x).$ Therefore the Riemann function is Riemann integrable.
\end{proof}












