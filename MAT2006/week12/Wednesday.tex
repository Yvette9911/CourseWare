
\chapter{Week12}
\section{Wednesday}
\subsection{Recap for Rank Theorem}
\paragraph{Inverse Function Theorem}
Given a $\mathcal{C}^p$ function $f:E(\subseteq\mathbb{R}^m)\subseteq\mathbb{R}^m$ and $Df(\bm x_0)$ is invertible. Then we imply that there is a neighborhood $U(\bm x_0)\times V(\bm y_0)$ of $(\bm x_0,f(\bm x_0))$ such that $f$ is a $\mathcal{C}^p$-diffeomorphism between $U(\bm x_0)$ and $V(\bm y_0)$; moreover,
\[
D(f^{-1})(\bm y_0)=(Df(\bm x_0))^{-1}
\]

\paragraph{Rank Theorem}
Given a $\mathcal{C}^p$ function $f:U(\bm x_0)\to\mathbb{R}^n$ of constant rank $k$ throughout $U(\bm x_0)$. Then there exists a neighborhood $N(\bm x_0)\times N(f(\bm x_0))$ and two $\mathcal{C}^p$-diffeomorphisms
\[
\begin{array}{ll}
\bm u=\phi(\bm x), \bm x\in N(\bm x_0)
&
\bm v=\psi(\bm y), \bm y\in N(\bm y_0), \bm y_0:=f(\bm x_0),
\end{array}
\]
such that the composition $\psi\circ f\circ\phi^{-1}$ takes the form
\[
(u_1,\dots,u_k,u_{k+1},\dots,u_m)\to
(u_1,\dots,u_k,0,0,\dots,0)
\]
\paragraph{Outline of proof}
Step 1:
\[
f=\begin{pmatrix}
f_1\\\vdots\\f_k\\\vdots\\f_n
\end{pmatrix},\qquad
Df=\frac{\partial(f_1,\dots,f_n)}{\partial(x_1,\dots,x_m)}
\]
w.l.o.g., assume the first $k\times k$ principal minors to be non-singular.

Step 2: Then construct the map $\phi(\bm x)$
\[
\phi(\bm x)=\begin{pmatrix}
f_1(\bm x)\\\vdots\\f_k(\bm x)\\x_{k+1}\\\vdots\\x_m
\end{pmatrix}\implies
D\phi=\begin{pmatrix}
\frac{\partial(f_1,\dots,f_k)}{\partial(x_1,\dots,x_k)}&
\frac{\partial(f_1,\dots,f_k)}{\partial(x_{k+1},\dots,x_m)}
\\
\bm0&\bm I
\end{pmatrix}
\]
which is invertible.

Step 3: define $g:=f\circ \phi^{-1}: \phi(N(\bm x_0))\to\mathbb{R}^n$, then rewrite $g$ as
\[
\begin{pmatrix}
y_1\\\vdots\\y_k\\y_{k+1}\\\vdots\\y_n
\end{pmatrix}
=
\begin{pmatrix}
u_1\\\vdots\\u_k\\g_{k+1}(\bm u)\\\vdots\\g_n(\bm u)
\end{pmatrix}
\implies
Dg=\begin{pmatrix}
\bm I&\bm0\\
&\frac{\partial (g_{k+1},\dots,g_n)}{\partial(u_{k+1},\dots,u_n)}
\end{pmatrix},
\]
which implies the lower right corner should be zero matrix, i.e., $(g_{k+1},\dots,g_n)(\bm u)$ depends only on the first $k$ variables. Thus rewrite $g$ as:
\[
\begin{pmatrix}
y_1\\\vdots\\y_k\\y_{k+1}\\\vdots\\y_n
\end{pmatrix}
=
\begin{pmatrix}
u_1\\\vdots\\u_k\\g_{k+1}(u_1,\dots,u_k)\\\vdots\\g_n(u_1,\dots,u_k)
\end{pmatrix}
\]

Step 4: Define the map $\bm v=\psi(\bm y)$:
\[
\begin{pmatrix}
v_1\\\vdots\\v_k\\v_{k+1}\\\vdots\\v_n
\end{pmatrix}
=
\begin{pmatrix}
y_1\\\vdots\\y_k\\y_{k+1}-g_{k+1}(y_1,\dots,y_k)\\\vdots\\
y_{k+1} - g_n(y_1,\dots,y_k)
\end{pmatrix}
\]
flatten out
\begin{example}
\begin{enumerate}
\item
Define $f(t)=(\cos t,\sin t), t\in\mathbb{R}$. Define $t_0=\frac{\pi}{4}$. Can we flatten out the curve near $(\frac{\sqrt{2}}{2},\frac{\sqrt{2}}{2})$? Note that
\[
Df(\frac{\pi}{4})=(-\sqrt{2}/2,\sqrt{2}/2)\ne0,
\]
with rank $1$. The answer is yes.

Choose $\phi(t)=\cos t$ and $\phi^{-1}(u)=t=\cos^{-1}u$, which follows that
\[
g(u)=f(\phi^{-1}(u))=\begin{pmatrix}
\cos(\phi^{-1}u)\\\sin(\phi^{-1}(u))
\end{pmatrix}
=\begin{pmatrix}
u\\
\sin(\cos^{-1}u)
\end{pmatrix}
\]
Choose $\psi(y)=\begin{pmatrix}
y_1\\y_2-\sin(\cos^{-1}y_1)
\end{pmatrix}$, which follows that
\begin{align*}
\psi\circ f\circ\phi^{-1}(u)
&=
\psi\circ f(\cos^{-1}u)\\
&=\psi\begin{pmatrix}
\cos\cos^{-1}u
\\
\sin\cos^{-1}u
\end{pmatrix}=
\psi\begin{pmatrix}
u
\\
\sin\cos^{-1}u
\end{pmatrix}
=\begin{pmatrix}
u\\0
\end{pmatrix}
\end{align*}
\item
$f(x_1,x_2)=(x_1+x_2,x_1-x_2,x_1x_2)$. Can we flatten out the curve of $f$ near $(0,0)$?
\[
Df(x_1,x_2)=\begin{pmatrix}
1&1\\
1&-1\\
x_2&x_1
\end{pmatrix},
\]
which is of rank $2$ throughout $\mathbb{R}^2$.

Note that
\[
\phi(x_1,x_2)=\begin{pmatrix}
f_1\\f_2
\end{pmatrix}=\begin{pmatrix}
x_1+x_2\\x_1-x_2
\end{pmatrix}
\]
and
\[
g=f\circ\phi^{-1}(u_1,u_2)=f\begin{pmatrix}
\frac{u_1+u_2}{2}\\\frac{u_1-u_2}{2}
\end{pmatrix}
=
\begin{pmatrix}
u_1\\u_2\\\frac{u_1^2-u_2^2}{4}
\end{pmatrix}
\]
and define
\[
\psi(y)=\begin{pmatrix}
y_1\\y_2\\y_3-\frac{y_1^2-y_2^2}{4}
\end{pmatrix}.
\]
Thus in summary, we have
\[
\psi\circ f\circ \phi^{-1}\begin{pmatrix}
u_1\\u_2
\end{pmatrix}=
\psi\begin{pmatrix}
u_1\\u_2\\\frac{u_1^2-u_2^2}{4}
\end{pmatrix}
=
\begin{pmatrix}
u_1\\u_2\\0
\end{pmatrix}
\]

\end{enumerate}
\end{example}
\subsection{Functional Dependence}
In linear algebra we have talked about the linear independence. Given $n$ vectors $v_1,\dots,v_n$, they are linear dependent if $\exists$ $a_1,\dots,a_n$ not all zero such that
\[
a_1v_1+\cdots+a_nv_n=0
\]
Then we talk about the dependence between functions.
\begin{definition}[Dependence]
A set of \emph{continuous} functions $f_1,\dots,f_n: U\to\mathbb{R}$, where $U\subseteq\mathbb{R}^m$ is a neighborhood of $\bm x_0\in\mathbb{R}^m$, is said to be \emph{functionally independent} if for any continuous function
\[
F(y)=F(y_1,\dots,y_n)
\]
in a neighborhood $V$ of $\bm y_0=f(\bm x_0)=(f_1(\bm x_0),\dots,f_n(\bm x_0))$, the relation $F(f_1(\bm x),\dots,f_n(\bm x))\equiv0$ for $\forall \bm x\in U$ is the only possible when $F\equiv0$ in $V$.
\end{definition}
\begin{proposition}
Let $\{f_1,\dots,f_n\}$ be $\mathcal{C}^1$ and the rank of
\[
\frac{\partial (f_1,\dots,f_n)}{\partial(x_1,\dots,x_m)}
\]
is $k$ at every $\bm x\in U$, then 
\begin{enumerate}
\item
$k=n$ implies $\{f_1,\dots,f_n\}$ is functionally independent
\item
$k<n$ implies there exists a neighborhood of $\bm x_0$ and $k$ functions $f_1,\dots,f_k$ such that the rest of $(n-k)$ functions can be written as
\[
f_j(\bm x)=g_i(f_1(\bm x),\dots,f_k(\bm x))
\]
for $\forall i=k+1,\dots,n$, where $g_i$ are $\mathcal{C}^1$ funcitions of $k$ variables.
\end{enumerate}
\end{proposition}




















