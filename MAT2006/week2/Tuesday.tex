
\chapter{Week2}
\section{Wednesday}\index{week2_Tuesday_lecture}
\subsection{Review and Announcement}
The quiz results will not be posted.

In this lecture we study the number theories.

The office hour is 2 - 4pm, TC606 on Wednesday
\subsection{Different Numbers}
\begin{definition}[Algebraic Number]
A number $x\in\mathbb{R}$ is said to be an \emph{algebraic number} if it satisfies the following equation:
\begin{equation}
a_nx^n+a_{n-1}x^{n-1}+\cdots+a_1x+a_0=0
\end{equation}
where $a_n,a_{n-1},\dots,a_0$ are integers and not all zero. We say $x$ is \emph{of degree $n$} if $a_n\ne0$ and $x$ is not the root of any polynomial with lower degree.
\end{definition}
\begin{definition}
A number $x\in\mathbb{R}$ is \emph{transcendental} if it is not an algebraic number.
\end{definition}

The first example is that all rational numbers are algebraic, since rational number $\frac{p}{q}$ satisfies $qx-p=0$. Also, $\sqrt{2}$ is algebraic. The exercise is that show $e$ and $\pi$ are all transcendental. In history joseph liouville (1844) have constructed the first transcendental number.

\begin{proposition}
The set of all algebraic numbers is countable.
\end{proposition}
\begin{proof}
\begin{enumerate}
\item
Let $\mathcal{P}_n$ denote the set of all polynomials of degree $n$ (Here we assert polynomials have all integer coefficients by default.), i.e., 
\[
\mathcal{P}_n=\{a_nx^n+a_{n-1}x^{n-1}+\cdots+a_1x+a_0\mid a_j\in\mathbb{Z}\}
\]
The set $\mathcal{P}_n$ have the one-to-one mapping to the set $\{(a_n,a_{n-1},\dots,a_0)\mid a_j\in\mathbb{Z}\}\subseteq\mathbb{Z}^{n+1}$, which implies $\mathcal{P}_n$ is \emph{countable}.
\item
Let $\mathcal{R}_n$ denote the set of all real roots of polynomials in $\mathcal{P}_n$, which can be mapped into the set 
\[
\mathcal{P}_n\times\cdots\times\mathcal{P}_n
\]
which is countable. Hence, $\mathcal{R}_n$ is also countable.
\item
Construct the set of all algebraic numbers $\bigcup_{n=1}^\infty\mathcal{R}_n$. Countably union of countable sets is also countable.
\end{enumerate}
\end{proof}

How fast to approximate rational numbers? How fast to approximate irrational numbers using rational numbers? How fast to approximate transcendental numbers using rational numbers? We need the definition for the rate of approximation first to answer these questions above.

\begin{definition}
A real number $\xi$ is \emph{approximable} by \emph{rational numbers to order $n$} if $\exists$ a constant $K = K(\xi)$ such that the inequality
\[
\left|\frac{p}{q} - \xi\right|\le\frac{K}{q^n}
\]
has \emph{infinitely many} solutions $\frac{p}{q}\in\mathbb{Q}$ with $q>0$.
\end{definition}

\begin{example}
\begin{itemize}
\item
Intuitively, a rational number is approximable by rational numbers. Now we study its rate of approximation. It suffices to choose $(p_k,q_k)$ such that
\[
\left|
\frac{p_k}{q_k}-\frac{p}{q}
\right|\le\frac{K}{q^\alpha}
\]
Note that
\begin{align*}
\left|
\frac{p_k}{q_k}-\frac{p}{q}
\right|=\left|
\frac{p_kq-pq_k}{q_kq}
\right|\ge\frac{1}{qq_k}=\frac{1/q}{q_k}
\end{align*}
$\frac{p}{q}$ is approximable by rational numbers to order $1$ and no higher than $1$. (Exercise)
\end{itemize}
\end{example}

liouville had shown that the transcendental number has the higher approxiamtion rate than rational and algebraic numbers, which is counter-intuitive.

\begin{theorem}[Liouville, 1844]
A real algebraic number $\xi$ of degree $n$ is not approximable by rational numbers to any order greater than $n$.
\end{theorem}
\begin{example}[1st Constructed Transcendental Number]
We aim to show that
\[
\xi=\frac{1}{10^{1!}}+\frac{1}{10^{2!}}+\cdots
\]
is transcendental. We construct the first $n$ tails of $\xi$:
\[
\xi_n=\frac{1}{10^{1!}}+\frac{1}{10^{2!}}+\cdots+\frac{1}{10^{n!}}
\]
It follows that
\begin{align*}
|\xi-\xi_n|&=\frac{1}{10^{(n+1)!}}+\frac{1}{10^{(n+2)!}}+\cdots\\
&=\frac{1}{10^{(n+1)!}}\left[1+\frac{1}{10^{n+2}}+\frac{1}{10^{(n+2)(n+3)}}+\cdots\right]\\
&\le\frac{1}{10^{(n+1)!}}\cdot 2 = \frac{2}{(10^{n!})^{n+1}}=\frac{2}{q^{n+1}}
\end{align*}
which implies $|\xi-\frac{p}{q}|\le\frac{c}{q^{n+1}}$ has one solution $\xi_n$. However, we can construct infinitely many solutions from this solution.
\end{example}

\begin{proof}
For given algebraic number $\xi$ of degree $n$, there $\exists$ a polynomial whose roots contain $\xi$:
\[
f(x) \equiv a_nx^n+\cdots+a_1x+a_0=0.
\]
It follows that
\[
\left|f(\frac{p}{q})\right|=\left|a_n\frac{p^n}{q^n}+\cdots+a_1\frac{p}{q}+a_0\right|
=\left|\frac{a_np^n+a_{n-1}p^{n-1}q+\cdots+a_0q^n}{q^n}\right|\ne0
\]
Hence, $\left|f(\frac{p}{q})\right|\ge\frac{1}{q^n}$. Or equivalently,
\[
\frac{1}{q^n}\le\left|f(\frac{p}{q})\right| = \left|f(\frac{p}{q})-f(\xi)\right|\le
|f'(\eta)||\xi-\frac{p}{q}|\le M|\xi-\frac{p}{q}|\implies
|\xi-\frac{p}{q}|\ge\frac{1/M}{q^n}
\]
applies for any $\frac{p}{q}$ inside the interval $(\xi-1,\xi+1)$. Hence, $\xi$ is not approximable to any order greater than $n$ (why?).
\end{proof}

Is the transcendental number approximable by rational numbers to any order? (Question)

We are going to use Continued Fractions to approximate any transcendental numbers.
\begin{theorem}[Continued Fractions]
\[
a_0+\frac{1}{a_1};
\qquad
a_0+\frac{1}{a_1+\frac{1}{a_2}};
\qquad
a_0+\frac{1}{a_1+\frac{1}{a_2+\frac{1}{a_3}}};
\cdots
\]
\end{theorem}
We use the notation $[a_0,a_1]$ to denote the first transcendental number, $[a_0,a_1,a_2]$ to denote the second, ...

To apprximate $0<x\notin\mathbb{Q}$, we set
\[
a_0=\lceil{x}\rceil\implies
x = a_0+\xi_0=a_0+\frac{1}{\frac{1}{\xi_0}}
\]
for $0<\xi_0<1$.  We set $a_1=\lceil{\frac{1}{\xi_0}}\rceil$. It follows that
\[
x=a_0+\frac{1}{a_1+\xi_1}=a_0+\frac{1}{a_1+\frac{1}{\frac{1}{\xi}}}
\]
We set $a_2=\lceil{\frac{1}{\xi_1}}\rceil$. Such a process will continue without end as $x$ is irational.
\[
x = [a_0,a_1,\dots,a_n,a_{n+1}+\xi_{n+1}]
\]
we denote $a_{n+1}':=a_{n+1}+\xi_{n+1}$.
\begin{remark}
\begin{enumerate}
\item
$a_n\ge1$ for $\forall n\ge1$
\item
The sequence $\{[a_0,a_1,\dots,a_n]\}$ is a Cauchy sequence. (Exercise)
\item
$[a_1,\dots,a_n]=\frac{p_n}{q_n}$, where (by induction)
\[
p_0=a_0,
\quad
p_1=a_1a_0+1,
\quad
p_n=a_np_{n-1}+p_{n-2}\ge q_{n-1}+q_{n-2}>q_{n-1}
\]
and
\[
q_0 = 1,
\quad
q_1 = a_1,
\quad
q_n = a_nq_{n-1}+q_{n-2}
\]
which implies $q_n\ge n$ for $\forall n$.
\item
By 3, we obtain
\[
x=\frac{a_{n+1}'p_n + p_{n-1}}{a_{n+1}'q_n+q_{n-1}}
\]
By computation,
\[
|x-\frac{p}{q}| = |\frac{p_{n-1}q_n - p_nq_{n-1}}{q_n(a_{n+1}'q_n + q_{n-1})}|=|\frac{(-1)^n}{??}|\le\frac{1}{q_n(a_{n+1}q_n + q_{n-1})}=\frac{1}{q_nq_{n+1}}\le\frac{1}{q_n^2}
\]
\end{enumerate}
\end{remark}
In next lecture, we will apply Liouville Theorem to construct the first transcdental number.

$\pi$ is apprixmable by rational number of order 42.

completeness of real numbers and barin-caombe theorem.










