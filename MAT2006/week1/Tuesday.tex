
\chapter{Week1}

\section{Wednesday}\index{Wednesday_lecture}
 
\paragraph{Recommended Reading}
Zorich, Analysis I, II

Rudin, Principles of Mathematical Analysis

Bous, R. A primery ...

T.Tao Analysis I, II

A. Knapp (Advanced) Basics Real Analysis

\subsection{Introduction to Set}
For a set $\mathcal{A}=\{1,2,3\}$, we have $2^3=8$ subsets of $\mathcal{A}$. We are interested to study the collection of sets.
\begin{definition}[Collection of Subsets]
The the collection of subsets of $\mathcal{A}$ is denoted as $2^{\mathcal{A}}$.
\end{definition}

We use Candinal to describe number of elements in a set. 
\begin{definition}
Given two sets $\mathcal{A}$ and $\mathcal{B}$, $\mathcal{A}$ and $\mathcal{B}$ are said to have the same \emph{candinal} (or $\mathcal{A}$ and $\mathcal{B}$ are said to be \emph{equivalent}) if there exists a 1-1 onto mapping from elements of $\mathcal{A}$ to that of $\mathcal{B}$.
\end{definition}
\begin{definition}[Countability]
$\mathcal{A}$ is said to be \emph{countable} if $\mathcal{A}\sim\mathbb{N}=\{1,2,3,\dots\}$; an infinite $\mathcal{A}$ is \emph{uncountable} if it is not equivalent to $\mathbb{N}$
\end{definition}
\begin{remark}
Note that the set of integers, i.e., $\mathbb{Z}=\{\cdots,-2,-1,0,1,2,\cdots\}$ is also countable; the set of rational numbers, i.e., $\mathbb{Q}=\{p/q\mid p,q\in\mathbb{Z}, q\ne0\}$ is countable.
\end{remark}
We skip the process to define real numbers.
\begin{proposition}\label{Pro:1:1}
The set of real numbers $\mathbb{R}$ is \emph{uncountable}.
\end{proposition}
For example, $\sqrt{2}\notin\mathbb{Q}$. Some inrational numbers are the roots of some polynomials, such a number is called \emph{algebraic} numbers. However, some inrational numbers are not, such a number is called \emph{transcendental}. For example, $\pi$ is \emph{not} algebraic. We will show that the collection of algebraic numbers are countable in the future.

There are two steps for the proof for proposition(\ref{Pro:1:1}):

\begin{proof}
\begin{enumerate}
\item
$2^{\mathbb{N}}$ is \emph{uncountable}
\item
$\mathbb{R}\sim 2^{\mathbb{N}}$.
\end{enumerate}
\end{proof}


















