
\chapter{Week1}

\section{Wednesday}\index{Wednesday_lecture}
 
\paragraph{Recommended Reading}
\begin{enumerate}
\item
(Springer-Lehrbuch) V. A. Zorich, J. Schüle-Analysis I-Springer (2006).
\item
(The Carus mathematical monographs 13) Ralph P. Boas, Harold P. Boas, A primer of real functions-Mathematical Association of America (1996).
\item
(International series in pure and applied mathematics) Walter Rudin, Principles of Mathematical Analysis-McGraw-Hill (1976).
\item
Terence Tao, Analysis I,II-Hindustan Book Agency (2006)
\item
(Cornerstones) Anthony W. Knapp, Basic real analysis-Birkhäuser (2005)
\end{enumerate}
\subsection{Introduction to Set}
For a set $\mathcal{A}=\{1,2,3\}$, we have $2^3=8$ subsets of $\mathcal{A}$. We are interested to study the collection of sets.
\begin{definition}[Collection of Subsets]
Given a set $\mathcal{A}$, the the collection of subsets of $\mathcal{A}$ is denoted as $2^{\mathcal{A}}$.
\end{definition}

We use Candinal to describe the order of number of elements in a set. 
\begin{definition}
Given two sets $\mathcal{A}$ and $\mathcal{B}$, $\mathcal{A}$ and $\mathcal{B}$ are said to be \emph{equivalent} (or have the same\emph{candinal}) if there exists a 1-1 onto mapping from $\mathcal{A}$ to $\mathcal{B}$.
\end{definition}
\begin{definition}[Countability]
The set $\mathcal{A}$ is said to be \emph{countable} if $\mathcal{A}\sim\mathbb{N}=\{1,2,3,\dots\}$; an infinite set $\mathcal{A}$ is \emph{uncountable} if it is not equivalent to $\mathbb{N}$.
\end{definition}
\begin{remark}
Note that the set of integers, i.e., $\mathbb{Z}=\{\cdots,-2,-1,0,1,2,\cdots\}$ is also countable; the set of rational numbers, i.e., $\mathbb{Q}=\{p/q\mid p,q\in\mathbb{Z}, q\ne0\}$ is countable.
\end{remark}
We skip the process to define real numbers.
\begin{proposition}\label{Pro:1:1}
The set of real numbers $\mathbb{R}$ is \emph{uncountable}.
\end{proposition}
For example, $\sqrt{2}\notin\mathbb{Q}$. Some inrational numbers are the roots of some polynomials, such a number is called \emph{algebraic} numbers. However, some inrational numbers are not, such a number is called \emph{transcendental}. For example, $\pi$ is \emph{not} algebraic. We will show that the collection of algebraic numbers are countable in the future.

There are two steps for the proof for proposition(\ref{Pro:1:1}):

\begin{proof}
\begin{enumerate}
\item
$2^{\mathbb{N}}$ is \emph{uncountable}:

Assume $2^{\mathbb{N}}$ is countable, i.e.,
\[
2^{\mathbb{N}} = \{A_1,A_2,\dots,A_k,\dots\}
\]

Define $B:=\{k\in\mathbb{N}\mid k\notin A_k\}$, it is a collection of subscripts such that the subscript $k$ does not belong to the corresponding subsets $A_k$.

It follows that $B\in2^{\mathbb{N}}\implies B=A_n$ for some $n$. Then it follows two cases:
\begin{itemize}
\item
If $n\in A_n$, then $n\notin B=A_n$, which is a contradiction
\item
Otherwise, $n\in B=A_n$, which is also a contradiction.
\end{itemize}
The proof for the claim $2^{\mathbb{N}}$ is \emph{uncountable} is complete.
\item
$\mathbb{R}\sim 2^{\mathbb{N}}$:

\paragraph{Firstly we have $\mathbb{R}\sim(0,1)$} This can be shown by constructing a one-to-one mapping:
\[
\begin{array}{ll}
f:\mathbb{R}\mapsto(0,1)
&
f(x)=\frac{1}{\pi}\arctan x+\frac{1}{2},\forall x\in\mathbb{R}
\end{array}
\]

\paragraph{Secondly, we show that $2^{\mathbb{N}}\sim(0,1)$} We construct a mapping $f$ such that
\[
f:2^{\mathbb{N}}\mapsto(0,1), 
\]
where for $\forall A\in2^{\mathbb{N}}$, 
\[
\begin{array}{ll}
f(A)=0.a_1a_2a_3\dots,
&
a_j=\left\{
\begin{aligned}
2,&\quad\mbox{if }j\in A\\
4,&\quad\mbox{if }j\notin A
\end{aligned}
\right.
\end{array}
\]
This function is only 1-1 mapping but not onto mapping.

Reversely, we construct a 1-1 mapping from $(0,1)$ to $2^{\mathbb{N}}$. We construct a mapping $g$ such that
\[
g:(0,1)\mapsto2^{\mathbb{N}}
\]
where for any real number from $(0,1)$, we can write it into binary expansion:
\[
\begin{array}{lll}
\mbox{binary form: }
&
0.a_1a_2\dots
&
\mbox{where }a_j=0\mbox{ or }1.
\end{array}
\]
Hence, we construct $g(0.a_1a_2\dots) = \{j\in\mathbb{N}\mid a_j=0\}\subseteq\mathbb{N}$, which implies $g(\cdot)\in2^{\mathbb{N}}$.
\begin{remark}
Our intuition is that two 1-1 mappings in the reverse direction will lead to a 1-1 \emph{onto} mapping. If this is true, then we complete the proof. This intuition is the \emph{Schroder-Bernstein Theorem}.
\end{remark}
\end{enumerate}
\end{proof}
\paragraph{Defining Binary Form} However, during this proof, we must be careful about the binary form of a real number from $(0,1)$. Now we give a clear definition of Binary Form:

For a real number $a$, to construct its binary form, we define
\[
a_1=\left\{
\begin{aligned}
0,&\quad\mbox{if }a\in(0,\frac{1}{2})\\
1,&\quad\mbox{if }a\in[\frac{1}{2},1).
\end{aligned}
\right.
\]
After having chosen $a_1,a_2,\dots,a_{j-1}$, we define $a_j$ to be the largest integer such that
\[
\frac{1}{2}a_1+\frac{1}{2^2}a_2+\cdots+\frac{a_j}{2^j}\le a
\]
Then the binary form of $a$ is $a:=0.a_1a_2\dots$.
\begin{theorem}[Schroder-Bernstein Theorem]
If $f:A\mapsto B$ and $g:B\mapsto A$ are both 1-1 mapping, then there exists a 1-1 onto mapping from $A$ to $B$, i.e., card $\# A$ equals to card $\# B$.
\end{theorem}
Exercise: Show that $(0,1)$ and $[0,1]$ have 1-1 onto mapping without applying Schroder-Bernstein Theorem.

The next lecture we will take a deeper study into the proof of Schroder-Bernstein Theorem and the real number.













