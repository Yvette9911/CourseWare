\section{Monday for MAT4002}\index{Monday_lecture}
\paragraph{Reviewing}
\begin{enumerate}
\item
Homotopy: $f\cong g$
\item
If $Y\subseteq\mathbb{R}$ is convex, then the set of continuous functions $f:X\to Y$ form a single equivalence class, i.e., $\{\text{continuous functions $f:X\to Y$}\}/\sim$ has only one element
\end{enumerate}
\subsection{Remarks on Homotopy}
\begin{proposition}
Consider
\[
W\to_fX,\quad
X\to_gY,\quad
X\to_hY,\quad
Y\to_k Z
\]
where $f,g,h,k$ are all continuous. If $g\cong h$, then
\[
g\circ f\cong h\circ f,\quad
k\circ g\cong k\circ h
\]
\end{proposition}
\begin{proof}
Suppose $H:g\cong h$, then $k\circ H:X\times I\to Z$ givs the momotopy between $k\circ g$ and $k\circ h$.

Simiarly, $H\circ(f\times\text{id}_I):W\times I\to Y$ gives homotopy $g\circ f\simeq h\circ f$.

\end{proof}

\begin{definition}[Homotopy Equivalent]
Two topological spaces $X$ and $Y$ are \emph{homotopy equivalent} if there are continuous maps $f:X\to Y$, and $g:Y\to X$ such that
\begin{align*}
g\circ f&\simeq \text{id}_{X\to X}\\
f\circ g&\simeq\text{id}_{Y\to Y}
\end{align*}
denoted as $X\simeq Y$.
\end{definition}
\begin{remark}
\begin{enumerate}
\item
If $X\cong Y$ are homeomorphic, then they are homotopic equivalent.
\item
$X\simeq Y$ gives a bijection between $\{\phi:\text{continuous }W\to X\}/\sim$ and $\{\phi:\text{continuous }W\to Y\}/\sim$
\item
exercise: $X\simeq Y$ forms an equivalence relation between topological spaces
\end{enumerate}
\end{remark}
Some properties are lost when we only study spaces up to homotopy equivalence.

\begin{definition}[Contractible]
$X$ is \emph{contractible} if it is homotopy equivalent to a point $\{*\}$.

Equivalently, we need $f,g$ such that
\begin{align*}
\{*\}\to_fX\to_g\{*\},\ g\circ f\simeq\text{id}_{\{*\}}\\
X\to_g\{*\}\to_fX,\ f\circ g\simeq\text{id}_{X}
\end{align*}
\end{definition}
Note that $f\circ g=c_y$ for some $y\in X$, where $c_y:X\to X$ is $c_y(x)=y,\forall x\in X$.
Therefore, to check $X$ is contractible, we only need to check $c_y\simeq\text{id}_{X},\forall y\in X$

\begin{example}
\begin{enumerate}
\item
$X=\mathbb{R}^2$ is contractible.

WTS $f(\bm x)=\bm x$ (i.e., $f=\text{id}$) is homotopic to the constant function $g(x)=(0,0),\forall x\in\mathbb{R}^2$, i.e., $g=c_{(0,0)}$

Consider $H(\bm x,t) = tf(\bm x)$, which implies
\[
H(\bm x,0)=c_{(0,0)},\qquad
H(\bm x,1)=\text{id}
\]
Therefore, $\text{id}\simeq c_{(0,0)}$, and $c_{(0,0)}\simeq c_y,\forall y\in\mathbb{R}^2$

Therefore, $X$ is contractible.

More ggenerally, any convex $X\subseteq\mathbb{R}^n$ is contractible.
\end{enumerate}
\end{example}

\begin{remark}
$S^1$ is not contractible, and we will see it in 3 weeks' time.

We are not able to construct 
\[
H:S^1\to[0,1]\to S^1
\]
such that $H(e^{2\pi ix},0)=e^{2\pi i x}$ and $H(e^{2\pi i x},1)=e^{2\pi i(0)} = 1$ ($c_1$)

But how about $H(e^{2\pi ix},t)=e^{2\pi ixt}$?
What's wrong with $H$?
Such a funciton is not well-defined:
\[
H(e^{2\pi i(1)},t)=e^{2\pi i(1-t)}=H(e^{2\pi i(0)},t)=1
\]
Therefore, $H$ is not well-defined for $t\ne0,1$.
\end{remark}


\begin{definition}
Let $A\subseteq X$ and $i:\hookrightarrow X$ be an inclusion.
We say $A$ is a homotopy retract of $X$ if there exists continuous mapping $r:X\to A$ such that 
\[
A\hookrightarrow X\to^r A\implies 
r\circ i=\text{id}_{A\to A}
\]
\[
X\to^rA\hookrightarrow^iX\implies i\circ r\simeq\text{id}_X
\]
In particualr, $A\simeq X$.
\end{definition}

\begin{example}
$S^1$ is a homotopy retract of $M=\text{Mobius Band}$

Here $M=[0,1]^2/\sim$ and $S^1=[0,1]/\sim$.
Define
\[
i:S^1\hookrightarrow M,\quad
[x]\mapsto [(x,\frac{1}{2})]
\]
\[
\begin{array}{ll}
r:&M\to S^1\\
&[(x,y)]\mapsto[x]
\end{array}
\]
Then $r\circ i = \text{id}_{S^1}$; $i\circ r([(x,y)]) = [(x,1/2)]$; $\text{id}_M([(x,y)]) = [(x,y)]$.

Define $H:M\times I\to M$ by
\[
H([(x,y)],t):=[(x,(1-t)y+t/2)]
\]
We really need to check
\[
H([(0,y)],t)=H([(1,1-y)],t),\quad \forall y\in[0,1]
\]
Therefore, $H$ gives a homotopy between $i\circ r$ and $\text{id}_{M}$, i.e., $i\circ r\simeq\text{id}_M$
\end{example}

$S^{n-1}$ is a homotopy retract of $\mathbb{R}^n\setminus\{\bm0\}$.

We have $i:\text{id}=S^{n-1}\to\mathbb{R}^n\setminus\{0\}$
and
\[
\begin{array}{ll}
r:&\mathbb{R}^n\setminus\{0\}\to \mathbb{S}^{n-1}\\
&x\mapsto\frac{x}{\|x\|}
\end{array}
\]
Therefore, $r\circ i =\text{id}_{S^{n-1}}$ and $i\circ r(x)=x/\|x\|$.

WTS $i\circ r\simeq\text{id}_{\mathbb{R}^n\setminus\{0\}}$
Consider $H(x,t)=t\bm x+(1-t)\bm x/\|\bm x\|$:
\[
H(\bm x,0)=i\circ r(\bm x),\quad
H(\bm x,1)=\bm x=\text{id}(\bm x)
\]
However, we need to check that $H(x,t)\in\mathbb{R}^n\setminus\{\bm0\}$ for all $\bm x\in\mathbb{R}^n\setminus\{\bm0\}$ and $t\in[0,1]$.

\begin{definition}[Homotopic Relative]
Let $A\subseteq X$ be topological spaces.
We say $f,g:X\times I\to Y$ are homotopic relative to $A$ if there eixsts $H:X\times I\to Y$ such that
\[
\left\{
\begin{aligned}
H(x,0)&=f(x)\\
H(x,1)&=g(x)
\end{aligned}
\right.
\]
and
\[
H(a,t)=f(a)=g(a),\forall a\in A
\]
\end{definition}














