
\section{Wednesday for MAT3006}\index{Wednesday_lecture}
\paragraph{Reviewing}
\begin{itemize}
\item
All null sets are measurable
\item
If $E\subseteq \mathbb{R}$ is measurable, then $E^c:=R\setminus E$ is measurable.
\item
$E_i$ is measurable implies $\cup_{i=1}^nE_i$ is measurable. 
\end{itemize}
\subsection{Remarks on Lebesgue Measurability}
\begin{proposition}
If $E_i$ is measurable for $\forall i\in\mathbb{N}$, then so is $\cup_{i=1}^\infty E_i$.
Moreover, if further $E_i$'s are pairwise disjoint, then
\[
m^*\left(\bigcup_{i=1}^\infty E_i\right)
=
\sum_{i=1}^\infty m^*(E_i)
\]
\end{proposition}
\begin{proof}
\begin{itemize}
\item
Consider the case where $E_i$'s are measurable, pairwise disjoint first.
For all subsets $A\subseteq\mathbb{R}$, and all $n\in\mathbb{N}$,
\begin{subequations}
\begin{align}
m^*(A)&=
m^*(A\cap (\cup_{i=1}^n E_i))+
m^*(A\cap (\cup_{i=1}^n E_i)^c)\label{Eq:8:4:a}\\
&=[m^*(A\cap (\cup_{i=1}^n E_i)\cap E_n)
+
m^*(A\cap (\cup_{i=1}^n E_i)\cap E_n^c)]
+m^*(A\cap (\cup_{i=1}^n E_i)^c)\label{Eq:8:4:b}\\
&=[m^*(A\cap E_n)
+
m^*(A\cap (\cup_{i=1}^{n-1} E_i)]
+m^*(A\cap (\cup_{i=1}^n E_i)^c)\label{Eq:8:4:c}
\end{align}
\end{subequations}
where (\ref{Eq:8:4:a}) is by the measurability of $\cup_{i=1}^n E_i$;
(\ref{Eq:8:4:b}) is by the measurability of $E_n$;
(\ref{Eq:8:4:c}) is by direct calculation.

Proceeding these trick similarly, we obtain:
\begin{subequations}
\begin{align}
m^*(A) &=[m^*(A\cap E_n)+m^*(A\cap (\cup_{i=1}^n E_i)^c)]+m^*(A\cap (\cup_{i=1}^{n-1} E_i)]\\
&=
\sum_{\ell=1}^n
m^*(A\cap E_{\ell})
+
m^*(A\cap (\cup_{i=1}^{\ell} E_i)^c)\\
&\ge
\sum_{\ell=1}^n
m^*(A\cap E_{\ell})
+
m^*(A\cap (\cup_{i=1}^\infty E_i)^c)\label{Eq:8:5:c}
%&\ge
%m^*(\cup_{i=1}^\infty(A\cap E_i))+m^*(A\cap (\cup_{i=1}^\infty E_i)^c)
\end{align}
for any $n\in\mathbb{N}$, where (\ref{Eq:8:5:c}) is by lower bounding $(\cup_{i=1}^{\ell} E_i)^c\supseteq
(\cup_{i=1}^{\infty} E_i)^c
$.
Taking $n\to\infty$ in (\ref{Eq:8:5:c}), we imply
\begin{align}
m^*(A)&\ge 
\sum_{\ell=1}^\infty
m^*(A\cap E_{\ell})
+
m^*(A\cap (\cup_{i=1}^\infty E_i)^c)\label{Eq:8:5:d}\\
&\ge m^*(\cup_{i=1}^\infty(A\cap E_i))+m^*(A\cap (\cup_{i=1}^\infty E_i)^c)\label{Eq:8:5:e}\\
&=m^*(A\cap(\cup_{i=1}^\infty E_i))+m^*(A\cap (\cup_{i=1}^\infty E_i)^c)\label{Eq:8:5:f}
\end{align}
\end{subequations}
where (\ref{Eq:8:5:e}) is by the countable sub-addictivity of $m^*$.
Therefore, $\cup_{i=1}^\infty E_i$ is measurable.
\item
Moreover, taking $A=\cup_{i=1}^\infty E_i$ in (\ref{Eq:8:5:d}) gives
\[
m^*(\cup_{i=1}^\infty E_i) = \sum_{i=1}^\infty m^*(E_i)+m^*(\emptyset)= \sum_{i=1}^\infty m^*(E_i)+0.
\]
\item
Now suppose that $E_i$'s are measurable but not necessarily pairwise disjoint.
We need to show $\bigcup_{i=1}^\infty E_i$ is measurable. The way is to construct the disjoint sequence of sets first:
\[
\left\{
\begin{aligned}
F_1 &= E_1,\\
F_{k+1}&=E_k\setminus \left(\cup_{i=1}^kE_i\right),\ \forall k>1
\end{aligned}
\right.\implies
\cup_{i=1}^\infty F_i = \cup_{i=1}^\infty E_i
\]
\end{itemize}
It's clear that $F_i$'s are pairwise disjoint and measurable, which implies $\cup_{i=1}^\infty E_i=\cup_{i=1}^\infty F_i$ is measrable. 
The proof is complete.
\end{proof}
\paragraph{Notations}
We denote $\mathcal{M}$ as the collection of all \emph{(Lebesgue) measurable} subsets of $\mathbb{R}$, and
\[
m(E) = m^*(E),\ \quad \forall E\in\mathcal{M}
\]

\subsection{Measures In Probability Theory}

\begin{definition}[$\sigma$-Algebra]
\begin{itemize}
\item
Let $\Omega$ be any set, and $\mathbb{P}(\Omega)$ (\emph{power set}) denotes the collection of all subsets of $\Omega$
\item
A family of subsets of $\Omega$, denoted as $\mathcal{T}$, is a $\sigma$-algebra if it satisfies
\begin{enumerate}
\item
$\emptyset,\Omega\in\mathcal{T}$
\item
If $E_i\in\mathcal{T}$ for $\forall i\in\mathbb{N}$,
then $\cup_{i=1}^\infty E_i\in\mathcal{T}$ (and therefore $\cap_{i=1}^\infty E_i\in\mathcal{T}$).
\end{enumerate}
\end{itemize}
\end{definition}

\begin{definition}[Measure]
A \emph{measure} on a $\sigma$-algerba $(\Omega,\mathcal{T})$ is a function 
$\mu:\mathcal{T}\to[0,\infty]$ 
such that
\begin{itemize}
\item
$\mu(\emptyset) = 0$
\item
$\mu(\cup_{i=1}^\infty E_i)=\sum_{i=1}^\infty\mu(E_i)$ whenever $E_i$'s are pairwise disjoint in $\mathcal{T}$.
\end{itemize}
As a result, $(\Omega,\mathcal{T},\mu)$ is called a \emph{measurable space}.
\end{definition}

\begin{example}
\begin{enumerate}
\item
Let $\Omega$ be any set, $\mathcal{T}=\mathcal{M}$, and $\mu(E)=|E|$ (the number of elements in $E$). Then $(\Omega,\mathbb{P}(\Omega),\mu)$ is a measure space, and $\mu$ is called a counting measure on $\Omega$.
\end{enumerate}
\end{example}


\begin{definition}[Borel $\sigma$-algebra]
Let $\bm{B}$ be a collection of all intervals in $\mathbb{R}$.
Then there is a \emph{unique} $\sigma$-algebra $\mathcal{B}$ of $\mathbb{R}$, such that
\begin{enumerate}
\item
$\bm B\subseteq\mathcal{B}$
\item
For all $\sigma$-algebra $\mathcal{T}$ containing $\bm B$, we have $\mathcal{B}\subseteq\mathcal{T}$
\end{enumerate}
This $\mathcal{B}$ is called a \emph{Borel $\sigma$-algebra}
\end{definition}
\begin{remark}
\begin{enumerate}
\item
In particular, $C_i\in\mathcal{B}$ implies $\cup_{i=1}^\infty C_i$ and $\cap_{i=1}^\infty C_i\in\mathcal{B}$.
\item
$\mathcal{B}\subseteq\mathcal{M}$, since $\bm B\subseteq\mathcal{M}$ and $\mathcal{M}$ is a $\sigma$-algebra.
\item
However, $\mathcal{M}$ and $\mathcal{B}$ are not equal.
% (see example below).
The element $C\in\mathcal{B}$ is called \emph{Borel measurable subsets}
\end{enumerate}
\end{remark}
%
%
%In Hw3, for all $E\in\mathcal{M}$, there exists $B\in\mathcal{B}$ with $B\supseteq E$ such that 
%\[
%B\setminus E\text{ is null , i.e., }m(B\setminus E)=0.
%\]

\begin{definition}[complete]
Let $(\Omega, \mathcal{T},\mu)$ be a measurable space.
Then we say it is \emph{complete} if for any $E\in\mathcal{T}$ with $\mu(E)=0$,
$N\subseteq E$ implies $N\in\mathcal{T}$. (and therefore $\mu(N)=0$)
\end{definition}

\begin{example}
\begin{enumerate}
\item
$(\mathbb{R},\mu,m^*)$ is complete.

Reason: if $m^*(E)=0$, then $m^*(N)=0$, $\forall N\subseteq E$
\item
$(\mathbb{R},\mu,m)$ is complete.

Reason: the same as in (1)
\item
However, $(\mathbb{R},\mathcal{B},m\mid_{\mathcal{B}})$ is not complete. (left as exercise)
\end{enumerate}
\end{example}

Then we study the difference between $\mathcal{B}$ and $\mathcal{M}$:

\begin{definition}[Completion]
Let $(\Omega,\mathcal{T},\mu)$ be measurable space.
The \emph{completion} of $(\Omega,\mathcal{T},\mu)$ with respect to $\mu$ 
is the smallest complete $\sigma$-algebra containing $\mathcal{T}$, denoted as $\overline{\mathcal{T}}$.
More precisely, 
\[
\overline{\mathcal{T}} = \{G\cup N\mid G\in\mathcal{T}, N\subseteq F\in\mathcal{T}, \text{with }\mu(F)=0\}
\]
e.g., take $G = \emptyset\in\mathcal{T}$.
For all $F\in\mathcal{T}$ such that $\mu(F)=0$, $N\subseteq F$ implies $N\in\overline{\mathcal{T}}$.
\end{definition}
\begin{remark}
If further define $\overline{\mu}:\overline{\mathcal{T}}\to[0,\infty]$ by
\[
\overline{\mu}(G\cup N) = \mu(G), 
\]
then $(\Omega,\overline{\mathcal{T}},\overline{\mu})$ is a measurable space.
\end{remark}

\begin{theorem}
The completion of $(\mathbb{R},\mathcal{B},m\mid_{\mathcal{B}})$ is $(\mathbb{R},\mathcal{M},m)$
\end{theorem}

\begin{remark}
Another completion of $(\mathbb{R},\mu,m)$ is as follows:

Define $\ell(\{a,b\}) = b-a$ for all intervals $\{a,b\}\in\bm B$
Then by Caratheodory extension theorem, we can extend $\ell:\bm B\to[0,\infty]$ to $\ell:\mathcal{B}\to[0,\infty]$.


Complete $\ell:\mathcal{B}\to[0,\infty]$ to $\bar{\ell}:\mathcal{M}\to[0,\infty]$.
Then $\bar{\ell}=m$ as in our course.

\end{remark}











