\section{Wednesday for MAT4002}\index{Monday_lecture}
\paragraph{Reviewing}
Let $K'=(V',\Sigma')$ be a \emph{simplicial subcomplex}, then $K=(V,\Sigma)$.
\[
|K'| = D_{\Sigma'}/\sim_{\Sigma'}\implies
|K| = D_{\Sigma}/\sim_{\Sigma}
\]
Thus $D_{\Sigma'}\to D_{\Sigma}\to^PD_{\Sigma}/\sim_{\Sigma}$.
For all $x,y\in D_{\Sigma}$, 
\begin{equation}\label{Eq:8:2}
x\sim_{\Sigma'}y\Longleftrightarrow
i(x)\sim_{\Sigma}i(y)
\end{equation}
The whole map $f$ descends to a continuous map
\[
\tilde{f}:D_{\Sigma'}/\sim_{\Sigma'}\to D_{\Sigma}/\sim_{\Sigma}
\]
Indeed, the $\Longleftarrow$ part of (\ref{Eq:8:2}) guarantess that $\tilde{f}$ is injective.

For $i:|K'|\to |K|$ continuous and injective, $i(|K'|)$ is closed in $|K|$.


\begin{proposition}
For each $K=(V,\Sigma)$, and finite $V$, there is a continuous injection $g:|K|\to\mathbb{R}^n$ for some $n$.
\end{proposition}

\begin{proof}
Consider $K^p:=(V,\Sigma^p)$, where $\Sigma^p$ is the power set of $V$.
Therefore, $|K^p| = \Delta^{|V|-1}\subseteq\mathbb{R}^{|V|}$. Also, we have the injection
\[
|K'|\xrightarrow{l}|K^p|\xrightarrow{l}\mathbb{R}^{|V|}
\]
Since $K=(V,\Sigma)$ is a simplicial subcomplex of $K^p=(V,\Sigma^p)$, the proof is complete.
\end{proof}

\begin{proposition}[Hausdorff]
If $K=(V,\Sigma)$ with fintie $V$, then $|K|$ is Hausdorff.
\end{proposition}

\begin{proof}
Let $g:|K|\xrightarrow{l}\mathbb{R}^n$.
Consider the bijective $g:|K|\to g(|K|)$, which is continuous.
Sicne $|K|$ is compact, $g(|K|)\subseteq \mathbb{R}^n$ is Hausdorff.
Therefore, we imply that $|K|$ and $g(|K|)$ are homeomorphic, i.e., $|K|$ is Hausdorff.
\end{proof}

\begin{definition}[Edge Path]
An \emph{edge path} of $K=(V,\Sigma)$ is a sequence of vertices $(v_1,\dots,v_n), v_i\in V$ such that $\{v_i,v_{i+1}\}\in\Sigma,\forall i$.
\end{definition}

\begin{proposition}[Connectedness]
Let $K=(V,\Sigma)$ be a simplicial complex. TFAE:
\begin{enumerate}
\item
$|K|$ is connected
\item
$|K|$ is path-connected
\item
Any 2 vertices in $(V,\Sigma)$ can be joined by an edge path, i.e., for $\forall u,v\in V$, there exists $v_1,\dots,v_k\in V$ such that $(u,v_1,\dots,v_k,v)$ is an edge path.
\end{enumerate}
\end{proposition}

\begin{proof}[Sketch of Proof]
\begin{enumerate}
\item
(3) implies (2):
For every $x,y\in|K|$, 
\[
\left\{
\begin{aligned}
x\in\Delta_{\sigma_1}\text{ for some $\sigma_1\in\Sigma$.}\\
y\in\Delta_{\sigma_2}\text{ for some $\sigma_2\in\Sigma$.}\\
\end{aligned}
\right.
\]
Take a path joining $x$ to a vertex $v_1\in\sigma_1$ and a path joining $y$ to a vertex $v_2\in\sigma_2$.

By (3), we have a path joninig $v_1$ and $v_2$.
\item
(1) implies (3):
Suppose on the contrary that there is a vertex $v$ not satisfying (3).
Take $V'$ as the set of vertexs that can be joined with $v$; and $V''$ as the set of vertexs that cannot be joinied with $v$.

Then $V',V''\ne\emptyset$.
Consider $K',K''$ be simplicial subcomplexes of $K$, spanned by $V'$ and $V''$.
Then $|K'|,|K''|$ are disjoint, closed in $|K|$.

$|K| = |K'|\cup|K''|$. 
If there exists $x\in |K|\setminus(|K'|\cup|K''|)$, then for any $\sigma\in\Sigma$ such that $x\in\Delta_{\sigma}$, we imply $\Delta_\sigma\not\subseteq |K'|$ or $|K''|$.

Therefore, $\sigma$ consists of vertices in both $V'$ and $V''$.
Then there is $v',v''\in\sigma$ joining $V'$ and $V''$.

Therefore, there is no such $x$ and hence $|K|=|K'|\cup|K''|$ is a disjoint union of two closed sets, i.e., not connected.
\end{enumerate}
\end{proof}

\subsection{Homotopy}
Yoneda's ``philosophy'':
To understand an object $X$ (in our case, $X$ denotes topological space), we should understand functions
\[
f:A\to X,
\text{ or }
g:X\to B
\]
One example is to let $B=\mathbb{R}$.

There are many continuous functions $g:X\to Y$.
We will group all these functions into equivalence classes.

\begin{definition}[Homotopy]
A \emph{Homotopy} between two continuous maps $f,g:X\to Y$ is a continuous map
\[
H:X\times[0,1]\to Y
\]
such that 
\[
H(x,0)=f(x),\quad
H(x,1)=g(y)
\]
If such $H$ exists, we say $f$ and $g$ are \emph{homotopic}, denoted as $f\cong g$
\end{definition}

\begin{example}
Let $Y\subseteq\mathbb{R}^2$ be a convex subset.
Consider two maps $f:X\to Y$ and $g:X\to Y$.
They are always homotopic since we can define
\[
H(x,t) = tg(x) + (1-t)f(x)
\]
\end{example}

\begin{proposition}
Homotopy is an equivalent relation
\end{proposition}
\begin{enumerate}
\item
$f\cong f$ is obvious: let $H(x,t) = f(x)$, for $\forall 0\le t\le1$
\item
If $f\cong g$, then $g\cong f$:
For homotopic from $f$ to $g$, say $H(x,t)$, construct
\[
H'(x,t):=H(x,1-t)
\]
Therefore, $H'(x,0)=g(x)$ and $H'(x,1)=f(x)$.
\item
If $f\cong g$ and $g\cong h$, then $f\cong h$:
suppose $H:f\cong g$, and $K:g\cong h$.
Consider
\[
J(x,t)=\left\{
\begin{aligned}
H(x,2t),&\quad 0\le t\le1/2\\
K(x,2t-1),&\quad 1/2\le t\le 1
\end{aligned}
\right.
\]
Note that $J(x,1/2)$ are well-defined.
Then $J$ is continuous, since for all closed $V\subseteq Y$,
\[
J^{-1}(V)=(J^{-1}(V)\cap(X\times[0,1/2]))\cup(J^{-1}(V)\cap(X\times[1/2,1]))
=
H^{-1}(V)\cup K^{-1}(V)
\]
Since $H^{-1}(V)$ and $K^{-1}(V)$ are both closed, we imply $J^{-1}(V)$ is closed.
\end{enumerate}

Therefore, there is only one equivalence class in example~(1).
This reflects the fact that $Y\subseteq\mathbb{R}^2$ is a ``simple'' object.

\begin{proof}
Tkae $y_0\in Y$.
Consider $C_y:X\to Y$ by $C_y(x)=y_0,\forall x$.
For all continuous maps $f:X\to Y$, $f\cong C_y$.

Therefore, there is only one equivalence class since every continuous map is homotopic to $C_y$
\end{proof}





















