
\chapter{Week3}


\section{Tuesday}\index{week3_Tuesday_lecture}
\begin{definition}[Cartesian Product]
\[
\prod_{i=1}^nS_i=S_1\times S_2\times\cdots\times S_n
=
\{
(a_1,a_2,\dots,a_n)\mid a_i\in S_i
\}
\]
\end{definition}
\begin{theorem}
$\prod_{i=1}^nG_i$ is a group under the operation
\[
(g_1,\dots,g_n)(h_1,\dots,h_n)=(g_1h_1,\dots,g_nh_n)
\]
\end{theorem}
\begin{proof}
\begin{itemize}
\item
It's obvious that the operation is closed.
\item
Check inverse and identity.
\[
\mbox{identity} = (e_1,e_2,\dots,e_n)
\]
\item
Check the operation is associate:
\begin{align*}
\left[(g_1,\dots,g_n)(h_1,\dots,h_n)\right](k_1,\dots,k_n)&=(g_1h_1,\dots,g_nh_n)(k_1,\dots,k_n)\\
&=(g_1h_1k_1,\dots,g_nh_nk_n)\\
&=(g_1,\dots,g_n)(h_1k_1,\dots,h_nk_n)\\
&=(g_1,\dots,g_n)\left[(h_1,\dots,h_n)(k_1,\dots,k_n)\right]
\end{align*}
\end{itemize}
\end{proof}
\begin{remark}
If the operation of each $G_i$ is the \emph{addition}, then 
\[
\prod_{i=1}^nG_i:=\oplus_{i=1}^nG_i
\]
\end{remark}
\begin{example}
\begin{enumerate}
\item
$G=(S_3\times\mathbb{Z}_2,\cdot)$ is not abelian, e.g.,
\[
((12),0)\cdot((23),0)
\]
\item
$G=(\mathbb{Z}_2\times\mathbb{Z}_3,+)=\mathbb{Z}_2\oplus\mathbb{Z}_3$ is cyclic
\[
d(1,1) = (0,0)\implies d=6k
\]
\item
The \emph{Klein} 4-group $V=\mathbb{Z}_2\times\mathbb{Z}_2$ is not cyclic
\[
d(x,y) = (0,0)
\]
\end{enumerate}
\end{example}
\begin{theorem}
$G=\mathbb{Z}_m\times\mathbb{Z}_n$ is \emph{cyclic} iff $gcd(m,n)=1$.
\end{theorem}
\begin{proof}
Let $k=lcn(m,n)=\frac{mn}{gcd(m,n)}\le mn$.

Necessity. Consider $(a,b)\in G$:
\[
k(a,b)=(ka,kb):=(msa,ntb)=(0,0),
\]
i.e., $|(a,b)|\le k$. In particular, $mn\le k$, thus $k=mn$, i.e., $gcd(m,n)=1$.

Sufficiency. Consider $(1,1)\in G$: $d(1,1) = (0,*)\implies d=xm$; and $d(1,1)=(*,0)\implies d=yn$. Thus $|(1,1)| = lcm(m,n)=mn$, i.e., this group is cyclic.
\end{proof}
\begin{corollary}
$
\prod_{i=1}^n\mathbb{Z}_{m_i}
$ is cyclic iff $(m_i,m_j)$ are mutually coprime.
\end{corollary}
\begin{definition}
Let $G$ be a group, $S$ a non-empty subset.
\[
<S>:=\{a_1^{m_1},\dots,a_n^{m_n}\mid n\in\mathbb{Z}^+,m_i\in\mathbb{Z},a_i\in S\}
\]
If $S$ is finite, then $<S>$ is \emph{finitely generated}.
\end{definition}
Verify that this is a group, i.e., a subgroup of $G$. Note that $a_i$'s need not to be distinct. e.g.,
\[
S=\{a,b\}\implies a^{-1}bab^2\in<S>
\]
\begin{proposition}
\[
<S>=\bigcap_{\{H\mid S\subseteq H\subseteq G\}}H
\]
\end{proposition}
\begin{example}
\begin{enumerate}
\item
$<\mbox{cycles in $S_n$}> = S_n=<\mbox{transpositions}>$
\item
$S_n=<(12),(1,2,\dots,n)>$.

hint: $(i,i+1)\in S_n, (i,j)\in S_n$
\item
$D_n=<r,s>$
\end{enumerate}
\end{example}
\begin{proposition}
$\mathbb{Q}$ is not finitely generated.
\end{proposition}
\begin{theorem}[Fundamental Theorem of Finitely Generated Abelian Groups]
Any finitely generated abelian group (is isomorphic to)
\[
\prod_{i=1}^m\mathbb{Z}_{p_i^{r_i}}\times\mathbb{Z}^n,
\]
$r_i,n\in\mathbb{N}$.
\end{theorem}
\begin{example}
abelian group of order $360=2^33^25$:
\[
G_2\times G_3\times G_5
\]
$G_5=\mathbb{Z}_5$, $G_3=\mathbb{Z}_3\times \mathbb{Z}_3,\mathbb{Z}_9$, $G_2=\mathbb{Z}_2\times\mathbb{Z}_2\times\mathbb{Z}_2,\mathbb{Z}_2\times\mathbb{Z}_4,\mathbb{Z}_8$.

Thus there are 6 possible abelian groups of order $360$.
\end{example}
How about abelian group of order $7^5$?

\begin{definition}[Partition]
Let $S\ne\emptyset$. A \emph{partition} $P$ of $S$ is $\{S_i\mid i\in I\}$ such that
\begin{enumerate}
\item
$S_i\ne\emptyset,\forall i\in I$
\item
$S_i\bigcap S_j=\emptyset,\forall i\ne j$
\item
$\bigcup_{i\in I}S_i=S$
\end{enumerate}
Also, we denote $S=\bigsqcup_{i\in I}S_i$
\end{definition}
\begin{definition}[Equivalence Relation]
An \emph{equivalence relation} on $S$ is a relation $\sim $ such that
\begin{enumerate}
\item
Reflexive: $a\sim a,\forall a\in S$
\item
Symmetric: $a\sim b$ implies $b\sim a$
\item
Transitive: $a\sim b,b\sim c$ implies $a\sim c$.
\end{enumerate}
\end{definition}
Equivalence relation is essentially the same meaning of partition:
\begin{itemize}
\item
Partition implies equivalence relation: Define $a\sim b$ if $a,b\in S_i$
\item
Equivalence relation implies partition: Define $C_a:=\{b\in S\mid b\sim a\}$. (For the symmeticity part, show that $C_a\bigcap C_b\ne\emptyset$ implies $C_a=C_b$.)
\end{itemize}

We call $C_a$ the \emph{equivalence class} with the representative $a$. If $b\in C_a$, then $C_b=C_a$, so any element in an equivalence class can be its representative.
\begin{proposition}
Any $\sigma\in S_n$ is a product of disjoint cycles.
\end{proposition}
\begin{proof}
Given $a,b\in X=\{1,2,\dots,n\}$, define $a\sim b$ if $b=\sigma^k(a)$ for some $k\in\mathbb{Z}$.
\end{proof}










