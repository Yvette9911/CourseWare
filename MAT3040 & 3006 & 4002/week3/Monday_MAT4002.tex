\section{Monday for MAT4002}\index{Monday_lecture}

\paragraph{Reviewing}
\begin{enumerate}
\item
$A\subseteq A_S\subseteq\overline{A}$, where $A_S$ is sequential closure and $\overline{A}$ denotes closure.
\item
Subspace topology
\item
Homeomorphism
\[
f:(0,1)\to(0,1]
\]
whether it is continuos? think of $f^{-1}(1)$.
\item
Base of a topology $B\subseteq\mathcal{T}$
\end{enumerate}

Let $\mathbb{R}^n$ be equipped with usual topology, then
\[
B=\{B_q(x)\mid x\in\mathbb{Q}^n,q\in\mathbb{Q}^+\}
\]
is a basis of $\mathbb{R}^n$.

It suffices to show $U\subseteq\mathbb{R}^n$ can be written as 
\[
U=U_{x\in\mathcal{A}}B_{q_x}(x)
\]

\begin{proposition}
Let $X,Y$ be topological spaces, and $\mathcal{B}$ a basis for topology on $Y$. Then
\[
f:X\to Y\text{ is continuous}
\Longleftrightarrow
f^{-1}(B)\text{ is open in X, }\forall B\in\mathcal{B}
\]
Therefore there is no need to exam $f^{-1}(U)$ is open for all $U\in\mathcal{T}_Y$.
\end{proposition}

\begin{proof}
Since $B\subseteq\mathcal{T}_Y$, the proof for necessity follows.

Let $U\in\mathcal{T}_Y$, then $U=\bigcup_{i\in I}B_i$, where $B_i\in\mathcal{B}$, which implies
\[
f^{-1}(U)=f^{-1}\left(
\bigcup_{i\in I}B_i
\right)
=
\bigcup_{i\in I}f^{-1}(B_i)
\]
which is open in $X$ by our hypthesis.
\end{proof}

\begin{corollary}\label{cor:3:1}
Let $f:X\to Y$ be a bijection. Suppose there is a basis $\mathcal{B}_X$ of $\mathcal{T}_X$ such that $\{f(B)\mid B\in\mathcal{B}_X\}$ forms a basis of $\mathcal{T}_Y$. Then $X\cong Y$.
\end{corollary}
\begin{proof}
Suppose $W\in\mathcal{T}_Y$, then
\[
W=\bigcup_{i\in I}f(B_i),\qquad
B_i\in\mathcal{B}_X\implies
f^{-1}(W)=\bigcup_{i\in I}B_i,
\]
which implies $f$ is continuous due to $f^{-1}(W)\in\mathcal{T}_X$

Suppose $U\in\mathcal{T}_X$, then
\[
U=\bigcup_{i\in I}B_i\implies
f(U)=\bigcup_{i\in I}f(B_i)\in\mathcal{T}_Y\implies
[f^{-1}]^{-1}(U)\in\mathcal{T}_Y,
\]
i.e., $f$ is continuous.
\end{proof}

Question: given a family of subsets $\mathcal{B}$, how can we guarantee it is a basis of a topology?

\begin{proposition}\label{pro:3:4}
Let $X$ be a set, $\mathcal{B}$ is a collection of subsets satisfying
\begin{enumerate}
\item
Every $x\in X$ lies in some $B_x\in\mathcal{B}$
\item
For each $B_1,B_2\in\mathcal{B}$, and $x\in B_1\cap B_2$, then there exists $B_3\in\mathcal{B}$ such that $x\in B_3\subseteq B_1\cap B_2$
\end{enumerate}
Then $B$ defines a basis of a topology of $X$.
\end{proposition}
\begin{proof}
Let $\mathcal{T}$ be the family of subsets obtained by taking union of elements (i.e., subsets) in $\mathcal{B}$. It suffices to show $\mathcal{T}$ is a topology in $X$.
\begin{enumerate}
\item
$\emptyset\in\mathcal{T}$ (taking nothing from $\mathcal{B}$); for $x\in X,B_x\in\mathcal{B}$, by (1),
\[
X=\bigcup_{x\in X}B_x\in\mathcal{T}
\]
\item
Suppose $T_1,T_2\in\mathcal{T}$. Let $x\in T_1\cap T_2 $, where $T_i$ is a union of subsets in $\mathcal{B}$. Therefore,
\[
\left\{
\begin{aligned}
x\in B_1\subseteq T_1,\qquad B_1\in\mathcal{B}\\
x\in B_2\subseteq T_2,\qquad B_2\in\mathcal{B}
\end{aligned}
\right.
\]
which implies $x\in B_1\cap B_2$, i.e., $x\in B_x\subseteq B_1\cap B_2$ for some $B_x\in\mathcal{B}$.

Therefore,
\[
\bigcup_{x\in B_1\cap B_2}\{x\}\subseteq
\bigcup_{x\in B_1\cap B_2}B_x\subseteq B_1\cap B_2,
\]
i.e., $B_1\cap B_2=\bigcup_{x\in B_1\cap B_2}B_x$, i.e., $B_1\cap B_2\in\mathcal{T}$.
\item
The property (3) directly follows definition
\end{enumerate}
Therefore, $\mathcal{B}$ is a basis of the topology formed in the beginning.
\end{proof}

\subsection{Product Space}

\begin{definition}
Let $(X,\mathcal{T}_X),(Y,\mathcal{T}_Y)$ be topological spaces. Consider
\[
\mathcal{B}_{X\times Y}=\{U\times V\mid U\in\mathcal{T}_X,V\in\mathcal{T}_y\}
\]
It is a family of subsets in $X\times Y$. Then $\mathcal{B}_{X\times Y}$ forms a basis of a topology on $X\times Y$. This is called \emph{product topology}.
\end{definition}

e.g., for $X=\mathbb{R},Y=\mathbb{R}$, the elements in $\mathcal{B}_{X\times Y}$ are rectangles.

So we can only hope $B_{X\times Y}$ forms a basis of a topology on $X\times Y$.

In order to see $B_{X\times Y}$ forms a basis, we need proposition~(\ref{pro:3:4}):
\begin{enumerate}
\item
for $\forall (x,y)\in X\times Y, X\in\mathcal{T}_X,Y\in\mathcal{T}_Y$, we imply $(x,y)\in X\times Y\in\mathcal{B}_{X\times Y}$
\item
Suppose $U_1\times V_1,U_2\times V_2\in\mathcal{B}_{X\times Y}$,
\[
(U_1\times V_1)\cap(U_2\times V_2)=(U_1\cap U_2)\times (V_1\cap V_2),
\]
where $U_1\cap U_2\in\mathcal{T}_X,V_1\cap V_2\in\mathcal{T}_Y$. Therefore, $(U_1\times V_1)\cap(U_2\times V_2)\in\mathcal{B}_{X\times Y}$
\end{enumerate}
Therefore, $B_{X\times Y}$ forms a basis of a topology $\mathcal{T}_{X\times Y}$, which is the product topology.

\begin{example}
$\mathbb{R}\times\mathbb{R}\cong\mathbb{R}^2$, where the left is the product topolgy, and the right is the usual topology.

Obviously, $f:\mathbb{R}\times\mathbb{R}\to\mathbb{R}^2$ is a bijection.

Take the basis of the topology on $\mathbb{R}$ as open intervals,
\[
B_X=\{(a,b)\mid a<b\text{ in $\mathbb{R}$}\}
\]
Therefore, 
\[
B_{X\times Y}=\{(a,b)\times(c,d)\mid a<b,c<d\}
\]
It forms a basis of the usual topology in $\mathbb{R}^2$.

By Corollary~(\ref{cor:3:1}), we imply $\mathbb{R}\times\mathbb{R}\cong\mathbb{R}^2$.
\end{example}
\begin{example}
Let $S^1=\{(\cos x,\sin x\mid x\in[0,2\pi])\}$ be a unit circle on $\mathbb{R}^2$.

Consider $f: S^1\times(0,\infty)\to\mathbb{R}^2\setminus\{\bm0\}$ defined as
\[
f(\cos x,\sin x,r)\mapsto(r\cos x,r\sin x)
\]
It's clear that $f$ is a bijection, and $f$ is continuous. 

Moreover, $g:=f^{-1}$ with
\[
g(a,b)=(\frac{a}{\sqrt{a^2+b^2}},\frac{b}{\sqrt{a^2+b^2}},\sqrt{a^2+b^2})
\]
is continuous.

Therefore, the $f:\mathcal{S}^1\times(0,\infty)\to\mathbb{R}^2\setminus\{\bm0\}$ is a homomorphism.
\end{example}
















