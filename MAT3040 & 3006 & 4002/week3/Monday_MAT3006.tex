\section{Monday for MAT3006}\index{Monday_lecture}
\paragraph{Reviewing}
\begin{enumerate}
\item
Compactness/Sequential Compactness:
\begin{itemize}
\item
Equivalence for metric space
\item
Stronger than closed and bounded
\end{itemize}
\item
Completeness: 
\begin{itemize}
\item
The metric space $(E,d)$ is complete if every Cauchy sequence on $E$ is convergent.
\item
$\mathbb{P}[a,b]\subseteq\mathcal{C}[a,b]$ is not complete:
\[
f_N(x)=\sum_{n=0}^N(-1)^n\frac{x^{2n}}{(2n)!}\to\cos x,
\]
while $\cos x\notin\mathcal{P}[a,b]$.
\end{itemize}
\end{enumerate}


\subsection{Remarks on Completeness}

\begin{proposition}
Let $(X,d)$ be a metric space.
\begin{enumerate}
\item
If $X$ is complete and $E\subseteq X$ is closed, then $E$ is complete.
\item
If $E\subseteq X$ is complete, then $E$ is closed in $X$.
\item
If $E\subseteq X$ is compact, then $E$ is complete.
\end{enumerate}
\end{proposition}
\begin{proof}
\begin{enumerate}
\item
Every Cauchy sequence $\{e_n\}$ in $E\subseteq X$ is also a Cauchy sequence in $E$.

Therefore we imply $\{e_n\}\to x\in X$, due to the completeness of $X$.

Due to the closedness of $E$, the limit $x\in E$, i.e., $E$ is complete.
\item
Consider any convergent sequence $\{e_n\}$ in $E$, with some limit $x\in X$.

We imply $\{e_n\}$ is Cauchy and thus $\{e_n\}\to e\in E$, due to the completeness of $E$.

By the uniqueness of limits, we must have $x=z\in E$, i.e., $E$ is closed.
\item
Consider a Cauchy sequence $\{e_n\}$ in $E$.
There exists a subsequence $\{e_{n_j}\}\to e\in E$, 
due to the sequential compactness of $E$.

It follows that for large $n$ and $j$,
\[
d(e_n,e)
\makebox[1cm][c]{$\overset{\text{(a)}} \leq$}
 d(e_n,e_{n_j})+d(e_{n_j},e)
 \makebox[1cm][c]{$\overset{\text{(b)}} <$}
 \varepsilon
\]
where (a) is due to triangle inequality
and (b) is due to the Cauchy property of $\{e_n\}$ and the convergence of $\{e_{n_j}\}$. 

Therefore, we imply $\{e_n\}\to e\in E$, i.e., $E$ is complete.
\end{enumerate}
\end{proof}

\begin{remark}
Given any metric space that may not be necessarily complete, we can make the union of it with another space to make it complete, e.g., just like the completion from $\mathbb{Q}$ to $\mathbb{R}$.
\end{remark}


\subsection{Contraction Mapping Theorem}

The motivation of the contraction mapping theorem comes from solving an equation $f(x)$. More precisely, such a problem can be turned into a problem for fixed points, i.e., it suffices to find the fixed points for $g(x)$, with $g(x)=f(x)+x$.

\begin{definition}
Let $(X,d)$ be a metric space. 
A map $T:(X,d)\to (X,d)$ is a \emph{contraction} 
if there exists a constant $\tau\in(0,1)$ such that
\[
\begin{array}{ll}
d(T(x),T(y))<\tau\cdot d(x,y),
&
\forall x,y\in X
\end{array}
\]
A point $x$ is called a fixed point of $T$ if $T(x)=x$.
\end{definition}
\begin{remark}
All contractions are continuous: 
Given any convergence sequence $\{x_n\}\to x$, 
for $\varepsilon>0$, take $N$ such that 
$d(x_n,x)<\frac{\varepsilon}{\tau}$ for $n\ge N$. It suffices to show the convergence of $\{T(x_n)\}$:
\[
d(T(x_n),T(x))<\tau\cdot T(x_n,x)<\tau\cdot\frac{\varepsilon}{\tau}=\varepsilon.
\]
Therefore, the contraction is Lipschitz continuous with Lipschitz constant $\tau$.
\end{remark}

\begin{theorem}[Contraction Mapping Theorem / Banach Fixed Point Theorem]
Every contraction $T$ in a \emph{complete} metric space $X$ has a unique fixed point.
\end{theorem}

\begin{example}
\begin{enumerate}
\item
The mapping $f(x)=x+1$ is not a contraction in $X=\mathbb{R}$, and it has no fixed point.
\item
Consider an in-complete space $X=(0,1)$ and a contraction $f(x)=\frac{x+1}{2}$. It doesn't admit a fixed point on $X$ as well.
\end{enumerate}
\end{example}

\begin{proof}
Pick any $x_0\in X$, and define $x_{n+1}=T(x_n)$ for $n=0,1,\dots$.

Therefore, $x_n=T^n(x_0)$. It suffices to show the sequence $\{x_n\}$ is Cauchy. For any $n\ge m$,
\begin{align*}
d(x_n,x_m)&=d(T^n(x_0),T^m(x_0))\\
&=d(T(T^{n-1}(x_0)),T(T^{m-1}(x_0)))\\
&\le \tau\cdot d(T^{n-1}(x_0),T^{m-1}(x_0))\\
&\cdots\\
&\le \tau^md(T^{n-m}(x_0),x)\\
&\le\tau[d(T^{n-m}(x_0),T^{n-m-1}(x_0))+\cdots+d(T^2(x_0),T(x_0))+d(T(x_0),x_0)]\\
&\le\tau^m
\left[
\tau^{n-m-1}d(T(x_0),x_0)
+
\tau^{n-m-2}d(T(x_0),x_0)
+
\cdots+
d(T(x_0),x_0)
\right]\\
&<\tau^m
\frac{1-\tau^{n-m}}{1-\tau}d(x_1,x_0)\\
&\le\frac{\tau^m}{1-\tau}d(x_1,x_0)
\end{align*}
When $m$ is large, $d(x_n,x_m)\to0$.

Therefore, $\{x_n\}$ is Cauchy. By the completeness of $X$, we imply $\{x_n\}\to x\in X$

Now we imply
\[
x=\lim_{n\to\infty}x_{n+1}=\lim_{n\to\infty}T(x_n)=T(\lim_{n\to\infty}x_n)=T(x),
\]
i.e., $x$ is a fixed point of $T$.

Suppose $T(x)=x$ and $T(y)=y$, we imply
\[
d(T(x),T(y))<rd(x,y)\implies d(x,y)<rd(x,y)
\]
Therefore $d(x,y)=0$, i.e., $x=y$.

\end{proof}

\begin{example}[Newton's Method]
Define
\[
\frac{y-f(x_n)}{x-x_n}=f'(x_n)
\]
and put $x=x_{n+1}$ into the intersect $y=0$, i.e.,
\[
x_{n+1}=x_n-\frac{f(x_n)}{f'(x)}
\]
Suppose$f(x)=0$ is not a root, then
\begin{enumerate}
\item
$f'(r)\ne0$
\item
$f\in\mathcal{C}^2$ on some neighborhood of $r$
\end{enumerate}
Then there exists $[r-\varepsilon,r+\varepsilon]$ such that the mapping
\[
T:\mathcal{C}[r-\varepsilon,r+\varepsilon]\to\mathbb{R}
\]
satisfying $|T'(x)|<1$ for all $x\in [r-\varepsilon,r+\varepsilon]
$.


Assume this is true for any $x,y\in [r-\varepsilon,r+\varepsilon]$,
\[
d(T(x),T(y))=|T(x)-T(y)|=|T'(z)||x-y|
\]
Since $|T'(w)|<1$ for all $w\in[r-\varepsilon,r+\varepsilon]$, we claim that $\max_{w\in[r-\varepsilon,r+\varepsilon]}|T'(w)|=r<1$.

Then, MVT implies $|T(x)-T(y)|<r|x-y|$ for $\forall x,y\in[r-\varepsilon,r+\varepsilon]$.

Therefore, $T\in\mathcal{C}[r-\varepsilon,r+\varepsilon]$ is a contraction, and thereforewe have $T(x)=x$, which implies
\[
x-\frac{f(x)}{f'(x)}=x\implies\frac{f(x)}{f'(x)}=0\implies f(x)=0,
\]
i.e., we obtain a root $x=r$.

Conclusion: if we use Newton's method on any point between $[r-\varepsilon,r+\varepsilon]$ where $f(r)=0$, then we will eventually get close to $r$.


So we need to show that $|T'(x)<1|$ for $x\in[r-\varepsilon,r+\varepsilon]$.

Indeed,
\[
T'(x)=\frac{f(x)}{[f'(x)]^2}f''(x)\implies
|T'(x)|=h(x)
\]
Note that $h(r)=0$ and $h(x)$ is continuous, which implies
\[
r\in h^{-1}((-1,1))\implies B_\rho(r)\subseteq h^{-1}((-1,1))
\]
Therefore, $h((r-\rho,r+\rho))\subseteq(-1,1)$. Take $\varepsilon=\frac{\rho}{2}$, the proof is complete.
\end{example}

\subsection{Picard Lindelof Theorem}
We will use Banach fixed point theorem to show the existence and uniqueness of the solution of ODE
\[
\left\{
\begin{aligned}
\frac{\diff y}{\diff x}&=f(x,y(x))\\
y(x_0)&=y_0
\end{aligned}
\right.\qquad
\textbf{Initial Value Problem, IVP}
\]

The separation of variables can be applied to solve
\[
\left\{
\begin{aligned}
\frac{\diff y}{\diff x}&=x^{2}y^{1/5}\\
y(x_0)&=c>0
\end{aligned}
\right.
\implies 
y=\left(
\frac{4x^3}{15}+c^{4/5}
\right)^{5/4}
\]
This problem has a unique solution. However, when $c=0$, the ODE does not have a unique solution:
\[
y_1=(\frac{4x^3}{15})^{5/4},\qquad
y_2=0
\]










