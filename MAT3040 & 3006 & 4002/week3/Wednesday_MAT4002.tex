\section{Wednesday for MAT4002}\index{Monday_lecture}
\paragraph{Reviewing}
\begin{itemize}
\item
Product Topology:
\[
\mathcal{B}_{X\times Y}=\{U\times V\mid U\in\mathcal{T}_X,V\in\mathcal{T}_Y\}
\]
e.g., $X=Y=\mathbb{R}$.
\[
S^1\times S^1\cong T
\]
\end{itemize}
\begin{proposition}
If $f:[0,2\pi]\to X$ is continuous and $f(0)=f(2\pi)$, then
\[
f:S^1\to X
\]
is continuous. For example, $f(\theta)=(\cos\theta,\sin\theta)$ is continuous for $f:S^1\to\mathbb{R}^3$.
\end{proposition}

The mapping $f:[0,2\pi]\times[0,2\pi]\to T\subseteq\mathbb{R}^3$ is continuous,
\[
f(\theta,\phi)=\begin{pmatrix}
(R+r\cos\theta)\cos\phi,
&
(R+r\cos\theta)\sin\phi,
&
r\sin\theta
\end{pmatrix}
\]
Reason: define $i:T\to\mathbb{R}^3$, and
\[
i\circ f:[0,2\pi]\times[0,2\pi]\to\mathbb{R}^3
\]
Note that $i\circ f$ is continuous iff $f:[0,2\pi]\times[0,2\pi]\to T$ is continuous.
\[
\left\{
\begin{aligned}
f(0,y)&=f(2\pi,y)\\
f(x,0)&=f(x,2\pi)
\end{aligned}
\right.
\]
Therefore, $f:S^1\times S^1\to T$ is continuous. We can also show it is bijective. We can also show $f^{-1}$ is continuous.

\begin{proposition}
\begin{enumerate}
\item
Let $X\times Y$ be endowed with product topology. The $p_X:X\times Y\to X$ and $p_Y:X\times Y\to Y$ with $(x,y)\mapsto x$ and $(x,y)\mapsto y$ are continuous
\item
The product topology is the \emph{coarest topology} on $X\times Y$ such that $p_X$ and $p_Y$ are both continuous.
\item
Let $Z$ be a topological space, then the product topology is the unique topology such that 
\[
F:Z\to X\times Y\text{ is continuous}
\]
if and only if both $P_X\circ F:Z\to X$ and $P_Y\circ F:Z\to Y$ are continuous.
\end{enumerate}
\end{proposition}
\begin{proof}
\begin{enumerate}
\item
For $\forall U\in\mathcal{T}_X$, we imply\[
p_X^{-1}(U)=U\times Y\in\mathcal{B}_{X\times Y}\subseteq\mathcal{T}_{X\times Y}
\]
The same goes for $p_Y$
\item
For any topology $\mathcal{T}$ on $X\times Y$ such that  $p_X,p_Y$ are continuous, we imply
\[
\begin{array}{ll}
p_X^{-1}(U)=U\times X\in\mathcal{T},
&
p_Y^{-1}(V)=X\times V\in\mathcal{T},
\end{array}
\]
for $\forall U\in\mathcal{T}_X,V\in\mathcal{T}_Y$, which implies
\[
(U\times Y)\cap(X\times V)\in\mathcal{T},
\]
o.e., $U\times V\in\mathcal{T}$ for all $U\in\mathcal{T}_X,V\in\mathcal{T}_Y$, which implies
\[
\mathcal{B}_{X\times Y}\subseteq\mathcal{T},
\]
i.e., $\mathcal{T}_{\text{product topology}}\subseteq\mathcal{T}$.
\item
Exercise: If $\mathcal{T}=\mathcal{T}_{\text{product}}$, then $\mathcal{T}_{\text{product}}$ satisfies (*).

Then we show the uniqueness. Suppose $\mathcal{T}$ is another topology $X\times Y$ satisfying (*).

Take $Z=(X\times Y,\mathcal{T})$, then $F=\text{id}:Z\to Z$ is continuous.

It implies that $p_X\circ\text{id}$ and $p_Y\circ\text{id}$ are continuous, i.e., $p_X$ and $p_Y$ are continuous, by (2) implies that $\mathcal{T}_{\text{product}}\subseteq\mathcal{T}$.


Take $Z=(X\times Y,\mathcal{T}_{\text{product}})$ and $F=\text{id}$. Then
\[
p_X\circ F=p_X
\]
is continuous by (1). SImilarly, $p_Y\circ F$ is continuous. Therefore $F:Z\to(X\times Y,\mathcal{T})$ is continuous, i.e., the identity mapping
\[
(X\times Y,\mathcal{T}_{\text{product}})
\times
(X\times Y,\mathcal{T})\text{ is continuous}
\]
for $\forall U\in\mathcal{T}$, which implies
$U=\text{id}^{-1}(U)\subseteq\mathcal{T}_{\text{product}}$. Therefore
\[
\mathcal{T}\subseteq\mathcal{T}_{\text{product}}
\implies
\mathcal{T}=\mathcal{T}_{\text{product}}
\]
\end{enumerate}
\end{proof}

\begin{definition}[Disjoint Union]
Let $X\times Y$ be two topological spaces, then the \emph{disjoint union} is
\[
\displaystyle
X\prod Y:=(X\times\{0\})\cup(Y\times\{1\})
\]
\end{definition}
Then $U$ is open in $X\prod Y$ if
\begin{enumerate}
\item
$U\cap(X\times\{0\})$ is open in $X\times\{0\}$; and
\item
$U\cap(Y\times\{1\})$ is open in $Y\times\{1\}$.
\end{enumerate}

$S$ is open in $X\perp Y$ iff 
\[
S=(U\times\{0\})\cup(V\times\{1\})
\]
where $U\subseteq X$ is open and $V\subseteq Y$ is open.

\subsection{Properties of Topological Spaces}
\subsubsection{Hausdorff Property}
\begin{definition}
A topological space $X$ satisfies the \emph{first separation axiom} if for any two distinct points $x\ne y\in X$, there exists open $U\ni x$ but not including $y$.
\end{definition}

\begin{proposition}
A topological space $X$ has first separation property if and only if for $\forall x\in X$, $\{x\}$ is closed in $X$.
\end{proposition}
\begin{proof}
Sufficiency.
Suppose that $x\ne y$, then $x\in U\setminus\{y\}$ is open but not includes $y$.

Necessity.
Take any $x\in X$, then for $\forall y\ne x$, there exists $y\in U_y$ that is open and $x\notin U_y$. Thus 
\[
\{y\}\subseteq U_y\subseteq X\setminus\{x\}
\]
which implies
\[
\bigcup_{y\in X\setminus\{x\}}\{y\}\subseteq
\bigcup_{y\in X\setminus\{x\}}U_y\subseteq
X\setminus\{x\},
\]
i.e., $X\setminus\{x\}=\bigcup_{y\in X\setminus\{x\}}U_y$ is open in $X$, i.e., $\{x\}$ is closed in $X.$
\end{proof}

\begin{definition}[Second separation]
A topological space satisfies the \emph{second separation axiom} (or $X$ is Hausdorff) if for all $x\ne y$ in $X$, there exists open sets $U,V$ such that
\[
\begin{array}{lll}
x\in U,
&
y\in V,
&
U\cap V=\emptyset
\end{array}
\]
\end{definition}

\begin{example}
All metrizable topological spaces are Hausdorff.

Suppose $d(x,y)=r>0$, then take $B_{r/2}(x)$ and $B_{r/2}(y)$
\end{example}

\begin{example}
Given the $\mathcal{T}_{\text{co-finite}}$, then $X$ is first separable but not Hausdorff.
\[
X\setminus(U\cap V)
=
(X\setminus U)\cup(X\setminus V)
\]
\end{example}

















