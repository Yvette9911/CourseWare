\section{Monday for MAT3006}\index{Monday_lecture}
\paragraph{Reviewing}
\begin{enumerate}
\item
Equivalent Metric:
\[
d_1(\bm x,\bm y)\le Kd_2(\bm x,\bm y)\le K'd_1(\bm x,\bm y)
\]

In $\mathcal{C}[0,1]$, the metric $d_1$ and $d_\infty$ are not equivalent: 

For $f_n(x)=x^nn^2(1-x)$, $d_1(f_n,0)\to1$ and $d_\infty(f_n,0)\to\infty$. Suppose on contrary that 
\[
d_1(f_n,0)\le Kd_\infty(\bm x,\bm y)\le K'd_1(\bm x,\bm y).
\]
Taking limit both sides, we imply the immediate term goes to infinite, which is a contradiction.
\item
Continuous functions: the function $f$ is continuous is equivalent to say for $\forall x_n\to x$, we have $f(x_n)\to f(x)$.
\item
Open sets: Let $(X,d)$ be a metric space. A set $U\subseteq X$ is open if for each $x\in U$, there exists $\rho_x>0$ such that $B_{\rho_x}(x)\subseteq U$.
\end{enumerate}

\begin{remark}
Unless stated otherwise, we assume that 
\[
\mathcal{C}[a,b]\longleftrightarrow
(\mathcal{C}[a,b],d_\infty)
\]
\[
\mathbb{R}^n\longleftrightarrow
(\mathbb{R}^n,d_2)
\]
\end{remark}
\subsection{Remark on Open and Closed Set}

\begin{example}\label{Exp:2:6}
Let $X=\mathcal{C}[a,b]$, show that the set 
\[
U:=\{f\in X\mid f(x)>0,\forall x\in[a,b]\}\quad\text{is open}.
\]
Take a point $f\in U$, then
\[
\inf_{[a,b]}f(x)=m>0.
\]
Consider the ball $B_{m/2}(f)$, and for $\forall g\in B_{m/2}(f)$,
\begin{align*}
|g(x)|&\ge |f(x)|-|f(x)-g(x)|\\
&\ge\inf_{[a,b]}|f(x)|-\sup_{[a,b]}|f(x)-g(x)|\\
&\ge m-\frac{m}{2}\\&=\frac{m}{2}>0,\ \forall x\in[a,b]
\end{align*}
Therefore, we imply $g\in U$, i.e., $B_{m/2}(f)\subseteq U$, i.e., $U$ is open in $X$.
\end{example}

\begin{proposition}
Let $(X,d)$ be a metric space. Then 
\begin{enumerate}
\item
$\emptyset,X$ are open in $X$
\item
If $\{U_\alpha\mid\alpha\in\mathcal{A}\}$ are open in $X$, then $\bigcup_{\alpha\in\mathcal{A}}$ is also open in $X$
\item
If $U_1,\dots,U_n$ are open in $X$, then $\bigcap_{i=1}^nU_i$ are open in $X$
\end{enumerate}
\end{proposition}

\begin{remark}
Note that $\bigcap_{i=1}^\infty U_i$ is not necessarily open if all $U_i$'s are all open:
\[
\bigcap_{i=1}^\infty
\left(
-\frac{1}{i},1+\frac{1}{i}
\right)
=
[0,1]
\]
\end{remark}

\begin{definition}[Closed]
The closed set in metric space $(X,d)$ are the complement of open sets in $X$, i.e., any closed set in $X$ is of the form $V=X\setminus U$, where $U$ is oepn.
\end{definition}

For example, in $\mathbb{R}$, 
\[
[a,b]=\mathbb{R}\setminus\{(-\infty,a)\bigcup(b,\infty)\}
\]

\begin{proposition}
\begin{enumerate}
\item
$\emptyset,X$ are closed in $X$
\item
If
$\{V_\alpha\mid\alpha\in\mathcal{A}\}$ are closed subsets in $X$, then $\bigcap_{\alpha\in\mathcal{A}}V_\alpha$ is also closed in $X$
\item
If $V_1,\dots,V_n$ are closed in $X$, then $\bigcup_{i=1}^nV_i$ is also closed in $X$.
\end{enumerate}
\end{proposition}

\begin{remark}
Whenever you say $U$ is open or $V$ is closed, you need to specify the underlying space, e.g.,
\begin{align*}
\textbf{Wrong}: &\text{$U$ is open}\\
\textbf{Right}: &\text{$U$ is open in $X$}
\end{align*}
\end{remark}

\begin{proposition}
The following two statements are equivalent:
\begin{enumerate}
\item
The set $V$ is closed in metric space $(X,d)$.
\item
If the sequence $\{v_n\}$ in $V$ converges to $x$, then $x\in V$
\end{enumerate}
\end{proposition}

\begin{proof}
\textit{Necessity.}

Suppose on the contrary that $\{v_n\}\to x\notin V$. Since $X\setminus V\ni x$ is open, there exists an open ball
$
B_\varepsilon(x)\subseteq X\setminus V.
$

Due to the convergence of sequence, there exists $N$ such that $d(v_n,x)<\varepsilon$ for $\forall n\ge N$, i.e., $v_n\in B_\varepsilon(x)$, i.e., $v_n\notin V$, which contradicts to $\{v_n\}\subseteq V$.

\textit{Sufficiency.}

Suppose on the contrary that $V$ is not closed in $X$, i.e., $X\setminus V$ is not open, i.e., there exists $x\notin V$ such that for all open $U\ni x$, $U\bigcap V\ne\emptyset$. In particular, take 
\[
U_n=B_{1/n}(x),\implies\exists v_n\in B_{1/n}(x)\bigcap V,
\]
i.e., $\{v_n\}\to x$ but $x\notin V$, which is a contradiction.
\end{proof}

\begin{proposition}\label{Pro:2:5}
Given two metric space $(X,d)$ and $(Y,\rho)$, the following statements are equivalent:
\begin{enumerate}
\item
A function $f:(X,d)\to(Y,\rho)$ is continuous on $X$
\item
For $\forall U\subseteq Y$ open in $Y$, $f^{-1}(U)$ is open in $X$.
\item
For $\forall V\subseteq Y$ closed in $Y$, $f^{-1}(V)$ is closed in $X$.
\end{enumerate}
\end{proposition}

\begin{example}
The mapping $\Psi:\mathcal{C}[a,b]\to\mathbb{R}$ is defined as:
\[
f\mapsto f(c)
\]
where $\Psi$ is called a \emph{functional}. 

Show that $\Psi$ is continuous by using $d_\infty$ metric on $\mathcal{C}[a,b]$:
\begin{enumerate}
\item
Any open set in $\mathbb{R}$ can be written as countably union of open disjoint intervals, and therefore suffices to consider the pre-image $\Psi^{-1}(a,b)=\{f\mid f(c)\in(a,b)\}$. Following the similar idea in Example~(\ref{Exp:2:6}), it is clear that $\Psi^{-1}(a,b)$ is open in $(\mathcal{C}[a,b],d_\infty)$. Therefore, $\Psi$ is continuous. 
\item
Another way is to apply definition.
\end{enumerate}
\end{example}

We now study open sets in a subspace $(Y,d_Y)\subseteq(X,d_X)$, i.e.,
\[
d_Y(y_1,y_2):=d_X(y_1,y_2).
\]

Therefore, the open ball is defined as
\begin{align*}
B_\varepsilon^Y(y)&=\{y'\in Y\mid d_Y(y,y')<\varepsilon\}\\
&=\{y'\in Y\mid d_X(y,y')<\varepsilon\}\\
&=\{y'\in X\mid d_X(y,y')<\varepsilon, y'\in Y\}\\
&=B_\varepsilon^X(y)\bigcap Y
\end{align*}
\begin{proposition}
All open sets in the subspace $(Y,d_Y)\subseteq (X,d_X)$ are of the form $U\bigcap Y$, where $U$ is open in $X$.
\end{proposition}
\begin{corollary}
For the subspace $(Y,d_Y)\subseteq (X,d_X)$, the mapping $i:(Y,d_Y)\to(X,d_X)$ with $i(y)=y,\forall y\in Y$ is continuous.
\end{corollary}
\begin{proof}
$i^{-1}(U)=U\bigcap Y$ for any subset $U\subseteq X$. The results follows from proposition~(\ref{Pro:2:5}).
\end{proof}

\begin{remark}
It's important to specify the underlying space to describe an open set. 

For example, the interval $[0,\frac{1}{2})$ is not open in $\mathbb{R}$, while $[0,\frac{1}{2})$ is open in $[0,1]$, since
\[
[0,\frac{1}{2})
=
(-\frac{1}{2},\frac{1}{2})
\bigcap
[0,1].
\]
\end{remark}

\subsection{Boundary, Closure, and Interior}

\begin{definition}
Let $(X,d)$ be a metric space, then
\begin{enumerate}
\item
A point $x$ is a \emph{boundary point} of $S\subseteq X$ (denoted as $x\in\partial S$)
if for any open $U\ni x$, then both $U\bigcap S,U\setminus S$ are non-empty.

(one can replace $U$ by $B_{1/n}(x)$, with $n=1,2,\dots$)
\item
The \emph{closure} of $S$ is defined as $\overline{S}=S\bigcup\partial S$.
\item
A point $x$ is an \emph{interior point} of $S$ (denoted as $x\in S^\circ$) 
if there $\exists U\ni x$ open such that $U\subseteq S$. We use $S^\circ$ to denote the set of interior points.
\end{enumerate}
\end{definition}

\begin{proposition}
\begin{enumerate}
\item
The closure of $S$ can be equivalently defined as
\[
\overline{S}=\bigcap\{C\in X\mid\text{$C$ is closed and $C\supseteq S$}\}
\]

Therefore, $\overline{S}$ is the smallest closed set containing $S$.
\item
The interior set of $S$ can be equivalently defined as
\[
S^\circ=\bigcup\{U\subseteq X\mid\text{$U$ is open and $U\subseteq S$}\}
\]

Therefore, $S^\circ$ is the largest open set contained in $S$.
\end{enumerate}
\end{proposition}
\begin{example}
For $S=[0,\frac{1}{2}]\subseteq X$, we have
\begin{enumerate}
\item
$\partial S=\{0,\frac{1}{2}\}$
\item
$\overline{S}=[0,\frac{1}{2}]$
\item
$S^\circ=(0,\frac{1}{2})$
\end{enumerate}
\end{example}
\begin{proof}
\begin{enumerate}
\item
\begin{enumerate}
\item
Firstly, we show that $\overline{S}$ is closed, i.e., $X\setminus\overline{S}$ is open.
\begin{itemize}
\item
Take $x\notin\overline{S}$. Since $x\notin\partial S$, there $\exists B_r(x)\ni x$ such that 
\[
\begin{array}{lll}
B_r(x)\bigcap S,
&
\text{or}
&
B_r(x)\setminus S\text{ is }\emptyset.
\end{array}
\]
\item
Since $x\notin S$, the set $B_r(x)\setminus S$ is not empty. Therefore, $B_r(x)\bigcap S=\emptyset$.
\item
It's clear that $B_{r/2}(x)\bigcap S=\emptyset$. We claim that $B_{r/2}(x)\bigcap \overline{S}$ is empty.

Suppose on the contrary that
\[
y\in B_{r/2}(x)\bigcap\partial S,
\]
which implies that $B_{r/2}(y)\bigcap S\ne\emptyset$. Therefore,
\[
B_{r/2}(y)\subseteq B_r(x)\implies 
B_r(x)\bigcap S\supseteq B_{r/2}(y)\bigcap S\ne\emptyset,
\]
which is a contradiction.
\end{itemize}
Therefore, $x\in X\setminus\overline{S}$ implies $B_{r/2}(x)\bigcap\overline{S}=\emptyset$, i.e., $X\setminus\overline{S}$ is open, i.e., $\overline{S}$ is closed.
\item
Secondly, we show that $\overline{S}\subseteq C$, for any closed $C\supseteq S$, i.e., suffices to show $\partial S\subseteq C$.

Take $x\in\partial S$, and construct a sequence
\[
x_n\in B_{1/n}(x)\bigcap S.
\] 
Here $\{x_n\}$ is a sequence in $S\subseteq C$ converging to $x$, 
which implies $x\in C$, 
due to the closeness of $C$ in $X$.
\end{enumerate}
Combining (a) and (b), the result follows naturally. (Question: do we need to show the well-defineness?)
\item
Exercise. Show that 
\[
S^\circ=S\setminus \partial S=X\setminus(\overline{X\setminus S}).
\]
Then it's clear that $S^\circ$ is open, and contained in $S$.
\end{enumerate}
\end{proof}

The next lecture we will talk about compactness and sequential compactness.











