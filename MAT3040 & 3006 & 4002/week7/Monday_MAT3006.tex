\section{Monday for MAT3006}\index{Monday_lecture}
Our first mid-term will be held on this Wednesday.
\paragraph{Reviewing}
\begin{itemize}
\item
Outer Measure: For $E\subseteq\mathbb{R}$,
\[
m^*(E)=\inf\left\{
\sum_{n=1}^\infty m(I_n)\middle|
E\subseteq\bigcup_{n=1}^\infty I_n,\ \text{$I_n$ are open intervals}
\right\}
\]
\end{itemize}
\begin{proposition}
\begin{enumerate}
\item
$m^*(\phi)\cap m^*(\{x\})=0$
\item
$m^*(E+x) = m^*(E)$
\item
$m^*(\{a,b\}) = b-a$, where the brackets can be replaced by $($ or $]$.
\item
If $A\subseteq B$, then $m^*(A)\le m^*(B)$
\item
$m^*(kE) = |k|m^*(E)$
\item
$m^*(\cup_{m=1}^\infty E_n)\le \sum_{n=1}^\infty m^*(E_n)$
\end{enumerate}
\end{proposition}
The trick to show $x\le y$ is by argument $x\le y+\varepsilon,\forall \varepsilon>0$.
\begin{proof}
(1),(2),(5) is clear.

(4) is a one-line argument: suppose $B\subseteq\cup_{n=1}^\infty I_n$, then
\[
A\subseteq\cup_{n=1}^\infty I_n
\]
(3): Consider $m^*([a,b])$ first:
\[
[a,b]\subseteq(a-\frac{\varepsilon}{2},b+\frac{\varepsilon}{2})\cup(a,a)\cup\cdots
\]
which implies 
\[
m^*([a,b])\le m((a-\frac{\varepsilon}{2},b+\frac{\varepsilon}{2}))+0+\cdots+0=(b-a)+\varepsilon,\ \forall\varepsilon>0
\]
In particular, $m^*([a,b])\le b-a$.

Conversely, to show $b-a\le m^*([a,b])$, for all $\varepsilon>0$, there exists $I_n, \ n\in\mathbb{N}$ such that
\[
[a,b]\subseteq \cup_{n=1}^\infty I_n,\qquad
\sum_{n=1}^\infty m(I_n)\le m^*([a,b])+\varepsilon
\]
By Heine-Borel Theorem, there exists finite subcover $[a,b]\subseteq \cup_{n=1}^k I_n$.
Let $I_n = (\alpha_n,\beta_n)$, consider $\alpha:=\min\{\alpha_n\mid a\in I_n\}$ and $\beta:=\max\{\beta_n\mid b\in I_n\}$.
Then we imply 
\[
[a,b]\subseteq(\alpha,\beta)\subseteq \cup_{n=1}^kI_n.
\]
We claim that
\[
\beta - \alpha\le \sum_{n=1}^km(I_n)
\]
Therefore, $b-a\le \beta-\alpha\le \sum_{n=1}^km(I_n)\le\sum_{n=1}^\infty m(I_n)\le m^*([a,b])+\varepsilon$.
The proof is complete.

For $m^*((a,b))$, 
\[
a-b = m^*([a+\frac{\varepsilon}{2},b-\frac{\varepsilon}{2}]) + \frac{\varepsilon}{2}+\frac{\varepsilon}{2}
\le
m^*((a,b))+\varepsilon
\]
The proof is complete.
\end{proof}
\begin{proof}[Proof for (6)]
When there exists $m^*(E_n)=\infty$, check both sides equal to infinite.

Consider the case where $m^*(E_n)<\infty$ only.
Then $E_n\subseteq\cup_{k=1}^\infty I_{n,k}$, with $\sum_{k=1}^\infty m(I_{n,k})\le m^*(E_n) + \frac{\varepsilon}{2^n}$.

Note that 
\begin{itemize}
\item
$\cup_{n=1}^\infty\cup_{k=1}^\infty I_{n,k}$ is a countable open cover of $\cup_{n=1}^\infty E_n$
\item
$\sum_{n,k}m(I_{n,k})\le \sum_{n=1}^\infty m^*(E_n)+\varepsilon$
\end{itemize}
Therefore,
\[
m^*(\cup_{n=1}^\infty E_n)\le \sum_{n,k}m(I_{n,k})\le\sum_{n=1}^\infty m^*(E_n)+\varepsilon
\]


\end{proof}


\begin{definition}[Null Set]
The set $E\subseteq\mathbb{R}$ is a \emph{null set} if
$m^*(E)=0$.

These are the points we can ignore.
\end{definition}
\begin{corollary}
\begin{enumerate}
\item
If $E$ is null, so is any subset $E'\subseteq E$
\item
If $E_n$ is null for all $n\in\mathbb{E}$, so is $\cup_{n=1}^\infty E_n$
\item
All countable subsets of $\mathbb{R}$ are null.
\end{enumerate}
\end{corollary}
\begin{proof}
(1) is clear; (2) follows the same trick as in (6);
(3) follows from (2) and part (1) of previous proposition
\end{proof}

\begin{remark}
\begin{enumerate}
\item
Are there any uncountable null sets?
\item
We have another notion of ``meagre''. Is null = meagre?
\end{enumerate}
\end{remark}
\begin{example}[Cantor Set]
$C_0=[0,1]$;
$C_1=[0,1/3]\cup[2/3,1]$;
$C=\cap_{n=1}^\infty C_n$.

Note that $C$ is null, since $C\subseteq C_n\forall n$,
\[
m^*(C_n) = (2/3)^n\implies
m^*(C)\le m^*(C_n) = (2/3)^n\implies
m^*(C)=0.
\]
$C$ is uncountable:
every element in $C$ can be expressed uniquely in ternary expression, i.e., only use 0,1,2 as digits.

Suppose on the contrary that $C$ is countable, i.e., $C=\{c_n\}_{n\in\mathbb{N}}.$
Then construct a new number such that $c\notin \{c_n\}_{n\in\mathbb{N}}$ (diagonal argument).
\end{example}
For question 2, if $E$ is countable set, then $E$ is null and $E$ is meagre (by Baire Category Theorem).

Define $E = C$ (cantor set), then $E$ is null. Is $E$ meagre?
\begin{proposition}
$C$ is nowhere dense, i.e., $C$ is meagre.
\end{proposition}
\begin{proof}
Firstly, $C$ is closed, since intersection of closed sets is closed.

Suppose on the contrary that $(\alpha,\beta)\subseteq C$, then $(\alpha,\beta)\subseteq C_n = \sqcup_{k=1}^{2^n}[a_{n,k},b_{n,k}]$ for all $n$

Therefore, $(\alpha,\beta)\subseteq[a_{n,k},b_{n,k}]$ for some $k$
\[
\beta-\alpha<b_{n,k} - a_{n,k}=\frac{1}{3^n},\ \forall n\in\mathbb{N}
\]
Therefore, $\beta-\alpha=0$, which is a contradiction.
\end{proof}

There exists a mergre set $S$ with $m^*(S)=\infty$; a null set that is co-meagre.

\begin{definition}[Measure]
A meaasure of length for all subsets in $\mathbb{R}$ is a function $m$ satisfying
\begin{enumerate}
\item
$m(\emptyset) = m(\{x\})=0$
\item
$m(\{a,b\}) = b-a$
\item
$m(A+x) = m(A),\forall x\in\mathbb{R}$
\item
If $A\subseteq B$, then $m(A)\le m(B)$
\item
$m(kA) = |k|m(A)$
\item
If $E_i\cap E_j = \emptyset,\forall i\ne j$, then 
\[
\sum_{i=1}^\infty m(E_i) = m(\cup_{i=1}^\infty E_i)
\]
\end{enumerate}
\end{definition}
Question: $m^*$ satisfies (1) to (5), does all subsets satisfying $(6)$?

Answer: no.









