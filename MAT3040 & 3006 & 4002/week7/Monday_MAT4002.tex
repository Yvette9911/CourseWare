\section{Monday for MAT4002}\index{Monday_lecture}
\subsection{Quotient Map}
\begin{definition}[Quotient Map]
A map $q:X\to Y$ between topological spaces is a \emph{quotient map} if 
\begin{enumerate}
\item
$q$ is surjective
\item
For any $U\subseteq Y$, $U$ is open iff $q^{-1}(U)$ is open.
\end{enumerate}
\end{definition}
For example, 
\begin{enumerate}
\item
the mapping $p:X\to X/\sim$ is a quotient map
\item
For $f$ is continuous and maps open sets to open sets. Then $f$ satisfies (2).
\end{enumerate}

\begin{proposition}
Suppose $q:X\to Y$ is a quotient map, and $\sim$ is an equivalence relation given by the partition $\{q^{-1}(y)\mid y\in Y\}$ of $X$.
Then $X$ and $Y$ are \emph{homeomorphic}.
\end{proposition}
\begin{proof}
Let $h:X/\sim\to Y$ defined as $h([x]) = q(x)$.
\begin{enumerate}
\item
The mapping $h$ is well-defined and injective.
\item
Surjective is easy to shown
\item
$q = h\circ p$, by (2), is continuous. Therefore, $h$ is continuous.
\end{enumerate}
It suffices to show $h^{-1}$ is continuous:
For any open $\tilde{U}\subseteq X/\sim$, it suffices to show $h(\tilde{U})$ is open in $Y$. Note that 
\[
q^{-1}[h(\tilde{U})] = p^{-1}h^{-1}(h(\tilde{U}))=p^{-1}(\tilde{U}),
\]
which is open by definition of quotient topology.
By (2), $h(\tilde{U})$ is open
\end{proof}

\begin{example}
Define the mapping
\[
\begin{array}{ll}
q:&\mathbb{R}\to S^1\\
&x\mapsto e^{2\pi i x}
\end{array}
\]
Note that $q$ open balls to open balls. (Check what $q(a,b)$ is)

$q^{-1}(e^{2\pi i x}) = \{x+z\mid z\in\mathbb{Z}\}$. Therefore, 
\[
\mathbb{R}/\sim\cong S^1,
\]
where $x\sim y$ iff $x-y\in\mathbb{Z}$, i.e..,
\[
\mathbb{R}/\mathbb{Z}\cong S^1
\]
\end{example}


\subsection{Simplicial Complex}
The idea is to build some new spaces from some fundamental objects.
Then we can use the combinatorics of these fundamental objects to study topology.

\begin{definition}[$n$-simplex]
The standard $n$-simplex is the set
\[
\Delta^n = \{(x_1,\dots,x_{n+1})\in\mathbb{R}^{n+1}\mid x_i\ge0, \sum_{i}x_i=1\}
\]
\begin{enumerate}
\item
vertices of $\Delta^n$ are the points on $\Delta^n$ with
$x_i=1$ for some $i$.
\item
face of $\Delta^n:$ For $\mathcal{A}\subseteq\{1,\dots,n+1\}$, a face is given by
\[
\{(x_1,\dots,x_{n+1})\in\Delta^n\mid x_i=0,\ \forall i\notin\mathcal{A}\}
\]
\item
inside of $\Delta^n$ is 
\[
\{(x_1,\dots,x_{n+1})\in\Delta^n\mid x_i>0,\forall i\}
\]
(Inside of $\Delta^0$ is $\Delta^0$)
\end{enumerate}
\end{definition}

\begin{definition}[face inclusion]
A face inclusion of $\Delta^m$ into $\Delta^n$ ($m<n$) is a function $\Delta^m\to\Delta^n$ which comes from the restriction of an \emph{injective linear} map
\[
f:\mathbb{R}^{n+1}\to\mathbb{R}^{n+1},
\]
which maps vertices into vertices.
\end{definition}
\[
\begin{pmatrix}
1\\0
\end{pmatrix}\mapsto\begin{pmatrix}
0\\1\\0
\end{pmatrix}\quad
\begin{pmatrix}
0\\1
\end{pmatrix}\mapsto\begin{pmatrix}
0\\0\\1
\end{pmatrix}
\]
Any injection from $\{1,\dots,m+1\}\to\{1,\dots,n+1\}$ gives a face inclusion $\Delta^m\to\Delta^n$, and vice versa.

\begin{definition}[Simplicial Complex]
An (abstract) \emph{simplicial complex} is a pair $(V,\Sigma)$,
where $V$ is a set of vertices and $\Sigma$ is a collection of non-empty finite subsets of $V$ (simplices) such that
\begin{enumerate}
\item
$\forall \bm v\in V$, $\{\bm v\}\in\Sigma$
\item
If $\sigma\in\Sigma$, then any non-empty subset of $\sigma$ must lie in $\Sigma$.

e.g., $V=\{1,2,3,4\}$, then 
\[
\Sigma=\{\{1\},\{2\},\{3\},\{4\},\{1,3,4\},\{2,4\},\{1,3\},\{3,4\},\{1,4\}\}
\]
\end{enumerate}
\end{definition}

\begin{definition}[Topological Realization]
The \emph{topological realization} of $K=(V,\Sigma)$ is 
a topological space $|K|$ (or denoted as $|(V,\Sigma)|$), where
\begin{enumerate}
\item
For each $\sigma\in\Sigma$ with $|\sigma|=n+1$, 
take a copy of $n$-simplex and denote it as $\Delta_\sigma$
\item
Whenever $\sigma\subset\tau\in\Sigma$, identify $\Delta_{\sigma}$ with a face of $\Delta_{\tau}$ through face inclusion.
\end{enumerate}
\end{definition}
\begin{example}
Take 

\end{example}

\begin{example}
Take $V=\{1,2,3,4\}$ and 
\[
\Sigma=\{\text{all subsets of $V$ except $V$}\}
\]
\end{example}
\begin{example}
Take $V=\{1,\dots,n+1\}$ and
\[
\Sigma=\{\text{All subsets of $V$}\}
\]
Then $|(V,\Sigma)| = \Delta^n$.
\end{example}

\begin{definition}[Triangulation]
A \emph{triangulation} of a topological spcae $X$ is a simplicial complex $(V,\Sigma)=K$ with a choice of homeomorphism $|(V,\Sigma)|\to X$.
\end{definition}





















