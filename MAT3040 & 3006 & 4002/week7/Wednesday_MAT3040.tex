
\section{Wednesday for MAT3040}\index{Wednesday_lecture}
\paragraph{Recall}
Let $f(x)\in\mathbb{F}[x]$ with $f(T):V\to V$.
$m_T(x)$ is defined to be the polynomial of smallest degree such that
\[
m_T(T)=\bm0_{V\to V}\Longleftrightarrow
m_T(T)\bm v=0_{\bm V}, \ \forall\bm v\in V
\]
If $f(T) = 0_{V\to V}$, then we imply $m_T(x)\mid f(x)$.

If $g(T)_{\bm w}=0_V$, then we imply $m_{T,\bm w}(x)\mid g(x)$.

In particular, $m_T(T)\bm w=\bm0$, which implies $m_{T,\bm w}(x)\mid m_T(x)$.

\begin{example}
Let $A=\begin{pmatrix}
1&0\\0&2
\end{pmatrix}:\mathbb{R}^2\to\mathbb{R}^2$, if $m_A(x) = x-k$, then
\[
m_A(A) = A - kI=\begin{pmatrix}
1-k&0\\0&2-k
\end{pmatrix}\ne\begin{pmatrix}
0&0\\0&0
\end{pmatrix},\ \forall k,
\]
which is a contradiction.

Consider $(A-I)(A-2I)=\begin{pmatrix}
0&0\\0&1
\end{pmatrix}\begin{pmatrix}
-1&0\\0&0
\end{pmatrix}=\begin{pmatrix}
0&0\\0&0
\end{pmatrix}$, and therefore
\[
f(x)= (x-1)(x-2)\implies
m_A(x) = (x-1)(x-2)
\]
Let $\bm w = \bm e_2 = [0,1]\trans$, and 
\[
(A-2I)\bm w=\bm0\implies
m_{A,\bm w}(x) = x-2
\]
\end{example}
More generally, if $\lambda$ is an eigenvalue of $T:V\to V$ with eigenvector $\bm v$, then
\[
m_{T,\bm v}(x) =  (x-\lambda)\implies
(x-\lambda)\mid m_T(x)
\]


\subsection{Cayley-Hamiton Theorem}

The goal is to relate $m_T(x)$ with $\mathcal{X}_T(x)$.
Suppose that
\[
\mathcal{X}_T(x) = (x-\lambda_1)^{e_1}\cdots(x-\lambda_k)^{e_k}\in\mathbb{F}[x]
\]
Then we imply
\begin{itemize}
\item
$\lambda_i$ is an eigenvalue of $T$.
\item
$(x-\lambda_i)\mid m_T(x)$
\end{itemize}
Therefore, $(x-\lambda_1)\cdots(x-\lambda_k)\mid m_T(x)$.

Question: Does $m_T(x)$ possess other factors?
Does $(x-\lambda_i)^{f_i}\mid m_T(x)$ when $f_i>e_i$?

Answer: No.

\begin{theorem}[Cayley-Hamiton]
$m_T(x)\mid\mathcal{X}_T(x)$. In particular, $\mathcal{X}_T(T)=\bm0$.
\end{theorem}
Whtat if $\mathcal{X}_{T}(x)$ cannot be factorized into linear factors in $\mathbb{F}[x]$? e.g., $A=\begin{pmatrix}
0&-1\\1&0
\end{pmatrix}$ in $\mathbb{R}$.

Fact: for every $f(x)\in\mathbb{F}[x]$, we can extend $\mathbb{F}$ into $\overline{F}\supseteq\mathbb{F}$ such that
\[
f(x) = (x-\lambda_1)^{e_1}\cdots(x-\lambda_k)^{e_k}
\]
where $\lambda_i\in\overline{\mathbb{F}}$.

e.g., suppose $\mathbb{F}=\mathbb{R}$, $f(x) = x^2+1$, then
\[
f(x) = (x+i)(x-i).
\]
If $\mathbb{F}= \mathbb{R}$, one can always pick $\overline{\mathbb{F}}=\mathbb{C}$, since $\overline{\mathbb{F}}$ is algebraically closed.

To study $m_T(x)$ and $\mathcal{X}_T(x)$ in $\mathbb{F}[x]$, we will
\begin{itemize}
\item
study $m_T(x),\mathcal{X}_T(x)$ in $\overline{F}[x]$
\item
show $m_T(x)\mid\mathcal{X}_T(x)$ in $\overline{F}[x]$
\item
show that $m_T(x)\mid\mathcal{X}_T(x)$ in $\overline{F}[x]$.
\end{itemize}

Recall that for linear transformation $T:V\to V$, $U\le V$ is $T$-invariant if $T(U)\le U$
\begin{example}
\begin{enumerate}
\item
$U=\ker(T-\lambda I)$ ($\lambda$-eigenspace) is $T$-invariant.
\item
More generally, $U=\ker(g(T))$ is $T$-invariant for polynomial $g$:
if $\bm v\in\ker(g(T))$, i.e., $g(T)\bm v=\bm0$, it suffices to show $T(\bm v)\in\ker(g(T))$
\begin{align*}
g(T)T(\bm v) &= (a_mT^m+\cdots+a_0I)T(\bm v)\\
&=
(a_mT\circ T^m+\cdots+a_1T\circ T+a_0T\circ I)(\bm v)\\
&=
T(g(T)\bm v)=T(\bm 0)=\bm0
\end{align*}
\item
For $\bm v\in\ker(T-\lambda I)$, $U=\Span\{\bm v\}$ is $T$-invariant.
\end{enumerate}

\end{example}

\begin{proposition}
Suppose that $T:V\to V$ is a linear transformation and $W\le V$ is $T$-invariant, then
\[
T\mid_U:U\to U
\]
and
\[
\mathcal{X}_T(x) = \mathcal{X}_{T\mid u}(x)\circ\mathcal{X}_{\tilde{T}}(x)
\]
where $\tilde{T}:V/U\to V/U$ is given in Hw2, Q4 ($\tilde{T}(\bm v+U) = T(\bm v)+U$).  
\end{proposition}
\begin{proof}
Let $\mathcal{C} = \{\bm v_1,\dots,\bm v_k\}$ be a basis of $U$, and extend it into a basis of $V$, say
\[
\mathcal{B}=\{\bm v_1,\dots,\bm v_k,\bm v_{k+1},\dots,\bm v_n\}
\]
Therefore, $\overline{\mathcal{B}} = \{\bm v_{k+1}+U,\dots,\bm v_n+U\}$ is a basis of $V/U$.

By Hw2, Q5,
\[
(T)_{\mathcal{B},\mathcal{B}} = \begin{pmatrix}
(T|_{U})_{\mathcal{C},\mathcal{C}}&*\\
\bm0&(\tilde{T})_{\overline{\mathcal{B}},\overline{\mathcal{B}}}
\end{pmatrix}
\]
Therefore,
\[
\det((T)_{\mathcal{B},\mathcal{B}} - xI) = \det((T|_{U})_{\mathcal{C},\mathcal{C}} - xI)\cdot\det((\tilde{T})_{\overline{\mathcal{B}},\overline{\mathcal{B}}} - xI)
\]
\end{proof}

\begin{proposition}
Suppose that 
\[
\mathcal{X}_T(x) = (x-\lambda_1)\cdots(x-\lambda_n)
\]
where $\lambda_i$'s are not necessarily distinct.
Then there exists a basis of $V$, say $\mathcal{A}$, such that
\[
(T)_{\mathcal{A},\mathcal{A}}=\diag(\lambda_1,\dots,\lambda_n)
\]
\end{proposition}
\begin{proof}
Use induction on $n$.
By 2040, there exists $\bm y\in\mathbb{R}^n$ such that
\[
(T)_{\mathcal{B},\mathcal{B}}\bm y = \lambda_1\bm y
\]
Recall that $(\cdot)_{\mathcal{B}}:V\to\mathbb{R}^n$ is an isomorphism, there exists $\bm v\in V\setminus\{\bm0\}$ such that $(\bm v)_{\mathcal{B}} = \bm y$.

Therefore, we imply
\[
(T)_{\mathcal{B},\mathcal{B}}(\bm v)_{\mathcal{B}} = \lambda_1(\bm v)_{\mathcal{B}}
\]
which implies $(T\bm v)_{\mathcal{B}} = (\lambda_1\bm v)_{\mathcal{B}}$, i.e., $T\bm v = \lambda_1\bm v$.

Let $U=\Span\{\bm v\}$.
Since $U$ is $T$-invariant, the proof of proposition implies that there exists $\mathcal{B}$, a basis of $V$ such that
\[
(T)_{\mathcal{B},\mathcal{B}}=\begin{pmatrix}
T|_U&*\\\bm0&(\tilde{T})_{\mathcal{B},\mathcal{B}}
\end{pmatrix}_{[1+(n-1)]\times[1+(n-1)]}
\]
$T|_U:\Span\{\bm v\}\to\Span\{\bm v\}$ with $\bm v\mapsto T\bm v=\lambda_1\bm v$. Therefore
\[
(T)_{\mathcal{B},\mathcal{B}}=\begin{pmatrix}
\lambda_1&*\\\bm0&(\tilde{T})_{\overline{\mathcal{B}},\overline{\mathcal{B}}}
\end{pmatrix}
\]
By proposition, $\tilde{T}:V/U\to V/U$ has 
\[
\mathcal{X}_{\tilde{T}}(x) = (x-\lambda_2)\cdots(x-\lambda_n)
\]
By inducition, there exists basis $\overline{\mathcal{C}}$ of $V/U$, i.e.,
\[
\overline{\mathcal{C}} = \{\bm w_2+U,\dots,\bm w_n+U\}
\]
such that
\[
(\tilde{T})_{\overline{\mathcal{C}},\overline{\mathcal{C}}}
=
\diag(\lambda_2,\dots,\lambda_n)
\]

Consider
\[
\mathcal{A}:=
\{\bm v,\bm w_2,\dots,\bm w_n\},\ \text{where $\bm v\in U$}
\]
We claim that 
\begin{itemize}
\item
$\mathcal{A}$ is a basis of $V$
\item
\[
(T)_{\mathcal{A},\mathcal{A}}=\begin{pmatrix}
\lambda_1&*\\*&(\tilde{T})_{\overline{\mathcal{C}},\overline{\mathcal{C}}}
\end{pmatrix}=\diag(\lambda_1,\dots,\lambda_n)
\]
\end{itemize}
\end{proof}

\begin{proposition}
Suppose that $\mathcal{X}_T(x) = (x-\lambda_1)\cdots(x-\lambda_n)$, then $\mathcal{X}_T(T)=\bm0$.
\end{proposition}
If $\bm A = \diag(\lambda_1,\dots,\lambda_n)$, then
\[
(A-\lambda_1\bm I)\cdots(A-\lambda_n\bm I)\text{ is a zero matrix}
\]
















