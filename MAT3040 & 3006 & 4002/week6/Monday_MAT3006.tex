\section{Monday for MAT3006}\index{Monday_lecture}

\subsection{Compactness in Functional Space}
In functional space, previous study have shown that closedness and boundedness is not equivalent to compactness. 
We need the equi-continuity to rescue the situation:
\begin{definition}[Equi-continuity]
Let $X\subseteq\mathbb{R}^n$.
A subset $\mathcal{T}\subseteq\mathcal{C}(X)$ is called \emph{equi-continuous} if 
for any $\varepsilon>0$, there exists $\delta>0$ such that whenever $d(x,y)<\delta, x,y\in X$
\[
d_\infty(f(x),f(y))<\varepsilon,\forall f\in\mathcal{T}
\]
\end{definition}
\begin{example}
\begin{enumerate}
\item
Let $\mathcal{T}$ be a collection of Lipschitz continuous functions with the same Lipschitz constant $L$, i.e., $\forall f\in\mathcal{T}$, $|f(x)-f(y)|<L|x-y|$ for $\forall x,y\in X$.
It's clear that $\mathcal{T}$ is equi-continuous.
\item
Let $\mathcal{T}\subseteq\mathcal{C}[a,b]$ be such that
\[
\sup_{x\in[a,b]}|f'(x)|<M,\quad
\forall f\in\mathcal{T},
\]
then for any $\forall x,y\in[a,b]$, we imply $|f(y)-f(x)|=|f'(\xi)||y-x|$ for some $\xi\in[a,b]$. Therefore,
\[
|f(y)-f(x)|<M|y-x|,\quad\forall f\in\mathcal{T},
\]
i.e., $\mathcal{T}$ reduces to the space studied in (1) with Lipschitz constant $M$, thus is equi-continuous.
\end{enumerate}
\end{example}

\begin{theorem}
Let $K\subseteq\mathbb{R}^n$ be a compact set, and $\mathcal{T}\subseteq\mathcal{C}(K)$.
Then $\mathcal{T}$ is \emph{compact} if and only if $\mathcal{T}$ is \emph{closed}, \emph{uniformly bounded}, and \emph{equicontinuous}.
\end{theorem}
\begin{proof}
To be added.
\end{proof}

\begin{corollary}
Let $K\subseteq\mathbb{R}^n$ be compact, and $\{f_n\}$ be a sequence of uniformly bounded, equi-continuous functions on $K$.
Then $\{f_n\}$ has the \emph{Bolzano-Weierstrass property}, i.e.,
it has a convergent subsequence.
\end{corollary}
\begin{proof}
To be added.
\end{proof}
\subsection{An Application of Ascoli-Arzela Theorem}
The Ascoli-Arzela Theorem has a novel application on the ODE. Consider the IVP problem again:
\begin{equation}\label{Eq:6:2}
\left\{
\begin{aligned}
\frac{\diff y}{\diff x}&=f(x,y)\\
y(\alpha)&=\beta
\end{aligned}
\right.
\end{equation}
where $f$ is continuous on a rectangle $R$ containing $(\alpha,\beta)$.
Now we show the existence of Picard-Lindelof Theorem without the Lipschitz condition:
\begin{theorem}[Cauchy-Peano Theorem]
Consider the problem~(\ref{Eq:6:2}).
Then there exists a solution of this ODE on some rectangle $R'\subseteq R$.
\end{theorem}
\begin{proof}
To be added.
\end{proof}
%
%
%
%
%\subsection{Arzela-Ascoli Theorem}
%\begin{theorem}
%Let $\mathcal{F}$ be a closed set in $\mathcal{C}(K)$, where $K$ is closed and bounded in $\mathbb{R}^n$.
%Then $\mathcal{F}$ is compact iff it is bounded and equi-continuous.
%\end{theorem}
%\begin{definition}[Equi-continuous]
%For any $\varepsilon>0$, there exists $\delta>0$ such that whenever $d(x,y)<\delta, x,y\in X$
%\[
%d_\infty(f(x),f(y))<\varepsilon,\forall f\in\mathcal{F}
%\]
%\end{definition}
%\begin{example}
%Let $X$ be a bounded and convex set in $\mathbb{R}^n$.
%Show that a family of equi-continuous functions, say $\mathcal{F}$,
%is bounded in $C(X)$,
%if there exists a point $x_0\in X$ and a constant $M>0$ such that $|f(x_0)|\le M$ for all $f\in\mathcal{F}$.
%\begin{proof}
%By equi-continuity for $\varepsilon:=1$, there exists $\delta_0$ such that $|f(x)-f(y)|<1$ whenever $|x-y|<\delta_0$.
%
%Consider the open ball $B_{R}(x_0)$ containing $E$.
%Then $|x-x_0|\le R$ for all $x\in E$, we can find $x_0,x_1,\dots,x_n=x$ where $n\delta_0\le R\le (n+1)\delta_0$ such taht
%\[
%|x_{n+1} - x_0|\le \delta_0\implies
%|f(x)-f(x_0)|\le \sum_{j=0}^{n-1}|f(x_{j+1}) - f(x_j)|\le n\le R/\delta_0
%\]
%Therefore,
%\[
%|f(x)|\le |f(x_0)|+n+1\le M+R/\delta_0,\forall x\in X,\forall f\in\mathcal{F}
%\]
%\end{proof}
%\end{example}
%
%\begin{example}
%Let $\phi:[0,1]\times\mathbb{R}\to\mathbb{R}$ be continuous. Suppose that there is $M>1$ such that 
%$|\phi(r,s)|\le M$ for $r\in[0,1]$ and $s\in\mathbb{R}$.
%Let $\mathbb{Z} = C[0,1]$ and $\psi$ be the following:
%\[
%\psi(f(x)) = c + \int_0^x\phi(t,f(t))\diff t,\forall x\in[0,1]
%\]
%where $c$ is a constant.
%Then define
%\[
%E = \left\{f\in\mathbb{Z}\middle|
%|f(x)-c|\le M,\quad
%|f(x)-f(y)|\le M|x-y|,\ \forall x,y\in[0,1]
%\right\}
%\]
%Show that $\psi(E)\subseteq E$ and $E$ is compact.
%\end{example}
%\begin{proof}
%Let $f\in E$, then
%\[
%|\psi(f(x)) - c|=|\int_0^x\phi(t,f(t))\diff t|\le 1\cdot M = M
%\]
%Moreover,
%\[
%|\psi(f(x)) - \psi(f(y))|=|\int_y^x\phi(t,f(t))\diff t|\le M|x-y|=M|x-y|
%\]
%Therefore, $\psi(E)\subseteq E$.
%
%We need to show that $E$ is bounded and equi-continuous:
%
%Since $|f(x)-c|\le M,\forall f\in E$, we imply $|f(x)|\le M+c$.
%
%Given $\varepsilon>0$, take $\delta = \frac{\varepsilon}{M}$, then 
%\[
%|x-y|<\delta\implies|f(x) - f(y)|\le M|x-y|=\varepsilon,\forall f\in E
%\]
%
%The closedness of $E$:
%for any $\{f_n\}\in E$ and let $f_n\to f$, and show that $f\in E$, i.e., $E'\subseteq E$, i.e., $E$ is closed.
%
%Applying Arzela-Ascoli Theorem, $E$ is compact.
%\end{proof}
%
%\subsection{Baire Category Theorem}
%Let $\{E_j\}_{j=1}^\infty$ be a sequence of nowhere dense subsets of $(X,d)$, where $(X,d)$ is complete.
%Then
%\begin{itemize}
%\item
%$\cup_{j=1}^\infty \overline{E}_j$ has empty interior.
%\end{itemize}
%
%\begin{definition}[Nowhere Dense]
%A set $E$ is said to be \emph{nowhere dense} if $X\setminus E$ is dense in $X$.
%\end{definition}
%Cator Set: nowhere dense.
%
%Boundary point is nowhere dense.
%
%\begin{proposition}
%The intersection of dense sequence of open sets is dense.
%\end{proposition}
%\begin{proof}
%Let $\{G_j\}$ be open and $\overline{G}_j = X$.
%Let $F_j = X\setminus G_j$.
%Then $F_j$ is nowhere dense, which follows that
%\[
%\bigcap_j G_j = \left(\bigcup_{j}F_j\right)^c,
%\]
%which is comeagre, and therefore dense.
%\end{proof}
%\begin{proposition}
%The set $\mathbb{Q}$ of rationals is an $F_\delta$ but not $G_\delta$, where $F_\delta = \cup_{n=1}^NF_n$, where $F_n$ is closed, and $G_\delta = \cap_{n=1}^NG_n$, where $G_n$ is open.
%\end{proposition}
%\begin{proof}
%If $\mathbb{Q}$ were a $G_\delta$, then the complement of $\mathbb{Q}$ is a countable union of closed sests 
%\[
%\mathbb{Q}^c=\cup_{n=1}^NG_n^c
%\]
%Since $\mathbb{Q}^c$ has no interiors, $\text{int}(\mathbb{Q}^c)$.
%However, $\mathbb{R} = \mathbb{Q}\cup(\cup_{n=1}^NG_n^c)$, a countable union of nowhere dense sets, which contradicts to the BCT.
%\end{proof}
%
%
%
%
%
%
%
%
%
%
%
%
%
%
%
