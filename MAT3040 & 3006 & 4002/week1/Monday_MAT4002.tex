\section{Monday for MAT4002}\index{Monday_lecture}

\subsection{Introduction to Topology}
We will study global properties of a geometric object, i.e., the distrance between 2 points in an object is totally ignored.

For example, consider the topology illustrated below. 

Therefore, the difference relies on the connectedness. Why the last graph is intrinsically different?  

In this course, we are going to introduce the mathematics to describe the phenomenon above.

First you learn about distances, and then you forget about distances.
\subsection{Metric Spaces}
\begin{definition}[Metric Space]
Metric space is a set $X$ where one can measure distance between any two objects in X.

Specifically speaking, a metric space $X$ is a non-empty set endowed with a function (distance function) $d:X\times X\to\mathbb{R}$ such that
\begin{enumerate}
\item
$d(\bm x,\bm y)\ge0$ for $\forall\bm x,\bm y\in X$ with equality iff $\bm x=\bm y$
\item
$d(\bm x,\bm y)=d(\bm y,\bm x)$
\item
$d(\bm x,\bm z)\le d(\bm x,\bm y)+d(\bm y,\bm z)$ (triangular inequality)
\end{enumerate}
\end{definition}

\begin{example}
\begin{enumerate}
\item
Let $X=\mathbb{R}^n$, with 
\[
d_2(\bm x,\bm y)=\sqrt{\sum_{i=1}^n(x_i-y_i)^2}
\]
\[
d_\infty(\bm x,\bm y)=\max_{i=1,\dots,n}|x_i-y_i|
\]
\item
Let $X$ be any set, and define the discrete metric 
\[
d(\bm x,\bm y)=\left\{
\begin{aligned}
0,&\quad\mbox{if }x=y\\
1,&\quad\mbox{if }x\ne y
\end{aligned}
\right.
\]
\end{enumerate}
Homework: Show that (1) and (2) defines a metric.
\end{example}
\begin{definition}[Open Ball]
An \emph{open ball} of radius $r$ centered at $\bm x\in X$ is defined as
\[
B_r(\bm x)=\{\bm y\in X\mid d(\bm x,\bm y)<r\}
\]
\end{definition}
\begin{example}
\begin{enumerate}
\item
Under the metric $(X=\mathbb{R}^2,d_2)$, the set $B_1(0,0)$ defines an open ball

\item
Under the metric $(X=\mathbb{R}^2,d_\infty)$, the set $B_1(0,0)$ defines an open ball

\item
Under the metric $(X=\mathbb{R}^2,\text{discrete metric})$, the set $B_1(0,0)$ is one single point, also defines an open ball.

\end{enumerate}
\end{example}

\begin{definition}[Open Set]
Let $X$ be a metric space, $U\subseteq X$ is an open set in $X$ if $\forall u\in U$, there exists $\epsilon_u>0$ such that $B_{\epsilon_u}(u)\subseteq U$.
\end{definition}
We denote $\mathcal{T}$ to be the collection of all open sets in $(X,d)$, which is so called the \emph{topology} induced from $(X,d)$.
\begin{example}
\begin{enumerate}
\item
All open balls $B_r(\bm x)$ are open in $(X,d)$, e.g., for $X=\mathbb{R}$ and $d_2$, we have $B_r(x)=(x-r,x+r)$. Take $\bm y\in B_r(\bm x)$ such that $d(\bm x,\bm y)=q<r$ and consider $B_{(r-q)/2}(\bm y)$, then for all $z\in B_{(r-q)/2}(\bm y)$, we have
\[
d(\bm x,\bm z)\le d(\bm x,\bm y)+d(\bm y,\bm z)<q+\frac{r-q}{2}<r,
\]
which implies $\bm z\in B_r(x)$.
\end{enumerate}
\end{example}
\begin{proposition}
Let $(X,\bm d)$ be metric space, and $\mathcal{T}$ is the topology induced from $(X,d)$, then
\begin{enumerate}
\item
let the set $\{G_\alpha\mid\alpha\in\mathcal{A}\}$ be a collection of (uncountable) open sets, i.e., $G_\alpha\in\mathcal{T}$, then $\bigcup_{\alpha\in\mathcal{A}}G_\alpha\in\mathcal{T}$.
\item
Let $G_1,\dots,G_n\in\mathcal{T}$, then $\bigcap_{i=1}^nG_i\in\mathcal{T}$. The finite intersection of open sets is open.
\end{enumerate}
\end{proposition}
\begin{proof}
\begin{enumerate}
\item
Take $x\in\bigcup_{\alpha\in\mathcal{A}}G_\alpha$, 
then $x\in G_\beta$ for some $\beta\in\mathcal{A}$. 
Since $G_\beta$ is open, there exists $\epsilon_x>0$ s.t.
\[
B_{\epsilon_x}(x)\subseteq G_\beta\subseteq\bigcup_{\alpha\in\mathcal{A}}G_\alpha
\]
\item
Take $x\in\bigcap_{i=1}^nG_i$, i.e., $x\in G_i$ for $i=1,\dots,n$, i.e., there exists $\epsilon_i>0$ such that $B_{\epsilon_i}(x)\subseteq G_i$ for $i=1,\dots,n$. Take $\epsilon=\min\{\epsilon_1,\dots,\epsilon_n\}$, then
\[
B_\epsilon(x)\subseteq B_{\epsilon_i}(x)\subseteq G_i,\forall i
\]
which implies $B_\epsilon(x)\subseteq\bigcap_{i=1}^nG_i$



\end{enumerate}
\end{proof}
Exercise: let $\mathcal{T}_2,\mathcal{T}_\infty$ be topologies induced from the metrices $d_2,d_\infty$ in $\mathbb{R}^2$. Then $J_2=J_\infty$, i.e., every open set in $(\mathbb{R}^2,d_2)$ is open in $(\mathbb{R}^2,d_\infty)$, and every open set in $(\mathbb{R}^2,d_\infty)$ is open in $(\mathbb{R}_2,d_2)$.

Let $\mathcal{T}$ bbe topology induced from the discrete metric $(X,d_{\text{discrete}}).$ What is $\mathcal{T}$?












