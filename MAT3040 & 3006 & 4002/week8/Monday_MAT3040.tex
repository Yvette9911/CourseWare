\chapter{Week8}
\section{Monday for MAT3040}\index{Monday_lecture}
\paragraph{Reviewing}
\begin{itemize}
\item
If $\mathcal{X}_T(x) =(x-\lambda_1)\cdots(x-\lambda_n)$, then 
\[
(T)_{\mathcal{A},\mathcal{A}}=\begin{pmatrix}
\lambda_1&\times&\times&\times\\
0&\lambda_2&\cdots&\times\\
0&\cdots&\ddots&\times\\
0&0&\cdots&\lambda_n
\end{pmatrix}
\]
for some basis $\mathcal{A}$.
In other words, $T$ is \emph{triangularizable} with the diagonal entries $\lambda_1,\dots,\lambda_n$.
\end{itemize}
\begin{remark}
I hope you appreciate this result. Consider the example below:
In linear algebra we have studied that the matrix $\bm A=\begin{pmatrix}
1&1\\0&1
\end{pmatrix}$ is not diagonalizable, and the characteristic polynomial is given by
\[
\mathcal{X}_{A}(x)=(x-1)^2.
\]
However, the theorem above claims that $\bm A$ is \textit{triangularizable}, with diagonal entries 1 and 1.
The diagonalization of $\bm A$ only uses the eigenvector of $\bm A$, but the $1$-eigenspace has only 1 dimension.
Fortunately, the triangularization gives a rescue such that we can make use of the generalized eigenvector $(0,1)\trans$ (but not an eigenvector) of $\bm A$ by considering the mapping below:
\[
\begin{array}{ll}
&U=\Span\left\{\begin{pmatrix}
1\\0
\end{pmatrix}\right\}\\
\bar{A}:&V/U\to V/U
\end{array}
\]
Here $(0,1)\trans+U$ is an eigenvector of $\bar{ A}$, with eigenvalue $1$.
\end{remark}

\begin{theorem}
The linear operator $T$ is triangularizable with diagonal entries $(\lambda_1,\dots,\lambda_n)$ if and only if 
\[
\mathcal{X}_{T}=(x-\lambda_1)\cdots(x-\lambda_n)
\]
\end{theorem}
\begin{proof}
It suffices to show only the sufficiency.
Suppose that there exists basis $\mathcal{A}$ such that 
\[
(T)_{\mathcal{A},\mathcal{A}}=\begin{pmatrix}
\lambda_1&\times&\times&\times\\
0&\lambda_2&\cdots&\times\\
0&\cdots&\ddots&\times\\
0&0&\cdots&\lambda_n
\end{pmatrix}
\]
Then we compute the characteristic polynomial directly:
\begin{align*}
\mathcal{X}_T(x)&=\det[(xI-T)_{\mathcal{A},\mathcal{A}}]\\
&=\det\begin{pmatrix}
x-\lambda_1&\times&\times&\times\\
0&x-\lambda_2&\cdots&\times\\
0&\cdots&\ddots&\times\\
0&0&\cdots&x-\lambda_n
\end{pmatrix}\\
&=(x-\lambda_1)\cdots(x-\lambda_n)
\end{align*}
\end{proof}
\subsection{Cayley-Hamiton Theorem}
\begin{proposition}[A Useful Lemma]\label{pro:8:1}
Suppose that $\mathcal{X}_T(x)=(x-\lambda_1)\cdots(x-\lambda_n)$, then $\mathcal{X}_T(T)=0$.
\end{proposition}
\begin{proof}
Since $\mathcal{X}_T(x)=(x-\lambda_1)\cdots(x-\lambda_n)$, we imply $T$ is triangularizable under some basis $\mathcal{A}$.
Note that
\begin{itemize}
\item
$T\mapsto(T)_{\mathcal{A},\mathcal{A}}$ is an isomorphism between $\text{Hom}(V,V)$ and $M_{n\times n}(\mathbb{F})$,
\item
$(\underbrace{T\circ T\circ\cdots\circ T}_{\text{$m$ times}})_{\mathcal{A},\mathcal{A}} = [(T)_{\mathcal{A},\mathcal{A}}]^m$, for any $m$,
\end{itemize}
It suffices to show $\mathcal{X}_T((T)_{\mathcal{A},\mathcal{A}})$ is the zero matrix (why?):
\[
\mathcal{X}_T((T)_{\mathcal{A},\mathcal{A}})=
((T)_{\mathcal{A},\mathcal{A}}-\lambda_1\bm I)\cdots
((T)_{\mathcal{A},\mathcal{A}}-\lambda_n\bm I).
\]
Observe the matrix multiplication
\[
((T)_{\mathcal{A},\mathcal{A}}-\lambda_i\bm I)\begin{pmatrix}
x_1\\\vdots\\x_i\\0\\\vdots\\0
\end{pmatrix}
=
\begin{pmatrix}
\lambda_1-\lambda_i&\times&\times&\times\\
0&\lambda_2-\lambda_i&\cdots&\times\\
0&\cdots&\ddots&\times\\
0&0&\cdots&\lambda_n-\lambda_i
\end{pmatrix}
\begin{pmatrix}
x_1\\\vdots\\x_i\\0\\\vdots\\0
\end{pmatrix}
\in\Span\{\bm e_1,\dots,\bm e_{i-1}\}
\]
Therefore, for any $\bm v\in V$,
\[
((T)_{\mathcal{A},\mathcal{A}}-\lambda_n\bm I)\bm v\in\Span\{\bm e_1,\dots,\bm e_{n-1}\}.
\]
Applying the same trick, we conclude that 
\[
((T)_{\mathcal{A},\mathcal{A}}-\lambda_1\bm I)\cdots
((T)_{\mathcal{A},\mathcal{A}}-\lambda_n\bm I)\bm v
=\bm0,\quad\forall \bm v\in V,
\]
i.e., $\mathcal{X}_T((T)_{\mathcal{A},\mathcal{A}})=((T)_{\mathcal{A},\mathcal{A}}-\lambda_1\bm I)\cdots
((T)_{\mathcal{A},\mathcal{A}}-\lambda_n\bm I)$ is a zero matrix.
\end{proof}

Now we are ready to give a proof for the Cayley-Hamiton Theorem:
\begin{proof}
Suppose that $\mathcal{X}_T(x)=x^n+a_{n-1}x^{n-1}+\cdots+a_0\in\mathbb{F}[x]$. By considering algebrically closed field $\overline{\mathbb{F}}\supseteq\mathbb{F}$, we imply
\begin{subequations}
\begin{align}
\mathcal{X}_T(x)&=x^n+a_{n-1}x^{n-1}+\cdots+a_0\label{Eq:8:1:a}\\
&=(x-\lambda_1)\cdots(x-\lambda_n),
\quad
\lambda_i\in\overline{\mathbb{F}}\label{Eq:8:1:b}
\end{align}
\end{subequations}
By applying proposition~(\ref{pro:8:1}), we imply $\mathcal{X}_T(T)=0$, where the coefficients in the formula $\mathcal{X}_T(T)=0$ w.r.t. $T$ are in $\overline{\mathbb{F}}$.

Then we argue that these coefficients are essentially in $\mathbb{F}$. Expand the whole map of $\mathcal{X}_T(T)$:
\begin{subequations}
\begin{align}
\mathcal{X}_T(T)
&=
(T-\lambda_1I)\cdots(T-\lambda_nI)
\\&=
T^n-(\lambda_1+\cdots+\lambda_n)T^{n-1}+\cdots+(-1)^n\lambda_1\cdots\lambda_nI\\
&=
T^n+a_{n-1}T^{n-1}+\cdots+a_0I\label{Eq:8:2:c}
\end{align}
\end{subequations}
where the derivation of (\ref{Eq:8:2:c}) is because that the polynomial coefficients for (\ref{Eq:8:1:a}) and (\ref{Eq:8:1:b}) are all identical.

Therefore, we conclude that $\mathcal{X}_T(T)=0$, under the field $\mathbb{F}$.
\end{proof}

\begin{corollary}\label{cor:8:1}
$m_T(x)\mid\mathcal{X}_T(x)$.
More precisely,
if
\[
m_T(x) = [p_1(x)]^{e_1}\cdots[p_k(x)]^{e_k},\ e_i>0,\forall i
\]
where $p_i$'s are distinct, monic, and irreducible polynomials.
Then
\[
m_T(x)=[p_1(x)]^{f_1}\cdots[p_k(x)]^{f_k},\ \text{for some }0<f_i\le e_i,\forall i
\]
\end{corollary}
\begin{proof}
The statement $m_T(x)\mid\mathcal{X}_T(x)$ is from Cayley-Hamiton Theorem.
Therefore, $0\le f_i\le e_i,\forall i$.
Suppose on the contrary that $f_i=0$ for some $i$.
w.l.o.g., $i=1$.

It's clear that $\text{gcd}(p_1,p_j)=1$ for $\forall j\ne 1$, which implies
\[
a(x)p_1(x)+b(x)p_j(x)=1,\text{ for some }a(x),b(x)\in\mathbb{F}[x].
\]

Considering the field extension $\overline{\mathbb{F}}\supseteq\mathbb{F}$, we have
$p_1(x)=(x-\mu_1)\cdots(x-\mu_\ell)$.
For any root $\mu_m$ of $p_1$, $m=1,\dots,\ell$, we have
\[
a(\mu_m)p_1(\mu_m)+b(\mu_m)p_j(\mu_m)=1
\implies
b(\mu_m)p_j(\mu_m)=1
\implies
p_j(\mu_m)\ne0,
\]
i.e., $\mu_m$ is not a root of $p_j$, $\forall j\ne1$.

Therefore, $\mu_m$ is a root of $\mathcal{X}_T(x)$, but not a root of $m_T(x)$.
Then $\mu_m$ is an eigenvalue of $T$, e.g., $T\bm v=\mu_m\bm v$ for some $\bm v\ne\bm0$.
Recall that $m_{T,\bm v}=x-\mu_m$,
we imply $m_{T,\bm v}=x-\mu_m\mid m_T(x)$, which is a contradiction.
\end{proof}





\begin{example}
We can use Corollary~(\ref{cor:8:1}), a stronger version of Cayley-Hamiltion Theorem to determine the minimal polynomials:
\begin{enumerate}
\item
For matrix $\bm A=\begin{pmatrix}
0&-1\\1&1
\end{pmatrix}$, we imply $\mathcal{X}_{A}(x)=(x^2+x+1)^1$.
Since $x^2+x+1$ is irreducible in $\mathbb{R}$, we have $m_A(x)=x^2+x+1$.
\item
For matrix 
\[
\bm A=\begin{pmatrix}
1&1&0&0\\0&1&0&0\\0&0&2&0\\0&0&0&2
\end{pmatrix},
\]
we imply $\mathcal{X}_A(x)=(x-1)^2(x-2)^2$.

By Corollary~(\ref{cor:8:1}), we imply both $(x-1)$ and $(x-2)$ should be roots of $m_T(x)$, i.e., $m_A(x)$ may have the four options:
\begin{align*}
(x-1)^2(x-2)^2,\text{ or }\\
(x-1)(x-2)^2,\text{ or }\\
(x-1)^2(x-2),\text{ or }\\
(x-1)(x-2).
\end{align*}
\end{enumerate}
\end{example}
\subsection{Primary Decomposition Theorem}

We know that not every linear operator is diagonalizable, but diagonalization has some nice properties:
\begin{definition}[diagonalizable]
The linear operator $T:V\to V$ is diagonalizable over $\mathbb{F}$ if and only if there exists a basis $\mathcal{A}$ of $V$ such that 
\[
(T)_{\mathcal{A},\mathcal{A}}=\diag(\lambda_1,\dots,\lambda_n),
\]
where $\lambda_i$'s are not necessarily distinct.
\end{definition}

\begin{proposition}\label{pro:8:2}
If the linear operator $T:V\to V$ is diagonalizable, then
\[
m_T(x) = (x-\mu_1)\cdots(x-\mu_k),
\]
where $\mu_i$'s are \emph{distinct}.
\end{proposition}
\begin{proof}
Suppose $T$ is diagonalizable, then there exists a basis $\mathcal{A}$ of $V$ such that
\[
(T)_{\mathcal{A},\mathcal{A}}=\diag(\mu_1,\dots,\mu_1,\mu_2,\dots,\mu_2,\dots,\mu_k,\dots,\mu_k)
\]
It's clear that $((T)_{\mathcal{A},\mathcal{A}}-\mu_1\bm I)\cdots((T)_{\mathcal{A},\mathcal{A}}-\mu_k\bm I)=\bm0$, i.e., $m_T(x)\mid (x-\mu_1)\cdots(x-\mu_k)$.

Then we show the minimality of $(x-\mu_1)\cdots(x-\mu_k)$.
In particular, if $(x-\mu_i)$ is omitted for any $1\le i\le k$, then it's easy to show
\begin{align*}
(T_{\mathcal{A},\mathcal{A}}-\mu_1\bm I)&\cdots
(T_{\mathcal{A},\mathcal{A}}-\mu_{i-1}\bm I)
(T_{\mathcal{A},\mathcal{A}}-\mu_{i+1}\bm I)
\cdots
(T_{\mathcal{A},\mathcal{A}}-\mu_{k}\bm I)
\ne\bm0,
\end{align*}
since all $\mu_i$'s are distinct.
Therefore, $m_T(x)$ will not divide $(x-\mu_1)\cdots(x-\mu_{i-1})(x-\mu_{i+1})\cdots(x-\mu_k)$ for any $i$, i.e.,
\[
m_T(x)=(x-\mu_1)\cdots(x-\mu_k)
\]

\end{proof}
\begin{remark}
The converse of proposition~(\ref{pro:8:2}) is also true, which is a special case for the Primary Decomposition Theorem.
\end{remark}

\begin{theorem}[Primary Decomposition Theorem]
Let $T:V\to V$ be a linear operator with
\[
m_T(x) = [p_1(x)]^{e_1}\cdots[p_k(x)]^{e_k},
\]
where $p_i$'s are distinct, monic, and irreducible polynomials.
Let $V_i=\ker([p_i(x)]^{e_i})\le V, i=1,\dots,k$,
then
\begin{enumerate}
\item
Each $V_i$ is $T$-invariant ($T(V_i)\le V_i$)
\item
$V=V_1\oplus V_2\oplus\cdots\oplus V_k$
\item
Consider $T\mid_{V_i}:V_i\to V_i$, then
\[
m_{T\mid V_i}(x) = [p_i(x)]^{e_i}
\]
\end{enumerate}
\end{theorem}






















