\section{Monday for MAT3006}\index{Monday_lecture}
Our first quiz will be held on next Wednesday.
\paragraph{Reviewing}
\begin{itemize}
\item
Picard Lindelof Theorem on ODEs. e.g., consider
\[
\left\{
\begin{aligned}
\frac{\diff y}{\diff x} &= \frac{x}{1-y},\ (x,y)\in G:=(-\infty,\infty)\times(-\infty,1)\\
y(0)&=2
\end{aligned}
\right.
\]
Since $f\in\mathcal{C}^1(G)$ satisfies the Lipschitz condition on some closed ball of the point $(x_0,y_0)$, the setting for Picard Lindelof Theorem is satisfied, and the solution is uniquely given by:
\[
y = 1+\sqrt{1-x^2},\ -1<x<1.
\]
Therefore, the maximal interval of existence is given by $(-1,1)$. 
In order to restrict $G$ to be open to construct a closed ball of $(x_0,y_0)$, 
we need the initial condition $y(0)\ne1$.
\end{itemize}

\subsection{Generalization into System of ODEs}
\paragraph{Formal Setting of System of ODEs}
Consider the system of ODEs
\[
\left\{
\begin{aligned}
y_1'(x)&=f_1(x,y_1(x),\dots,y_n(x))\\
\vdots\\
y_n'(x)&=f_n(x,y_1(x),\dots,y_n(x))
\end{aligned}
\right.\quad
\left\{
\begin{aligned}
y_1(\alpha)&=\beta_1\\
\vdots\\
y_n(x)&=\beta_n
\end{aligned}
\right.
\]
It's convenient to denote
\[
\begin{array}{lll}
\bm y(x)=\begin{pmatrix}
y_1(x)\\\vdots\\y_n(x)
\end{pmatrix}\in\mathcal{C}(\mathbb{R},\mathbb{R}^n),
&
\bm f(x,\bm y)=\begin{pmatrix}
f_1(x,\bm y)\\
\vdots\\
f_n(x,\bm y)
\end{pmatrix},
&
\bm\beta:=\begin{pmatrix}
\beta_1\\\vdots\\\beta_n
\end{pmatrix}
\end{array}
\]
Here the notation $\mathcal{C}(X,Y)$ denotes the set of bounded continuous mapping from $X$ to $Y$. Therefore we can express the system of ODE as a compact form:
\[
\left\{
\begin{aligned}
\bm y'&=\bm f(x,\bm y)\\
\bm y(\alpha)&=\bm\beta
\end{aligned}
\right.
\]
\paragraph{Generalization of Picard Lindelof Theorem}
Consider the rectangle
\[
S = \{(\bm x,\bm y)\in\mathbb{R}\times\mathbb{R}^n\mid \alpha-a\le x\le\alpha+a,\beta_i-b_i\le y_i\le \beta_i+b_i, i=1,\dots,n\}
\]
Suppose that 
\begin{itemize}
\item
$\|\bm f(x,\bm y)\|\le M, \forall (x,\bm y)\in S$
\item
$\|\bm f(x,\bm y) - \bm f(x,\bm y')\|\le L\cdot\|\bm y-\bm y'\|$ for $\forall x\in[\alpha-a,\alpha+a]$
\end{itemize}
Then consider the complete metric space
\[
X=\{\bm y\in \mathcal{C}([\alpha-a,\alpha+a],\mathbb{R}^n)\mid \beta_i - b_i\le y_i(x)\le\beta_i + b_i\}
\]
(Verification of completeness: if $Y$ is complete, then $\mathcal{C}(X,Y)$ is complete.) 
Under this setting, the similar argument gives the Picard-Lindelof for system of ODEs.


\paragraph{Higher Order ODEs}
Note that there is a standard way to transform the ODE with higher order derivatives into a system of first order ODEs. Suppose we want to solve the initival value problem
\[
\left\{
\begin{aligned}
y^{(m)}&=f(x,y,y',\dots,y^{(m-1)})\\
y(\alpha)&=\beta_0,\ y'(\alpha)=\beta_1,\dots,y^{(m-1)}(\alpha)=\beta_{m-1}
\end{aligned}
\right.
\]
We can define the variables
\[
\begin{pmatrix}
y_{m-1}(x)\\\vdots\\y_1(x)\\y_0(x)
\end{pmatrix}=\begin{pmatrix}
y^{(m-1)}(x)\\\vdots\\y'(x)\\y(x)
\end{pmatrix}
\]
which gives an equivalent system of ODE:
\[
\left\{
\begin{aligned}
y_{m-1}'&=f(x,y_0,\dots,y_{m-1})\\
y_{m-2}'&=y_{m-1}\\
\vdots\\
y_0' &= y_1
\end{aligned}\right.,\ \text{with }
\left\{
\begin{aligned}
y_{m-1}(\alpha)&=\beta_{m-1}\\
y_{m-2}(\alpha)&=\beta_{m-2}\\
\vdots\\
y_0(\alpha) &= \beta_0
\end{aligned}
\right.
\]

\subsection{Stone-Weierstrass Theorem}
Under the compact metric space $X$, the goal is to approximate \emph{any} functions in $\mathcal{C}(X)$. For example, under $X=[a,b]$, one can apply Taylor polynomials $p_n(x)$ to approximate differentiable functions:
\[
\|f(x) - p_n(x)\|_\infty<\varepsilon,\ \text{for large $n$}.
\]
To formally describe the phenomenon for the approximation of \emph{any} functions in $\mathcal{C}(X)$, we need to describe the set of approximate functions, which usually obtains a common property:

\begin{definition}[Algebra]
A subset $\mathcal{A}\subseteq\mathcal{C}(X)$ (where $X$ is a general space) is an \emph{algebra} if the following holds:
\begin{itemize}
\item
If $f_1,f_2\in\mathcal{A}$, then $\alpha f_1 + \beta f_2\in\mathcal{A}$
\item
If $f_1,f_2\in\mathcal{A}$, then $f_1\cdot f_2\in\mathcal{A}$
\end{itemize}
\end{definition}
\begin{example}
\begin{enumerate}
\item
$\mathcal{A} = \mathcal{C}(X)$ is an algebra.
\item
$X=[a,b]$, then $\mathcal{A}=P[a,b] = \{\text{All polynomials $p(x)$}\}$ is an algebra.
\end{enumerate}
\end{example}

The goal is to approximate any $f\in\mathcal{C}(X)$ by $p\in\mathcal{A}$, i.e., for $\forall f\in\mathcal{C}(X)$, there exists $p\in\mathcal{A}$ such that
\[
\|f - p\|_\infty<\varepsilon,\ \forall \varepsilon>0.
\]
In other words, we aim to find an algebra $\mathcal{A}\subseteq\mathcal{C}(X)$ such that 
$\overline{\mathcal{A}}=\mathcal{C}(X)$, i.e., $\mathcal{A}$ is dense in $\mathcal{C}(X)$.

\begin{theorem}[Weierstrass Approximation]
$\mathcal{P}[a,b]$ is dense in $\mathcal{C}[a,b]$.
\end{theorem}

\begin{proof}
Consider any function $f\in\mathcal{C}[0,1]$.
By rescaling, assume that $f\in\mathcal{C}[0,1]$.
By subtracting a linear function $\ell(x)$, 
assume that $f(0) = f(1) = 0$. 
Then we extend $f(x)$ into $\mathbb{R}$ by setting $f(x)=0,\forall x\notin [0,1]$.

\begin{itemize}
\item
\textbf{Step 1: Construction of approximate function}: Consider the \emph{Landaus kernel function}
\[
Q_n(x)
=
\left\{
\begin{aligned}
c_n\cdot(1-x^2)^n,&\quad -1\le x\le 1\\
0,&\quad |x|>1
\end{aligned}
\right.
\]
where $c_n$ is chosen such that $\int Q_n(x)\diff x = 1$. Then construct the approximation of $f$ by defining
\[
p_n(x):=Q_n * f = \int_{-1}^1f(x+t)Q_n(t)\diff t
\]
The intuition behind this construction is that as $n\to\infty$, $Q_n(x)\to\delta(x)$, where
\[
\delta(x)=\left\{
\begin{aligned}
\infty,&\quad x=0\\
0,&\quad x\ne0
\end{aligned}
\right.
\implies
\int_{-1}^1f(x+t)\delta(t)\diff t = f(x).
\]
\textbf{Step 2: Argue that $p_n(x)\in \mathcal{P}[a,b]$: }Now it's clear that
\begin{subequations}
\begin{align}
p_n(x)&=\int_{-1}^1f(x+t)Q_n(t)\diff t\label{Eq:4:2:a}\\
&=\int^{1-x}_{-x} f(x+t)Q_n(t)\diff t \label{Eq:4:2:b}\\
&=\int_{-1}^1f(u)\cdot Q_n(u-x)\diff u\label{Eq:4:2:c}\\
&=\int_{-1}^1f(u)\cdot (1 - (u - x)^2)^n\diff u,\label{Eq:4:2:d}
\end{align}
\end{subequations}
where (\ref{Eq:4:2:b}) is because that $f=0$, for $x\notin[0,1]$ and $Q_n=0$ for $|x|>1$;
(\ref{Eq:4:2:c}) is by change of variables; and (\ref{Eq:4:2:d}) is by substitution of $Q_n(x)$. Therefore, $p_n$ is still a polynomial of $x$.
\item
\textbf{Step 3: Construct an upper bound on $c_n$: }It's clear that
\begin{align*}
c_n^{-1}&=\int_{-1}^1(1-x^2)^n\diff x\\
&=2\int_{0}^1(1-x^2)^n\diff x\\
&\ge2\int_0^1(1-nx^2)\diff x\\
&\ge2\int_{0}^{1/\sqrt{n}}(1-nx^2)\diff x\\
&=2(\frac{1}{\sqrt{n}} - \frac{1}{3\sqrt{n}}) >\frac{1}{\sqrt{n}}
\end{align*}
and therefore $c_n<\sqrt{n}$. As a result, for any fixed $\delta\in(0,1)$, we imply
\[
Q_n(x)\le\sqrt{n}(1-\delta^2)^n,\qquad \forall x\in[\delta,1],
\]
which implies $Q_n(x)\to0$ uniformly on $[\delta,1]$.
\item
\textbf{Step 4: Show that $\|p_n-f\|_\infty\to0$.}
Since $f$ is continuous, for given $\varepsilon>0$, there exists $\delta\in(0,1)$ such that
\[
|f(x)-f(y)|<\varepsilon,\quad\text{when }|x-y|<\delta,x,y\in[0,1].
\]
Therefore, for any $x\in[0,1]$, and for sufficiently large $n$,
\begin{subequations}
\begin{align}
|p_n(x)-f(x)|&=\left|\int_{-1}^1f(x+t)Q_n(t) - \int_{-1}^1f(x)Q_n(t)\diff t\right|\\
&\le\int_{-1}^1|f(x+t)-f(x)|Q_n(t)\diff t\\
&\le 2M\int_{-1}^{-\delta}Q_n(t)\diff t+\frac{\varepsilon}{2}\int_{-\delta}^{\delta}Q_n(t)\diff t
+2M\int_{\delta}^1Q_n(t)\diff t\label{Eq:4:3:c}\\
&\le 4M\sqrt{n}(1-\delta^2)^n+\frac{\varepsilon}{2}\label{Eq:4:3:d}\\
&\le\varepsilon\label{Eq:4:3:e}
\end{align}
\end{subequations}
where (\ref{Eq:4:3:c}) is by separating the integrand into three parts, and then upper bounding $|f(x+t)-f(x)|$ by $2M:=2\sup_{x}|f(x)|$ for the integrand $t\in[-1,\delta)\cup(\delta,1]$, and upper bounding $|f(x+t)-f(x)|$ by $\frac{\varepsilon}{2}$ due to the continuity of $f$ for the integrand $t\in[\delta,\delta]$; (\ref{Eq:4:3:e}) is by choosing $n$ sufficiently enough to make $4M\sqrt{n}(1-\delta^2)^n$ sufficiently small.

Therefore $\|p_n-f\|_\infty=\max_{x\in[0,1]}|p_n(x)-f(x)|<\varepsilon$ for large $n$. The proof is complete.
\end{itemize}
\end{proof}





















