
\chapter{Week5}

\section{Monday}\index{week5_Tuesday_lecture}

Let $G$ be a finite group with $H\le G$. Then $G$ can be partitioned into the left cosets or the right cosets of $H$. However, the left cosets and the right cosets are usually different.
\begin{example}
Let $G=S_3$ and $H=\{(),(12)\}$, it is eaily seen that
\[
G = H\sqcup (13)H\sqcup(23)H
=
H\sqcup H(13)\sqcup H(23)
\]
However, we see that
\[
\begin{array}{ll}
(13)H\ne H(13),
&
(23)H\ne H(23)
\end{array}
\]
We are interested in the case when the left coset of H and the right coset of H are always the same.

\end{example}
\begin{definition}[normal subgroup]
Let $G$ be a group. A subgroup $H\le G$ is \emph{normal} if
\[
\begin{array}{ll}
aH=Ha,
&
\forall a\in G
\end{array}
\]
We denote this by $H\trianglelefteq G$ and $H\triangleleft G$ when $H<G$.
\end{definition}
Normal subgroups have several equivalent definitions:
\begin{theorem}
Let $G$ be a group and $H\le G$. The following statements are equivalent:
\begin{enumerate}
\item
$H\trianglelefteq G$
\item
$a^{-1}Ha\subseteq H$, for $\forall a\in G,h\in H$
\item
$a^{-1}Ha = H$, for $\forall a\in G$.
\end{enumerate}
\end{theorem}
\begin{proof}
The non-trivial case is (2) implies (3). Since (2) holds for all $a\in G$, it holds for $a^{-1}$, i.e.,
\[
(a^{-1})^{-1}Ha^{-1}\subseteq H\implies
aHa^{-1}\subseteq H\implies
H\subseteq a^{-1}Ha
\]
\end{proof}
\begin{example}
\begin{enumerate}
\item
Any group $G$ contains the trivial normal subgroups, i.e. ,$1$ and $G$
\item
Let $G=S_3, N=\{(),(123),(132)\}, H_1=\{(),(12)\},H_2=\{(),(13)\},H_3=\{(),(23)\}$, then $N\triangleleft G$ but $H_i$'s are not.
\item
Let $n\in\mathbb{N}^+$, then $\mbox{SL}(n,\mathbb{R})\triangleleft\mbox{GL}(n,\mathbb{R})$
\item
Let $n\in\mathbb{N}^+$, then $A_n\triangleleft S_n$. (question)
\item
Let $H,K$ be groups and $G=H\times K$. Then $H\times 1$ and $1\times K$ are normal subgroups of $G$.
\end{enumerate}
\end{example}
\begin{proposition}
Let $i,j,k$ be such that
\[
i^2=j^2=k^2=ijk=-1\in\mathbb{R},
\]
show that the \emph{quaternion group}
\[
Q_8=\gen{i,j,k}
\]
has order $8$, and every its subgroup is normal.
\end{proposition}
\begin{proof}
Since $i^2=j^2=k^2=-1,ij=k,jk=i,ki=j$, any element from the set $Q_8$ can be written as the form $\pm i^{m_1}j^{m_2}k^{m_3}$ for $m_i\in\mathbb{N}$ with $m_1+m_2+m_3=1$. Hence, the set $Q_8$ has at most $8$ elements, i.e., the order should be no more than $8$. Furthermore, note that $\pm1,\pm i,\pm j,\pm k\in Q_8$, which means the $|Q_8|=8$.

Also, due to Lagrange's theorem, every subgroup can only have order $1,2,4,8$, and the subgroup with order $1$ or $8$ are trivial normal subgroups; the subgroup with order $4$ has index $2$, i.e., is normal. After computation, we find the only one subgroup with order $2$ is $\{1,-1\}$, which is normal obviously.
\end{proof}
\begin{remark}
Every subgroup of an abelian group is normal, but the converse is not true (e.g., see example above). In general, a group $G$ is \emph{Dedekind} if every its subgroups is normal; and if $G$ is non-abelian but with all normal subgroups, then $G$ is \emph{Hamiltonian group}.
\end{remark}

\begin{theorem}
Let $G$ be a group with $H\trianglelefteq G$, then the set $[G:H]$ forms a \emph{quotient group} (factor group) $G/H$ under the operation defined as:
\[
(aH)(bH):=(ab)H,\qquad
\forall a,b\in G
\]
\end{theorem}
Note that the proof is incomplete, we need to check the well-defineness of operation.
\begin{proof}
To examine that $G/H$ is indeed a group:
\begin{itemize}
\item
$(ab)H$ is also a left cosets
\item
associative
\item
$H$ is identity
\item
$a^{-1}H$ is inverse
\end{itemize}
\end{proof}
\begin{example}
For $n\in\mathbb{N}^+$, the abelian group $\mathbb{Z}$ contains a normal subgroup $n\mathbb{Z}$, and $\mathbb{Z}/n\mathbb{Z}$ is a cyclic group of order $n$.
\end{example}
\begin{proposition}
Let $G$ be a group, then the \emph{center}
\[
Z(G):=\{z\in G\mid zg = gz,\forall g\in G\}
\]
forms a normal subgroup of $G$.
\end{proposition}
Question for Proposition 3.5
\begin{proof}
First, show that $z_1,z_2\in Z(G)$ implies $z_1z_2^{-1}\in Z(G)$. Next, show that $g^{-1}zg=z$ for all $g\in G$ and $z\in Z(G)$.
\end{proof}
\begin{example}
\begin{enumerate}
\item
Let $G$ be an abelian group, then $Z(G)=G$, i.e., $Z(G) $ is essentially the largest abelian subgroup of $G$
\item
Let $n\ge3$ be an integer, then $Z(S_n)=1$.
\item
Let $n\ge 3$ be an integer, then $Z(\mathbb{Z}_n\times S_n) = \mathbb{Z}_n\times 1$.
\item
\[
Z(\mbox{GL}(2,\mathbb{R}))=\{\diag(a,b):ab\ne0\}
\]
\end{enumerate}
\end{example}
\subsection{Derived subgroups}
\begin{definition}[derived subgroup]
Let $G$ be a group and $a,b\in G$. The \emph{commutator} of $a,b$ is:
\[
[a,b]:=a^{-1}b^{-1}ab
\]
The  \emph{derived subgroup} (\emph{commutator subgroup}) of $G$ is
\[
G':=\gen{[a,b]| a,b\in G}
\]
\end{definition}
\begin{proposition}
The $G'$-coset partition defines an equivalence relation on $G$ such that $ab\sim_{G'}ba$ for $\forall G$.
\end{proposition}
\begin{proof}
First show that $x\sim_{G'}y$  iff $xy^{-1}\in G'$.

Then it's trivial that $aba^{-1}b^{-1}\in G'$.
\end{proof}
Note that the $L$-coset parititon $a\sim_Lb$ means that $aH=bH$.
\begin{remark}
If $G'\triangleleft G$, then $G/G'$ is an abelian group. Note that $G'$ is normal since
\[
a^{-1}ha=[a,h^{-1}]h\in G'
\]
\end{remark}
\begin{theorem}
Let $G$ be a group, then $G'\triangleleft G$ and $G/G'$ is abelian.
\end{theorem}
\begin{corollary}
Let $G$ be a group such that $G''=1$, then $G$ is abelian
\end{corollary}
\begin{proof}
$\{\{a\}\mid a\in G\}$ is abelian implies $G$ is abelian.
\end{proof}
\begin{remark}
The derived subgroup is the smallest normal subgraoup such that the quotient group $G/G'$ is abelian, i.e., any quotient group $G/H$ is abelian iff $H$ contains $G'$.
\end{remark}
\begin{theorem}
Let $G$ be a group and $H\triangleleft G$, then $G/H$ is abelian iff $G'\le H$.
\end{theorem}
\begin{proof}
Necessity. Since $G/H$ is abelian, we have
\[
ab H=baH\implies
abh_1=bah_2\implies
[a,b]=h_2h_1'\in H\implies
\gen{[a,b]|a,b\in G}\in H
\]

Suffiency. Note that
\[
a^{-1}b^{-1}ab\in G'\subseteq H\implies
a^{-1}b^{-1}ab=h\implies
ab\sim_Hba,\forall a,b\in G
\]
\end{proof}
\begin{theorem}
Let $n\in\mathbb{N}^+$, then $A_n=S_n'$ ($A_n$ denotes the group of even permutations). Moreover, when $n\ge5$, $A_n'=A_n$.
\end{theorem}
Recall that a permutation is called an even permutation if it can be written as a product of an even number of transpositions.

\begin{proof}
Note that $S_n/A_n=A_n\sqcup\tau A_n$, and therefore abelian. Thus $A_n\ge S'_n$. It suffices to show $S_n'\ge A_n$, Note that
\[
A_n=\gen{(12i)|i=3,\dots,n}
\]
Therefore $(12i)=(12)^{-1}(1i)^{-1}(12)(1i)\in S_n'$ implies $A_n\le S_n'$.

When $n\ge 5$, note that $A_n'\le A_n$. On the other hand,
\[
(12i) = (1a2)^{-1}(1bi)^{-1}(1a2)(1bi)\in A_n'\implies
A_n\le A_n'
\]
\end{proof}
\begin{remark}
In general, a group satisfying $G'=G$ is perfect. The alternating groups are concrete examples of perfect groups.
\end{remark}
\begin{proposition}
The group $\mbox{SL}(2,\mathbb{R})$ is perfect.
\end{proposition}
\begin{proof}
Note that any element is a product of $\begin{pmatrix}
1&x\\0&x
\end{pmatrix}$ and $\begin{pmatrix}
1&0\\y&1
\end{pmatrix}$ and these two basis can be written as the form $[a,b]$.,e.g.,
\[
\begin{pmatrix}
1&x\\0&1
\end{pmatrix}=[\begin{pmatrix}
1&x\\0&1
\end{pmatrix}\begin{pmatrix}
(\sqrt{2})^{-1}&0\\0&\sqrt{2}
\end{pmatrix}]
\]
\end{proof}
\begin{definition}[Simple]
Every group $G$ contains the trivial normal subgroups $1$ and $G$. If these are only normal subgroups contained in $G$, then we say $G$ is \emph{simple}.
\end{definition}
\begin{definition}[Conjugacy Class]
Let $G$ be a group and $a,b\in G$. If there exists $g\in G$ such that $g^{-1}ag=b$, then $a,b$ are \emph{conjugate}, and $b$ is a conjugate of $a$.  The conjugacy class with representative $a$ is a collectio nof all conjugates of $a$:
\[
\mbox{Cl}(a)=\{g^{-1}ag\mid g\in G\}
\]
\end{definition}
\begin{proposition}
The conjugacy class defines an equivalence relation on $G$; and $\mbox{Cl}(z)=\{z\}$ for each $z\in G$.
\end{proposition}
\begin{theorem}
Let $G$ be a finite group with $r$ disjoint conjugacy classes of size $c_1,\dots,c_r\ge2$. Let $|Z(G)|=c_0$, then
\[
|G| = \sum_{i=0}^rc_i
\]
\end{theorem}
\begin{proof}
Note that $x\in Cl(z)$ with $z\in Z(G)$ iff $x=z$. Hence the conjugacy class with only one element must be of the form $\{z\}$, $z\in Z(G)$. Thererfore,
\[
|G|=\sum_{i=0}^rc_i
\]
\end{proof}

\begin{theorem}
The alternating group $A_5$ is simple.
\end{theorem}
\begin{proof}[Solution.]
Let $\sigma\in N\triangleleft A_5$ be non-identity, then if we can show that $N=A_5$, which is a contradiction, then we show that $A_5$ is simple.

Note that $A_5$ is generated by the $3-$cycles, i.e., every element $\sigma$ of $A_n$ can be written as
\[
\sigma=C_1C_2\cdots C_k,
\]
with $C_i$ to be $3$-cycles.

Note that $N$ contains a non-trivial even permutation $\sigma$, which must be of the form $(abcde)$ or $(ab)(cd)$ or $(abc)$.
\begin{itemize}
\item
When $\sigma=(abcde)$, let $\alpha = (ab)(cd)$. then $N$ also contains:
\[
\alpha\sigma\alpha^{-1}=(ab)(cd)(abcde)(ab)(cd)=(adceb)
\]
and therefore contains
\[
\sigma\sigma^{-1}=(aec)
\]
\item
When $\sigma=(ab)(cd)$, let $\beta=(abe)$, then $N$ also contains
\[
\sigma' =\beta\sigma\beta^{-1}=(becd)
\]
and therefore contains
\[
\sigma\sigma^{-1}=(abe)
\]
\end{itemize}
If $N$ contains a signle $3$-cycles, since $3$-cycles are mutually conjugate, $N$ will contain any other $3$-cycles. Therefore $N=A$, which is a contradiction.
\end{proof}
\begin{proof}[Solution 2]
Let $N$ be a normal subgroup of $A_5$, then it is a union of some of the conjugacy classes of $A_5$. Since the order of $N$ must divide $60$, a short calculation shows that no union of some of these conjugacy classes that includes $\{e\}$ has order a divisor of $60$, unless $A_5=\{e\}$ or $A_5$.
\end{proof}


















