\subsection{Solution to Assignment Six}
\begin{enumerate}
%q1
\item
\begin{proof}[Solution.]
One basis for $\mathbb{P}_2$ is $\{t^2,t,1\}.$ And we obtain:
\begin{align*}
T(t^2)&=(3t-2)^2&=9t^2-6t+4\times 1\\
T(t)&=3t-2&=0t^2+3t+(-2)\times 1\\
T(1)&=1&=ot^2+0t+1\times 1\\
\end{align*}
Hence the matrix representation is given by:
\[
\bm A=\begin{bmatrix}
9&-6&4\\0&3&-2\\0&0&1
\end{bmatrix}
\]
We croos the column 1 to compute determinant:
\[
\det(\bm A)=9\begin{vmatrix}
3&-2\\0&1
\end{vmatrix}=27.
\]
\end{proof}
%q2
\item
\begin{proof}
We only need to show $\bm x\trans\bm y=0$:\\
By \textit{postmultiplying} $\bm x\trans$ for $\bm A\trans\bm y=2\bm y$ both sides we obtain:
\[
\bm x\trans\bm A\trans\bm y=2\bm x\trans\bm y
\]
Or equivalently,
\[
(\bm{Ax})\trans\bm y=2\bm x\trans\bm y
\implies
\bm 0\trans\bm y=2\bm x\trans\bm y
\implies
\bm x\trans\bm y=0.
\]
\end{proof}
%q3
\item
\begin{proof}[Solution.]
\begin{enumerate}
%part a
\item
\textit{True.}\\
\textbf{Reason: }Assume $\bm Q$ is a $n\times n$ matrix s.t. 
\[
\bm Q=\begin{bmatrix}
q_1&q_2&\dots&q_n
\end{bmatrix}
\]
Then the product of $\bm Q\trans\bm Q$ is
\[
\bm Q\trans\bm Q=\begin{bmatrix}
q_1\trans\\q_2\trans\\\vdots\\q_n\trans
\end{bmatrix}\begin{bmatrix}
q_1&q_2&\dots&q_n
\end{bmatrix}=\begin{bmatrix}
q_1\trans q_1&q_1\trans q_2&\dots&q_1\trans q_n\\
q_2\trans q_1&q_2\trans q_2&\dots&q_2\trans q_n\\
\vdots&\vdots&\ddots&\vdots\\
q_n\trans q_1&q_n\trans q_2&\dots&q_n\trans q_n
\end{bmatrix}
\]
Due to the orthonormality of $q_1,\dots,q_n$, we obtain:
\[
\bm Q\trans\bm Q=\bm I_{n}.
\]
Hence $\bm Q^{-1}=\bm Q\trans$. If we define $\bm Q^{-1}=\begin{bmatrix}
q_1^*&q_2^*&\dots&q_n^*
\end{bmatrix}$, then we obtain:
\[
(\bm Q^{-1})\trans\bm Q^{-1}=
\begin{bmatrix}
(q_1^*)\trans\\(q_2^*)\trans\\\vdots\\(q_n^*)\trans
\end{bmatrix}\begin{bmatrix}
q_1^*&q_2^*&\dots&q_n^*
\end{bmatrix}=\begin{bmatrix}
(q_1^*)\trans q_1^*&(q_1^*)\trans q_2^*&\dots&(q_1^*)\trans q_n^*\\
(q_2^*)\trans q_1^*&(q_2^*)\trans q_2^*&\dots&(q_2^*)\trans q_n^*\\
\vdots&\vdots&\ddots&\vdots\\
(q_n^*)\trans q_1^*&(q_n^*)\trans q_2^*&\dots&(q_n^*)\trans q_n^*
\end{bmatrix}
=\bm I
\]
Hence for columns $q_1^*,q_2^*,\dots,q_n^*$ we have:
\[
\inp{\bm q_i^*}{\bm q_j^*}=\begin{cases}
0&\text{when $i\ne j$}\qquad\text{(\emph{orthogonal} vectors)},\\
1&\text{when $i=j$}\qquad\text{(\emph{unit} vectors: $\|\bm q_i^*\|=1$)}.
\end{cases}
\]
for $i,j\in\{1,2,\dots,n\}$.\\ By definition, $q_1^*,q_2^*,\dots,q_n^*$ are orthonormal. Hence $\bm Q^{-1}$ is a orthogonal matrix.\\
\textbf{Example:}
\[
\bm Q=\begin{bmatrix}
1&0\\0&1
\end{bmatrix}\implies\bm Q^{-1}=\begin{bmatrix}
1&0\\0&1
\end{bmatrix}
\]
which is obviously orthonormal.
%part b
\item
\textit{True.}\\
\textbf{Reason: }
Assume $\bm Q=\begin{bmatrix}
q_1&q_2&\dots&q_n
\end{bmatrix}$, where $q_i\in\mathbb{R}^{m}$ for $i=1,\dots,n$.\\
\begin{itemize}
\item
Firstly we show $\bm Q\trans\bm Q=\bm I$:
\[
\bm Q\trans\bm Q=\begin{bmatrix}
q_1\trans\\q_2\trans\\\vdots\\q_n\trans
\end{bmatrix}\begin{bmatrix}
q_1&q_2&\dots&q_n
\end{bmatrix}=\begin{bmatrix}
q_1\trans q_1&&&\\
&q_2\trans q_2&&\\
&&\ddots&\\
&&&q_n\trans q_n
\end{bmatrix}=\bm I_{n}.
\]
\item
Hence we derive
\begin{align*}
\|\bm{Qx}\|^2&=\bm x\trans\bm Q\trans\bm Q\bm x\\
&=\bm x\trans(\bm Q\trans\bm Q)\bm x=\bm x\trans\bm I\bm x\\
&=\bm x\trans\bm x\\
&=\|\bm x\|^2
\end{align*}

\end{itemize}
Hence $\|\bm{Qx}\|=\|\bm x\|$.\\
\textbf{Example: }\\
If $\bm Q=\begin{bmatrix}
1\\0
\end{bmatrix}_{2\times 1}$, then for any $\bm x=\begin{bmatrix}
\bm\alpha
\end{bmatrix}$ ($\bm\alpha$ is a row vector), 
\begin{gather}
\|\bm{Qx}\|=\|\begin{bmatrix}
\bm\alpha\\\bm 0
\end{bmatrix}\|=\sqrt{|\inp{\bm\alpha}{\bm\alpha}|+\bm 0^2}=\sqrt{|\inp{\bm\alpha}{\bm\alpha}|}\\
\|\bm x\|=\sqrt{|\inp{\bm\alpha}{\bm\alpha}|}.
\end{gather}
Hence we obtain $\|\bm{Qx}\|=\|\bm x\|$ for $\forall\bm x$.
%part c
\item
\textit{False.}\\
\textbf{Example: }\\
$\bm Q=\begin{bmatrix}
1&0\\0&0\\0&1
\end{bmatrix},\bm y=\begin{bmatrix}
0\\1\\0
\end{bmatrix}$, then note that
\[
\bm Q\trans\bm y=\begin{bmatrix}
1&0&0\\0&0&1
\end{bmatrix}\begin{bmatrix}
0\\1\\0
\end{bmatrix}=\begin{bmatrix}
0\\0
\end{bmatrix}.
\]
Thus $\|\bm Q\trans\bm y\|=0\ne 1=\|\bm y\|$.
\end{enumerate}
\end{proof}
%q4
\item
\begin{proof}[Solution.]
\begin{itemize}
\item
Firstly we show $\bm W_1\subset\bm W_2^{\perp}$:\\
For $\forall p\in\bm W_1,\forall q\in\bm W_2$, we only need to show $\inp{p}{q}=0:$\\
\begin{itemize}
\item
For $\forall f\in\bm W_2,$ we have
\begin{align*}
\int_{-1}^{1}f(x)\diff x&=\int_{-1}^{0}f(x)\diff x+\int_{0}^{1}f(x)\diff x\\
&=\int_{-1}^{0}-f(-x)\diff x+\int_{0}^{1}f(x)\diff x\\
&=\int_{-1}^{0}f(-x)\diff (-x)+\int_{0}^{1}f(x)\diff x\\
&=\int_{1}^{0}f(x)\diff (x)+\int_{0}^{1}f(x)\diff x\\
&=0.
\end{align*}
\item
And the product $pq\in\bm W_2$, this is because:
\begin{align*}
(pq)(x)&=p(x)q(x)=p(-x)-q(-x)\\
&=-p(-x)q(-x)\\
&=-(pq)(-x).
\end{align*}
Hence the inner product $\inp{p}{q}$ is given by:
\[\inp{p}{q}=\int_{-1}^{1}p(x)q(x)\diff x=\int_{-1}^{1}(pq)(x)\diff x=0
\]
\end{itemize}
Hence $\bm W_1\perp\bm W_2\implies$$\bm W_1\subset\bm W_2^{\perp}$.
\item
Then we show $\bm W_2^{\perp}\subset\bm W_1$:\\
Suppose $p^{*}\notin\bm W_1$, then we want to show $\inp{p^*}{q}\ne0$ for some $q\in\bm W_2$:\\
\begin{itemize}
\item
We decompose $p^{*}$ into
\[
p^*(x)=p_1(x)+p_2(x)
\]
where $p_1(x)=\frac{p^*(x)+p^*(-x)}{2}$ and $p_2(x)=\frac{p^*(x)-p^*(-x)}{2}$.
Since we have
\begin{gather*}
p_1(-x)=\frac{p^*(-x)+p^*(x)}{2}=p_1(x)\\
p_2(-x)=\frac{p^*(-x)-p^*(x)}{2}=-p_2(x),
\end{gather*}
we derive $p_1(x)\in\bm W_1,p_2(x)\in\bm W_2$. ($p^*\notin \bm W_1\implies p_2\ne0$.)
\item
Thus the inner product for $\inp{p^{*}}{p_2}$ is positive:
\begin{align*}
\inp{p^*}{p_2}&=\inp{p_1+p_2}{p_2}\\
&=\inp{p_1}{p_2}+\inp{p_2}{p_2}\\
&=0+\int_{-1}^{1}p_2^2(x)\diff x>0.
\end{align*}
\end{itemize}
Hence given $\forall p^*\notin\bm W_1$, there exists $q=p_2\in\bm W_2$ s.t. $\inp{p^*}{q}\ne0$.\\
Thus $p^*\notin\bm W_2^{\perp}\implies\bm W_2^{\perp}\subset\bm W_1.$
\end{itemize}
Hence we obtain $\bm W_1=\bm W_2^{\perp}$.
\end{proof}
%q5
\item\begin{proof}[Solution.]
\begin{itemize}
\item
Firstly we find a basis for $\bm U$:\\
The space $\Span\left\{\begin{bmatrix}
1\\2\\-5
\end{bmatrix}\right\}$ is the row space for matrix 
\[
\bm A=\begin{bmatrix}
1&2&-5
\end{bmatrix}
\]
Hence $\bm U=(C(\bm A))^{\perp}=N(\bm A)$. We only need to find the basis for $N(\bm A)$:
\[
\bm{Ax}=\bm 0\implies
x_1+2x_2-5x_3=0.
\]
Hence the solution to $\bm{Ax}=\bm 0$ is
\[
\begin{pmatrix}
x_1\\x_2\\x_3
\end{pmatrix}=\begin{pmatrix}
-2x_2+5x_3\\x_2\\x_3
\end{pmatrix}=x_2\begin{pmatrix}
-2\\1\\0
\end{pmatrix}+x_3\begin{pmatrix}
5\\0\\1
\end{pmatrix}
\]
where $x_2,x_3$ are arbitrary scalars.\\
Hence $\bm U$ is spanned by $\left\{\begin{pmatrix}
-2\\1\\0
\end{pmatrix},\begin{pmatrix}
5\\0\\1
\end{pmatrix}\right\}.$
And obviously, $\begin{pmatrix}
-2\\1\\0
\end{pmatrix}$ and $\begin{pmatrix}
5\\0\\1
\end{pmatrix}$ are ind.\\
Hence one basis for $\bm U$ is $\left\{\begin{pmatrix}
-2\\1\\0
\end{pmatrix},\begin{pmatrix}
5\\0\\1
\end{pmatrix}\right\}.$
\item
Let's do Gram-Schmidt Process to convert this basis into \textit{orthonormal}:\\
We set $\bm a=\begin{bmatrix}
-2\\1\\0
\end{bmatrix}$ and $\bm b=\begin{bmatrix}
5\\0\\1
\end{bmatrix}$.
\begin{itemize}
\item
Then we set $\bm A=\begin{bmatrix}
-2\\1\\0
\end{bmatrix}$.
\item
Next step, we compute 
\begin{align*}
\bm B&=\bm b-\Proj_{\bm A}(\bm b)=\bm b-\frac{\inp{\bm A}{\bm b}}{\inp{\bm A}{\bm A}}\bm A\\
&=\begin{pmatrix}
5\\0\\1
\end{pmatrix}-\frac{-10}{5}\begin{pmatrix}
-2\\1\\0
\end{pmatrix}\\
&=\begin{pmatrix}
1\\2\\1
\end{pmatrix}
\end{align*}
\item
Then we convert orthogonal sets $\{\bm A,\bm B\}$ into orthonormal:
\[
\bm q_1:=\frac{\bm A}{\|\bm A\|}=\begin{pmatrix}
-\frac{2}{\sqrt{5}}\\\frac{1}{\sqrt{5}}\\0
\end{pmatrix}\qquad
\bm q_2:=\frac{\bm B}{\|\bm B\|}=\begin{pmatrix}
\frac{1}{\sqrt{6}}\\\frac{2}{\sqrt{6}}\\\frac{1}{\sqrt{6}}
\end{pmatrix}
\]
\end{itemize}
\end{itemize}
In conclusion, one orthonormal basis for $\bm U$ is $\left\{\begin{pmatrix}
-\frac{2}{\sqrt{5}}\\\frac{1}{\sqrt{5}}\\0
\end{pmatrix},\begin{pmatrix}
\frac{1}{\sqrt{6}}\\\frac{2}{\sqrt{6}}\\\frac{1}{\sqrt{6}}
\end{pmatrix}\right\}$.
\end{proof}
%q6
\item
\begin{proof}[Solution.] 
We only need to find \textit{least squares solution} $\bm x^*$ to $\bm{Ax}=\bm b$, where
\[
\bm A=\begin{bmatrix}
1&-2\\1&-1\\1&0\\1&1\\1&2
\end{bmatrix}\qquad\bm x=\begin{bmatrix}
C\\D
\end{bmatrix}\qquad\bm b=\begin{bmatrix}
4\\2\\-1\\0\\0
\end{bmatrix}.
\]
Take on trust that we only need to solve $\bm A\trans\bm A\bm x=\bm A\trans\bm b$.\\
\begin{itemize}
\item
But before that, let's do QR factorization for $\bm A$:
\[
\text{Define }\bm A:=\begin{bmatrix}
\bm a_1&\bm a_2
\end{bmatrix}\qquad
\inp{\bm a_1}{\bm a_2}=0
\implies
\text{Columns of $\bm A$ are orthogonal.}
\]
So we obtain orthonormal vectors:
\[
\bm q_1:=\frac{\bm a_1}{\|\bm a_1\|}=\begin{bmatrix}
\frac{1}{\sqrt{5}}\\\frac{1}{\sqrt{5}}\\\frac{1}{\sqrt{5}}\\\frac{1}{\sqrt{5}}\\\frac{1}{\sqrt{5}}
\end{bmatrix}\qquad
\bm q_2=\frac{\bm a_2}{\|\bm a_2\|}=\begin{bmatrix}
-\frac{2}{\sqrt{10}}\\-\frac{1}{\sqrt{10}}\\0\\\frac{1}{\sqrt{10}}\\\frac{2}{\sqrt{10}}
\end{bmatrix}
\]
Thus the factor is given by
\[
\bm Q=\begin{bmatrix}
\bm q_1&\bm q_2
\end{bmatrix}\qquad
\bm R=\bm Q\trans\bm A=\begin{bmatrix}
\bm q_1\trans\bm a_1&\bm q_1\trans\bm a_2\\0&\bm q_2\trans\bm a_2
\end{bmatrix}=\begin{bmatrix}
\sqrt{5}&0\\0&\sqrt{10}
\end{bmatrix}.
\]
\item
Hence we could compute the least squares solution more easily:
\[
\bm A\trans\bm A\bm x=\bm A\trans\bm b\Longleftrightarrow
\bm R\trans\bm Q\trans\bm Q\bm R\bm x=\bm R\trans\bm Q\trans\bm b\Longleftrightarrow\bm R\trans\bm R\bm x=\bm R\trans\bm Q\trans\bm b
\]
\begin{align*}
\implies\bm x=\bm R^{-1}\bm Q\trans\bm b&=\frac{1}{5\sqrt{2}}\begin{bmatrix}
\sqrt{10}&0\\0&\sqrt{5}
\end{bmatrix}\begin{bmatrix}
\frac{1}{\sqrt{5}}&\frac{1}{\sqrt{5}}&\frac{1}{\sqrt{5}}&\frac{1}{\sqrt{5}}&\frac{1}{\sqrt{5}}\\
-\frac{2}{\sqrt{10}}&-\frac{1}{\sqrt{10}}&0&\frac{1}{\sqrt{10}}&\frac{2}{\sqrt{10}}
\end{bmatrix}\\
&=\begin{bmatrix}
1\\-1
\end{bmatrix}
\end{align*}
\end{itemize}
Hence we have $\begin{cases}
C=1\\D=-1.
\end{cases}$ The best line is $\hat y=1-x$.
\end{proof}
\end{enumerate}