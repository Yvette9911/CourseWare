
\chapter{Week1}

\section{Wednesday}\index{Wednesday_lecture}
\subsection{Introduction to Imaginary System}

Actually, $(\mathbb{C},+)$ forms a group:
\begin{align*}
(z_1+z_2)+z_3&=z_1+(z_2+z_3)\\
z_1+z_2&=z_2+z_1\\
z+0&=0+z=z\\
z+(-z)&=(-z)+z=0
\end{align*}

Also, $(\mathbb{C}\setminus\{0\},\cdot)$ forms a group.

The product for imaginary numbers is different from vector product:
\[
\vec v\cdot\vec w = v_1w_1+v_2w_2
\]

Also, we can define the crossover product $\vec v\times\vec w$.

modulus:
\[
|z| = \sqrt{x^2+y^2}
\]

direction angle (Argument):
\[
\tan\theta=\frac{y}{x}
\]

Using the polar coordination, we find $z=x+iy$ can be transformed into
\begin{align*}
z&=r\cos\theta+ir\sin\theta\\
&=r(\cos\theta+i\sin\theta)\\
&=r\left[\cos(\theta+2n\pi) + i\sin(\theta+2n\pi)\right]
\end{align*}

Principal argument:
\[
-\pi<\mbox{Arg}z\le\pi
\]

Conjugate form of imaginary number:
\[
\bar z = x-iy
\]

\begin{align*}
\overline{z+w} &=\bar z + \bar w\\
z\cdot\bar z&=|z|^2\\
\frac{1}{z}&=\frac{\bar z}{|z|^2}
\end{align*}
\begin{proposition}
$|z+w|\le|z|+|w|$
\end{proposition}
\begin{proposition}
$|z+w|^2+|z-w|^2=2|z|^2+2|w|^2$.
\end{proposition}
\begin{proposition}
\begin{eqnarray*}
\Re z=\frac{z+\bar z}{2},
&
\Im z=\frac{z-\bar z}{2i}
\end{eqnarray*}
\end{proposition}


















