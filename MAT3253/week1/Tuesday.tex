
\chapter{Week1}

\section{Wednesday}\index{Wednesday_lecture}
\subsection{Introduction to Imaginary System}
\begin{definition}[Complex Number]
A complex number $z$ is a pair of real numbers:
\[
z = (x,y),
\]
where $x$ is the \emph{real} part and $y$ is the \emph{imaginary part} of $z$, denoted as
\[
\begin{array}{ll}
\mbox{Re}z = x
&
\mbox{Im}z = y
\end{array}
\]
\end{definition}
\begin{remark}
Note that the complex multiplication does not correspond to any standard vector operation. However, $(\mathbb{C},+)$ and $(\mathbb{C}\setminus\{0\},\cdot)$ forms a field:
\begin{align*}
(z_1+z_2)+z_3&=z_1+(z_2+z_3)\\
z_1+z_2&=z_2+z_1\\
z+0&=0+z=z\\
z+(-z)&=(-z)+z=0
\end{align*}
There is no other Eucliean space that can form a field.
\end{remark}
\begin{proposition}
$zz'=0$ if and only if $z=0$ or $z'=0$.
\end{proposition}
\begin{proof}
Rewrite the product as a linear system
\[
\begin{pmatrix}
x&-y\\y&x
\end{pmatrix}\begin{pmatrix}
x'\\y'
\end{pmatrix}=\begin{pmatrix}
0\\0
\end{pmatrix}
\]
and discuss the determinant of the coefficient matrix.
\end{proof}
\paragraph{Solving quadratic equation with one unknown}We can apply the imaginary number to solve the quadratic equations. For example, to solve $z^2-2z+2=0$, the first method is to substitute $z$ with $x+iy$; the second method is to simplify it into standard form to solve it.
\begin{definition} 
If $z\ne0$, then $z^{-1}$ is the complex number satisfying $z\cdot z^{-1}=1$.
\end{definition}
Suppose $z=(x,y)$ and $z^{-1}=(u,v)$. After simplification, we derive
\[
\left\{
\begin{aligned}
xu - yv &=1\\
xv + yu &=0
\end{aligned}
\right.\implies
\left\{
\begin{aligned}
u&=\frac{x}{x^2+y^2}\\
v&=\frac{-y}{x^2+y^2}
\end{aligned}
\right.
\]
\begin{definition}[Division]
The division between complex numbers is defined as:
\[
\begin{array}{ll}
\frac{z_1}{z_2}=z_1\cdot z_2^{-1},
&
\mbox{when }z_2\ne0
\end{array}
\]
\end{definition}
\begin{example}
\[
\frac{3-4i}{1+i} = (3-4i)\left(\frac{1}{2} - \frac{1}{2}i\right)
=-\frac{1}{2} - \frac{7}{2}i
\]
\[
\frac{10}{(1+i)(2+i)(3+i)}=\frac{10}{(1+3i)(3+i)}=\frac{10}{10i}=\frac{1}{i}=-i
\]
\end{example}
\begin{definition}[Complex Conjugate]
The complex number $x-iy$ is called the \emph{complex conjugate} of $z = x+iy$, which is denoted by $\bar z$.
\end{definition}
The following properties hold for complex conjugate:
\[
\begin{array}{lll}
\overline{z_1\pm z_2} = \bar z_1\pm\bar z_2,
&
\overline{z_1z_2} = \bar z_1\bar z_2.
&
\overline{\frac{z_1}{z_2}}=\frac{\bar z_1}{\bar z_2}
\end{array}
\]
\[
\begin{array}{ll}
\mbox{Re}z=\frac{z+\bar z}{2},
&
\mbox{Im}z = \frac{z - \bar z}{2i}
\end{array}
\]
\subsection{Algebraic and geometric properties}
\begin{definition}[Algebraic Region]
\begin{enumerate}
\item
The complex plane: the $z$-plane, i.e., $\mathbb{C}$
\item
Vector in $\mathbb{R}^2$: $(x,y) = x+iy=z\in\mathbb{C}$
\item
Modulus of $z$:
\[
\begin{array}{ll}
|z|=\sqrt{x^2+y^2}
&
\mbox{distance to the origin}
\end{array}
\]
\end{enumerate}
\end{definition}
Note that
\[
\begin{array}{ll}
|z|=0\Longleftrightarrow z=0,
&
|z_1 - z_2| = 0\Longleftrightarrow
z_1=z_2
\end{array}
\]
\begin{definition}[Circle in plane]
A circle with center $z_0$ and radius $R$ is defined as follows in $\mathbb{C}$:
\[
\{z\in\mathbb{C}\mid |z-z_0|=R\}
\]
\end{definition}
\begin{proposition}
Complex roots of polynomials with real coefficients appear in conjugate pairs.
\end{proposition}
\begin{proof}
Given $P(z_0)=0$, we derive
\[
P(z_0)=\overline{P(z_0)}=0.
\]
\end{proof}
Note that a polynomial with real coefficients of degree 3 must have at least one real root.

\paragraph{Conjugate Product}
Note that the conjugate product leads to the square of modulus:
\[
z\cdot\bar z =  |z|^2
\Longleftrightarrow
(x+iy)(x-iy) = x^2 + y^2
\]
Such a property can be used to simplify quotient of two complex numbers:
\[
\frac{z_1}{z_2}=\frac{z_1\bar z_2}{|z_2|^2}=\frac{x_1x_2 + y_1y_2 + (y_1x_2 - x_1y_2)i}{x_2^2+y_2^2}
\]
\begin{example}
\[
\frac{-1+3i}{2-i}=\frac{(-1+3i)(2+i)}{(2-i)(2+i)}=\frac{-5+5i}{5}=-1+i
\]
\[
|z_1+z_2|^2 + |z_1-z_2|^2 = 2(|z_1|^2+|z_2|^2)
\]
\end{example}
We can use conjugate to show the \emph{triangle inequality}:
\begin{proposition}[Triangle Inequality]
$|z_1+z_2|\le|z_1|+ |z_2|$.
\end{proposition}
\begin{proof}
\begin{align*}
|z_1+z_2|^2&=(z_1+z_2)\overline{(z_1+z_2)}\\
&=|z_1|^2+|z_2|^2+z_1\bar z_2 + \overline{z_1\bar z_2}\\
&=|z_1|^2+|z_2|^2+2\mbox{Re}(z_1\bar z_2)\\
&\le |z_1|^2+|z_2|^2+2|z_1\bar z_2|\\
&=|z_1|^2+|z_2|^2+2|z_1z_2|=(|z_1|+|z_2|)^2.
\end{align*}
\end{proof}
\begin{corollary}
\begin{enumerate}
\item
$||z_1| - |z_2||\le |z_1\pm z_2|$.
\item
If $|z|\le 1$, then $|z^2+z+1|\le 3$
\end{enumerate}
\end{corollary}
\begin{proof}
\begin{enumerate}
\item
Note that 
\[
|z_1| = |z_1\pm z_2\mp z_2|\le |z_1\pm z_2|+|z_2|\implies
|z_1|-|z_2|\le |z_1\pm z_2|
\]
Similarly, $|z_2| - |z_1|\le |z_1\pm z_2|$.
\item
\[
|z^2+z+1|\le |z^2|+|z+1|\le |z|^2 + |z|  + 1\le1+1+1=3.
\]
\end{enumerate}
\end{proof}
\begin{proposition}[Cauchy-Schwarz inequality]
If $z_1,\dots,z_n$ and $w_1,\dots,w_n$ are complex numbers, then
\[
\left[
\sum_{k=1}^nz_kw_k
\right]^2
\le
\left[
\sum_{k=1}^n|z_k|^2
\right]
\left[
\sum_{k=1}^n|w_k|^2
\right]
\]
\end{proposition}
\subsection{Polar and exponential forms}
\begin{definition}[Polar Form]
The polar form of a nonzero complex number $z$ is:
\[
z = r(\cos\theta + i\sin\theta)
\]
where $(r,\theta)$ is the polar coordinates of $(x,y)$.
\[
(r,\theta)\implies (x,y):\left\{
\begin{aligned}
x&=r\cos\theta\\
y&=r\sin\theta
\end{aligned}
\right.
\]
\[
(x,y)\implies (r,\theta):\left\{
\begin{aligned}
r&=\sqrt{x^2+y^2}\\
\tan\theta&=\frac{y}{x}
\end{aligned}
\right.
\]
Note that $\theta$ is said to be the \emph{argument} of $z$, i.e., $\theta=\arg z$. The augument is not unique, i.e.,
\[
z = r(\cos\theta + i\sin\theta) 
 r(\cos(\theta+2\pi) + i\sin(\theta + 2\pi))
\]
\end{definition}
If given an argument of $z$, then we form the set of arguments of $z$:
\[
\{\theta+2n\pi\mid n\in\mathbb{Z}\}
\]
\begin{definition}[Principal Value]
The principal value of $\arg z$, denoted by $\mbox{Arg} z$, is the unique value of $\arg z$ such that $-\pi<\arg z\le\pi$
\end{definition}
\begin{example}
\begin{enumerate}
\item
$\mbox{Arg}z=\pi$ implies $z = r(\cos\pi + i\sin\pi) = -r<0$, which is a negative real number.
\item
$\mbox{Arg}z=0$ implies $z=r(\cos0+i\sin0)=r>0$m which is a positive real number.
\item
$\mbox{Arg}z=-\frac{\pi}{2}$ implies $z=r(\cos(-\frac{\pi}{2})+i\sin(-\frac{\pi}{2}))=-ri$
\item
$\mbox{Arg}z = \frac{\pi}{2}$ implies $z = ri$
\item
Particularly, $\pm i = \cos(\pm \frac{\pi}{2}) + i\sin(\pm\frac{\pi}{2})$
\end{enumerate}
\end{example}





Actually, $(\mathbb{C},+)$ forms a group:


Also, $(\mathbb{C}\setminus\{0\},\cdot)$ forms a group.

The product for imaginary numbers is different from vector product:
\[
\vec v\cdot\vec w = v_1w_1+v_2w_2
\]

Also, we can define the crossover product $\vec v\times\vec w$.

modulus:
\[
|z| = \sqrt{x^2+y^2}
\]

direction angle (Argument):
\[
\tan\theta=\frac{y}{x}
\]

Using the polar coordination, we find $z=x+iy$ can be transformed into
\begin{align*}
z&=r\cos\theta+ir\sin\theta\\
&=r(\cos\theta+i\sin\theta)\\
&=r\left[\cos(\theta+2n\pi) + i\sin(\theta+2n\pi)\right]
\end{align*}

Principal argument:
\[
-\pi<\mbox{Arg}z\le\pi
\]

Conjugate form of imaginary number:
\[
\bar z = x-iy
\]

\begin{align*}
\overline{z+w} &=\bar z + \bar w\\
z\cdot\bar z&=|z|^2\\
\frac{1}{z}&=\frac{\bar z}{|z|^2}
\end{align*}
\begin{proposition}
$|z+w|\le|z|+|w|$
\end{proposition}
\begin{proposition}
$|z+w|^2+|z-w|^2=2|z|^2+2|w|^2$.
\end{proposition}
\begin{proposition}
\begin{eqnarray*}
\Re z=\frac{z+\bar z}{2},
&
\Im z=\frac{z-\bar z}{2i}
\end{eqnarray*}
\end{proposition}


















