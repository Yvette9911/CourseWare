
\chapter{Week1}

\section{Monday}\index{Monday_lecture}
\subsection{Introduction to Numerical Analysis}
\paragraph{Solving Nonlinear Equations}
For example, we want to solve a nonlinear equation $w(1)$ with $w$ to be the LambertW function:
\[
we^w = 1.
\]
This topic will be taught in chapter 2.

\paragraph{Interpolation}
Given a list of data points, our aim is to recover/approximate the origin function over a function class, i.e., piecewise linear functions or polynomials. This topic will be taught in chapter 3.

\paragraph{Numerical Integration}
The cdf of the standard normal distribution is given by:
\[
N(x)=\frac{1}{\sqrt{2\pi}}\int_{-\infty}^x\exp\left(-\frac{t^2}{2}\right)\diff t
\]
To approximate the values for $N(x)$, we need to apply a numerical method (quadrature rules) to evaluate. This topic will be taught in chapter 4.

\paragraph{Solving Linear Systems}
To find the solutions of a linear system of equations 
\[
\bm{Ax}=\bm b,
\]
e.g., when we use \textit{finite difference} method to solve a differential equation, it is necessary to apply some numerical method to solve it in computer. This topic will be taught in chapter 5.

\paragraph{Least Squares}
If we have more time, we will teach how to fit a set of data points by a function from a function class.

\subsection{Basic Concepts}
\begin{definition}[Truncation Error]
The error made by numerical algorithms that arises from taking finite number of steps in computation
\end{definition}
For example, consider the taylor's theorem
\[
f(x)=P_n(x)+R_n(x)
\]
where
\begin{align*}
P_n(x)&=f(x_0)+f'(x_0)(x-x_0)+\cdots+\frac{f^{(n))}(x_0)}{n!}(x-x_0)^n\\
R_n(x)&=\frac{f^{(n+1)}(\xi)}{(n+1)!}(x-x_0)^{n+1}
\end{align*}
If we use $P_n(x)$ to approximate $f(x)$, the error $R_n(x)$ is called the \emph{truncation error}.

\begin{definition}[Round-off Error]
The error produced when a computer is used to
perform real number calculations, i.e., the computer only gives approximate value for some real numebers.
\end{definition}

For example, for numbers $\frac{1}{3},\pi,\sqrt{2}$, they cannot be represented exactly in the computation by a computer.

Such errors are invertible, but  we can try to minimize the negative impact of these errors by rewriting the formula that are wish to compute.

\begin{definition}[Binary Floating Point Number]
A 64-bit (binary digit) representation is used for a real number:
\[
(-1)^s2^{c-1023}(1+f)
\]
where 
\begin{itemize}
\item
The first bit $s$ is said to be a \emph{sign indicator};
\item
The 11-bit exponent $c$ is called the \emph{characteristic}
\item
The 52-bit binray $f$ is called the \emph{mantissa}.
\end{itemize}
\end{definition}
The \emph{underflow} is given by (??):
\[
2^{-1022}\cdot(1+0)\approx 0.22251\times 10^{-307}
\]
The \emph{overflow} is given by (??):
\[
2^{1023}\cdot(2-2^{-52})\approx 0.17977\times 10^{309}
\]
\begin{definition}[Decimal floating-point number]
Any positive real number within the numerical range of the machine can be converted into the form
\[
y = 0.d_1d_2\cdots d_k\cdots\times10^n
\]
The \emph{floating-point} form of $y$, denoted by $fl(y)$, is obtained by terminating the mantissa of $y$ at $k$ decimal digits.
\end{definition}
There are two common ways to perform such termination.
\paragraph{Chopping} Simply chops off the digits $d_{k+1}d_{k+2}\dots$, which produces the floating-point of the form
\[
fl(y) = 0.d_1d_2\cdots d_k\times 10^k
\]
\paragraph{Rounding} Adds $5\times 10^{n-(k+1)}$ to $y$ and then chops the result to obtain a number of the form
\[
fl(y) = 0.\delta_1\delta_2\cdots\delta_k\times 10^n.
\]
Or equivalently, for rounding, we add $1$ to $d_k$ to obtain $fl(y)$ for $d_{k+1}\ge5$ (round up); otherwise we simply chop off at $k$ digits (round down).

\paragraph{Round-Off Error} The errors that results from replacomg a number with its floating-point form is called \emph{round-off error}.

\begin{example}
Determine the five-digit chopping and rounding values of $\pi$:
\begin{itemize}
\item
Firstly we write $\pi$ in normalized decimal form:
\[
\pi=0.3141592\cdots\times 10^1
\]
\item
The floating-point form of $\pi$ using five-digit chopping is
\[
fl(\pi)=0.31415\times10^1=3.1415
\]
\item
The floating-point form of $\pi$ using five-digit rounding is
\[
fl(\pi)=(0.31415+0.00001)\times10^1=3.1416
\]
\end{itemize}
\end{example}



\begin{definition}
Suppose $p^*$ is an approximation to $p$. The acutal error is $p-p^*$, the absolute error is $|p-p^*|$, and the relative error is $\frac{|p-p^*|}{|p|}$ provided that $p\ne0$.
\end{definition}
\begin{definition}
Suppose $p^*$ is an approximation to $p$, then $p^*$ is said to approximate $p$ to $t$ \emph{significant digits} if $t$ is the largest nonnegative integer for which
\[
\frac{|p-p^*|}{|p|}\le5\times 10^{-t}.
\]
\end{definition}
\subsection{Examples in Numerical Calculations}
\subsubsection{Inaccuracy Issues}
\paragraph{Inaccuracy of floating-point numbers}
The relative error for floating-point numbers is
\[
\left|\frac{y-fl(y)}{y}\right|.
\]
For $k$-digit chopping arithmetic, its relative error is give by:
\[
\left|\frac{y-fl(y)}{y}\right|=\left|\frac{0.d_{k+1}d_{k+2}\cdots\times 10^{n-k}}{0.d_1d_2\cdots\times 10^n}\right|
\le\frac{1}{0.1}\times 10^{-k}=10^{-k+1}
\]
Similarly, the bound for the relative error when using $k$-digit rounding arithmetic is $0.5\times 10^{-k+1}$.

\paragraph{Inaccuray of finite-digit arithmetics}
We assume that the finite-digit arithmetics are given by:
\[
\begin{array}{ll}
x
\end{array}
\]

\subsubsection{Reduce the loss of accuracy due to round-off error}
\paragraph{Reformulate the problem} One common error during calculations is the cancelation of significant digits due to the \emph{subtraction of nearly equal numbers}. Reformulation can avoid such a problem.
\begin{example}
The two roots of $ax^2+bx+c=0$ are given by:
\[
x_1=\frac{-b+\sqrt{b^2-4ac}}{2a},
\quad
x_2=\frac{-b-\sqrt{b^2-4ac}}{2a}.
\]
For the equation $x^2+62.10x+1=0$, the two roots are
\[
\begin{array}{ll}
x_1= 0.01610723,
&
x_2= 62.08390.
\end{array}
\]
In this equation, the numerator of $x_1$ is a subtraction of two nearly equal numbers. If using 4-digit rounding we derive
\[
fl(x_1)=\frac{-62.10+62.06}{2.000}=-0.02000\implies
\mbox{relative error}=0.24\times 10^0.
\]
To improve the accuracy of calculation, we should change the formula by \emph{rationalizing the numerator}:
\[
x_1=\frac{-2c}{b+\sqrt{b^2-4ac}}
\]
It follows that
\[
fl(x_1)=\frac{-2.000}{62.10+62.06}=-0.01610\implies
\mbox{relative error}=6.2\times 10^{-4}.
\]
\end{example}
\paragraph{Rewrite polynomials into nested form}
For example, evaluating $f(x)=x^3-6.1x^2+3.2x+1.5$ with $x=4.71$ directly will result in large relative errors. To improve the calculation, we change the polymomial to the following nested form:
\[
f(x)=((x-6.1)x+3.2)x+1.5,
\]
which results in small relative erros.
\begin{quotation}
Polynomials should always be expressed in nested form before evaluation, since this form will minimize the number of arithmetic calculations
\end{quotation}

\paragraph{Avoid large numbers eat small numbers}
We should add those numbers with small magnitude first when we do the summation
\[
fl\left(\sum_{i=1}^nfl(x_i)\right).
\]
\begin{example}
For $x_1=100.5$ and $x_i=0.01, i=2,\dots,101$, using 4-digit arithmetic, we have $fl(x_1)=100.5$ and $fl(x_i)=0.0100, i=2,\dots,101$.
\begin{itemize}
\item
For doing the summation $s=fl\left(\sum_{i=1}^{101}fl(x_i)\right)$, since $100.5+0.0100=100.5100=100.5$, finally we have $s=100.5$
\item
If we define $x_i=0.01, i=1,\dots,100$ and $x_{101}=100.5$, then we have $s=101.5$, which is the true value of the summation.
\end{itemize}
\end{example}





\subsection{Convergence and Stability}
\begin{definition}[Convergence]
Suppose a sequence $\{\beta_n\}_{n=1}^\infty$ is known to converge to zero, and $\{\alpha_n\}_{n=1}^\infty$ converges to a number $\alpha$. If there exists $K>0$ such that
\[
\begin{array}{ll}
|\alpha_n - \alpha|\le K|\beta_n|
&
\mbox{for large $n$,}
\end{array}
\]
then $\{\alpha_n\}_{n=1}^\infty$ is said to converge to $\alpha$ with rate of convergence $\mathcal{O}(\beta_n)$, which is denoted as
\[
\alpha_n = \alpha + \mathcal{O}(\beta_n).
\]
\end{definition}











