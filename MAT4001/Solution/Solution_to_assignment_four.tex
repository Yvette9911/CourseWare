\subsection{Solution to Assignment Four}
\begin{enumerate}
%%%%%%q1%%%%%%%%%
\item
\begin{proof}[Solution.]
\begin{enumerate}
%$$$$$parta
\item
\[
\begin{bmatrix}
1&2&3&1&-3\\2&5&5&4&9\\3&7&8&5&6
\end{bmatrix}\xLongrightarrow[\text{Add $(-3)\times$Row 1 to Row 3}]{\text{Add $(-2)\times$Row 1 to Row 2}}
\begin{bmatrix}
1&2&3&1&-3\\0&1&-1&2&15\\0&1&-1&2&15
\end{bmatrix}\xLongrightarrow{\text{Add $(-1)\times$Row 2 to Row 3}}
\]
\[
\begin{bmatrix}
1&2&3&1&-3\\0&1&-1&2&15\\0&0&0&0&0
\end{bmatrix}
\xLongrightarrow{\text{Add $(-2)\times$Row 2 to Row 1}}
\begin{bmatrix}
1&0&5&-3&-33\\0&1&-1&2&15\\0&0&0&0&0
\end{bmatrix}\text{(rref)}
\]
%part b
\item
We write $\bm{Ax} = \bm b$ in argumented matrix form:
\[
\left[\begin{array}{ccccc|c}
1&2&3&1&-3&1\\2&5&5&4&9&1\\3&7&8&5&6&2
\end{array}\right]
\]
We convert $\bm A$ into $\bm U$(rref):
\[
\left[\begin{array}{ccccc|c}
1&0&5&-3&-33&3\\0&1&-1&2&15&-1\\0&0&0&0&0&0
\end{array}\right]
\]
Hence we only need to solve
\[
\left\{\begin{aligned}
x_1+5x_3-3x_4-33x_5&=3\\
x_2-x_3+2x_4+15x_5&=-1
\end{aligned}\right.
\implies
\left\{\begin{aligned}
x_1&=3-5x_3+3x_4+33x_5\\
x_2&=-1+x_3-2x_4-15x_5
\end{aligned}\right.
\]
Hence all solutions is given by
\[
\bm x = \begin{pmatrix}
x_1\\x_2\\x_3\\x_4\\x_5
\end{pmatrix} = \begin{pmatrix}
3-5x_3+3x_4+33x_5\\-1+x_3-2x_4-15x_5\\x_3\\x_4\\x_5
\end{pmatrix} = \begin{pmatrix}
3\\-1\\0\\0\\0
\end{pmatrix}+
x_3\begin{pmatrix}
-5\\1\\1\\0\\0
\end{pmatrix}+
x_4\begin{pmatrix}
3\\-2\\0\\1\\0
\end{pmatrix}+x_5\begin{pmatrix}
33\\-15\\0\\0\\1
\end{pmatrix}
\]
where $x_3,x_4,x_5$ can be taken arbitrarily.
%%%part c
\item
We write $\bm{Ax} = \bm b$ in argumented matrix form:
\[
\left[\begin{array}{ccccc|c}
1&2&3&1&-3&b_1\\2&5&5&4&9&b_2\\3&7&8&5&6&b_3
\end{array}\right]
\]
We convert $\bm A$ into $\bm U$(rref):
\[
\left[\begin{array}{ccccc|c}
1&0&5&-3&-33&4b_1-b_2
\\0&1&-1&2&15&-2b_1+b_2
\\0&0&0&0&0&-b_1-b_2+b_3
\end{array}\right]
\]
\begin{itemize}
\item
When $-b_1-b_2+b_3\ne 0$, there is \emph{no solution.}
\item
When $-b_1-b_2+b_3=0$, we only need to solve 
\[
\left\{\begin{aligned}
x_1+5x_3-3x_4-33x_5&=5b_1-2b_2\\
x_2-x_3+2x_4+15x_5&=-2b_1+b_2
\end{aligned}\right.
\implies
\left\{\begin{aligned}
x_1&=4b_1-b_2-5x_3+3x_4+33x_5\\
x_2&=-2b_1+b_2+x_3-2x_4-15x_5
\end{aligned}\right.
\]
Hence all solutions is given by
\[
\bm x = \begin{pmatrix}
x_1\\x_2\\x_3\\x_4\\x_5
\end{pmatrix} =
\begin{pmatrix}
4b_1-b_2-5x_3+3x_4+33x_5\\-2b_1+b_2+x_3-2x_4-15x_5\\x_3\\x_4\\x_5
\end{pmatrix}=\begin{pmatrix}
4b_1-b_2\\-2b_1+b_2\\0\\0\\0
\end{pmatrix}+x_3\begin{pmatrix}
-5\\1\\1\\0\\0
\end{pmatrix}+x_4\begin{pmatrix}
3\\-2\\0\\1\\0
\end{pmatrix}+x_5\begin{pmatrix}
33\\-15\\0\\0\\1
\end{pmatrix}
\]
\end{itemize}
\end{enumerate}
\end{proof}
%q2
\item
\begin{proof}
\begin{enumerate}
%part a
\item
We set $v_1=\begin{pmatrix}
1\\-2\\2
\end{pmatrix},v_2=\begin{pmatrix}
2\\-2\\4
\end{pmatrix},v_3=\begin{pmatrix}
-3\\3\\6
\end{pmatrix}.$ Then we claim that $\dim(\Span\{v_1,v_2,v_3\}) = 3$. Hence we only need to show that $v_1,v_2,v_3$ forms the basis for $\Span \{v_1,v_2,v_3\}.$ Hence we only need to show they are ind. Thus we only need to show $\bm{Ax} = \begin{bmatrix}
v_1&v_2&v_3
\end{bmatrix}x = \bm 0$ has unique solution. Thus we only need to show $\bm A = \begin{bmatrix}
v_1&v_2&v_3
\end{bmatrix}$ is invertible:
\[
\bm A = \begin{bmatrix}
1&2&-3\\-2&-2&3\\2&4&6
\end{bmatrix}\xLongrightarrow[\text{Add $(-2)\times$Row 1 to Row 3}]{\text{Add $2\times$Row 1 to Row 2}}
\begin{bmatrix}
1&2&-3\\0&2&-3\\0&0&12
\end{bmatrix}\xLongrightarrow[\text{Row 3$\times\frac{1}{12}$}]{\text{Row 2$\times\frac{1}{2}$}}
\begin{bmatrix}
1&2&-3\\0&1&-\frac{3}{2}\\0&0&1
\end{bmatrix}(\text{rref})
\]
Hence $\rank(\bm A) = 3$. Thus $\bm A$ is full rank, which means $\bm A$ is invertible.
\item
We do elimination to convert $\bm A$ into its rref form:
\[
\begin{bmatrix}
1&-2&3&2\\-1&2&-2&-1\\2&-4&5&3
\end{bmatrix}\xLongrightarrow[\text{Add $(-2)\times$Row 1 to Row 3}]{\text{Add $1\times$Row 1 to Row 2}}
\begin{bmatrix}
1&-2&3&2\\0&0&1&1\\0&0&-1&-1
\end{bmatrix}
\]
\[
\xLongrightarrow[\text{Add $(-3)\times$Row 2 to Row 3}]{\text{Add $1\times$Row 2 to Row 3}}
\begin{bmatrix}
1&-2&0&-1\\0&0&1&1\\0&0&0&0
\end{bmatrix}\text{(rref)}
\]
Hence $\rank(\bm A)=\dim(\col(\bm A)) =2$. Hence dimension of $\col(\bm A)$ is 2.
%%part c
\item
We convert $\bm B$ into rref:
\[
\bm B=\begin{bmatrix}
1&3&2\\2&1&4\\4&7&8
\end{bmatrix}\implies
\bm R = \begin{bmatrix}
1&0&2\\0&1&0\\0&0&0
\end{bmatrix}\text{(rref)}
\]
Thus we only need to compute the solution to $\bm{Ux} = \bm 0$.\\
If $x_3=1$, then $x_1=-2,x_2=0$.\\
Hence the basis for $N(\bm R)$ is $\begin{pmatrix}
-2\\0\\1
\end{pmatrix}$. Hence $\dim(N(\bm B)) = \dim(N(\bm R))=1.$
%%part d
\item
The linear combination of $(x-2)(x+2),x^2(x^4-2),x^6-8$ is given by:
\[
m_1(x-2)(x+2)+m_2x^2(x^4-2)+m_3(x^6-8) = (m_2+m_3)x^6 + (m_1-2m_2)x^2+(-4m_1-8m_3)
\]
where $m_1,m_2,m_3\in\mathbb{R}$.\\
\begin{itemize}
\item
Firstly we show $\{x^4-4,x^6-8\}$ span the space $\Span\{(x-2)(x+2),x^2(x^4-2),x^6-8\}$:\\
Given any vector \[(m_2+m_3)x^6 + (m_1-2m_2)x^2+(-4m_1-8m_3)\in\Span\{(x-2)(x+2),x^2(x^4-2),x^6-8\}\]  for $\forall m_1,m_2,m_3\in\mathbb{R}$,\\ we construct $a_1 = m_2+m_3,a_2=m_1-2m_2$. Then the linear combination of $x^6-8$ and $x^4-4$ with coefficient $a_1,a_2$ is exactly
\[
a_2(x^4-4)+a_1(x^6-8)=(m_2+m_3)x^6 + (m_1-2m_2)x^2+(-4m_1-8m_3)
\]
Hence 
\[
(m_2+m_3)x^6 + (m_1-2m_2)x^2+(-4m_1-8m_3)\in\Span\{x^4-4,x^6-8\}
\]
\[\implies \Span\{(x-2)(x+2),x^2(x^4-2),x^6-8\}\subset\Span\{x^4-4,x^6-8\}\]
Conversely, by setting $m_1=2a_1+a_2,m_2=a_1,m_3=0$ we can show $\Span\{x^4-4,x^6-8\}\subset\Span\{(x-2)(x+2),x^2(x^4-2),x^6-8\}$.\\
Hence $\Span\{x^4-4,x^6-8\}=\Span\{(x-2)(x+2),x^2(x^4-2),x^6-8\}$
\end{itemize}
Then we show $x^4-4,x^6-8$ are ind.:\\
\[\text{Given }
a_1(x^4-4)+a_2(x^6-8) =0\implies
a_2x^6+a_1x^4+(-4a_1-8a_2)=0
\]
\[
\implies \left\{\begin{aligned}
a_2=0\\a_1=0\\-4a_1-8a_2=0
\end{aligned}\right.
\implies \left\{\begin{aligned}
a_1=0\\a_2=0
\end{aligned}\right.
\]
Hence $x^4-4,x^6-8$ are ind. They form the basis for the space $\Span\{(x-2)(x+2),x^2(x^4-2),x^6-8\}$.\\
Hence $\dim(\Span\{(x-2)(x+2),x^2(x^4-2),x^6-8\})=2.$
%%part e
\item
Firstly, it's easy to verify that $5$ and $\cos^2x$ are ind.\\
Next, let's show $\Span\{5,\cos^2x\}=\Span\{5,\cos 2x,\cos^2x\}$:\\
Any linear combination of $\{5,\cos 2x,\cos^2x\}$ is given by:
\[
5m_1+m_2\cos 2x+m_3\cos^2x = (2m_2+m_3)\cos^2x+(5m_1-m_2)
\]
where $m_1,m_2,m_3\in\mathbb{R}$.\\
Any linear combination of $\{5,\cos^2x\}$ is given by:
\[
5n_1+n_2\cos^2x
\]
where $n_1,n_2\in\mathbb{R}$.
\begin{itemize}
\item
if we construct $n_1=m_1-\frac{1}{5}m_2,n_2=2m_2+m_3$, then it means any linear combination of $\{5,\cos 2x,\cos^2x\}$ can be expressed in form of $\{5,\cos^2x\}$. \\Hence $\Span\{5,\cos 2x,\cos^2x\}\subset\{5,\cos^2x\}$.
\item
if wr construct $m_1=n_1+\frac{1}{10}n_2,m_2=\frac{1}{2}n_2,m_3=0$, then it means any linear combination of $\{5,\cos^2x\}$ can be expressed in form of $\{5,\cos 2x,\cos^2x\}$.\\
Hence $\Span\{5,\cos^2x\}\subset\{5,\cos 2x,\cos^2x\}$.
\end{itemize}
Hence $\Span\{5,\cos^2x\}=\{5,\cos 2x,\cos^2x\}$. $\{5,\cos^2x\}$ is the basis for $\Span\{5,\cos 2x,\cos^2x\}.$ Hence $\dim(\Span\{5,\cos 2x,\cos^2x\})=2.$
\end{enumerate}
\end{proof}
%%q3
\item
\begin{proof}[Solution.]
\begin{enumerate}
\item
It can have \emph{no} or \emph{infinitely many} solutions.\\
Since $r<m$ and $r<n$, matrix $\bm A$ is not full rank. When reducing $\bm A$ into rref, there must exist row that contains all zero entries. For its augmented matrix which is rref, when the right hand side is zero for the zero row in the left, it has \emph{infinitely many} solutions; when the right hand side is nonzero for the zero rwo in the left, it has \emph{no} solutions.
\item
It has \emph{infinitely many} solutions.\\
Since $r=m$ and $r<n$, $\bm A$ is full rank. Hence $\bm{Ax} = \bm b$ has at least one solutions. Since $\dim(N(\bm A)) = n-r>0$, there exists \emph{infinitely many} solutions for $\bm{Ax} = \bm 0$. Sicne $\bm x_{\text{complete}} = \bm x_p + \bm x_{\text{special}}$, $\bm{Ax} = \bm b$ has \emph{infinitely many} solutions.
\item
It has \emph{no} or \emph{unique} solution.\\
Since $r<m$ and $r=n$, the rref of $\bm A$ must be of the form $\bm R = \begin{bmatrix}
\bm I\\\bm 0
\end{bmatrix}$. If $\bm d$ has nonzero entries for the zero rows in the left side equation, then $\bm{Rx} = \bm d$(And the orignal $\bm{Ax} = \bm b$) has no solution. If $\bm d$ has all zero entries for the zero rows in the left side equation, then $\bm{Rx} = \bm d$(And the orignal $\bm{Ax} = \bm b$) has unique solution.
\end{enumerate}
\end{proof}
%%q4
\item
\begin{proof}
\begin{enumerate}

%%part a
\item
For any given ind. vectors $v_1,v_2,\dots,v_n$, suppose $v$ is the any vector in $\bm V$.\\
\begin{itemize}
\item
Let's show $v_1,v_2,\dots,v_n,v$ must be dep:\\
It suffices to show $c_1v_1+\dots+c_nv_n+c_{n+1}v=\bm 0$ has nontrival solution for $c_1,\dots,c_{n+1}\in\mathbb{R}$.
\[
\Longleftrightarrow \bm{Ax} = \bm 0\text{ has nontrival solution, where $\bm A=\left[\begin{array}{c|c|c|c}
v_1&\dots&v_n&v
\end{array}\right]$}
\]
which is obviously true since $\bm A$ is a $n\times n+1$ matrix $(n<n+1)$
\item
Hence there exists $(c_1,c_2,\dots,c_{n+1})\ne(0,0,\dots,0)$ such that \[c_1v_1+\dots+c_nv_n+c_{n+1}v=\bm 0\]
If $c_{n+1}=0$, then we have $(c_1,c_2,\dots,c_{n})\ne(0,0,\dots,0)$ such that \[c_1v_1+\dots+c_nv_n=\bm 0,\] which contradicts that $v_1,\dots,v_n$ are ind.\\
Hence $c_{n+1}\ne0$. Then any $v\in\bm V$ could be expressed as:
\[
v = -\frac{c_1}{c_{n+1}}v_1-\frac{c_2}{c_{n+1}}v_2-\dots-\frac{c_n}{c_{n+1}}v_n
\]
which means $v_1,v_2,\dots,v_n$ spans $\bm V$. And they are ind. 
\end{itemize}
So they form a basis for $\bm V$.
%%part b
\item
Suppose $v_1\dots,v_n$ spans $\bm V$. We assume that they are dep. Hence there exists $(c_1,c_2,\dots,c_n)\ne(0,0,\dots,0)$ such that 
\[
c_1v_1+c_2v_2+\dots+c_nv_n=\bm 0
\]
WLOG, we set $c_n\ne0$. Hence we could express $v_n$ as:
\[
v_n=-\frac{c_1}{c_n}v_1-\frac{c_2}{c_n}v_2-\dots-\frac{c_{n-1}}{c_n}v_{n-1}
\]
\begin{itemize}
\item
We claim that $v_1,v_2,\dots,v_{n-1}$ still spans $\bm V$:\\
For any vector $v\in\bm V$, since $v_1,\dots,v_n$ spans $\bm V$, $v$ could be expressed in form of $v_1,\dots,v_n$:
\[
v=m_1v_1+\dots+m_nv_n
\]
where $m_1,\dots,m_n\in\mathbb{R}.$\\
Hence it could also  be expressed in form of $v_1,\dots,v_{n-1}$:
\[
\begin{aligned}
v&=m_1v_1+\dots+m_n(-\frac{c_1}{c_n}v_1-\frac{c_2}{c_n}v_2-\dots-\frac{c_{n-1}}{c_n}v_{n-1})\\
&=(m_1-\frac{m_nc_1}{c_n})v_1+(m_2-\frac{m_nc_2}{c_n})v_2-\dots-(m_{n-1}-\frac{m_nc_{n-1}}{c_n})v_{n-1}
\end{aligned}
\]
Hence $v_1.v_2,\dots,v_{n-1}$ still spans $\bm V$.
\item
If $v-_1,v_2,\dots,v_n$ still dep, we continue eliminating vectors until we get ind. vectors, say, $v_1,v_2,\dots,v_k$. Hence $\dim(\bm V) = k<n$. which contradicts $\dim(\bm V)=n$.
\end{itemize}
\end{enumerate}
\end{proof}
%%q5
\item
\begin{proof}
\begin{enumerate}
%%parta
\item
Suppose $u_1+v_1$ is one vector in $\bm{U+V}$ s.t. $u_1\in\bm U,v_1\in\bm V$; $u_2+v_2$ is one vector in $\bm{U+V}$ s.t. $u_2\in\bm U,v_2\in\bm V$.\\ Hence we claim addition and scalar multiplication is still closed under $\bm{U+V}:$
\[
(u_1+v_1)+(u_2+v_2)=(u_1+u_2)+(v_1+v_2)\qquad c(u_1+v_1)=cu_1+cv_1
\]
where c is a scalar.
\begin{itemize}
\item
Since $u_1,u_2\in\bm U$, $u_1+u_2\in\bm U$. Similarly, $v_1+v_2\in\bm V$. 
\\Hence $(u_1+u_2)+(v_1+v_2)=(u_1+v_1)+(u_2+v_2)\in\bm{U+V}$.
\item
Since $u_1\in\bm U$, $cu_1\in\bm U$. Similarly, $cv_1\in\bm U$.\\
Hence $cu_1+cv_1=c(u_1+v_1)\in\bm{U+V}$
\end{itemize}
Hence addition and scalar multiplication is still closed under $\bm{U+V}$. Hence $\bm{U+V}$ is still a subspace of $W.$
%%part b
\item
If $w_1,w_2\in\bm{U\cap V}$, then $w_1,w_2\in\bm U$ and $w_1,w_2\in\bm V$. Thus the linear combintation of $w_1,w_2$ is still in $\bm U$ and $\bm V$:
\[
a_1w_1+a_2w_2\in\bm U\qquad a_1w_1+a_2w_2\in\bm V
\]
where $a_1,a_2$ is a scalar.\\
Hence $a_1w_1+a_2w_2\in\bm{U\cap V}$. Hence $\bm{U\cap V}$ is also a subspace of $\bm W$.
%%part c
\item
$\dim(\bm U)=2$. The set $\{\bm e_1,\bm e_2\}$ is a basis for $\bm U$.\\
$\dim(\bm V)=2$. The set $\{\bm e_2,\bm e_3\}$ is a basis for $\bm V$.\\
$\dim (\bm{U\cap V})=1$. The set $\{\bm e_2\}$ is a basis for $\bm{U\cap V}$.\\
$\dim(\bm U+\bm V)=3$. The set $\{\bm e_1,\bm e_2,\bm e_3\}$ is a basis for $\bm U+\bm V$.
\item
%%part d
Let $\bm U$ and $\bm V$ be subspaces of $\mathbb{R}^{n}$ such that $\bm{U\cap V}=\{\bm 0\}$.\\
If either $\bm U = \{0\}$ or $\bm U=\{0\}$ the result is obvious.\\
Assume that both subspaces are nontrivial with $\dim(\bm U) = m > 0$ and $\dim(\bm V) = n > 0$. \\
Let $\{u_1, . . . , u_m\}$ be a basis for $\bm U$ and let $\{v_1, . . . , v_n\}$ be a basis for $\bm V$. 
These vectors $u_1,u_2,\dots,u_m,v_1,v_2,\dots,v_n$ spans $\bm{U+V}$. \\
\begin{itemize}
\item
We claim that these vectors form a basis for $\bm{U+V}$. It suffices to show they are ind:\\
If we have the condition
\[
c_1u_1+c_2u_2+\dots+c_mu_m+c_{m+1}v_1+\dots+c_{m+n}v_n=\bm 0
\]
where $c_1,\dots,c_{m+n}$ are scalars,\\
if we set $\bm u = c_1u_1+c_2u_2+\dots+c_mu_m$ and $\bm v = c_{m+1}v_1+\dots+c_{m+n}v_n$, then we have 
\[
\bm u + \bm v = \bm 0
\]
Hence $\bm u=-\bm v$. Then $\bm u,\bm v\in\bm U$ and $\bm u,\bm v\in\bm V$. Hence $\bm u,\bm v\in\bm{U\cap V}$.\\
Hence $\bm u,\bm v = \bm 0$ since $\bm{U\cap V} = \{\bm 0\}$. Thus we have
\[\begin{aligned}
c_1u_1+c_2u_2+\dots+c_mu_m&=\bm 0\\
c_{m+1}v_1+c_{m+2}v_2+\dots+c_{m+n}v_n&=\bm 0
\end{aligned}
\]
By the independence of $u_1,\dots,u_m$ and the independence of $v_1,\dots,v_n$ it
follows that
\[
c_1=c_2=\dots=c_{m+n}=0
\]
\item
Thus $\{u_1,u_2,\dots,u_m,v_1,v_2,\dots,v_n\}$ form a basis for $\bm U+\bm V$.
\end{itemize}
Hence $\dim(\bm U+\bm V) = m+n$.
\end{enumerate}
\end{proof}
\item
\begin{proof}
For any vector $\bm y\in\range(\bm{A+B})$, there exists vector $\bm x$ such that 
\[(\bm{A+B})\bm x = \bm y\]
Also, we can express $\bm y$ as sum of vectors in range of $\bm A$ and $\bm B$:
\[
\begin{aligned}
\bm y &=(\bm{A+B})\bm x&=\bm{Ax}+\bm{Bx}
\end{aligned}
\]
Hence we obtain
\[
\range(\bm{A+B})\subset\range(\bm A) + \range(\bm B)
\]
Assume one basis for $\range(\bm A)$ is $\{a_1,\dots,a_s\}$; $\bm B = \left[\begin{array}{c|c|c}
B_1&\dots&B_n
\end{array}\right]$ one basis for $\range(\bm B)$ is $\{b_1,\dots,b_t\}$. Thus we obtain:
\[
\begin{aligned}
\dim(\range(\bm A)+\range(\bm B))&=\dim(a_1,\dots,a_s,b_1,\dots,b_t)\\&\le s+t\\&=\dim(\range(\bm A))+\dim(\range(\bm B))\\&=\rank(\bm A)+\rank(\bm B)
\end{aligned}
\]
Hence we have 
\[
\begin{aligned}
\rank(\bm{A+B}) &= \dim(\range(\bm{A+B}))\\
&\le\dim(\range(\bm A)+\range(\bm B))\\
&\le\rank(\bm A)+\rank(\bm B)
\end{aligned}
\]
\end{proof}
\item
\begin{proof}
\begin{enumerate}
\item
We assume $\bm{A} = \left[\begin{array}{c|c|c}
A_1&\dots&A_n
\end{array}\right]$, $\bm{B} = \left[\begin{array}{c|c|c}
B_1&\dots&B_n
\end{array}\right]\trans$.
\\Hence $\bm{AB}$ could be expressed as:
\[
\bm{AB} = A_1B_1+\dots+A_nB_n
\]
which means every column of $\bm{AB}$ is a linear combination of columns of $\bm A$. Assume one basis for $\col(\bm A)$ is $a_1,\dots,a_s$. Then $\{a_1,\dots,a_s\}$ can also span $\col(\bm{AB})$. 
\\Hence $\rank(\bm{AB})=dim(\col(\bm{AB}))\le\dim(\col(\bm A))=\rank(\bm A)$
\item
We use the conclusion of part(a) to derive this statement:\\
If $\rank(\bm B)=n$, then $\bm B$ is invertible, $\bm A = \bm{AB}\bm B^{-1}$.\\
Since product $\bm{AB}$ is a $m\times n$ matrix, $\bm B^{-1}$ is a $n\times n$ matrix, by part(a), $\rank(\bm{AB}\bm B^{-1})\le\rank(\bm{AB})$.\\
In conclusion,
\[
\rank(\bm A) = \rank(\bm{AB}\bm B^{-1})\le\rank(\bm{AB})\le\rank(\bm A)
\]
The equality must be satisfied, hence we have $\rank(\bm{AB})=\rank(\bm A)$.
\end{enumerate}
\end{proof}
\item
\begin{proof}
We assume $\{v_1,\dots,v_{n-1}\}$ form a basis for $\mathbb{R}^{n}$. \\
It is equivalent to $\bm{Ax} = \bm b$ must have a solution $\forall \bm b\in\mathbb{R}^{n}$ and $\bm A = \left[\begin{array}{c|c|c}
v_1&\dots&v_{n-1}
\end{array}\right]$.
However, since $\bm A$ is $n\times (n-1)$ matrix, the number of equations is greater than number of unknowns, this system may not have a solution, which forms a contradiction!
\end{proof}
\end{enumerate}