\subsection{Midterm Exam Solution}
\begin{enumerate}
%q1
\item
\begin{enumerate}
\item
We can write this system as:
\[
\left\{\begin{aligned}
x-y+3z&=1\\
2x+y&=5\\
-x-5y+9z&=-7
\end{aligned}\right.
\]
We can convert it into matrix form:
\[
\begin{bmatrix}
1&-1&3\\
2&1&0\\
-1&-5&9
\end{bmatrix}\begin{bmatrix}
x\\y\\z
\end{bmatrix}=\begin{bmatrix}
1\\5\\-7
\end{bmatrix}.
\]
\item
The \textit{augmented matrix} is given by:
\[
\left[\begin{array}{@{}ccc|c@{}}
1&-1&3&1\\
2&1&0&5\\
-1&-5&9&-7
\end{array}\right]
\]
And we perform \textit{row transformation} on this matrix:
\[
\left[\begin{array}{@{}ccc|c@{}}
1&-1&3&1\\
2&1&0&5\\
-1&-5&9&-7
\end{array}\right]
\xLongrightarrow[\text{Row 3}=\text{Row 3}+\text{Row 1}]{\text{Row 2}=\text{Row 2}-2\x\text{Row 1}}
\left[\begin{array}{@{}ccc|c@{}}
1&-1&3&1\\
0&3&-6&3\\
0&-6&12&-6
\end{array}\right]
\xLongrightarrow{\text{Row 3}=\text{Row 3}+2\x\text{Row 2}}
\]
\[
\left[\begin{array}{@{}ccc|c@{}}
1&-1&3&1\\
0&3&-6&3\\
0&0&0&0
\end{array}\right]
\xLongrightarrow{\text{Row 1}=\text{Row 1}+\frac{1}{3}\x\text{Row 2}}
\left[\begin{array}{@{}ccc|c@{}}
1&0&1&2\\
0&3&-6&3\\
0&0&0&0
\end{array}\right]
\xLongrightarrow{\text{Row 2}=\text{Row 2}\x\frac{1}{3}}
\]
\[
\qquad\qquad\left[\begin{array}{@{}ccc|c@{}}
1&0&1&2\\
0&1&-2&1\\
0&0&0&0
\end{array}\right](\text{rref})
\]
The reduced row echelon form of the augmented matrix for this system is
\[
\left[\begin{array}{@{}ccc|c@{}}
1&0&1&2\\
0&1&-2&1\\
0&0&0&0
\end{array}\right].
\]
\item
We convert this system into:
\[
\left\{
\begin{aligned}
x+z&=2\\
y-2z&=1
\end{aligned}
\right.
\implies
\left\{
\begin{aligned}
x&=2-z\\
y&=1+2z
\end{aligned}
\right.
\]
Hence the complete set of solutions is given by
\[
\bm x_{\text{complete}}=\begin{pmatrix}
2-z\\1+2z\\z
\end{pmatrix}=\begin{pmatrix}
2\\1\\0
\end{pmatrix}+z\begin{pmatrix}
-1\\2\\1
\end{pmatrix}.
\]
\item
\[
\bm A=\begin{bmatrix}
1&-1&3\\2&1&0\\-1&-5&9
\end{bmatrix}
\]
From part $(b)$, we know that $\bm A$ is \textit{singular}. Hence $\bm A^{-1}$ doesn't exist.
\item
From part $(b)$, we know that $\bm A$ has $2$ \textit{pivot variables.} Hence $\rank(\bm A)=2.$
\end{enumerate}
\item
\begin{enumerate}
\item
The \textit{coefficient matrix} for this equation is given by:
\[
\begin{bmatrix}
2&-1&3&0
\end{bmatrix}
\]
Hence $x_1$ is \textit{pivot variable}, $x_2,x_3,x_4$ are \textit{free variables.}\\
Moreover, $2x_1-x_2+3x_3=0\implies x_1=\frac{x_2-3x_3}{2}.$\\
Hence the complete set of solutions is given by
\[
\bm x_{\text{complete}}=\begin{pmatrix}
\frac{x_2-3x_3}{2}\\x_2\\x_3\\x_4
\end{pmatrix}
=x_2\begin{pmatrix}
\frac{1}{2}\\1\\0\\0
\end{pmatrix}+x_3\begin{pmatrix}
-\frac{3}{2}\\0\\1\\0
\end{pmatrix}+x_4\begin{pmatrix}
0\\0\\0\\1
\end{pmatrix}.
\]
\item
Obviously, the three vectors $\begin{pmatrix}
\frac{1}{2}\\1\\0\\0
\end{pmatrix},\begin{pmatrix}
-\frac{3}{2}\\0\\1\\0
\end{pmatrix},\begin{pmatrix}
0\\0\\0\\1
\end{pmatrix}$ are ind.\\
Hence one basis for $\bm V$ is $\left\{\begin{pmatrix}
\frac{1}{2}\\1\\0\\0
\end{pmatrix},\begin{pmatrix}
-\frac{3}{2}\\0\\1\\0
\end{pmatrix},\begin{pmatrix}
0\\0\\0\\1
\end{pmatrix}\right\}.$\\
Hence $\dim(\bm V)=3.$
\item
The columns of $\bm A$ form a basis for $\bm A$.\\
Hence one matrix $\bm A$ is given by:
\[
\bm A=\begin{bmatrix}
\frac{1}{2}&-\frac{3}{2}&0\\
1&0&0\\
0&1&0\\
0&0&1
\end{bmatrix}.
\]
\item
We only need to find $\bm B$ such that
\[
\bm{Bx}=\bm 0\quad\text{where $\bm x=\begin{pmatrix}
x_1\\x_2\\x_3\\x_4
\end{pmatrix}.$}
\]
Thus one possible matrix is $\bm B=\begin{bmatrix}
4&-2&6&0
\end{bmatrix}.$\\
In this case, $\bm{Bx}=2(2x_1-x_2+3x_3)=0.$
\end{enumerate}
\item
\begin{enumerate}
\item
$\bm B=\begin{bmatrix}
2&0&0\\0&2&0\\0&0&2
\end{bmatrix}$.\\
\textbf{Verify: }In this case, $\bm B=2\bm I$.\\
Thus $\bm{BA}=2\bm{IA}=2\bm A.$ for every $\bm A.$
\item
$\bm B=\begin{bmatrix}
0&0&0\\0&0&0\\0&0&0
\end{bmatrix}$.\\
\textbf{Verify: }In this case, $\bm{BA}=\bm0\bm A=\bm 0;2\bm B=\bm0.$\\
Hence $\bm{BA}=2\bm B$ for every $\bm A$.
\item
$\bm B=\begin{bmatrix}
0&0&1\\0&1&0\\1&0&0
\end{bmatrix}$.\\
\textbf{Verify: }In this case, $\bm B$ is an \textit{elementary matrix}. It interchanges the first and the last rows of $\bm A.$
\item
Such $\bm B$ doesn't exist.\\
\textbf{Reason: }Suppose $\bm A=\begin{bmatrix}
a&b&c\\d&e&f\\g&h&i
\end{bmatrix}$, then $\bm{BA}=\begin{bmatrix}
c&b&a\\f&e&d\\i&h&g
\end{bmatrix}.$\\
However, if the first row of $\bm B$ is $\begin{bmatrix}
\alpha_1&\alpha_2&\alpha_3
\end{bmatrix}$, then the $(1,1)$th entry of $\bm{BA}$ is
\[
\alpha_1a+\alpha_2d+\alpha_3g,
\]
which makes it impossible to equal to $c$.\\
Hence such $\bm B$ doesn't exist.
\end{enumerate}
\item
\begin{enumerate}
%q a
\item
%first part of a
\begin{enumerate}
\item
\begin{itemize}
\item
\textit{Sufficiency.} If there exists an $n\x m$ matrix $\bm C$ such that $\bm{AC}=\bm I_m$, then for $\forall\bm b\in\mathbb{R}^{m}$ we obtain:
\[
\bm{AC}\bm b=\bm I_m\bm b=\bm b.
\]
If we set $\bm x_0=\bm{Cb}$, then we derive $\bm A\bm x_0=\bm b.$ Hence $\bm x_0$ is one solution to $\bm{Ax}=\bm b$, which means $\bm{Ax}=\bm b$ has at least one solution for $\forall\bm b\in\mathbb{R}^m.$
\item
\textit{Necessity.} If $\bm{Ax}=\bm b$ has at least one solution for $\forall\bm b\in\mathbb{R}^m$, then we construct $\bm b=\bm e_i$ for $i=1,2,\dots,m.$\\
For $\forall i\in\{1,2,\dots,m\}$, there exists $\bm x_i$ such that $\bm A\bm x_i=\bm e_i$.\\
Thus we construct $\bm C=\left[\begin{array}{@{}c|c|c|c@{}}
\bm x_1&\bm x_2&\cdots&\bm x_m
\end{array}\right]$. $\bm C$ is an $n\x m$ matrix and 
\begin{align*}
\bm{AC}&=\bm A\left[\begin{array}{@{}c|c|c|c@{}}
\bm x_1&\bm x_2&\cdots&\bm x_m
\end{array}\right]\\&=\left[\begin{array}{@{}c|c|c|c@{}}
\bm A\bm x_1&\bm A\bm x_2&\cdots&\bm A\bm x_m
\end{array}\right]
\\&=\left[\begin{array}{@{}c|c|c|c@{}}
\bm e_1&\bm e_2&\cdots&\bm e_m
\end{array}\right]=\bm I.
\end{align*}
Thus $\bm C$ is \textit{right inverse} of $\bm A$.
\end{itemize}
\item
The rank of $\bm A$ is the number of \textit{nonzero} rows in the rref($\bm A$).\\
The linear system $\bm{Ax}=\bm b$ always has solution for $\forall\bm b$. We convert it into \textit{augmented matrix form:}
\[
\left[\begin{array}{@{}c|c@{}}
\bm A&\bm b
\end{array}\right]
\xLongrightarrow{\text{Row transform}}
\left[\begin{array}{@{}c|c@{}}
\text{rref($\bm A$)}&\bm b^*
\end{array}\right]
\]
Once the rref($\bm A$) has zero rows and the corresponding $\bm b^*$ has nonzero entries, this system has no solution. Hence rref($\bm A$) has \emph{no} zero rows.\\
Since $\bm A$ is a $m\x n$ matrix, we have $m$ nonzero rows for $\bm A$.\\
Thus $\rank(\bm A)=m.$
\end{enumerate}
\item
\begin{itemize}
\item
For $1\x 3$ matrix $\bm A=\begin{pmatrix}
1&2&7\pi
\end{pmatrix},$ $\rank(\bm A)=1.$\\
And there exists $\bm x_1=\begin{pmatrix}
1\\0\\0
\end{pmatrix}$ such that $\bm A\bm x_1=\bm e_1$.\\
Hence we construct $\bm C=\begin{bmatrix}
\bm x_1
\end{bmatrix}$. We find that
$
\bm{AC}=\begin{pmatrix}
1&2&7\pi
\end{pmatrix}\begin{pmatrix}
1\\0\\0
\end{pmatrix}=1=\bm I.
$
Hence $\bm C=\begin{pmatrix}
1\\0\\0
\end{pmatrix}$ is the \textit{right inverse} of $\bm A$.
\item
For $3\x 1$ matrix $\bm B=\begin{pmatrix}
1\\2\\7\pi
\end{pmatrix}$, we find $\rank(\bm B)=1\ne 3.$ \\From part $(a)$ we derive $\bm B$ has no \textit{right inverse}.
\end{itemize}
\end{enumerate}
\item
\begin{enumerate}
\item
No, let's raise a counter-example:
\[
\bm A=\begin{bmatrix}
3&1\\5&3
\end{bmatrix}\implies
\rank(\bm A)=2.
\]
\[
\bm A\trans=\begin{bmatrix}
3&5\\1&3
\end{bmatrix}
\implies
\bm A+\bm A\trans=\begin{bmatrix}
6&6\\6&6
\end{bmatrix}
\]
Hence $\rank(\bm A+\bm A\trans)=1\ne2=\rank(\bm A).$
\item
\begin{itemize}
\item
Firstly, we show $N(\bm A)\subset N(\bm A\trans\bm A)$:\\
For any $\bm x_0\in N(\bm A)$, we have $\bm A\bm x_0=\bm 0$. Thus by postmultiplying $\bm A\trans$ we have $\bm A\trans\bm A\bm x_0=\bm 0$. Hence $\bm x_0\in N(\bm A\trans\bm A)$.
\item
Then we show $N(\bm A\trans\bm A)\subset N(\bm A)$:\\
For any $\bm x_0\in N(\bm A\trans\bm A)$, we have $\bm A\trans\bm A\bm x_0=\bm 0$. Thus by postmultiplying $\bm x_0\trans$ we have $\bm x_0\trans\bm A\trans\bm A\bm x_0=\bm 0$, which implies $\lVert \bm A\bm x_0\rVert^2=\bm x_0\trans\bm A\trans\bm A\bm x_0=\bm 0$. Hence $\bm A\bm x_0=\bm 0$. Hence $\bm x_0\in N(\bm A)$.
\end{itemize}
In conclusion, $N(\bm A)=N(\bm A\trans\bm A)$.
\item
\begin{itemize}
\item
Since $\bm A$ is $m\x n$ matrix, then $\rank(\bm A\trans\bm A)+\dim(N(\bm A\trans\bm A))=n=\rank(\bm A)+\dim(N(\bm A)).$
\item
Since $N(\bm A)=N(\bm A\trans\bm A)$, we derive $\dim(N(\bm A\trans\bm A))=\dim(N(\bm A))$.
\end{itemize}
Thus $\rank(\bm A\trans\bm A)=\rank(\bm A).$
\end{enumerate}
\item
\begin{enumerate}
\item
Verify by yourself that the following matrices are \textit{symmetric}:
\begin{align*}
(i)\quad&\bm A^2-\bm B^2\\
(iii)\quad&\bm{ABA}
\end{align*}
\item
There are \textit{infinitely} many solutions.\\
\textbf{Reason: }
\begin{itemize}
\item
Since $\bm A$ is $5\x 8$matrix, $\rank(\bm A)+\dim(N(\bm A))=8\implies \dim(N(\bm A))=3$.\\
Hence this system $\bm{Ax}=\bm b$ has \emph{special solutions}.
\item
Moverover, since $\rank(\bm A)=5$, we have 5 nonzero pivots, which means rref($\bm A$) has no zero rows.\\
Hence this system $\bm{Ax}=\bm b$ always has \emph{particular solution.}
\end{itemize}
In conclusion, there are \textit{infinitely} many solutions.
\item
False.\\
\textbf{Reason: }For example, if we have
\[
\bm A=\begin{bmatrix}
1&0\\0&0
\end{bmatrix}\qquad\bm B=\begin{bmatrix}
0&0\\0&1
\end{bmatrix}
\]
then $\bm A+\bm B=\begin{bmatrix}
1&0\\0&1
\end{bmatrix}$, which is obviously \textit{nonsingular}.
\item
False.\\
\textbf{Reason: }For example, the set of $2\x 2$ matrices with rank no more than $r=1$ is \emph{not} a vector space. Why?\\
$\bm A=\begin{bmatrix}
1&0\\0&0
\end{bmatrix},\bm B=\begin{bmatrix}
0&0\\0&1
\end{bmatrix}$ are both in this set since $\rank(\bm A)+\rank(\bm B)=1$.\\
However, $\bm A+\bm B=\begin{bmatrix}
1&0\\0&1
\end{bmatrix}$ doesn't belong to this set since $\rank(\bm A+\bm B)=2.$
\item
False.\\
\textbf{Reason: }This set doesn't satisfy \textit{vector addition rule} and \textit{scalar multiplication rule.}\\
If $f,g$ are both in this set, then $(f+g)(1)=f(1)+g(1)=2\ne1.$ Hence $f+g$ is not in this set.\\
Similarly, you can verify $\lambda f$ ($\lambda$ is a scalar that not equal to 1) is not in this set.\\
Hence it cannot be a vector space.
\end{enumerate}
\end{enumerate}