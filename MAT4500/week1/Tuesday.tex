
\chapter{Week1}

\section{Tuesday}\index{Tuesday_lecture}
\subsection{Analogs of deterministic differential equations}
\paragraph{Problem 1}
Consider the first order homogeneous ODE
\[
\left\{
\begin{aligned}
\frac{\diff N(t)}{\diff t} &= a(t)N(t)\\
N(0)&=N_0
\end{aligned}
\right.
\]

$N(t)$ is described as the \emph{size} of population at time $t$;
$a(t)$ is the given (deterministic) function describing the \emph{rate} of growth of population at time $t$;
$N_0$ is a given constant.

The question raises: What if $a(t)$ is no longer deterministic, instead $a(t)$ is subject to some random effect, e.g.,
\[
a(t) = r(t)\cdot\mbox{noise},
\mbox{ or }
r(t)+\mbox{noise},
\]
where $r(t)$ is deterministic, and the ``noise'' term is something random. Then how to solve the corresponding differential equation?

\paragraph{Problem 2}
Suppose $Q(t)$ describes the charge at time $t$ in an electricity circuit.
\[
\left\{
\begin{aligned}
LQ''(t)+RQ'(t)+\frac{1}{C}Q(t)&=F(t),\\
Q(0)=Q_0,\quad Q'(0)&=Q_0'
\end{aligned}
\right.
\]

$L$ is described as the \emph{inductance}, $R$ is the \emph{resistance}, $C$ is the \emph{capacity}, and $F(t)$ is the \emph{potential source}.

The Question raises: what if $F(t)$ involves some randomness? e.g.,
\[
F(t)=G(t)+\mbox{noise}
\]
where $G(t)$ is deterministic. How to solve the problem?
\begin{remark}
The differential equations with some coefficients involved randomness are called the stochastic differential equations. Clearly, solutions to SDEs should also involved ``randomness''.
\end{remark}
\subsection{Optimal Stopping}
\paragraph{Problem 3} Suppose someone holds an asset (e.g., stock, house, etc.) He plans to sell it at some future time. Denote $X(t)$ to be the price of the asset at time $t$, satisfying
\[
\frac{\diff X(t)}{\diff t}=rX(t)+\alpha X(t)\cdot\mbox{noise}
\]
where $r,\alpha$ are given constants. Our goal is to choose time $\tau$ to solve
\[
\max_{\tau\ge0}\mathbb{E}X(\tau)
\]
where the optimal solution $\tau^*$ is the optimal stopping time.

\subsection{Stochastic Control}
\paragraph{Problem 4 (Portfolio Selection)}
Suppose a person wants to invest into i) a riskless/safe asset (e.g., bond); or ii) a risky asset (e.g, stock).

The price of the saft asset $X_0(t)$ satisfies 
\[
\frac{\diff X_0(t)}{\diff t}=\rho X_0(t),
\]
where $\rho>0$ is a given constant. Therefore, $X(t)$ is exponentially growing function.

The price of risky assrt $X_1(t)$ satisfies
\[
\frac{\diff X_1(t)}{\diff t}=\mu X_1(t)+\sigma X_1(t)\cdot\mbox{noise}
\]
where $\mu,\sigma>0$ are the given constants.

Suppose $u(t)$ is the fraction of his wealth to be invested into the risky asset; the remaining $1-u(t)$ part to be invested into the saft asset. 
The wealth at time $t$ is denoted to be $v(t)$. 
Suppose the person has the utility function $U(\cdot)$.
The terminal time is $T$.
The objective function is
\[
\max_{u(t),0\le t\le T}\mathbb{E}[U(v^u(T))]
\]

If we impose no-shot selling constant, we further require
\[
0\le u(t)\le 1,\forall t\in[0,T]
\]

\paragraph{Problem 5 (Option Pricing)}
Suppose at time $0$, a person in the long position in an European call option has the right to buy the asset at a specified price $K$ at some future time $t$. How much the person should pay to the short position for the option? We can model this problem by Black-Sholes Formula.









