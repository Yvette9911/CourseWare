\section{Thursday}\index{Thursday_lecture}
\subsection{Reviewing for Probability Space}
A probability space is a triplet $(\Omega,\mathcal{F},\mathbb{P})$ defined as follows:
\begin{enumerate}
\item
$\Omega$: denotes the probability sample space. A point $w\in\Omega$ is called a sample point
\item
$\mathcal{F}$: A $\sigma$-algebra $\mathcal{F}$ on $\Omega$ contains a family of events. Each event $F\in\mathcal{F}$ is an $\mathcal{F}$-measurable subset of $\Omega$.
\begin{example}
Let the probability space $\Omega=\{0,1,2,3\}$, then one $\mathcal{F}$ can be $\{F_1,F_2,F_3\}$, where
\[
\begin{array}{lll}
F_1=\{0,1\},
&
F_2=\emptyset,
&
F_3=\{2,3\}
\end{array}
\]
\end{example}
\begin{definition}[$\sigma$-algebra]
A $\sigma$-algebra $\mathcal{F}$ on $\Omega$ is a family of subsets of $\Omega$ such that:
\begin{enumerate}
\item
$\emptyset\in\mathcal{F}$
\item
If $F\in\mathcal{F}$, then $F^c\in\mathcal{F}$
\item
If $A_n\in\mathcal{F}$ for $n\ge1$, then $\bigcup_{n=1}^\infty A_n\in\mathcal{F}$.
\end{enumerate}
\end{definition}
\item
$\mathbb{P}$ denotes a probability measure, which is a function $\mathcal{F}\to[0,1]$ such that:
\begin{enumerate}
\item
$\mathbb{P}(\emptyset)=0,\mathbb{P}(\Omega)=1$.
\item
$\mathbb{P}$ is countably addictive, i.e., if $A_n\in\mathcal{F}$ is a countable sequence of disjoint sets, then
\[
\mathbb{P}\left(\bigcup_{n=1}^\infty A_n\right)=\sum_{n=1}^\infty\mathbb{P}(A_n)
\]
\end{enumerate}
\begin{definition}[Almost Surely True]
A statement $S$ is said to be \emph{almost surely true} (a.s. with probability 1), if
\begin{enumerate}
\item
$F:=\{w\mid S(w)\mbox{is true}\}\in\mathcal{F}$
\item
$\mathbb{P}(F)=1$.
\end{enumerate}
\end{definition}
\begin{definition}[Borel $\sigma$-Algebra]
Let $\mathcal{U}$ be a collection of all open sets in a topological space $\Omega$ (e.g., $\Omega=\mathbb{R}^n$), then $\mathcal{B}(\Omega)$ denotes the \emph{smallest} $\sigma$-algebra that contains $\mathcal{U}$, which is called \emph{Borel $\sigma$-Algebra} on $\Omega$. The element $B\in\mathcal{B}(\Omega)$ is \emph{Borel subset}.
\end{definition}
\begin{remark}
We usually use the notation $\mathcal{B}$ to denote $\mathcal{B}(\mathbb{R}^n)$. Here $\mathcal{B}$ contains all the open sets, all the closed sets, and all the countable unions of such sets, as well as the countable intersection of such sets.
\end{remark}
\begin{definition}[$\mathcal{F}$-Measurable / Random Variable]
\begin{enumerate}
\item
A function $f:\Omega\to\mathbb{R}^n$ is called \emph{$\mathcal{F}$-measurable} if
\[
f^{-1}(\bm B)=\{w\mid f(w)\in\mathcal{B}\}\in\mathcal{F}
\]
for any $\bm B\in\mathcal{B}$.
\item
A random variable $X$ is a function $X:\Omega\to\mathbb{R}^n$ and is $\mathcal{F}$-measurable.
\end{enumerate}
\end{definition}
\begin{definition}[Generated $\sigma$-Algebra]
Suppose $X$ is a random variable on $(\Omega,\mathcal{F},\mathbb{P})$. Then the $\sigma$-algebra generated by $X$, say $\mathcal{H}_X$ is defined to be the \emph{smallest $\sigma$-algebra} on $\Omega$ containing $X^{-1}(U)$, where $U\subseteq\mathbb{R}^n$ is any open set.
\end{definition}
\begin{definition}[Distribution]
A probability measure $\mu_X$ on $\mathbb{R}^n$ induced by the random variabe $X$ isdefined as
\[
\mu_X(\bm B)=\mathbb{P}(X^{-1}(\bm B)),
\]
where $\bm B\in\mathcal{B}$. The $\mu_X$ is called the \emph{distribution} of $X$.
\end{definition}
\begin{definition}[Integrable]
The random variable $X$ is \emph{integrable}, denoted by $X\in\mathcal{L}^1(\Omega,\mathcal{F},\mathbb{P})$ ($X\in\mathcal{L}^1$), if 
\[
\int_\Omega|X(w)|\diff\mathbb{P}(w)<\infty.
\]
Then $\mathbb{E}X:=\int_\Omega|X(w)|\diff\mathbb{P}(w)=\int_{\mathbb{R}^n}X\diff\mu_X(x)$ is called the \emph{expectation} of $X$ (w.r.t. $\mathbb{P}$). 
\end{definition}
\begin{definition}[$L^p$ space]
Suppose $X$ is a random variable and $p\ge1$.
\begin{itemize}
\item
Define $L^p$-norm of $X$ as
\[
\|X\|_p=\left(\int_\Omega|X|^p\diff\mathbb{P}\right)^{1/p}
\]
If $p=\infty$, define
\[
\|X\|_\infty=\inf\{N\in\mathbb{R}\mid|X(w)|\le N,\text{ a.s.}\}
\]
\item
A random variable $X$ is in the $L^p$ space ($p$-th integrable) if
\[
\int_\Omega|X|^p\diff\mathbb{P}<\infty,
\]
denoted as $X\in\mathcal{L}^p(\Omega,\mathcal{F},\mathbb{P})$.
\end{itemize}
\end{definition}
\begin{proposition}
If $p\ge q$, then $\|X\|_q\le\|X\|_p$, and $\mathcal{L}^p(\Omega,\mathcal{F},\mathbb{P})\subseteq\mathcal{L}^q(\Omega,\mathcal{F},\mathbb{P})$
\end{proposition}
\begin{proof}
The inequality is shown by using Holder's inequality:
\[
\|X\|_q^q=\int_\Omega|X|^q\diff\mathbb{P}
\le
\left(\int_\Omega(|X|^q)^{p/q}\diff\mathbb{P}\right)^{q/p}
=
\left(
\int_\Omega|X|^p\diff\mathbb{P}
\right)^{\frac{1}{p}\cdot q}
=
\|X\|_p^q
\]
\end{proof}
\begin{definition}[Independence]
\begin{enumerate}
\item
Two events $A_1,A_2\in\mathcal{F}$ are said to be \emph{independent} if
\[
\mathbb{P}(A_1\bigcap A_2)=\mathbb{P}(A_1)\mathbb{P}(A_2)
\]
\item
Two $\sigma$-algebras $\mathcal{F}_1,\mathcal{F}_2$ are said to be \emph{independent} if $F_1,F_2$ are independent events for $\forall F_1\in\mathcal{F}_1,F_2\in\mathcal{F}_2$
\item
Two random variables $X,Y$ are said to be \emph{independent} if $\mathcal{H}_X,\mathcal{H}_Y$, the $\sigma$-algebra generated by $X$ and $Y$, respectively, are independent.
\end{enumerate}
\end{definition}


















\end{enumerate}