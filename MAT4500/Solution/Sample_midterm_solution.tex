\section{Midterm Exam Solutions}\index{Midterm Exam Solutions}
\subsection{Sample Exam Solution}
\begin{enumerate}
%q1
\item
\begin{enumerate}
\item
\[
\bm A=\begin{bmatrix}
1&1&c&1\\
0&-1&1&2\\
1&2&1&-1
\end{bmatrix}
\]
\item
The \textit{augmented matrix} is given by
\[
\left[\begin{array}{@{}cccc|c@{}}
1&1&c&1&c\\
0&-1&1&2&0\\
1&2&1&-1&-c
\end{array}\right]
\]
Then we compute its \textit{row-reduced form:}
\[
\left[\begin{array}{@{}cccc|c@{}}
1&1&c&1&c\\
0&-1&1&2&0\\
1&2&1&-1&-c
\end{array}\right]
\xLongrightarrow[\text{Row 3}=\text{Row 3}-\text{Row 1}]{\text{Row 1}=\text{Row 1}+\text{Row 2}}
\left[\begin{array}{@{}cccc|c@{}}
1&0&c+1&3&c\\
0&-1&1&2&0\\
0&1&1-c&-2&-2c
\end{array}\right]
\]
\[
\xLongrightarrow{\text{Row 2}=\text{Row 2}\x(-1)}
\left[\begin{array}{@{}cccc|c@{}}
1&0&c+1&3&c\\
0&1&-1&-2&0\\
0&1&1-c&-2&-2c
\end{array}\right]
\]
\[
\xLongrightarrow{\text{Row 3}=\text{Row 3}-\text{Row 2}}
\left[\begin{array}{@{}cccc|c@{}}
1&0&c+1&3&c\\
0&1&-1&-2&0\\
0&0&2-c&0&-2c
\end{array}\right]
\]
\begin{enumerate}
\item
If $c=2$, then we obtain:
\[
\xLongrightarrow{\text{Row 3}=\text{Row 3}\x(-\frac{1}{4})}
\left[\begin{array}{@{}cccc|c@{}}
1&0&3&3&2\\
0&1&-1&-2&0\\
0&0&0&0&1
\end{array}\right]
\xLongrightarrow{\text{Row 1}=\text{Row 1}-2\x\text{Row 3}}
\]
\[
\qquad\qquad\left[\begin{array}{@{}cccc|c@{}}
1&0&3&3&0\\
0&1&-1&-2&0\\
0&0&0&0&1
\end{array}\right]\text{(rref)}
\]
\item
Otherwise, we derive:
\[
\xLongrightarrow{\text{Row 3}=\text{Row 3}\x(\frac{1}{2-c})}
\left[\begin{array}{@{}cccc|c@{}}
1&0&c+1&3&c\\
0&1&-1&-2&0\\
0&0&1&0&\frac{2c}{c-2}
\end{array}\right]
\xLongrightarrow[\text{Row 2}=\text{Row 2}+\text{Row 3}]{\text{Row 1}=\text{Row 1}-\text{Row 3}\x(c+1)}
\]
\[
\qquad\qquad\left[\begin{array}{@{}cccc|c@{}}
1&0&0&3&-\frac{c^2+4c}{c-2}\\
0&1&0&-2&\frac{2c}{c-2}\\
0&0&1&0&\frac{2c}{c-2}
\end{array}\right](\text{rref})
\]
\end{enumerate}
%\[
%\xLongrightarrow{\text{Row 3}=\text{Row 3}\x\frac{1}{2-c}}
%\left[\begin{array}{@{}cccc|c@{}}
%1&0&c+1&3&c\\
%0&1&-1&-2&0\\
%0&0&1&0&0
%\end{array}\right]
%\]
\item
\begin{enumerate}
\item
If $c=2$, there is no solution to this system.
\item
Otherwise, we convert this system into:
\[
\left\{
\begin{aligned}
x_1+3x_4&=-\frac{c^2+4c}{c-2}\\
x_2-2x_4&=\frac{2c}{c-2}\\
x_3&=\frac{2c}{c-2}
\end{aligned}
\right.
\implies
\left\{
\begin{aligned}
x_1&=-\frac{c^2+4c}{c-2}-3x_4\\
x_2&=\frac{2c}{c-2}+2x_4\\
x_3&=\frac{2c}{c-2}
\end{aligned}
\right.
\]
Hence the complete set of solutions is given by
\[
\bm x_{\text{complete}}=\begin{pmatrix}
-\frac{c^2+4c}{c-2}-3x_4\\\frac{2c}{c-2}+2x_4\\
\frac{2c}{c-2}\\x_4
\end{pmatrix}=\begin{pmatrix}
-\frac{c^2+4c}{c-2}\\
\frac{2c}{c-2}\\
\frac{2c}{c-2}\\
0
\end{pmatrix}+x_4\begin{pmatrix}
-3\\2\\0\\1
\end{pmatrix}.
\]
\end{enumerate}
\item
\begin{enumerate}
\item
If $c=2$, obviously, the rref of $\bm A$ is
\[
\begin{bmatrix}
1&0&3&3\\0&1&-1&-2\\0&0&0&0
\end{bmatrix}
\]
Hence $\rank(\bm A)=2$.
\item
Otherwise, the rref of $\bm A$ is
\[
\begin{bmatrix}
1&0&0&3\\0&1&0&-2\\0&0&1&0
\end{bmatrix}
\]
Hence $\rank(\bm A)=3$.
\end{enumerate}
In conclusion, $\rank(\bm A)=\begin{cases}
3,&c\ne 2;\\
2,&c=2.
\end{cases}$
\item
When $c=0$, the complete solution is given by:
\[
\bm x_{\text{complete}}=x_4\begin{pmatrix}
-3\\2\\0\\1
\end{pmatrix}.
\]
where $x_4$ is a scalar.\\
Hence a basis for the subspace of solutions is $\left\{\begin{pmatrix}
-3\\2\\0\\1
\end{pmatrix}\right\}.$
\end{enumerate}
\item
\begin{itemize}
\item
For \textit{skew symmetric} matrix, once the lower triangular part is determined, the whole matrix is immediately determined. For example, if we know $a_{ij}=m(i>j)$, then the corresponding upper triangular entry is $a_{ji}=-m$. Thus our basis is given by: 
\[
\{\bm A_{ij}\}\text{ for }1\le j\le i\le n.
\]
where the entries $a_{st}$$(1\le s,t\le n)$ for $\bm A_{ij}$ is given by
\[
a_{st}=\left\{
\begin{aligned}
0,&\quad (s,t)\ne(i,j)\text{ and }(s,t)\ne(j,i);\\
1,&\quad (s,t)=(i,j);\\
-1,&\quad (s,t)=(j,i).
\end{aligned}\right.
\]
\item
Notice $ax^2+bx+2a+3b=a(x^2+2)+b(x+3)$. And $(x^2+2)$ and $(x+3)$ are obviously independent. Hence the basis is given by
\[
\{
(x^2+2),(x+3)
\}.
\]
\item
Firstly we show that $(x-1),(x+1),(2x^2-2)$ are independent:
\[
\begin{aligned}
\alpha_1(x-1)+\alpha_2(x+1)&+\alpha_3(2x^2-2)=0
\implies\\
&2\alpha_3x^2+(\alpha_1+\alpha_2)x+(-\alpha_1+\alpha_2-2\alpha-3)=0.
\end{aligned}
\]
Hence we derive
\[
\left\{
\begin{aligned}
2\alpha_3&=0\\\alpha_1+\alpha_2&=0\\-\alpha_1+\alpha_2-2\alpha_3&=0
\end{aligned}
\right.\implies
\left\{
\begin{aligned}
\alpha_1&=0\\\alpha_2&=0\\\alpha_3&=0
\end{aligned}
\right.
\]
which means $(x-1),(x+1),(2x^2-2)$ are independent.\\
Hence one basis for this space is $\{(x-1),(x+1),(2x^2-2)\}$.
\end{itemize}
\item
\begin{enumerate}
\item
Obviously, the entrie of $\bm D$ is
\[
d_{ij}=\left\{
\begin{aligned}
d_{ii},&\quad i=j;\\
0,&\quad i\ne j.
\end{aligned}
\right.
\]
We set $\bm E=\bm{AD},\bm F=\bm{DA}$. Hence the entries for $\bm E$ and $\bm F$ is given by:
\[
e_{ij}=\sum_{t=1}^{n}a_{it}d_{tj}=a_{ij}d_{jj}\qquad
f_{ij}=\sum_{t=1}^{n}d_{it}a_{tj}=d_{ii}a_{ij}
\]
where $1\le i,j\le n.$\\
In order to let $\bm E=\bm F$, we must let $e_{ij}=f_{ij}$ for $\forall 1\le i,j\le n.$
\[
\implies
a_{ij}d_{jj}=d_{ii}a_{ij}
\implies
a_{ij}(d_{jj}-d_{ii})=0.
\]
Since $d_{ii}\ne d_{jj}$ for $\forall i\ne j$, we derive $d_{jj}-d_{ii}\ne0.$ Hence $a_{ij}=0$ for $\forall i\ne j$.\\
Considering the case $i=j$, then $d_{jj}-d_{ii}=d_{ii}-d_{ii}=0.$ Thus the value of $a_{ij}$ is undetermined.\\
In conclusion, $\bm A$ could be any diagonal matrix.
\item
\begin{itemize}
\item
We construct $\bm B^{ij}$ such that the $(i,j)$th entry of $\bm B^{ij}$ is $1$, other entries are
all zero.
\item
We set $\bm A\bm B^{ij}=\bm E^{ij};\bm B^{ij}\bm A=\bm F^{ij}$. Hence the entries for $\bm E^{ij}$ and $\bm F^{ij}$ is given by:
\[
e_{pq}^{ij}=\sum_{t=1}^{n}a_{pt}b_{tq}\qquad
f_{pq}^{ij}=\sum_{t=1}^{n}b_{pt}a_{tq}
\]
where $1\le p,q\le n.$\\
Since $\bm{AB}=\bm{BA}$ is always true for any matrix $\bm B$, we have $\bm A\bm B^{ij}=\bm B^{ij}\bm A.$ Hence $e_{pq}^{ij}=f_{pq}^{ij}.$
\item
For $q\ne i$, we have $e_{iq}^{ii}=\sum_{t=1}^{n}a_{it}b_{tq}=0$ since $b_{tq}=0$ for $\forall t=1,2,\dots,n.$\\
Also, $f_{iq}^{ii}=\sum_{t=1}^{n}b_{it}a_{tq}=a_{iq}.$\\
Hence $0=a_{iq}$ for $\forall q\ne i.$
\item
For $i\ne j$, we have $e_{ij}^{ij}=\sum_{t=1}^{n}a_{it}b_{tj}=a_{ii}b_{ij}=a_{ii}$ and $f_{ij}^{ij}=\sum_{t=1}^{n}b_{it}a_{tj}=b_{ij}a_{jj}=a_{jj}.$\\
Hence $a_{ii}=a_{jj}.$
\end{itemize}
So, $\bm A$ is diagonal and all the diagonal entries of $\bm A$ are equal. Hence $\bm A=c\bm I$ for some scalar $c$.
\end{enumerate}
\item
\begin{enumerate}
\item
\[
\left[
\begin{array}{@{}cc|cc@{}}
5&4&1&0\\
4&5&0&1
\end{array}
\right]
\xLongrightarrow{\text{Row 2}=5\x\text{Row 2}-4\x\text{Row 1}}
\left[
\begin{array}{@{}cc|cc@{}}
5&4&1&0\\
0&9&-4&5
\end{array}
\right]
\]
\[
\xLongrightarrow{\text{Row 1}=9\x\text{Row 1}-4\x\text{Row 2}}
\left[
\begin{array}{@{}cc|cc@{}}
45&0&25&-20\\
0&9&-4&5
\end{array}
\right]
\xLongrightarrow[\text{Row 2}=\frac{1}{9}\x\text{Row 2}]{\text{Row 1}=\frac{1}{45}\x\text{Row 1}}
\]
\[
\qquad\qquad\left[
\begin{array}{@{}cc|cc@{}}
1&0&\frac{5}{9}&-\frac{4}{9}\\
0&1&-\frac{4}{9}&\frac{5}{9}
\end{array}
\right]
\]
Hence the inverse of the matrix $\begin{pmatrix}
5&4\\4&5
\end{pmatrix}$ is 
$
\left[
\begin{array}{@{}cc@{}}
\frac{5}{9}&-\frac{4}{9}\\
-\frac{4}{9}&\frac{5}{9}
\end{array}
\right]
$.
\item
\[
\left[
\begin{array}{@{}cc|cc@{}}
a&b&1&0\\
c&d&0&1
\end{array}
\right]
\xLongrightarrow{\text{Row 2}=a\x\text{Row 2}-c\x\text{Row 1}}
\left[
\begin{array}{@{}cc|cc@{}}
a&b&1&0\\
0&ad-bc&-c&a
\end{array}
\right]
\]
\[
\xLongrightarrow{\text{Row 1}=(ad-bc)\x\text{Row 1}-b\x\text{Row 2}}
\left[
\begin{array}{@{}cc|cc@{}}
a(ad-bc)&0&ad&-ab\\
0&ad-bc&-c&a
\end{array}
\right]
\]
\begin{enumerate}
\item
If $ad-bc=0$, then this process cannot continue, which means the inverse of $\begin{pmatrix}
a&b\\c&d
\end{pmatrix}$ doesn't exist.
\item
If $ad-bc\ne0$, without loss of generality, we assume $a\ne0$.\\ (If $a=0$, then $c$ must be nonzero. Then we only need to set the second row as pivot row to proceed similarly.)
\\Thus we obtain:
\[
\xLongrightarrow[{\text{Row 2}=\frac{1}{ad-bc}\x\text{Row 2}}]{\text{Row 1}=\frac{1}{a(ad-bc)}\x\text{Row 1}}
\left[
\begin{array}{@{}cc|cc@{}}
1&0&\frac{d}{ad-bc}&\frac{-b}{ad-bc}\\
0&1&\frac{-c}{ad-bc}&\frac{a}{ad-bc}
\end{array}
\right]
\]
Hence the inverse of the matrix $\begin{pmatrix}
a&b\\c&d
\end{pmatrix}$ is $\left[
\begin{array}{@{}cc@{}}
\frac{d}{ad-bc}&\frac{-b}{ad-bc}\\
\frac{-c}{ad-bc}&\frac{a}{ad-bc}
\end{array}
\right]$.
\end{enumerate}
\end{enumerate}
\item
\begin{enumerate}
\item
We set $\bm A=\bm I-\bm u\bm u\trans$.
\begin{itemize}
\item
Firstly, we find that $\bm u\in N(\bm A):$
\[
\bm A\bm u=(\bm I-\bm u\bm u\trans)\bm u=\bm u-\bm u(\bm u\trans\bm u)=\bm u-\bm u=\bm0.
\]
Moreover, $c\bm u\in N(\bm A)$, where $c$ is a scalar.\\
Hence any elements that parallel to $\bm u$ is in $N(\bm A).$
\item
Secondly, $\forall x\in N(\bm A),$ we notice:
\[
\bm{Ax}=\bm0\implies
(\bm I-\bm u\bm u\trans)\bm x=\bm x-\bm u\bm u\trans\bm x=\bm0
\implies
\bm x=\bm u(\bm u\trans\bm x).
\]
Since $\bm u\trans\bm x$ is a scalar, $\bm x$ is parallel to $\bm u.$\\
In other words, any elements in $N(\bm A)$ is parallel to $\bm u.$
\end{itemize}
In conclusion, $N(\bm A)=\Span\{\bm u\}.$ Hence $\dim(N(\bm A))=1.$\\
Hence $\rank(\bm A)=n-\dim(N(\bm A))=n-1.$
\item
We find that
\begin{align*}
\bm P^2&=\bm P\\
\bm P^5&=\bm P.
\end{align*}
Hence $\rank(\bm P^2)=\rank(\bm P)=n-1;\rank(\bm P^5)=\rank(\bm P)=n-1$.
\item
\begin{enumerate}
\item
If $\bm I-\bm x\bm y\trans=\bm 0$, (for example, $\bm x=\begin{bmatrix}
1
\end{bmatrix},\bm y=\begin{bmatrix}
1
\end{bmatrix}.$) then $\rank(\bm I-\bm x\bm y\trans)=0.$
\item
Otherwise, we set $\bm A=\bm I-\bm x\bm y\trans$.
\begin{itemize}
\item
Firstly, for $\forall\bm v\in N(\bm A)$, we notice:
\[
\bm A\bm v=(\bm I-\bm x\bm y\trans)\bm v=\bm0
\implies
\bm v=\bm x(\bm y\trans\bm v).
\]
Since $\bm y\trans\bm v$ is a scalar, $\bm v$ is parallel to $\bm x$.\\
In other words, any elements in $N(\bm A)$ is parallel to $\bm x.$
\item
Secondly, we discuss whether $\bm x$ is in $N(\bm A):$
\begin{equation}
\bm x\in N(\bm A)\Longleftrightarrow
\bm{Ax}=(\bm I-\bm x\bm y\trans)\bm x=\bm0
\Longleftrightarrow
\bm x=\bm x(\bm y\trans\bm x).\label{case_one}
\end{equation}
\begin{enumerate}
\item
If $\bm y\trans\bm x=1$, then condition $(\ref{case_one})$ is satisfied, then $\bm x$ is in $N(\bm A)$. Moreover, $c\bm x\in N(\bm A)$, where $c$ is a scalar.\\
Hence any elements that parallel to $\bm x$ is in $N(\bm A)$.\\
In this case, we derive $N(\bm A)=\Span\{\bm x\}$. Hence $\dim(N(\bm A))=1.$ $\rank(\bm A)=n-\dim(N(\bm A))=n-1.$
\item
Otherwise, then condition $(\ref{case_one})$ is  \emph{not} satisfied, thus $\bm x$ is not in $N(\bm A)$.\\
Obviously, $c\bm x\notin N(\bm A)$ for $\forall$ \emph{nonzero} scalar $c$.\\
Hence any nonzero elements that parallel to $\bm x$ is not in $N(\bm A)$.\\
In this case, we derive $N(\bm A)=\{\bm0\}.$ Hence $\dim(N(\bm A))=0.$ $\rank(\bm A)=n-\dim(N(\bm A))=n.$
\end{enumerate}
\end{itemize}
\end{enumerate}
In conclusion, 
\begin{itemize}
\item
When $\bm I-\bm x\bm y\trans=\bm0$, $\rank(\bm I-\bm x\bm y\trans)=0.$
\item
Otherwise,
\[
\rank(\bm I-\bm x\bm y\trans)=\begin{cases}
n&\bm y\trans\bm x\ne1;\\
n-1&\bm y\trans\bm x=1.
\end{cases}
\]
\end{itemize}
\end{enumerate}
\item
\begin{enumerate}
\item
No.\\
\textbf{Reason: }$(\bm A+\bm B)(\bm A-\bm B)=\bm A^2-\bm B^2+(\bm{BA}-\bm{AB})$.\\ But $(\bm{BA}-\bm{AB})$ cannot always be zero. For example,
\[
\bm A=\begin{bmatrix}
1&0\\1&0
\end{bmatrix}\qquad
\bm B=\begin{bmatrix}
-2&0\\2&0
\end{bmatrix}.
\]
\[
\text{But}\qquad\bm{AB}=\begin{bmatrix}
-2&0\\-2&0
\end{bmatrix},\quad\bm{BA}=\begin{bmatrix}
-2&0\\2&0
\end{bmatrix}.
\]
\item
False.\\
\textbf{Reason: } For example, 
\[
\bm A=\begin{bmatrix}
1&0\\0&1
\end{bmatrix}\qquad\bm B=\begin{bmatrix}
-1&0\\0&-1
\end{bmatrix}.
\]
Although $\bm A$ and $\bm B$ are invertible, $\bm A+\bm B$ is not invertible:
\[
\bm A+\bm B=\begin{bmatrix}
0&0\\0&0
\end{bmatrix}.
\]
\item
True.\\
\textbf{Reason: }If $f_1$ and $f_2$ is in this set, then the linear combination of $f_1$ and $f_2$ is also in this set. Why? \\
For function $\alpha_1f_1+\alpha_2f_2$, where $\alpha_1,\alpha_2$ are scalars, we obtain:
\begin{align*}
\alpha_1f_1+\alpha_2f_2(1)&=\alpha_1f_1(1)+\alpha_2f_2(1)\\
&=\alpha_1\x0+\alpha_2\x0\\
&=0.
\end{align*}
Hence $\alpha_1f_1+\alpha_2f_2$ is also in this set. Hence this set is a vector space.
\item
True.\\
\textbf{Reason: }If $\bm A$ and $\bm B$ are invertible, then for the product $\bm{AB}$, we find
\[
\bm{AB}\bm B^{-1}\bm A^{-1}=\bm A(\bm B\bm B^{-1})\bm A^{-1}=\bm A\bm I\bm A^{-1}=\bm I.
\]
Hence $\bm B^{-1}\bm A^{-1}$ is the inverse of $\bm{AB}$. Hence the product $\bm{AB}$ is invertible.
\item
False.\\
Don't mix up this statement with the proposition:\textit{ Row transforamtion doesn't change the row space}.\\
Actually, in most case, the two matrices that have the same \textit{reduced row echelon form} have \emph{different} \textit{column space.}\\
For example,
\[
\bm A=\begin{bmatrix}
1&3&3&4\\2&6&9&7\\-1&-3&3&4
\end{bmatrix}\xLongrightarrow{\text{Row transform}}
\bm U=\begin{bmatrix}
1&3&0&-1\\0&0&1&1\\0&0&0&0
\end{bmatrix}
\]
they have the same \textit{reduced row echelon form}. However, the first column of $\bm A$ is $\begin{pmatrix}
1\\2\\-1
\end{pmatrix}\notin\col(\bm U).$ They have \emph{different} \textit{column space.}
\item
True.\\
\textbf{Reason: }Suppose $\bm A$ is $n\x n$ square matrix, if two columns of $\bm A$ are the same, then $\dim(\col(\bm A))=\rank(\bm A)<n$. Since $\bm A$ is not \textit{full rank}, $\bm A$ \textit{cannot} be invertible.
\item
False.\\
Don't mix up this statement with the equality:
\[
\rank(\bm A)+\dim(N(\bm A))=n.
\]
Actually, $\rank(\bm A)=\dim(\row(\bm A))=\dim(\col(\bm A)).$
\end{enumerate}
\end{enumerate}
