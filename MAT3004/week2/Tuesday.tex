\chapter{Week2}
\section{Tuesday}\index{week2_Tuesday_lecture}
\subsection{Review}
Note that a group has the property of closeness, associatity, identity and its inverse
\subsection{Cyclic groups}
\begin{definition}[Ablian]
Let $(\mathcal{G},*)$ be a group, it is said to be \emph{ablian} if 
\[
\begin{array}{ll}
a*b = b*a,
&
\forall a,b\in\mathcal{G}
\end{array}
\]
\end{definition}
\begin{definition}[Order]
Let $\mathcal{G}$ be a group with the identity $e$. The \emph{order} if an element $g\in\mathcal{G}$ is denoted by $|g|$, i.e., the smallest $n\in\mathbb{N}^+$ such that $g^n=e$. If $|g|=\infty$, then $g$ has \emph{infinite order}.
\end{definition}
\begin{definition}[Periodic Group]
A group is said to be 
\begin{enumerate}
\item
\emph{periodic} (torsion) if every element from this group is of finite order.
\item
\emph{torsion-free} if every non-identity has infinite order.
\end{enumerate}
\end{definition}
Note that not torsion is not equivalent to torsion-free; not torsion-free is not equivalent to torsion.
\begin{proposition}
If $|\mathcal{G}|<\infty$, then $|g|<\infty$ for $\forall g\in\mathcal{G}$.
\end{proposition}
\begin{proof}
If $|g|=\infty$, then
\[
\{e,g,g^2,\cdots,g^n,\dots\}\subseteq\mathcal{G},
\]
which implies $|\mathcal{G}|=\infty$.
\end{proof}
\begin{proposition}
Let $\mathcal{G}$ be a group with identity $e$. If $g^n=e$ for some $n\in\mathbb{N}^+$, then $|g||n$.
\end{proposition}
\begin{proof}
Let $m:=|g|\le n$. Recall the ideas from discrete mathematics:
\begin{theorem}[well-ordering principle]
Any $S\subseteq\mathbb{N}$ has a least element (Axiom).
\end{theorem}
\begin{theorem}[Division Theorem]
For $\forall m\in\mathbb{Z}$ and $n\in\mathbb{N}^+$, there always $\exists$ $q,r\in\mathbb{Z}$ such that
\[
 m = nq+r,
\]
where $0\le r<n$.
\end{theorem}
Note that the power $g^n$ can be rewritten as:
\[
g^n:=g^{mq+r}=(g^m)^q\cdot g^r=e.
\]
Since $(g^m)^q$ equals to $e$, we imply $g^r = e, r<m$, i.e., $r=0$.
\end{proof}
\begin{remark}
Not that the condition $n\in\mathbb{N}^+$ can be relaxed into $n\in\mathbb{Z}$.
\end{remark}
\begin{definition}[cyclic]
A group $\mathcal{G}$ is \emph{cyclic} if there $\exists g\in\mathcal{G}$ such that for $\forall x\in\mathcal{G}$, there always $\exists n\in\mathbb{Z}$ such that
\[
x = g^n.
\]
We rewrite the group as $\mathcal{G} = <g>$, we call $g$ as the \emph{generator} of $\mathcal{G}.$ The notation $<g>$ means:
\[
<g>:=\{\cdots,g^{-2},g^{-1},e.g,g^2,\cdots\}
\]
\end{definition}
\begin{proposition}
Given a group $\mathcal{G}$ and $g\in\mathcal{G}$, we have $|<g>|=|g|$.
\end{proposition}
\begin{proof}
\begin{itemize}
\item
If $|g|=\infty$, the result is trivial.
\item
If $|g|=n$, we imply $|<g>| = |\{e,g,\dots,g^{n-1}\}|=n.$
\end{itemize}
\end{proof}
\begin{definition}
Let $a,b\in\mathbb{Z}$ not all zero. The greatest common divisor is defined as:
\[
\mbox{gcd}(a,b):=\mbox{the greatest integer that divides $a$ and $b$.}
\]
\end{definition}
\begin{theorem}[Bezout]
Provided with $a,b\in\mathbb{Z}$ not all zero. Then there exists $s,t\in\mathbb{Z}$ such that 
\[
sa+tb = \mbox{gcd}(a,b)
\]
\end{theorem}
\begin{example}
\begin{enumerate}
\item
$(\mathbb{Z},+)$ is \emph{cyclic} with generator $\pm1$
\item
$(\mathbb{Z}_n,+) = <k>$, where $\mbox{gcd}(k,n)=1$.  This is because we can always find $s>0$ and $t<0$ such that $sa+tb = 1$, i.e.,
\[
1=\underbrace{k+\cdots+k}_{\text{$s$ terms}}\in\mathbb{Z}_n
\]
\item
$(u_m,\cdot) = <\xi_{m}^k>$, where $\xi_{m} = \exp(\frac{2\pi i}{m})$ and $\mbox{gcd}(k,m)=1$. This is because we can similarly consturct $s>0$ s.t. $(\xi^k_m)^s=\xi_m$.
\end{enumerate}
\end{example}

\begin{proposition}\label{Pro:2:4}
Every cyclic group is abelian.
\end{proposition}
\begin{proof}
As $\mathcal{G} = <g>$, for $\forall x,y\in\mathcal{G}$, we have
\[
x\cdot y = g^m\cdot g^n = g^{m+n}=g^n\cdot g^m=y\cdot x.
\]
\end{proof}
\begin{remark}
The converse of proposition(\ref{Pro:2:4}) is not true. For example, $(\mathbb{Q},+)$ is abelian, but it is not cyclic, i.e., if $(\mathbb{Q},+)=<\frac{n}{m}>$, we find $\frac{n}{2m}\notin <\frac{n}{m}>$.
\end{remark}
\begin{definition}
Let $X$ be a set. A \emph{permutation} of $X$ is a \emph{bijection} of $X$. We denote 
\[
\mbox{Sym}(X) = \{\mbox{all permutations of }X\}
\]
\end{definition}
\begin{proposition}
$\mbox{Sym}(X)$ is a group under composition operation.
\end{proposition}
\begin{proof}
\begin{enumerate}
\item
For $\forall \alpha,\beta\in\mbox{Sym}(X)$, we have $\alpha\circ\beta\in\mbox{Sym}(X)$ as the composition of bijections is also bijection.
\item
For $\forall \alpha,\beta,\gamma\in\mbox{Sym}(X)$, we have $(\alpha\circ\beta)\circ\gamma = \alpha\circ(\beta\circ\gamma)$.
\item
$\mbox{identity = id$\in$Sym(X)}$
\item
For $\forall\sigma\in\mbox{Sym}(X)$, we choose $\rho\in\mbox{Sym}(X)$ s.t.
\[
\rho: \sigma(x)\mapsto x,\forall x\in X
\]
It follows that $\rho\circ\sigma=\mbox{id}$, since
\[
\sigma\circ\rho(\sigma(x)) = \sigma(\rho\circ\sigma(x))=\sigma(x)
\]
\end{enumerate}
\end{proof}
Let $X=\{1,2,\dots,n\}$, we denote $\mathbb{S}_n = \mbox{Sym}(X)$. Describe $\sigma\in\mathbb{S}_n$ by:
\[
\begin{pmatrix}
1&2&\cdots&n\\
\sigma(1)&\sigma(2)&\cdots&\sigma(n)
\end{pmatrix}
\]
Note that $|\mathbb{S}_n|=n!$
\begin{example}
Consider $\mathcal{G} := \mathbb{S}_3$, then $\sigma,\beta\in\mathcal{G}$:
\[
\sigma := \begin{pmatrix}
1&2&3\\
2&3&1
\end{pmatrix}:=(1,2,3)
\qquad
\beta := \begin{pmatrix}
1&2&3\\
2&1&3
\end{pmatrix}:=(1,2)
\]
Then we compute the composite $\sigma\circ\beta$:
\[
\sigma\circ\beta=\begin{pmatrix}
1&2&3\\
2&3&1
\end{pmatrix}\circ
\begin{pmatrix}
1&2&3\\
2&1&3
\end{pmatrix}
=
\begin{pmatrix}
1&2&3\\
3&2&1
\end{pmatrix}
\]
and $\beta\circ\sigma$:
\[
\beta\circ\sigma=\begin{pmatrix}
1&2&3\\
2&1&3
\end{pmatrix}
\circ
\begin{pmatrix}
1&2&3\\
2&3&1
\end{pmatrix}
=
\begin{pmatrix}
1&2&3\\
1&3&2
\end{pmatrix}
\]
and $\sigma\circ\sigma\circ\sigma$:
\[
\sigma\circ\sigma\circ\sigma = 
\begin{pmatrix}
1&2&3\\
1&2&3
\end{pmatrix}^3
=\mbox{id},
\]
which is said to be \emph{3-cycle}, which will be talked in future.
\end{example}
\begin{remark}
In general, $\mathbb{S}_n$ is not \emph{ablian} for $n\ge3$.
\end{remark}
In general, we write the $k$-cycle permutation as:
\[
\alpha=(i_1,\dots,i_k)
\]
where $i_1\mapsto i_2\mapsto i_3\mapsto\cdots\mapsto i_k\mapsto i_1$.
\begin{example}
Consider $\sigma = (15)(246)\in\mathbb{S}_6$, i.e.,
\[
\sigma=1\mapsto5\mapsto1;\qquad 2\mapsto4\mapsto6\mapsto2; 
\qquad 3\mapsto3
\]
and $\alpha = (13)(45)\in\mathbb{S}_6$. We study the composition $\sigma\circ\alpha$:
\[
\sigma\circ\alpha=[(15)(246)]\circ[(13)(45)]
=(135624)
\]
and
\[
\alpha\circ\sigma = (13)(45)(15)(246)=(146253)
\]

\end{example}
\begin{proposition}\label{Pro:2:6}
Each $\sigma\in\mathbb{S}_n$ is either a cycle or a product of disjoint cycle.
\end{proposition}
Disjoint cycles commute with one another.
\begin{definition}
2-cycle is called a \emph{transposition}
\end{definition}
\begin{proposition}
$\sigma\in\mathbb{S}_n$ can be written as a product of transpositions.
\end{proposition}
\begin{proof}
Due to proposition(\ref{Pro:2:6}) and
\[
(i_1i_2\cdots i_k) = (i_1i_k)\cdots(i_1i_3)(i_1i_2)
\]
\end{proof}
For $\sigma\in\mathbb{S}_n$
, we have
\[
\sigma(i_1,\dots,i_k)\sigma^{-1}=(\sigma(i_1),\dots,\sigma(i_k))
\]









