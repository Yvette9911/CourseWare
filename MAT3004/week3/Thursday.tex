
%\chapter{Week3}

\section{Thursday}\index{week3_Thursday_lecture}
\subsection{Cyclic Groups}
\begin{definition}[order]
Let $G$ be a group with \emph{identity} $e$. The \emph{order} of an element $g\in G$, denoted by $|g|$ is the smallest $n\in\mathbb{N}^+$ such that $g^n=e$; if no such $n$ exists, we say $g$ has \emph{infinite order} and $|g|=\infty$.
\end{definition}
\begin{proposition}
If $G$ has finite order, then every element of $G$ has finite order. (The order of $G$ is the number of elements in $G$).
\end{proposition}
\begin{proof}
Suppose $|G|=m$, consider the elements in $G$:
\[
g^0:=e,g^1,\dots,g^m,
\]
thus there exists $0\le i<j\le m$ such that $g^i=g^j$, otherwise there will be $m+1$ distinct elements in $G$. Therefore $g^{-i}g^i=e=g^{j-i}$.
\end{proof}
\begin{remark}
Note that the converse of this proposition is not true.
\end{remark}
\begin{definition}[Torsion]
A group is said to be a \emph{periodic group} (or \emph{torsion group}) if all its elements all have \emph{finite} order; A group is said to be \emph{torsion-free} if none of its non-trivial elements has finite order. Finite groups are always periodic.
\end{definition}
\begin{proposition}
Let $G$ be a group with identity $e$, and $g\in G$. If $g^n=e$ for some $n\in\mathbb{Z}$, then $|g|$ divides $n$, i.e., $|g| | n$.
\end{proposition}
\begin{proof}
Let $m=|g|$. There exists $q,r\in\mathbb{N}^+$ such that\[
\begin{array}{ll}
n = mq+r
&
0\le r<m
\end{array}
\]
It follows that $g^n=(g^m)^q\cdot g^r=e=g^r$, which follows that $r=0$.
\end{proof}
\begin{definition}[Cyclic]
A group $G$ is said to be 
\emph{cyclic} if there exists $g\in G$ such that every element of $G$ is equal to $g^n$ for some $n\in\mathbb{Z}$. In this case, we say $g$ is a \emph{generator} of $G$ and write $G=\gen{g}$.
\end{definition}
In general, $\gen{g} = \{g^n\mid n\in\mathbb{Z}\}$.
\begin{proposition}\label{Pro:3:6}
Let $G$ be a group and $g\in G$. Then $|\gen{g}| = |g|$.
\end{proposition}
Question.
\begin{proof}
Note that $m:=|g|\le |\gen{g}|$, since $g^0,g^1,\dots,g^{m-1}$ are distinct; and also $|\gen{g}|\le|g|$, since $g^i=g^j$ iff $i\equiv j(\bmod m)$
\end{proof}
\begin{remark}
The generator of a cyclic group may not be unique, i.e., there may exists distinct $g_1,g_2\in G$ s.t. $G=\gen{g_1}=\gen{g_2}$.
\end{remark}
\paragraph{Applications}
To illustrate some examples, we introduce some results from discrete mathematics:
\begin{definition}
Let $a,b\in\mathbb{Z}$ s.t. $a^2+b^2\ne0$. The \emph{greatest common divisor} is defined as:
\[
\mbox{gcd}(a,b):=\mbox{greatest integer that divides both $a$ and $b$}
\]
\end{definition}
\begin{theorem}[Bezout's identity]
Let $a,b\in\mathbb{Z}$ s.t. $a^2+b^2\ne0$. There exists $s,t\in\mathbb{Z}$ s.t.
\[
sa+tb = \mbox{gcd}(a,b)
\]
\end{theorem}
\begin{example}
\begin{enumerate}
\item
$(\mathbb{Z},+)$ is cyclic, generated by $1$ or $-1$.
\item
$(\mathbb{Z}_n,+)$ is cyclic, generated by $k$ with $\mbox{gcd}(k,n)=1$.
\item
$(U_m,\cdot)$ is cyclic, generated by $\zeta_m^k=\exp(\frac{2k\pi}{m}i)$ with $\mbox{gcd}(k,m)=1$.

Note that $U_m:=\{1,\zeta_m^1,\dots,\zeta_m^{m-1}\}$.
\end{enumerate}
\end{example}
\begin{proposition}\label{Pro:3:7}
A cyclic group $G$ has order $n$ iff its generators have order $n$.
\end{proposition}
Question: Can we apply proposition(\ref{Pro:3:6}) directly
\begin{proof}
\begin{itemize}
\item
Suppose $g$ is a generator of $G$ with order $n$, thus $|G| = |\gen{g}|=n$.
\item
Suppose $G$ is a cyclic group with order $n$ and $g$ is a generator of $G$, then $|g| = |\gen{g}|=|G|=n$.
\end{itemize}
\end{proof}
\begin{proposition}
 Let $p$ be a prime. Let $G=(\mathbb{Z}_p,+).$ Then $|g|=p$ for $\forall g\in G\setminus\{0\}$.
\end{proposition}
\begin{proof}
\begin{itemize}
\item
One way is to apply proposition(\ref{Pro:3:7}). The cyclic group $G$ has order $p$ and therefore its generators have order $p$. $g\in G\setminus\{0\}$ are generators since
\[
sp+tg=\mbox{gcd}(p,g)=1\implies
(sk)p+(tk)g=k,\quad 1\le k\le p
\]
\item
Another way is to assume $|g|=y$ for $1\le y<p$, since $g^i=g^{j}$ iff $i\equiv j(\bmod p)$ and $g^p\equiv0(\bmod p)$. Since $sp+tg=\mbox{gcd}(p,g)=1$, we derive for $1\le y<p$,
\[
(sy)p+(ty)g=y\implies y\equiv 0(\bmod p),
\]
which is a contradiction
\end{itemize}
\end{proof}

\begin{proposition}
Every cyclic group is abelian.
\end{proposition}
Recall that abelian means a group has commucative operation.
\begin{proof}
Suppose $G=\gen{g}$ for some $g\in G$, thus for any elements $g^{n_1},g^{n_2}$:
\[
g^{n_1}\cdot g^{n_2}
=
g^{n_1+n_2}
=
g^{n_2+n_1}
=
g^{n_2}\cdot g^{n_1},
\]
since the product of the elements is independent from adding parentheses.
\end{proof}
\begin{remark}
The converse is not true, e.g., the group $(\mathbb{Q},+)$. 

Verification: Assume $\mathbb{Q}=\gen{\frac{p}{q}}$, i.e., 
Choose $k>1$ such that $(k,q)=1$ and set $y =\frac{1}{k}\in G$. There exists $r\in\mathbb{Z}$ such that $(\frac{p}{q})^r=\frac{rp}{q}=y$, which implies $rp=qy=\frac{y}{k}$, which is a contradiciton since RHS is not an integer.
\end{remark}

\subsection{Symmetric Groups}
\begin{definition}
Let $X$ be a set. A \emph{permutation} of $X$ is a bijection $\sigma: X\mapsto X$/
\end{definition}
\begin{proposition}
The set of all permutations of a set $X$ forms a group under the operation $\circ$ (composition), denoted by $\mbox{Sym}(X)$. (\emph{symmetric group})
\end{proposition}
\begin{proof}
\begin{itemize}
\item
Note that $\alpha\circ\beta$ is a bijection of $X$, hence permutation of $X$
\item
$\alpha\circ(\beta\circ\gamma) = (\alpha\circ\beta)\circ\gamma$.
\item
$e$ is a permutation such that $e(x) = x$, $\forall x\in X$. Then
\[
\begin{array}{ll}
e\circ\sigma = \sigma\circ e=\sigma,
&
\forall\sigma\in\mbox{Sym}(X)
\end{array}
\]
\item
For a given $\sigma\in\mbox{Sym}(X)$, there exixts a bijection $\rho\in\mbox{Sym}(X)$ such that
\[
\rho\circ\sigma=\sigma\circ\rho=e
\]
\end{itemize}
\end{proof}
\paragraph{Notations}
For $X=\{1,2,\dots,n\}$, we write $\mbox{Sym}(X)$ as $S_n$ ($n$-th symmetric group). The element $\sigma\in S_n$ is denoted as:
\[
\sigma=\begin{pmatrix}
1&2&\cdots&n\\
\sigma(1)&\sigma(2)&\cdots&\sigma(n)
\end{pmatrix}
\]
\begin{proposition}
$|S_n| = n!$.
\end{proposition}
\begin{proof}
$\sigma(1)\in\{1,\dots,n\}$; for fixed $\sigma(1)$, $\sigma(2)\in \{1,\dots,n\}\setminus\{\sigma(1)\}$; so on and so forth. Hence, there are totoal $n*(n-1)*\cdots*1=n!$ choices of permutations.
\end{proof}
\begin{example}
For $\alpha,\beta\in S_3$ given by:
\[
\begin{array}{ll}
\alpha=\begin{pmatrix}
1&2&3\\2&3&1
\end{pmatrix},&
\beta = \begin{pmatrix}
1&2&3\\2&1&3
\end{pmatrix}
\end{array}
\]
we find
\[
\begin{array}{ll}
\alpha\beta=
\begin{pmatrix}
1&2&3\\
3&2&1
\end{pmatrix}
&
\beta\alpha=
\begin{pmatrix}
1&2&3\\
1&3&2
\end{pmatrix}
\end{array}
\]
Thus $\alpha\beta\ne\beta\alpha$, i.e., $S_3$ is non-abelian.
\end{example}
Note that $S_n$ is non-abelian for $n\ge3$: To show this property, construct
\[
\begin{array}{ll}
\alpha'=\begin{pmatrix}
1&2&3&4&\cdots&n\\
\alpha(1)&\alpha(2)&\alpha(3)&4&\cdots&n
\end{pmatrix}
&
\beta'=\begin{pmatrix}
1&2&3&4&\cdots&n\\
\beta(1)&\beta(2)&\beta(3)&4&\cdots&n
\end{pmatrix}
\end{array}
\]
Also, note that $|\alpha|=3$. Thus $S_3$ is not cyclic.

\paragraph{More on $S_n$}
Consider the element $\sigma$ in $S_6$:
\[
\sigma=\begin{pmatrix}
1&2&3&4&5&6\\5&4&3&6&1&2
\end{pmatrix}
\]
We re-write $\sigma$ as
\[
\sigma=(15)(246),
\]
where $(i_1,\dots,i_k)$ denotes the permutation
\[
\begin{array}{llll}
i_1\mapsto i_2,
&
i_2\mapsto i_3,
&\cdots&i_k\mapsto i_1,
\end{array}
\]
and $j\mapsto j$ for all remaining $j$.
\begin{definition}[cycle]
We call $(i_1,\dots,i_k)$ as a $k$-cycle or a \emph{cycle of length $k$}. In particular, we use $()$ to denote the identity $\epsilon\in S_n$, meaning that it fixes all numbers in $\{1,\dots,n\}.$
\end{definition}
\begin{proposition}\label{Pro:3:12}
Each permutation $\sigma\in S_n$ is either a cycle or a product of disjoint cycles.
\end{proposition}
We will prove proposition(\ref{Pro:3:12}) later.
\begin{proposition}
Disjoint cycles commute with each other.
\end{proposition}
\begin{proof}
Suppose $\tau=(i_1,\dots,i_k)$ and $\rho=(j_1,\dots,j_l)$. Show that for $i_m\in\{i_1,\dots,i_k\}$, $\tau(\rho(i_m))=\rho(\tau(i_m))$ and similarly for $j_n$; also show $\tau(\rho(s)) = \rho(\tau(s)) $ for $s\in\mbox{dom}\setminus(I\bigcup J)$
\end{proof}
\begin{definition}[Transposition]
A 2-cycle is called a transposition, for it swaps two elements with each other.
\end{definition}
\begin{proposition}
Each element of $S_n$ is a product of transpositions (not necessarily disjoint).
\end{proposition}
\begin{proof}
Consider $(i_1\dots i_k) = (i_1i_k)(i_1i_{k-1})\cdots(i_1i_3)(i_1i_2)$;
\end{proof}
\begin{example}
\[
\begin{pmatrix}
1&2&3&4&5&6\\
5&4&3&6&1&2
\end{pmatrix}
=
(15)(246)
=
(15)(26)(24)
=
(15)(46)(26)
\]
\end{example}
\begin{proposition}
Let $n\in\mathbb{N}^+$ and $\sigma\in S_n$. Show that
\[
\sigma(i_1\cdots i_k)\sigma^{-1} = (\sigma(i_1)\cdots\sigma(i_k))
\]
\end{proposition}
\begin{proof}
For $x:=\sigma(i_m)$, $1\le m\le k$, we have
\[
\sigma^{-1}(x)=i_m\implies
(i_1\cdots i_k)\sigma^{-1}(x)=\left\{
\begin{aligned}
i_1,&\quad\mbox{if }m=k\\
i_{m+1},&\quad\mbox{otherwise}
\end{aligned}
\right.
\]
adn thus $\sigma(i_1\cdots i_k)\sigma^{-1}(x)=\sigma(i_1)$ if $m=k$, and equals $\sigma(i_{m+1})$ otherwise.

For $x\in\{1,\dots,n\}\setminus\{\sigma(i_1),\dots,\sigma(i_k)\}$, we have $\sigma^{-1}(x)\notin\{i_1,\dots,i_k\}$, and thus
\[
(i_1\cdots i_k)(\sigma^{-1}(x))=\sigma^{-1}(x)\implies
\sigma(i_1\cdots i_k)(\sigma^{-1}(x))=x.
\]
\end{proof}
\begin{proposition}
In every factorization of $\sigma$ as a product of transpositions, the number of factors is either always even or always odd.
\end{proposition}
\begin{proof}
There is a one-to-one coorespondence between $\sigma$ and binary matrix $A_\sigma$, e.g.,
\[
\sigma=(15)(246)
\Leftrightarrow
A_\sigma=\begin{pmatrix}
0&0&0&0&1&0\\
0&0&0&0&0&1\\
0&0&1&0&0&0\\
0&1&0&0&0&0\\
1&0&0&0&0&0\\0&0&0&1&0&0\\
\end{pmatrix}
\]
Also, if $\sigma$ is a transposition then $\det(A_\sigma) = -1$.
\end{proof}

\subsection{Dihedral Groups}
Let $\mathcal{T}$ denote the set of \emph{transformations} of $\mathbb{R}^2$, consisting of all rotations by fixed angles about the origin, and all reflections over lines through the origin.
\begin{definition}
Consider a regular $n$-th polygon $P_n$ in $\mathbb{R}^2$ centered at origin. We represent the polygon by its vertices, e.g., $P_n=\{x_1,\dots,x_n\}\subseteq\mathbb{R}^2$. If $\tau(P_n) = P_n$ for some $\tau\in\mathcal{T}$, we say that $P_n$ is \emph{symmetric} w.r.t. $\tau$.
\end{definition}
\begin{remark}
Intuitively, $P_n$ is symmetric w.r.t. $n$ rotations $\{r_0,\dots,r_{n-1}\}$ and $n$ reflections $\{s_1,\dots,s_n\}$ in $\mathcal{T}$.
\end{remark}
\begin{theorem}
The set
\[
D_n:=\{r_0,r_1,\dots,r_{n-1},s_1,\dots,s_n\}
\]
forms the $n$-th \emph{dihedral group}, under the transformation composition operation $\tau*\gamma = \tau\circ\gamma$. In particular, $|D_n| = 2n$.
\end{theorem}
\begin{proof}
The set of rotations:
\[
\gen{r}=\{\mbox{id},r,r^2,\dots,r^{n-1}\}
\]
The set of reflections:
\[
\{s,rs,r^2s,\dots,r^{n-1}s\}
\]
Thus the elements in $D_n$:
\[
D_n = \{\mbox{id},r,r^2,\dots,r^{n-1},s,rs,r^2s,\dots,r^{n-1}s\}
\]
\end{proof}
\begin{proposition}
Show that $D_3=S_3$.
\end{proposition}


\subsection{Free Groups}
\begin{definition}[Free Groups]
\begin{enumerate}
\item
Let $S$ be a non-empty set (not necessarily finite). Let $1$ be an \emph{empty word}, i.e., a string of elements from $S$ with length $0$. For each $a\in S$, define an element $a^{-1}\notin S$ s.t. their juxtaposition is 1:
\[
\begin{array}{ll}
a^{-1}a=aa^{-1}=1
&
a^{-1}b\ne 1\ne ba^{-1},\forall b\ne a
\end{array}
\]
\item
Define $S^{-1}:=\{a^{-1}\mid a\in S\}$. A \emph{reduced word} $w$ of $S$ is a \emph{finite} string of elements from $S\bigcup S^{-1}$ s.t. no substring of $w$ contains $a^{-1}a$ or $aa^{-1}$ for $\forall a\in S$. Define an operation $*$ on the set of all reduced words $F_S$:
\[
w_1*w_2=w:=\left\{
\begin{aligned}
w_1w_2,&\quad\mbox{if $w_1w_2$ is a reduced word}\\
&\mbox{obtained by repeatedly removing $a^{-1}a$ or $aa^{-1}$}
\end{aligned}
\right.,\]
$\forall w_1,w_2\in F_S.$ It can be verified that $(F_S,*)$ forms a group, called the \emph{free group} generated by $S$.
\end{enumerate}
\end{definition}
\paragraph{Notations}
For convenience,
\[
\begin{array}{ll}
a^n:=\underbrace{a\cdots a}_{n}
&
a^{-n}:=\underbrace{a^{-1}\cdots a^{-1}}_{n}
\end{array}
\]
\begin{example}
Let $S=\{a,b,c\}$, then 
\[
w_1=ac^{-4}b^2c,w_2=c^{-1}b^{-2}c^2a^{-3},
w_3=a^3c^{-2}b^3
\]
are reduced words of $S$, while $w_1w_2$ and $w_2w_3$ are not. Also,
\[
(w_1*w_2)*w_3=ac^{-4}b^3=w_1*(w_2*w_3)
\]
\end{example}
\begin{example}
Let $S=\{a\}$ be a singleton. then the free group $F_S$ can be viewed as the group $(\mathbb{Z},+)$
\end{example}
In fact, any group can be viewed as a free group with some additional conditions. For instance, a cyclic group $G$ of order 6 can be viewed as the group $F_S$ with the condition that $a^6 = 1$, which is sometimes written as $G=\gen{a|a^6=1}$








