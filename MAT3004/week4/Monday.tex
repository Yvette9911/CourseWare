
\chapter{Week4}

\section{Subgroups}\index{week4_Friday_lecture}
\begin{definition}
Let $G$ be a group. A \emph{non-empty} subset $H\subseteq G$ is a \emph{subgroup} of $G$ (denoted by $H\le G$) if it satisfies the following:
\begin{enumerate}
\item
If $a,b\in H$, then $a*b\in H$
\item
If $a\in H$, then $a^{-1}\in H$
\end{enumerate}
Particularly, if $H$ is a \emph{proper subset} of $G$, then $H$ is a \emph{proper subgroup} of $G$, denoted by $H<G$.
\end{definition}
\begin{remark}
Any subgroup $H$ is a subgroup iff $H$ is a subset of $G$, and $H$ forms a group w.r.t. the induced binary operation from $G$ (called the induced operation on $H$)
\end{remark}
\begin{example}
\begin{enumerate}
\item
Any group $G$ contains the trivial subgroup $\{e\}$, (sometimes writeen as $1$ for such group). Any subgroup $H\ne 1$ is \emph{non-trivial}
\item
$\mathbb{Z}<\mathbb{Q}<\mathbb{R}<\mathbb{C}$ under addition; $\mathbb{Z}^{\#}<\mathbb{Q}^{\#}<\mathbb{R}^{\#}<\mathbb{C}^{\#}$ under multiplication
\item
For any $n\in\mathbb{Z}$, we have $n\mathbb{Z}\le\mathbb{Z}$ under addition
\item
Define $\mbox{SL}(n,\mathbb{R}):=\{A\in\mathbb{R}^{n\times n}\mid \det(A)=1\}$ and $\mbox{GL}(n,\mathbb{R}):=\{A\in\mathbb{R}^{n\times n}\mid \det(A)\ne0\}$, then $\mbox{SL}(n,\mathbb{R})<\mbox{GL}(n,\mathbb{R})$
\item
The set of all rotations (including the trivial rotation) in a dihedral group $D_n$ is a subgroup of $D_n$
\item
Viewing the elements in $D_n$ as permutations of the vertices of a regular $n$-gon $P_n$, we regard $D_n$ as a subgroup of $S_n$
\item
For any $n\in\mathbb{N}^+$, a permutation $\sigma\in S_n$ is \emph{even} (\emph{odd}) if it is a product of an even (odd) number of transpositions. The set of all even permutations in $S_n$ forms the $n$-th \emph{alternating group} $A_n\le S_n$.
\end{enumerate}
\end{example}

\begin{proposition}
Let $H$ be a non-empty subset of a group $G$. Then $H\le G$ iff $ab^{-1}\in H$ whenever $a,b\in H$
\end{proposition}
\begin{proof}
Necessity: 
\[
a,b\in H\implies b^{-1}\in H\implies ab^{-1}\in H
\]

Sufficiency: Let $e$ be the identity of $G$. As $H$ is non-empty, there exists $h\in H$ s.t.
\[
e=hh^{-1}\in H
\]
\begin{itemize}
\item
Since $e\in H$, for any $a\in H$, we have $a^{-1}=e\cdot a^{-1}\in H$
\item
For any $a,b\in H$, we have $b^{-1}\in H$. 
\[
ab = a(b^{-1})^{-1}\in H
\]
\end{itemize}
\end{proof}
\subsection{Cyclic subgroups}
Given a group $G$ and any element $g$, we have subset
\[
\gen{g}=\{g^n\mid n\in\mathbb{Z}\}
\]
It is indeed the cyclic subgroup generated by $g$.
\begin{proposition}
The intersection of any collection of subgroups of a group $G$ is also a subgroup of $G$.
\end{proposition}

\begin{corollary}
Let $G$ be a group. Then for any $g\in G$, we have
\[
\gen{g} =\bigcap_{\{H\mid g\in H\le G\}}H,
\]
i.e., $\gen{g}$ is the smallest subgroup of $G$ containing $g$.
\end{corollary}
\begin{proof}
Note that $\bigcap_{\{H\mid g\in H\le G\}}H\subseteq\gen{g}$.

On the other hand, note that $\bigcap_{\{H\mid g\in H\le G\}}H$ is a subgroup,
\[
\gen{g}\subseteq\bigcap_{\{H\mid g\in H\le G\}}H
\]
\end{proof}
\begin{proposition}
Every subgroup of a cyclic group is cyclic
\end{proposition}
\begin{proof}
Let $G=\gen{g}$ and $H\le G$
\begin{itemize}
\item
If $H=1$, then $H=\gen{e}$ is cyclic
\item
Otherwise, there exists a samllest $m\in\mathbb{N}^+$ s.t. $g^m\in H$. It suffices to show $H=\gen{g^m}$
\begin{itemize}
\item
It's clear that $\gen{g^m}\subseteq H$
\item
Take $g^n\in H$, there exists $q,r\in\mathbb{Z}$ s.t.
\[
n=mq+r,\quad
0\le r<m
\]
Therefore, $g^n = (g^m)^q\cdot g^r\implies g^r = (g^m)^{-q}\cdot g^n\in H\implies r=0$. Thus $g^n\in\gen{g^m}$, i.e., $H\subseteq\gen{g^m}.$
\end{itemize}
\end{itemize}
\end{proof}

\begin{corollary}\label{cor:4:2}
Any subgroup of $(\mathbb{Z},+)$ is of the form $n\mathbb{Z}$ for some $n\in\mathbb{N}$.
\end{corollary}
\begin{remark}
For $a,b\in\mathbb{Z}$, it's easy to verify
\begin{equation}
\gen{a,b}:=\{ma+nb\mid m,n\in\mathbb{Z}\}\le\mathbb{Z}
\end{equation}
Thus following corollary(\ref{cor:4:2}), $\gen{a,b}=d\mathbb{Z}$ for some $d\in\mathbb{N}^+$. In fact, $d$ is the greatest common divisor of $a$ and $b$. (proof relies on Bezout's identity.)
\end{remark}
\begin{theorem}[Bezout's identity]
Let $a,b\in\mathbb{Z}$ such that $a^2+b^2\ne0$. Then there exists $s,t\in\mathbb{Z}$ s.t.
\[
sa+tb=\mbox{gcd}(a,b)
\]
\end{theorem}
\begin{proof}
Consider the case that $ab\ne0$.

Note that $\gen{a,b}=d\mathbb{Z}$, there exists $s,t$ such that
\begin{equation}\label{Eq:4:2}
sa+tb=d
\end{equation}

Since $a,b\in d\mathbb{Z}$, we derive $d$ divides both $a$ and $b$. From (\ref{Eq:4:2}), for any $x$ dividing both $a$ and $b$, we have $x|d$, which implies $\mbox{gcd}(a,b)|d$, which implies $d=\mbox{gcd}(a,b)$.
\end{proof}
\begin{proposition}
Let $a,b\in\mathbb{Z}$ be such that $a^2+b^2\ne0$ and $k\in\mathbb{Z}^{\#}$. Show that
\[
\mbox{gcd}(ak,bk) = \mbox{gcd}(a,b)k
\]
\end{proposition}
\begin{proof}
First, since $\mbox{gcd}(a,b)$ divides both $a$ and $b$, we imply $\mbox{gcd}(a,b)k$ divides both $ak$ and $bk$. It follows that
\[
\mbox{gcd}(a,b)k|s(ak)+t(bk)=\mbox{gcd}(ak,bk),
\]
for some $s,t$.

Second, 
\[
\mbox{gcd}(ak,bk)|s(ak)+t(bk)=k\mbox{gcd}(a,b)
\]
\end{proof}
Let $a,b\in\mathbb{Z}$ s.t. $a^2+b^2\ne0$. The \emph{least common multiple} is
\[
\mbox{lcm}(a,b)=\left\{
\begin{aligned}
\mbox{least positive integer divisible by $a$ and $b$},&\quad ab\ne0\\
0,&\quad \mbox{otherwise}
\end{aligned}
\right.
\]
\begin{theorem}
Let $a,b\in\mathbb{Z}$ s.t. $a^2+b^2\ne0$. Then
\[
\mbox{lcm}(a,b)=\frac{|ab|}{\mbox{gcd}(a,b)}
\]
\end{theorem}
\begin{proof}
w.l.o.g., $a,b>0$.

Note that $a$ and $b$ divides $\frac{ab}{\mbox{gcd}(a,b)}$, and thus $k\le\frac{ab}{\mbox{gcd}(a,b)}$.

On the other hand, note that $ab$ divides both $a\mbox{lcm}(a,b)$ and $b\mbox{lcm}(a,b)$, which implies
\[
ab\le\mbox{gcd}(a\mbox{lcm}(a,b),b\mbox{lcm}(a,b))
=
\mbox{lcm}(a,b)\mbox{gcd}(a,b)
\]
\end{proof}
\begin{proposition}
Let $G=\gen{g}$ be a cyclic group of order $n$. Let $g^s\in G$, then
\[
|g^s|=\frac{n}{d},
\]
where $d=\mbox{gcd}(s,b)$. Moreover, $\gen{g^s} = \gen{g^t}$ iff $\mbox{gcd}(s,n) = \mbox{gcd}(t,n)$.
\end{proposition}
\begin{proof}
The idea is to transform the order of element into the order of a cyclic group.
\begin{itemize}
\item
For some $x,y\in\mathbb{Z}$,
\[
g^d=g^{xs+yn}=(g^{s})^x\in\gen{g^s}\implies
\gen{g^d}\subseteq\gen{g^s}
\]
On the other hand, $g^s$ is the power of $g^d$ since $d$ divides $s$:
\[
g^s = (g^d)^{s/d}\implies
\gen{g^s}\subseteq\gen{g^d}\implies\gen{g^s} = \gen{g^d}
\]
it follows that
\[
|g^s| = |\gen{g^s}| = |\gen{g^d}| = |g^d| = \frac{n}{d}
\]
\item
For second assrtion, the converse is clear. For the forward direction,
\[
|g^s|=\frac{n}{\mbox{gcd}(s,n)}=\frac{n}{\mbox{gcd}(t,n)}=|g^t|
\]
\end{itemize}


\end{proof}

\begin{corollary}
All generators of a cyclic group $G=\gen{g}$ of order $n$ are of the form $g^r$ with $\mbox{gcd}(r,n)=1$.
\end{corollary}
\begin{example}
Given the \emph{cyclic group} $G=\mathbb{Z}_{12}=\gen{g}$, then all the generators of $G$ are $g,g^5,g^7,g^{11}$; since $\mbox{gcd}(9,12)=\mbox{gcd}(3,12)=3$, we have $\gen{g^3} = \gen{g^9}$
\end{example}
\subsection{Direct Products}
\begin{definition}[Cartesian Product]
The \emph{Cartesian product} of given sets $S_1,\dots,S_n$ is
\[
\prod_{i=1}^nS_i=S_1\times S_2\times\cdots\times S_n:=\{(a_1,\dots,a_n)\mid a_i\in S_i\}
\]
If $S_1= S_2=\cdots= S_n$, we write $\prod_{i=1}^nS_i=S^n$
\end{definition}
\begin{proposition}
The product of groups also forms a group, under the operation induced from those groups.
\end{proposition}
\begin{remark}
If the operations of $G_i$ are all addition, then $\prod_{i=1}^nG_i$ is said to be the \emph{direct sum} of groups of $G_i$:
\[
\bigoplus_{i=1}^nG_i=G_1\oplus \cdots\oplus G_n
\]
Note that $\bigoplus_{i=1}^nG_i$ is always abelian.
\end{remark}
\begin{example}
\begin{enumerate}
\item
The group $G=S_3\times\mathbb{Z}_2$ is not abelian.
\item
The group $G=(\mathbb{Z}_2\times\mathbb{Z}_3,+) =\mathbb{Z}_2\oplus  \mathbb{Z}_3$ is cyclic
\item
The \emph{Klein 4-group} $V=\mathbb{Z}_2^2$ is \emph{not} cyclic.
\end{enumerate}
\end{example}
\begin{theorem}
The group $G=\mathbb{Z}_m\times\mathbb{Z}_n$ is \emph{cyclic} iff $\mbox{gcd}(m,n) = 1$.
\end{theorem}
\begin{proof}
Let $k=\mbox{lcm}(m,n) = mn/\mbox{gcd}(m,n)$, then
\[
k(a,b) = (ka,kb)=(0,0),\forall a\in\mathbb{Z}_m,b\in\mathbb{Z}_n,
\]
since $m$ and $n$ both divide $k$. Thus $|g|\le k,\forall g\in G$. Thus $G$ is cyclic implies $k=mn$, i.e., $\mbox{gcd}(m,n)=1$

To show the converse, consider the element $(1,1)\in G$:
\[
\begin{array}{lll}
d(1,1)=(0,*),&\implies&d=xm\\
d(1,1)=(*,0),&\implies&d=ym\\
\end{array}
\]
thus $|(1,1)|=\mbox{lcm}(m,n) = mn$, i.e., $G$ is cyclic.
\end{proof}
\begin{corollary}
The group $\prod_{i=1}^n\mathbb{Z}_{m_i}$ is cyclic iff $m_i,m_j$ are mutually co-prime.
\end{corollary}

\subsection{Generating Sets}
\begin{definition}[Generating Set]
Let $G$ be a group, $S$ be a non-empty subset of $G$, the set
\[
\gen{S}:=\{a_1^{m_1}a_2^{m_2}\cdots a_n^{m_n}\mid n\in\mathbb{N},a_i\in S,m_i\in\mathbb{Z}\}
\]
is a subgroup of $G$, called the subgroup of $G$ generated by $S$. If $G=\gen{S}$, then $S$ is the genreating set for $G$.
\end{definition}
\begin{remark}
The elements $a_i$ do not have to be distinct, e..g, if $S=\{a,b\}$, then $a^3b^{-1}a^{-1}\in\gen{S}$.
\end{remark}
\begin{proposition}
\[
\gen{S}=\bigcap_{\{H\mid S\subseteq H\le G\}}H,
\]
i.e., $\gen{S}$ is the smallest subgroup in $G$ containing the subset $S$.
\end{proposition}
\begin{proof}
It's clear that $\gen{S}\supseteq \bigcap_{\{H\mid S\subseteq H\le G\}}H.$

For any element $a$ in $\gen{S}$, we find $a\in \bigcap_{\{H\mid S\subseteq H\le G\}}H$, since $S\subseteq\bigcap_{\{H\mid S\subseteq H\le G\}}H.$

\end{proof}
When $S=\{a_1,\dots,a_l\}$ is a finite set, we write
\[
\gen{S} = \gen{a_1,\dots,a_l}
\]
\begin{example}
\begin{enumerate}
\item
The set of cycles and the set of transpositions are two examples of generating sets for $S_n$
\item
$S_n=\gen{(12)(12\cdots n)}$
\item
$D_n = \gen{r,s}$, where $r$ is the rotation by angle $\frac{2\pi}{n}$ in the anti-clock wise direction; and $s$ is any reflection
\end{enumerate}
\end{example}
A group is said to be \emph{finitely generated} if there are finite number of elements $a_1,\dots,a_l\in G$ s.t.
\[
G=\gen{a_1,\dots,a_l}
\]
Every finite group is finitely generated.
\begin{proposition}
The group $(\mathbb{Q},+)$ is not finitely generated.
\end{proposition}
\begin{proof}
Otherwise assume
\[
\mathbb{Q}=\gen{\frac{p_1}{q_1},\dots,\frac{p_l}{q_l}}
\]
Construct number $\frac{1}{q_1\dots q_l}\in\mathbb{Q}$, which is a contradiction.
\end{proof}
\begin{theorem}[Fundamental Theorem of Finitely Generated Abelian Groups]
Any finitely generated \emph{abelian} group is isomorphic to
\[
\prod_{i=1}^m\mathbb{Z}_{p_i^{r_i}}\times\mathbb{Z}^n
\]
where $m\in\mathbb{N}^+,n,r_i\in\mathbb{N}$, and $p_i$ are primes not necessarily distinct. This direct product is unique after re-arrangement.
\end{theorem}
\begin{example}
The abelian groups of order $360=2^33^25$ up to isomorphism are
\begin{enumerate}
\item
$\mathbb{Z}_2^3\times\mathbb{Z}_3^2\times\mathbb{Z}^5$
\item
$\mathbb{Z}_2\times\mathbb{Z}_{4}\times\mathbb{Z}_3^2\times\mathbb{Z}^5$
\item
$\mathbb{Z}_8\times\mathbb{Z}_3^2\times\mathbb{Z}^5$
\item
$\mathbb{Z}_2^3\times\mathbb{Z}_9\times\mathbb{Z}^5$
\item
$\mathbb{Z}_2\times\mathbb{Z}_{4}\times\mathbb{Z}_9\times\mathbb{Z}^5$
\item
$\mathbb{Z}_8\times\mathbb{Z}_9\times\mathbb{Z}^5$
\
\end{enumerate}
The abelian groups of order $7^5$ up to isomorphism are
\begin{enumerate}
\item
$\mathbb{Z}_7^5$
\item
$\mathbb{Z}_7\times \mathbb{Z}_{2401}$
\item
$\mathbb{Z}_{49}\times \mathbb{Z}_{343}$
\item
$\mathbb{Z}_{16807}$
\end{enumerate}
\end{example}









