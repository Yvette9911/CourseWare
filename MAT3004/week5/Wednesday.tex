\begin{theorem}
Let $n\ge5$, then $A_n$ is simple,and $A_n$ is the only non-trivial proper normal subgroup of $S_n$.
\end{theorem}
It suffices to show that $1<H\triangleleft S_n$ implies $H=A_n$.


\section{Thursday}\index{week5_Thursday_lecture}
\subsection{Homomorphisms}
\begin{definition}[Homomorphisms]
Let $G=(G,*)$ and $\hat G=(\hat G,\bigodot)$, then a \emph{homomorphisms} is a map $\phi: G\mapsto \hat G$ such that
\[
\phi(a*b) = \phi(a)\bigodot \phi(b),\quad
\forall a,b\in G
\]
If $\phi$ is a \emph{bijection}, then $\phi$ is said to be a \emph{isomorphism}. We denote $G\cong^{\phi} \hat G$.
\end{definition}
\begin{remark}
\begin{itemize}
\item
homomorphisms is not necessarily injective or surjective.
\item
The isomorphism from $G$ to $\hat G$ is not unique;
\item
 isomorphism admits symmetry, i.e., $G\cong \hat G$ iff $\hat G\cong G$.
\end{itemize}
\end{remark}
\begin{example}
\begin{itemize}
\item
Let $V,W$ be vector spaces over $\mathbb{R}$ (or $\mathbb{C}$), then ant linear transformation $\phi: V\mapsto W$ is a \emph{homomorphism} $\phi: (V,+)\mapsto (W,+)$.
\[
\phi(\lambda\bm u+\mu\bm v) = \lambda\phi(\bm u)+\mu(\bm v),
\]
and let $\lambda=\mu=1$, we derive the homomorphismness.
\item
The determinant $\det:\mbox{GL}(n,\mathbb{R})\mapsto\mathbb{R}^{\#}:=\mathbb{R}\setminus\{0\}$ is a group homomorphism:
\[
\phi:g\mapsto\det(g)\implies
\phi(gh)=\phi(g)*\phi(h)
\]
\item
For any $n\in\mathbb{Z}^+$, we have $n\mathbb{Z}\le\mathbb{Z}$. Define the map $\phi:n\mathbb{Z}\mapsto\mathbb{Z}$ as $nk\mapsto k$, then
\[
\phi(nh+nk)=\phi(n(h+k))=h+k=\phi(nh)+\phi(nk)
\]
Then we need to show it is bijection. Each element on the range has its input, i.e., surjective. Also, take $\phi(nh)=\phi(nk)$, then $n=k$, i.e., injective.

For $n>1$, we have $n\mathbb{Z}<\mathbb{Z}$, i.e., a proper subgroup can be isomorphic to its parent group.
\item
The map $\mathbb{Z}\mapsto\mathbb{Z}$ defined by $k\mapsto nk$ is a homomorphism but not isomorphism unless $n=\pm1$:
\[
\phi(h+k)=n(h+k)=\phi(h)+\phi(k)
\]
\item
The remainder map $\phi:\mathbb{Z}\mapsto\mathbb{Z}_n$ is defined as mapping $k$ to its remainder $\bar k$divided by $n$. It is a surjective homomorphism: $\bar k\in\{0,\dots,n-1\}$ always has its input
\item
The map $\phi$ defined as $k\mapsto k+1$ is not a homomorphism:
\[
\phi(0)=1,\phi(1)=2,\phi(0+1)=2
\]

\end{itemize}
\end{example}
\begin{proposition}
The group
\[
G=\left\{\begin{pmatrix}
\cos\theta&-\sin\theta\\
\sin\theta&\cos\theta
\end{pmatrix}|\theta\in\mathbb{R}\right\}
\]
is isomorphic to $H=\{z\in\mathbb{C}||z|=1\}$ under the map 
\[\begin{pmatrix}
\cos\theta&-\sin\theta\\
\sin\theta&\cos\theta
\end{pmatrix}\mapsto e^{i\theta}
\]
\end{proposition}
\begin{proof}
First is to check the well-defineness of $\phi$. i.e., different expression of the same input leads to the same output:
\[
\begin{pmatrix}
\cos\theta&-\sin\theta\\
\sin\theta&\cos\theta
\end{pmatrix}=
\begin{pmatrix}
\cos\theta'&-\sin\theta'\\
\sin\theta'&\cos\theta'
\end{pmatrix}\implies
\theta'=\theta+2n\pi\implies
e^{i\theta}=e^{i\theta'}
\]

Then check homomorphism and bijection.

\end{proof}
\begin{proposition}
Let $\phi:G\mapsto H$ be a group homomorphism, then
\begin{enumerate}
\item
$\phi(e_G)=e_H$
\item
$\phi(g^{-1})=[\phi(g)]^{-1}$ for $\forall g\in G$
\item
$\phi(g^n) = [\phi(g)]^n$ for  $\forall g\in G$ and $n\in\mathbb{Z}$
\end{enumerate}
\end{proposition}
\begin{proof}
\[
H\ni\phi(e_G)=\phi(e_G)\phi(e_G)\implies
e_H=\phi(e_G)
\]
\end{proof}
\begin{definition}[image]
Let $\phi:G\mapsto H$ be a group homomorphism, then the \emph{image} of $\phi$ is
\[
\mbox{Im }\phi=\phi(G)=\{\phi(g)\mid g\in G\}
\]
The \emph{kernel} of $\phi$ is
\[
\mbox{ker }\phi:=\{g\in G\mid \phi(g) = e_H\}
\]
In particular, if $\mbox{ker }\phi=G$, then we say the homomorphism is \emph{trivial}.
\end{definition}
\begin{remark}
$\mbox{im }\phi\le H$ and $\mbox{ker }G\triangleleft G$.
\end{remark}
\begin{proposition}
Let $\phi$ defined above, then $\mbox{im }\phi\le H$ and $\mbox{ker }\phi\le G$
\end{proposition}
\begin{proof}
\[
a,b\in\mbox{im }\phi\implies
ab^{-1}=\phi(g)[\phi(h)]^{-1}=\phi(gh^{-1})\in\mbox{im }\phi
\]
\end{proof}
\begin{proposition}
A group homomorphism $\phi:G\mapsto H$ is injective iff $\mbox{ker }\phi=\{e_G\}$
\end{proposition}
\begin{proof}
Necessity. 

Assume $a\ne e_G$ and $a\in\mbox{ker }\phi$, then
\[
\phi(g)=\phi(g)e_H=\phi(g)\phi(a)=\phi(g*a),
\]
but $g\ne g*a$, which is a contradiction.

Sufficiency.

For any $\phi(g)=\phi(h)$, it suffices to show $g=h$:
\[
\phi(g)[\phi(h)]^{-1}=e_H\implies
\phi(gh^{-1})=e_H\implies
gh^{-1}=e_G\implies g=h.
\]
\end{proof}
\begin{proposition}
Let $G,H$ be isomorphic groups, if $G$ is cyclic, then so is $H$
\end{proposition}
\begin{proof}
Let $G=\gen{g_0}\cong H$ and $\phi:G\mapsto H$. Define $h_0=\phi(g_0)$. Take $h\in H$, there exists $n\in\mathbb{Z}$ s.t.
\[
h=\phi(g_0^n)=[\phi(g_0)]^n:=h_0^n
\]
It follows that $H\subseteq\gen{h_0}\subseteq H$, i.e., $H=\gen{h_0}$
\end{proof}
\begin{proposition}
Let $G,H$ be isomorphic groups, if $G$ is abelian, then so is $H$
\end{proposition}
\begin{proof}
For any $h_1,h_2\in H$, there exists $g_1,g_2\in G$ such that
\[
h_1h_2=\phi(g_1)\phi(g_2)=\phi(g_2)\phi(g_1)=h_2h_1.
\]
\end{proof}
Note that $D_6$ is not isomorphic to $\mathbb{Z}_6\times\mathbb{Z}_2$, since $D_6$ is not abelian.
\begin{remark}
These two propositions above still remains true if replacing isomorphism by a surjective homomorphism.
\end{remark}
\begin{proposition}
The restriction of a homomorphism $\phi:G\mapsto\hat G$ to a subgroup $H\le G$ gives a homomorphism $\phi|_H:H\mapsto\hat G$ as well.
\end{proposition}
\begin{proof}
$\phi(g_1*g_2)=\phi(g_1)*\phi(g_2)$ for $g_1,g_2\in H$
\end{proof}
\begin{proposition}
Let $G,H$ be groups s.t. $G\cong_{\phi} H$, then $|\phi(g)| = |g|$ for each $g\in G$.
\end{proposition}
\begin{proof}
Note that $n=|g|$ implies
\[
[\phi(g)]^n=e_H,
\]
i.e., $|\phi(g)|\le n$. On the other hand, assume we can take a positive integer $m<n$ s.t.
\[
[\phi(g)]^m=e_H\implies \phi(g^m)=e_H,
\]
with $g^m\ne e_G$, which implies $\phi$ is not one-to-one, which is a contradiction.
\end{proof}

\subsection{Classification of cyclic groups}
\begin{proposition}
Let $r_1$ denote the anti-clockwise rotation by $\frac{2\pi}{n}$, then $H=\gen{r_1}\le D_n$. Then $H\cong\mathbb{Z}_n$.
\end{proposition}
\begin{proof}
Define $\phi: H\mapsto\mathbb{Z}_n$ with $\phi(r_1^k)=\bar k$, $k\in\mathbb{Z}$
\begin{itemize}
\item
$\phi$ is well-defined:
\[
r_1^{k_1}=r_1^{k_2}\implies
k_2=k_1+nd,
\]
which is well-defined since $\overline{k_1+nd}=\overline{k_1}$.
\item
$\phi$ is a homomorphism: for $i,j\in\{0,\dots,n-1\}$
\[
\phi(r_1^ir_1^j)=\phi(r_1^{i+j})=\overline{i+j}=i+_nj=\phi(r_1^i)+_n\phi(r_1^j)
\]
\item
To show $\phi$ is a bijection. It suffices to show $\mbox{ker }\phi=\{e_H\}$:
\[
\phi(r_1^i)=0\implies i=nd,d\in\mathbb{Z}\implies
r_1^i=r_0
\]
\end{itemize}



\end{proof}

\begin{theorem}
Let $G$ be a cyclic group, then
\begin{enumerate}
\item
If $|G|=\infty$, then $G\cong\mathbb{Z}$
\item
If $|G|=n$, then $G\cong\mathbb{Z}_n$
\end{enumerate}
 
\end{theorem}
\begin{proof}
Define $\phi:G\mapsto\mathbb{Z}$ with $g_0^k\mapsto k$

First show the well-defineness of $\phi$; then show $\phi$ is homomorphic:
\[
\phi(g_0^m*g_0^n)=\phi(g_0^m)+\phi(g_0^n)
\]

Then show that $\phi$ is bijection, i.e., $\mbox{ker }\phi=\{e_G\}.$

For the second case, define the map $\phi:\mathbb{Z}_n\mapsto G$ with $k\mapsto g_0^k$:

Check the well-defineness, which is clear since the expresison for $k$ is unique.

$\phi$ is homomorphism:
\[
\phi(h+_nk)=\phi(\overline{h+k})=g_0^{\overline{h+k}}
=g_0^{h+k}=g_0^hg_0^k=\phi(h)\phi(k)
\]

Then show that it is bijection. A one-to-one function from a finite set to itself is onto. Then check one-to-one mapping.

\end{proof}
\begin{corollary}
Let $G,\hat G$ be cyclic groups of the same order, then $G\cong \hat G$.
\end{corollary}

\subsection{Isomorphism Theorems}
The first and seond theorem is required in exam. (can we apply the corresponding theorem in the exam?)

\begin{theorem}[The First Isomorphism Theorem]
Let $G\mapsto H$ be a \emph{surjective} group homomorphism, then $\mbox{ker }\phi\triangleleft G$ and $G/\mbox{ker }\phi\cong\mbox{im }\phi$
\end{theorem}

















