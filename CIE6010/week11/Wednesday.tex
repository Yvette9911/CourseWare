
\section{Wednesday}\index{week7_Thursday_lecture}
\subsection{Review}
\begin{enumerate}
\item
To minimize the term $\|A - XX\trans\|_F^2$, just implement the algorithm in the paper. 
\item
The next question is to solve the linear system by Newton's method
\[
F(x+dx,y+dy,z+dz)=\begin{bmatrix}
A\trans y+z - c\\
Ax-b\\
x\circ z - \mu
\end{bmatrix}=\begin{bmatrix}
0\\0\\0
\end{bmatrix}\implies
F_\mu(x,y,z)+F'_{\mu}(x,y,z)(dx,dy,dz)\trans=0
\]
\end{enumerate}
\paragraph{Stochastic Gradient Method}
The objective function is
\[
f(x)=\frac{1}{2}(x-c_1)^2+\frac{1}{2}(x-c_2)^2,
\]
where $x^*=\frac{c_1+c_2}{2}$.
\[
\left\{
\begin{aligned}
%x^r(1) &= x^{r-1}(2) - \alpha (x^{r-1}(2)-c_2)\\
x^r(2) &= x^r(1) - \alpha (x^r(1)-c_1)=(1-\alpha)x^r(1) + \alpha c_1\\
x^{r+1}(1)&=x^r(2) - \alpha(x^r(2) - c_2)=(1-\alpha)x^r(2) + \alpha c_2
\end{aligned}
\right.
\]
After computation, we derive $x^{r+1}(1) = (1-\alpha)^2x^r(1) + (1-\alpha)\alpha c_1+\alpha c_2$.

How to guarantee convergence? We have generated two sequences
\[
\{
x^{r-1}(1).x^r(1),x^{r+1}(1),\dots
\}\to x_\alpha(1),\qquad
\{
x^{r-1}(2).x^r(2),x^{r+1}(2),\dots
\}\to x_\alpha(2),
\]
We can solve for $x_\alpha(1)$ and $x_\alpha(2)$:
\begin{align*}
x_\alpha(1)&=(1-\alpha)^2x_\alpha(1) + (1-\alpha)\alpha c_1+\alpha c_2\implies
x_\alpha(1)=\frac{(1-\alpha)c_1+c_2}{2-\alpha}\\
x_\alpha(2)&=\frac{(1-\alpha)c_2+c_1}{2-\alpha}
\end{align*}
For fixed $\alpha$ we can see that those two limits are not the same. From the formula above we can see that as $\alpha\to0$, $x_\alpha(1)\to\frac{c_1+c_2}{2}$ and $x_\alpha(2)\to\frac{c_1+c_2}{2}$.
\begin{remark}
Note that $\alpha^r$ cannot converge too slow or too fast, the right setting is $\alpha^r=\frac{1}{r}$.
\end{remark}
\subsection{Dual-Primal of LP}
Given the primal linear programming problem
\[
\begin{array}{ll}
\min&c\trans x\\
\mbox{such that}&Ax=b\\
&x\ge0
\end{array}
\]
\paragraph{Recap}
Recall how to solve the general optimization problem
\[
\begin{array}{ll}
\min&f(x)\\
\mbox{such that}&Ax=b\\
\end{array}
\]
The 1st order necessary condition is 
\[
\left\{
\begin{aligned}
\nabla f(x)&=A\trans\lambda\\
Ax&=b
\end{aligned}
\right.
\]
Or rewrite it into a system $F(x,\lambda)=0$, which can be solved by Newton's method.
\paragraph{Barrier Term}
But the LP problem has the inequality $x\ge0$, we add a log barrier $B_\mu(x)=-\mu\sum\log x_i$ with $\mu>0$ to sovle this problem:
\[
\begin{array}{ll}
\min&c\trans+B_\mu(x)\\
\mbox{such that}&Ax=b
\end{array}
\]
Solving this problem is equivalent to solving the system $F_\mu(x,y,z)=0$ in assignment 4. As $\mu\to0$, it suffices to solve the original problem, i.e., $(x(\mu),y(\mu),z(\mu))\to(x^*,y^*,z^*)$.
\begin{remark}
There is a faster method. Every iteration just approximate $\mu$, and after new iteration, $\mu$ is updated.
\end{remark}

\begin{enumerate}
\item
Change step-size to make convergence rate faster
\item
Appreciate the use of dense matrix $A$.
\end{enumerate}

\paragraph{Dual Problem of LP}
\begin{enumerate}
\item
We form the \emph{Lagrange function} of origin LP:
\[
L(x,y)=c\trans x-y\trans(Ax-b)
\]
(We take care of the constraint $x\ge0$ in next step).
\item
The \emph{dual function} is a function of $y$:
\begin{align*}
Q(y) &= \inf_{x\ge0}L(x,y)\\
&=\inf_{x\ge0}(c-A\trans y)\trans x+b\trans y\\
&=\left\{
\begin{aligned}
b\trans y,&\quad c-A\trans y\ge0\\
-\infty&\quad \mbox{otherwise}
\end{aligned}
\right.
\end{align*}
\item
The \emph{dual problem} is given by:
\[
\begin{array}{ll}
\max_{y\in\mathbb{R}^m}&Q(y)=\max_{y}b\trans y\\
\mbox{such that}&A\trans y\le c
\end{array}
\]
The constraint is because that violeting the constraint makes the objective goes to minus infinite.

Then we introduce a new variable $z\ge0$:
 \[
\begin{array}{ll}
\max&b\trans y\\
\mbox{such that}&A\trans y+z= c\\
&z\ge0
\end{array}\qquad\mbox{Dual}
\]
\end{enumerate}
\begin{itemize}
\item
The primal and dual problem of LP is that they share the same $(A,b,c)$
\item
Let $x$ be P-feasible, $(y,z)$ be D-feasible, then $c\trans x-b\trans y\ge0$ (\emph{weak duality}):
\begin{align*}
c\trans x-b\trans y&=x\trans c-x\trans A\trans y\\
&=x\trans(c-A\trans y)=x\trans z\ge0
\end{align*}
\item
For linear programming, the optimal solution satisfies $c\trans x^* = b\trans y^*$, i.e., $x\circ z=0$.
(\emph{strong duality})
\item
The optimality condition for the primial-dual problem is:
\begin{align*}
A\trans y+z-c&=0,\quad z\ge0\\
Ax-b&=0,\quad x\ge0\\
x\circ z&=0
\end{align*}
which means that the optimal solution of LP must satisfy both constraints from primal and dual. (The assignment 4 makes $x\circ z-\mu=0$ and let $\mu\to0$.)
\end{itemize}












