
\section{Wednesday}\index{week7_Thursday_lecture}
\subsection{Comments on Assignment 6}
We will have one more assignment this week. Then we are faced with the final project! Let's discuss the detail of the assignment 6.
\begin{equation}\label{Eq:11:8}
\begin{array}{ll}
\min&f(\bm x,\bm y)=\frac{1}{2}(\bm x\trans\bm A\bm x+\bm y\trans\bm A\bm y) - \bm b\trans\bm x-\bm c\trans\bm y\\
&\bm x\trans\bm x=1\\
&\bm y\trans\bm y=1\\
&\bm x\trans\bm y=0
\end{array}
\end{equation}
You can try quadratic penalty method, or Newton's method. The more general problem setting is
\begin{equation}
\begin{array}{ll}
\min&f(\bm Z)=\frac{1}{2}\trace(\bm Z\trans\bm A\bm Z) - \trace(\bm B\trans\bm Z)\\
&\bm Z\trans\bm Z=\bm I
\end{array}
\end{equation}
The (\ref{Eq:11:8}) is a special case for this problem with $\bm Z:=[\bm x, \bm y]$. 
\begin{itemize}
\item
For $\bm A=\bm0$, this problem becomes easy to solve, with the solution 
\[
\bm Z=\bm U\bm V\trans,
\]
where the matrices $\bm U$ and $\bm V$ are from the SVD: $\bm B=\bm U\bm\Sigma\bm V\trans$.
\item
For $\bm B=\bm0$, this problem is easy to solve as well. This problem is so called the \emph{eigenvalue problem}. It suffices to find the eigen-space for the minimum several eigen-values.
\item
For $\bm A,\bm B\ne\bm0$, one approach is to solve
\[
\begin{array}{ll}
\min&\frac{1}{2}(\bm x\trans\bm A\bm x+\bm y\trans\bm A\bm y) - \bm b\trans\bm x-\bm c\trans\bm y+\frac{c}{2}(\bm x\trans\bm y)^2\\
&\bm x\trans\bm x=1\\
&\bm y\trans\bm y=1
\end{array}
\]
We apply the \emph{Block Coordinate Descent Method}, i.e., to minimize $f(x_1,\dots,x_s)$, then in each iteration just minimize one-variable and fix others:
\[
x_i^{r+1}=\arg\min_{x_i}f(x_1^{r+1},\dots,x^{r+1}_{i-1},x_i,x_{i+1}^r,x_{i+2}^r,\dots)
\]
An alternative approach is by solving the trust region problem
\[
\begin{array}{ll}
\min&q(\bm x)\\
&\bm x\trans\bm x=1
\end{array}
\]
Suppose $q(\bm x)=\frac{1}{2}\bm x\trans\bm A\bm x-\bm b\trans\bm x$, then the necessary and sufficient optimality condition is
\[
\left\{
\begin{aligned}
(\bm A\bm x+\lambda\bm I)\bm x&=\bm b\qquad (\mbox{Complementarity})\\
\bm x\trans\bm x&=1\qquad(\mbox{Primal Feasible})\\
\lambda\ge0,\bm A+\lambda\bm I\succeq&0\qquad(\mbox{Dual Feasible})
\end{aligned}
\right.
\]
\end{itemize}
Btw, the final project in this semester will be interesting, as promised. It might be open-ending.

\subsection{Inequality Constraint Problem}
For the optimization problem
\begin{equation}\label{Eq:11:10}
\begin{array}{ll}
\min&f(x)\\
&h(x)=0\\
&g(x)\le0
\end{array}
\end{equation}
\begin{itemize}
\item
One approach is to apply \emph{penalty-barrier method}, i.e., solve the unconstraint problem
\begin{equation}
x(c)=\arg\min_{x} f(x)+\frac{c}{2}\|h(x)\|^2-\frac{1}{c}\sum\log(-g(x))
\end{equation}
Taking $c\to\infty$, we obtain $x(c)\to x^*$. However, as $c$ goes large, such a problem is difficult to solve.
\item
Another approach is to transform (\ref{Eq:11:7}) into equality constraint problem by setting $\phi(x):=\max(0,g(x))=0$, and then solve by penalty method. The function $\|\phi(x)\|^2$ is continuosly differentiable, but not second-order differentiable at $x$ such that $g_i(x)=0$.
\item
Alternative approach is to introduce the slack variable, i.e., $g(x)\le 0$ is equivalent to $g(x)+z=0$ and $z\ge0$. The inequality for such case is trivial.
\item
Heuristically, we add the squared slack variable, i.e., $g(x)\le0$ is equivalent to $g(x)+z^2=0$. In this case, no inequality is intriduced. The disadvantage is that the problem will become non-convex in most case.
\begin{quotation}
Even $\{x\mid g(x)\le0\}$ is convex, the set $\{(x,z)\mid g(x)+z^2=0\}$ is not convex.
\end{quotation}
\end{itemize}
Even we have so many methods to transform (\ref{Eq:11:10}) into equality constraint, in most case we still aim to solve the origin problem directly.
\begin{remark}
Nonlinear primal-dual interior-point method also work in some cases.
\end{remark}

\subsection{Non-smooth unconstraint problem}
For the unconstraint optimization problem $\min f(x)$ with non-smooth but convex function $f$, we don't obtain the usual optimality condition. Let's extend the definition of gradient. For convex case we have
\[
f(\bm y)\ge f(\bm x)+\nabla\trans f(\bm x)(\bm y-\bm x)
\]
\begin{definition}[Subgradient]
The subgradient of non-smooth function $f$, \emph{are} the vector $\bm g\in\mathbb{R}^n$, such that
\begin{equation}
f(\bm y)\ge f(\bm x)+\bm g\trans(\bm y-\bm x),\forall\bm y
\end{equation}
\end{definition}
Note that the subgradient may not be unique.
\begin{definition}[Sub-differential]
The Sub-differential of $f$ is the set
\begin{equation}
\partial f(\bm x)=\{g\mid \mbox{$g$ is a subgradient of $f$ at $\bm x$}\}
\end{equation}
\end{definition}
The sub-differential is always a closed and convex set.

For example, the sub-differential of $f(x)=|x|$ is given by:
\[
\partial f(\bm x)=\left\{
\begin{aligned}
\{-1\},&x<0\\
[-1,1],&x=0\\
\{1\},&x>0
\end{aligned}
\right.\]
\begin{proposition}
$\bm x$ is a global minimum of $f$ iff $0\in\partial f(\bm x)$.
\end{proposition}
\begin{remark}
The sub-differential gives a necessary and sufficient condition on global optimality, but it is hard to compute in general.
\end{remark}















