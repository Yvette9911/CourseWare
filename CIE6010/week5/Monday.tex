
\chapter{Week5}

\section{Monday}\index{week5_Friday_lecture}
\subsection{Review}
\paragraph{Optimality Condition}
Given a general problem
\[
\begin{array}{ll}
\min&f(x)\\
\mbox{s.t.}&x\in X\subseteq\mathbb{R}^n\\
&f\mbox{ is }\mathcal{C}^1\mbox{ or }\mathcal{C}^2
\end{array}
\]
One of the most important thing is the optimality condition.
\begin{itemize}
\item
Unconstrainted: $X=\mathbb{R}^n$. (First order and second order)
\item
Constrainted:
\begin{itemize}
\item
1st order necessary condition: Let $\bm x^*$ be a local minimum, then
\[
\inp{\nabla f(\bm x^*)}{(\bm x-\bm x^*)}\ge0,\forall x\in X
\]
\item
For convex function $f$, the above is also the sufficient condition, since
\[
f(\bm x)\ge f(\bm x^*)+\inp{\nabla f(\bm x^*)}{(\bm x-\bm x^*)},\forall x,x^*\in X.
\]
\end{itemize}
\end{itemize}

The optimality condition for constrainted problem is difficult to check. We want a more efficient way, which will be discussed later.
\paragraph{Iterative descent methods}
For unconstraint problem, we consider the iterative descent methods:
\[
\begin{array}{ll}
\bm x \leftarrow \bm x-\alpha\cdot\bm D\cdot\nabla f(\bm x)
&
(\bm D\succ0)
\end{array}
\]
\begin{itemize}
\item
If $\bm D=\bm I$, it is the first order (gradient) method.
\item
If $\bm D=\left(\nabla^2f(\bm x)\right)^{-1}$, it is the second order (Newton's) method
\begin{remark}
Sometimes it is difficult to compute $\nabla^2f(\bm x)$. We can apply finite difference method to accurately approximate the Hessian matrix.
\end{remark}
\item
If $\bm D=(\bm J\trans\bm J)^{-1}$ with Jacobian matrix for nonlinear least squares problem, it is the Gauss-Newton method.
\item
Sometimes we apply rough method to approximate the Hessian matrix in-accurately, which is called the \emph{Quasi-Newton} method. The most famous one is BFGS (L-BFGS).
\end{itemize}
There are more generalized iterative descent methods, such as the accelerated descent method tried in Assignment 3.
\paragraph{Reading materials} CG-conjugate gradient methods; and Nestorov's method (\emph{optimal} accelerated method in \emph{worse} case).

There is a method which is much faster than Nestorov's method in most cases:
\begin{align*}
\bm D^r&=\frac{1}{L}\bm I+\frac{\bm S^r (\bm S^r)\trans}{(\bm y^r)\trans(\bm y^r)}\succ0\\
\alpha&=1\\
\bm S^r&=\bm x^{r+1} - \bm x^r\\
\bm y^r&=\nabla f(\bm x^{r+1}) - \nabla f(\bm x^r)
\end{align*}
\paragraph{Step-size}
\begin{itemize}
\item
Back-tracking with Amijo condition:
\[
\begin{array}{ll}
\mbox{Amijo condition}
&
f(\bm x+\alpha\bm d)\le f(\bm x)+C_1\alpha\inp{\nabla f(\bm x)}{\bm d},0<C_1<1
\end{array}
\]
\item
Wolfe condition for line search:
\[
\begin{array}{ll}
\mbox{Wolfe condition}
&
\left\{
\begin{aligned}
f(\bm x+\alpha\bm d)&\le f(\bm x)+C_1\alpha\inp{\nabla f(\bm x)}{\bm d},&0<C_1<1\\
\inp{\nabla f(\bm x+\alpha\bm d)}{\bm d}&\ge C_2\inp{\nabla f(\bm x)}{\bm d}, &0<C_2<1
\end{aligned}
\right.
\end{array}
\]
Define $h(\alpha)=h(\bm x+\alpha\bm d)$, then the Wolfe condition is essentially
\[
\left\{
\begin{aligned}
h(\alpha)&\le h(0)+C_1h'(0)\\
h'(\alpha)&\ge C_2h'(0)
\end{aligned}
\right.
\]
\item
Constant step-size: $\alpha^r\equiv\frac{1}{L}$ with $L$ be the Lipschitz constant of $\nabla f(\bm x)$.
\item
$\alpha^r\to0$ with $\sum\alpha^r=+\infty$.
\end{itemize}
\begin{remark}
Amijo condition guarantees that $f(\bm x^r) - f(\bm x^{r+1})\ge-C_1\alpha^r\inp{\nabla f(\bm x^r)}{\bm d^r}$. Assume $f(\bm x)>-\infty$, then
\[
\alpha^r\inp{\nabla f(\bm x^r)}{\bm d^r}\to0
\]
We want $\nabla f(\bm x^r)\to0$, which means your direction $\bm d^r$ should not be perpendicular to $\nabla f(\bm x^r)$ after some iterations. If choosing $\bm d^r=-\nabla f(\bm x^r)$, then $\alpha^r\|\nabla f(\bm x^r)\|^2\to0$, which implies $\nabla f(\bm x^r)\to0$.
\end{remark}
Under reasonable conditions, applying first order condition we expect $\nabla f(\bm x^r)\to0$. Is $\bm x^r$ always convergent? not necessarily.

\paragraph{Local convergence rate}
The first order method has linear or sub-linear convergence rate; while the second order method has quadratic convergence rate.

\paragraph{Finite difference Method}
Given $F(\bm x): \mathbb{R}^n\mapsto\mathbb{R}^n$, its Jacobian is given by:
\[
F'(\bm x)=\begin{bmatrix}
\nabla\trans F_1(\bm x)\\
\vdots\\
\nabla\trans F_n(\bm x)
\end{bmatrix}
\]
Its $j$th column is given by:
\begin{align*}
F'(\bm x)\bm e_j:&=\lim_{h\to0}\frac{F(\bm x+h\bm e_j) - F(\bm x)}{h}\\
&\approx \frac{F(\bm x+h\bm e_j) - F(\bm x)}{h}\mbox{ for small }h
\end{align*}
where for $\varepsilon=10^{-8}$,
\[
h=\varepsilon\max\{1,|x_j|\}\mbox{sign}(x_j),
\]
more-multiplying the term $\mbox{sign}(x_j)$ means we avoid subtract between $\bm x$ and $h\bm e_j$.


















