
\chapter{Week5}

\section{Tuesday}\index{week5_Tuesday_lecture}
\subsection{Formulas for Determinant}
We want to use the \emph{3 basic properties} to derive the formula for determinant:
\begin{enumerate}
\item
\emph{The determinant of the $\bm n$ by $\bm n$ identity matrix is 1.}
\begin{gather*}
\begin{vmatrix}1&0\\0&1\end{vmatrix}=1\qquad\text{and}\qquad\begin{vmatrix}
1&&\\&\ddots&\\&&1\end{vmatrix}=1.
\end{gather*}
\item
\emph{The determianant changes sign when two rows are exchanged.} (sign reversal)
\begin{gather*}
\text{Check:   }\begin{vmatrix}c&d\\a&b\end{vmatrix}=-\begin{vmatrix}a&b\\c&d\end{vmatrix}\qquad\text{(both sides equal $bc-ad$)}.
\end{gather*}
\item
\emph{The determinant is a linear function of each row separately.} (all other rows stay fixed).
\begin{gather*}
\text{\emph{multiply row 1 by any number $\bm t$}}\qquad\begin{vmatrix}ta&tb\\c&d\end{vmatrix}=t\begin{vmatrix}a&b\\c&d\end{vmatrix}
\end{gather*}
\begin{gather*}
\text{\emph{Add row 1 of $\bm A$ to row 1 of $\bm B$:}}\qquad\begin{vmatrix}a_1+a_2&b_1+b_2\\c&d\end{vmatrix}=\begin{vmatrix}a_1&b_1\\c&d\end{vmatrix}+\begin{vmatrix}a_2&b_2\\c&d\end{vmatrix}
\end{gather*}
\end{enumerate}
Although we derive the formula for $\det\bm A$ is $\det\bm A=\pm\prod_i\text{pivots}_{i}$ (product of pivots), it is not \emph{explicit}. We begin some example to show how to derive the explicit formula for determinant.
\begin{example}
To derive the formula for determinant, let's start with $n=2$.

Given $\bm A=\begin{bmatrix}
a&b\\c&d
\end{bmatrix}$, our goal is to get $\det(\bm A) = ad-bc$.

We can break each row into two simpler rows:
\[
\begin{vmatrix}
a&b
\end{vmatrix}=\begin{vmatrix}
a&0
\end{vmatrix}+\begin{vmatrix}
0&b
\end{vmatrix}\qquad\text{and}\qquad
\begin{vmatrix}
c&d
\end{vmatrix}=\begin{vmatrix}
c&0
\end{vmatrix}+\begin{vmatrix}
0&d
\end{vmatrix}
\]
Now apply property 3, first in row 1(with row 2 fixed) and then in row 2(with row 1 fixed):
\[
\begin{aligned}
\begin{vmatrix}
a&b\\c&d
\end{vmatrix}&=\begin{vmatrix}
a&0\\c&d
\end{vmatrix}+\begin{vmatrix}
0&b\\c&d
\end{vmatrix}\\&=\begin{vmatrix}
a&0\\c&0
\end{vmatrix}+\begin{vmatrix}
a&0\\0&d
\end{vmatrix}+\begin{vmatrix}
0&b\\c&0
\end{vmatrix}+\begin{vmatrix}
0&b\\0&d
\end{vmatrix}
\end{aligned}
\]
The last line has $2^2=4$ determinants. The first and fourth are zero since their rows are \emph{dep.} (one row is a multiple of the other row.) We left two terms to compute:
\[
\begin{vmatrix}
a&0\\0&d
\end{vmatrix}+\begin{vmatrix}
0&b\\c&0
\end{vmatrix}=ad\begin{vmatrix}
1&0\\0&1
\end{vmatrix}+bc\begin{vmatrix}
0&1\\1&0
\end{vmatrix}=ad-bc
\]
The permutation matrices $\begin{bmatrix}
1&0\\0&1
\end{bmatrix}$ and $\begin{bmatrix}
0&1\\1&0
\end{bmatrix}$ have determinant $+1$ or $-1$.
\end{example}
\begin{example}
Now we try $n=3$. Each row splits into 3 simpler rows such as $\begin{bmatrix}
a_{11}&0&0
\end{bmatrix}$. Hence $\det\bm A$ will split into $3^3=27$ simple determinants. For simple determinant, if one column has two nonzero entries, (For example, $\begin{bmatrix}
a_{11}&0&0\\a_{21}&0&0\\0&0&a_{33}
\end{bmatrix}$), then its determinant will be zero.\\
Hence we only need to foucus on the matrix that \emph{the nonzero terms come from defferent columns}:
\[
\begin{aligned}
\begin{vmatrix}
a_{11}&a_{12}&a_{13}\\a_{21}&a_{22}&a_{23}\\a_{31}&a_{32}&a_{33}
\end{vmatrix}
&=
\begin{vmatrix}
a_{11}&&\\&a_{22}&\\&&a_{33}
\end{vmatrix}
+
\begin{vmatrix}
&a_{12}&\\&&a_{23}\\a_{31}&&
\end{vmatrix}
+
\begin{vmatrix}
&&a_{13}\\a_{21}&&\\&a_{32}&
\end{vmatrix}\\&+
\begin{vmatrix}
a_{11}&&\\&&a_{23}\\&a_{32}&
\end{vmatrix}
+
\begin{vmatrix}
&a_{12}&\\a_{21}&&\\&&a_{33}
\end{vmatrix}
+
\begin{vmatrix}
&&a_{13}\\&a_{22}&\\a_{31}&&
\end{vmatrix}
\end{aligned}
\]
There are $3!=6$ ways to permutate the three columns, so there leaves six determinants. The six permutations of $(1,2,3)$ is given by:
\[
\mbox{{\emph{Column numbers}}} = (1,2,3),(2,3,1),(3,1,2),(1,3,2),(2,1,3),(3,2,1).
\]
The last three are \textit{odd permutations} (One exchange from identity permutation $(1,2,3)$.) The first three are \textit{even permutations}. (zero or two exchange from identity permutation $(1,2,3)$.) When the column number is $(\alpha,\beta,\omega)$, we get the entries $a_{1\alpha},a_{2\beta},a_{3\omega}$.  The permutation $(\alpha,\beta,\omega)$ comes with a plus or minus sign. If you don't understand, look at example below:
\[
\begin{aligned}
\det\bm A&=
a_{11}a_{22}a_{33}\begin{vmatrix}
1&&\\&1&\\&&1
\end{vmatrix}
+
a_{12}a_{23}a_{31}\begin{vmatrix}
&1&\\&&1\\1&&
\end{vmatrix}
+
a_{13}a_{21}a_{32}\begin{vmatrix}
&&1\\1&&\\&1&
\end{vmatrix}\\&+
a_{11}a_{23}a_{32}\begin{vmatrix}
1&&\\&&1\\&1&
\end{vmatrix}
+
a_{12}a_{21}a_{33}\begin{vmatrix}
&1&\\1&&\\&&1
\end{vmatrix}
+
a_{13}a_{22}a_{31}\begin{vmatrix}
&&1\\&1&\\1&&
\end{vmatrix}
\end{aligned}
\]
The first three (even) permutation matrices have $\det\bm P=+1$, the last three (odd) permutation matrices have $\det\bm P=-1$. Hence we have:
\[
\begin{aligned}
\det\bm A&=a_{11}a_{22}a_{33}+
a_{12}a_{23}a_{31}+
a_{13}a_{21}a_{32}-
a_{11}a_{23}a_{32}-
a_{12}a_{21}a_{33}-
a_{13}a_{22}a_{31}\\
&=a_{11}(a_{22}a_{33} - a_{23}a_{32})+a_{12}(a_{23}a_{31}-a_{21}a_{33})+a_{13}(a_{21}a_{32}-a_{22}a_{31})
\end{aligned}
\]
\end{example}
\subsubsection{$n$ by $n$ formula of determinant}
Now we can see $n$ by $n$ formula. There are $n!$ permutations of columns, so we have $n!$ terms for determinant. 

Assuming $(\alpha,\beta,\dots,\omega)$ is the permutation of $(1,2,\dots,n)$. The coorsponding term is $a_{1\alpha}a_{2\beta}\dots a_{n\omega}\det\bm P$, where $\bm P$ is the permutation matrix with column number $\alpha,\beta,\dots,\omega$.

The complete determinant of $\bm A$ is the sum of these $n!$ simple determinants. $a_{1\alpha}a_{2\beta}\dots a_{n\omega}$ is obtained by choosing \emph{one entry from every row and every column:}
\begin{definition}[Big formula for determinant]
\[
\begin{aligned}
\det\bm A&=\text{sum of all $n!$ column permutations}\\
&=\sum(\det\bm P)a_{1\alpha}a_{2\beta}\dots a_{n\omega}=\text{\emph{BIG FORMULA}}
\end{aligned}
\]
where $\bm P$ is permutation matrix with column number $(\alpha,\beta,\dots,\omega)$. And $\{\alpha,\beta,\dots,\omega\}$ is a permutation of $\{1,2,\dots,n\}$.
\end{definition}
\begin{remark}
\paragraph{Complexity Analysis}
However, if we want to use big formula to compute matrix, we need to do $n!(n-1)$ multiplications. If we use formula $\det\bm A=\pm\prod pivots$, we only need to do $O(n^3)$ multiplications. Hence the letter one is more efficient.
\end{remark}
\subsubsection{Verify property}
We can also use the big formula to verify property 1 to property 3:
\begin{itemize}
\item $\det\bm I=1$:\\
Only when $(\alpha,\beta,\dots,\omega)$=$(1,2,\dots,n)$, there is no zero entries for $a_{1\alpha}a_{2\beta}\dots a_{n\omega}$. Hence $\det\bm A=a_{11}a_{22}\dots a_{nn}=1$.
\item \emph{sign reversal}:\\
If two rows are interchanged, then all determinant of permutation matrix will change its sign, hence the value for determinant $\bm A$ is opposite.
\item \emph{The determinant is a linear function of each row separately}.\\
\emph{If we separate out the fator $a_{11},a_{12},\dots,a_{1\alpha}$ that comes from the first row}, this property is easy to check. For 3 by 3 matrix, separate the usual 6 terms of the determinant into 3 pairs:
\[
\det\bm A=a_{11}(a_{22}a_{33}-a_{23}a_{32})+a_{12}(a_{23}a_{31}-a_{21}a_{33})+a_{13}(a_{21}a_{32}-a_{22}a_{31}).
\]
Those three quantities in parentheses are called \emph{cofactors}. They are $2\times 2$ determinant coming from matrices in row 2 and 3. The first row contributes the factors $a_{11},a_{12},a_{13}$. The lower rows contribute the cofactors $(a_{22}a_{33}-a_{23}a_{32}),(a_{23}a_{31}-a_{21}a_{33}),(a_{21}a_{32}-a_{22}a_{31})$. Certainly $\det\bm A$ depends \emph{linearly} on $a_{11},a_{12},a_{13}$, which is property 3.
\end{itemize}
\subsection{Determinant by Cofactors}
We could write the determinant in this form:
\[
\begin{vmatrix}
a_{11}&a_{12}&a_{13}\\a_{21}&a_{22}&a_{23}\\a_{31}&a_{32}&a_{33}
\end{vmatrix}=
\begin{vmatrix}
a_{11}&&\\&a_{22}&a_{23}\\&a_{32}&a_{33}
\end{vmatrix}+
\begin{vmatrix}
&a_{12}&\\a_{21}&&a_{23}\\a_{31}&&a_{33}
\end{vmatrix}+
\begin{vmatrix}
&&a_{13}\\a_{21}&a_{22}&\\a_{31}&a_{32}&
\end{vmatrix}.
\]
If we define $\bm A_{1j}$ to be \emph{the submatrix obtained by removing row 1 and column j}, We could compute $\det\bm A$ in this way:
\begin{gather*}
\text{The cofactors along row 1 are } C_{1j}=(-1)^{1+j}\det\bm A_{1j}\quad j=1,2,\dots,n.\\
\text{\emph{The cofactor expansion is }} \det\bm A=a_{11}C_{11}+a_{12}C_{12}+\dots+a_{1n}C_{1n}.
\end{gather*}
More generally, we can cross row $i$ to get the determinant:
\begin{definition}[Determinant]
The determinant is the \emph{dot product} of any row $i$ of $\bm A$ with its cofactors using other rows:
\[
\text{\emph{Cofactor Formula}}\qquad\det\bm A=
a_{i1}C_{i1}+a_{i2}C_{i2}+\dots+a_{in}C_{in}.
\]
Each cofactor $C_{ij}$ is defined as:
\[
\text{\emph{Cofactor}}\qquad C_{ij}=(-1)^{i+j}\det\bm A_{ij}
\]
where $A_{ij}$ is the submatrix obtained by removing row $i$ and column $j$.
\end{definition}

\paragraph{Cofactors down a column} Since we have $\det\bm A=\det\bm A\trans$, we can expand the determinant in cofactors \textit{down a column} instead of across a row. Down column $j$ the entries are $a_{1j}$ to $a_{nj}$, the cofactors are $C_{1j}$ to $C_{nj}$. The determinant is given by:
\[
\begin{array}{ll}
\mbox{\emph{Cofactors down column $j$:}}
&
\det\bm A=a_{1j}C_{1j}+a_{2j}C_{2j}+\dots+a_{nj}C_{nj}.
\end{array}
\]
\subsection{Determinant Applications}
\subsubsection{Inverse}
It's easy to check that the inverse of 2 by 2 matrix $\bm A$ is 
\[
\begin{bmatrix}
a&b\\c&d
\end{bmatrix}^{-1}=\frac{1}{ad-bc}\begin{bmatrix}
d&-b\\-c&a
\end{bmatrix}=\frac{1}{\det\bm A}\begin{bmatrix}
d&-b\\-c&a
\end{bmatrix}.
\]
We could use determinant to compute inverse! Before that let's define \emph{cofactor matrix}:
\begin{definition}[cofactor matrix]
The cofactor matrix of $n\times n$ matrix $\bm A$ is given by:
\[
\bm C=\begin{bmatrix}
C_{ij}
\end{bmatrix}_{1\le i,j\le n}
\]
where $C_{ij}$ is the cofactor of $\bm A$.
\end{definition}
Then we try to derive the inverse of matrix $\bm A$. 
\begin{quotation}
For $n\times n$ matrix $\bm A$, the product of $\bm A$ and the \emph{transpose} of \textit{cofactor matrix} is given by:
\begin{equation}\label{11.1}
\bm A\bm C\trans=\begin{bmatrix}
a_{11}&\dots&a_{1n}\\\vdots&\ddots&\vdots\\a_{n1}&\dots&a_{nn}
\end{bmatrix}\begin{bmatrix}
C_{11}&\dots&C_{n1}\\\vdots&\ddots&\vdots\\C_{1n}&\dots&C_{nn}
\end{bmatrix}=\begin{bmatrix}
\det\bm A&&\\&\det\bm A&\\&&\det\bm A
\end{bmatrix}
\end{equation}
\end{quotation}
\emph{Proofoutline:} 
\begin{itemize}
\item
Row 1 of $\bm A$ times the column 1 of $\bm C\trans$ yields the first $\det\bm A$ on the right:
\[
a_{11}C_{11}+a_{12}C_{12}+\dots+a_{1n}C_{1n}=\det\bm A
\]
Similarly, row $j$ of $\bm A$ times column $j$ of $\bm C\trans$ yields the determinant.
\item
\textit{How to explain the zeros off the main diagonal in equation (\ref{11.1})?} Rows of $\bm A$ are multiplying $\bm C\trans$ from \emph{different} columns. Why is the answer zero? For example, the (2,1)th entry of the result is given by
\begin{equation}
\begin{aligned}
\text{\emph{Row 2 of $\bm A$}}\\
\text{\emph{Row 1 of $\bm C$}}
\end{aligned}
\qquad\qquad
a_{21}C_{11}+a_{22}C_{12}+\dots+a_{2n}C_{1n}=0.\label{Eq:6:2}
\end{equation}

\textit{Explaination for Eq.(\ref{Eq:6:2}): }
If the second row of $\bm A$ is copied into its first row, we define this new matrix as $\bm A^{*}$. Thus the determinant of $\bm A^{*}$ is given by:
\[
\begin{vmatrix}
a_{21}&a_{22}&\dots&a_{2n}\\
a_{21}&a_{22}&\dots&a_{2n}\\
a_{31}&a_{32}&\dots&a_{3n}\\
\vdots&\vdots&\ddots&\vdots\\
a_{n1}&a_{n2}&\dots&a_{nn}
\end{vmatrix}
=
\begin{vmatrix}
a_{21}&&&\\
&a_{22}&\dots&a_{2n}\\
&a_{32}&\dots&a_{3n}\\
&\vdots&\ddots&\vdots\\
&a_{n2}&\dots&a_{nn}
\end{vmatrix}
+
\begin{vmatrix}
&a_{22}&&\\
a_{21}&&\dots&a_{2n}\\
a_{31}&&\dots&a_{3n}\\
\vdots&&\ddots&\vdots\\
a_{n1}&&\dots&a_{nn}
\end{vmatrix}
+\dots+
\begin{vmatrix}
&&&a_{2n}\\
a_{21}&a_{22}&a_{2(n-1)}&\\
a_{31}&a_{32}&a_{3(n-1)}&\\
\vdots&\vdots&\vdots&\\
a_{n1}&a_{n2}&a_{n(n-1)}&
\end{vmatrix}
\]
Or equivalently, we have
\[\det\bm A^{*}=
\begin{vmatrix}
a_{21}&a_{22}&\dots&a_{2n}\\
a_{21}&a_{22}&\dots&a_{2n}\\
a_{31}&a_{32}&\dots&a_{3n}\\
\vdots&\vdots&\ddots&\vdots\\
a_{n1}&a_{n2}&\dots&a_{nn}
\end{vmatrix}
=a_{21}C_{11}+a_{22}C_{12}+\dots+a_{2n}C_{1n}
\]
Since $\bm A^{*}$ has two equal rows, the determinant must be zero. Hence $a_{21}C_{11}+a_{22}C_{12}+\dots+a_{2n}C_{1n}=0.$

Similarly, \emph{all entries off the main diagonal in Eq.(\ref{11.1}) are zero.}
\end{itemize}	
Thus the equation $(\ref{11.1})$ is correct:
\[
\bm A\bm C\trans=\begin{bmatrix}
\det\bm A&&\\&\det\bm A&\\&&\det\bm A
\end{bmatrix}=\det(\bm A)\bm I
\implies
\bm A^{-1}=\frac{1}{\det\bm A}\bm C\trans.
\]
Hence we could compute the inverse by computing many determinants of submatrix:
\begin{definition}[Inverse]
\emph{The $(\bm i,\bm j)$th entry of $\bm A^{-1}$ is the cofactor $C_{ji}$ (not $C_{ji}$) divided by $\det\bm A$: }
\[
\mbox{\emph{Formula for $\bm A^{-1}$}}\qquad\qquad
(\bm A^{-1})_{ij}=\frac{C_{ji}}{\det\bm A}\qquad\text{and}\qquad\bm A^{-1}=\frac{\bm C\trans}{\det\bm A}.
\]
\end{definition}
\subsubsection{Cramer's Rule}
\paragraph{Cramer's Rule solves $\bm{Ax}=\bm b$}

Assume $\bm A$ is a $n\times n$ matrix that is \emph{nonsingular}. Then we can use determinant to solve this system:

Let's start with $n=3$. We could multiply $\bm A$ with a new matrix $\bm C_1$ to get $\bm B_1$:
\[
\begin{array}{ll}
\mbox{\emph{Key idea:}}
&
\bm A\bm C_1
=
\begin{bmatrix}
a_{11}&a_{12}&a_{13}\\a_{21}&a_{22}&a_{23}\\a_{31}&a_{32}&a_{33}
\end{bmatrix}\begin{bmatrix}
x_1&0&0\\x_2&1&0\\x_3&0&1
\end{bmatrix}=\begin{bmatrix}
b_1&a_{12}&a_{13}\\b_2&a_{22}&a_{23}\\b_3&a_{32}&a_{33}
\end{bmatrix}=\bm B_1
\end{array}
\]
Taking determinants both sides, then we have
\[
\det(\bm A\bm C_1)=\det(\bm A)\det(\bm C_1)=\det(\bm A)(x_1)=\det\bm B_1\implies
x_1=\frac{\det\bm B_1}{\det\bm A_1}.
\]
The matrix $\bm B_1$ is essentaily obtained by replacing the first column of $\bm A$ by the vector $\bm b$.

Similarly, we could get all $x_j$ in this way. $(i=1,\dots,n)$.
\begin{definition}[Cramer's Rule]
If $\det\bm A$ is not zero, $\bm{Ax}=\bm b$ could be solved by determinants:
\[
x_1=\frac{\det\bm B_1}{\det\bm A}\qquad
x_2=\frac{\det\bm B_2}{\det\bm A}\qquad
\dots\dots\qquad
x_n=\frac{\det\bm B_n}{\det\bm A}
\]
The matrix $\bm B_j$ has the $j$th column of $\bm A$ replaced by the vector $\bm b$. In other words,
\[
\begin{array}{ll}
\bm B_{j}=\begin{bmatrix}
a_{11}&\dots&b_1&\dots&a_{1n}\\
a_{21}&\dots&b_2&\dots&a_{2n}\\
\vdots&\ddots&\vdots&\ddots&\vdots\\
a_{n1}&\dots&b_n&\dots&a_{nn}\\
\end{bmatrix}\
&
j=1,\dots,n.
\end{array}
\]
\end{definition}
\subsection{Orthogonality}
\begin{definition}[Orthogonal vectors]
Two vectors $\bm x,\bm y\in\mathbb{R}^{n}$ are orthogonal when their inner product is zero:
\[
\inp{\bm x}{\bm y}=\sum_{i=1}^{n}x_{i}y_{i}=0.
\]
\end{definition}
\begin{remark}
Note that the inner product of two \emph{vectors} satisfies the \textit{commutative rule}. In other words, $\inp{\bm x}{\bm y}=\inp{\bm y}{\bm x}$ for vectors $\bm x$ and $\bm y$. The inner product defined for matrices may not satisfy the \textit{commutative rule}. Generally, if the result of inner product is a scalar, then inner product satisfies commutative rule.
\end{remark}
An important case is the inner product of a vector with \textit{itself}. The inner product $\inp{\bm x}{\bm x}$ gives the \textit{length of $\bm v$ squared}:
\begin{definition}[length/norm]
The \emph{length}(\emph{norm}) $\|\bm x\|$ of a vector $\bm x\in\mathbb{R}^n$ is the square root of $\inp{\bm x}{\bm x}$:
\[
\mbox{\emph{length}}=\|\bm x\|=\sqrt{\inp{\bm x}{\bm x}}=\sqrt{x_1^2+\dots+x_n^2}.
\]
\end{definition}
\subsubsection{Function space}
We can talk about inner product between functions under the function space. For example, if we define $V=\{f(t)\mid \int_{0}^{1}f^2(t)dt<\infty\}$, then we can define inner product and norm under $V$:
\begin{definition}[Inner product; norm]
The \emph{inner product} and the \emph{norm} of $f(x),g(x)$ under the function space $V=\{f(t)\mid \int_{0}^{1}f^2(t)dt<\infty\}$, are defined as:
\[
\inp{f}{g}=\int_{0}^{1}f(x)g(x)dx
\qquad\mbox{and}\qquad
\|f\|^2=\sqrt{\int_{0}^{1}f^2(x)dx}
\]
\end{definition}
Moreover, when $\inp{f}{g}=0$, we say two functions are \emph{orthogonal} and denote it as $f\perp g$.
\subsubsection{Cauchy-Schwarz Inequality}
In $\mathbb{R}^2$, suppose $\bm x=\begin{pmatrix}
x_1\\x_2
\end{pmatrix},\bm y=\begin{pmatrix}
y_1\\y_2
\end{pmatrix}$, then we set:
\[
\begin{array}{ll}
\left\{
\begin{aligned}
x_1=\|\bm x\|\cos\theta\\
x_2=\|\bm x\|\sin\theta
\end{aligned}
\right.
&
\left\{
\begin{aligned}
y_1=\|\bm y\|\cos\varphi\\
y_2=\|\bm y\|\sin\varphi
\end{aligned}
\right.
\end{array}
\]
The inner product of $\bm x$ and $\bm y$ is given by:
\[
\begin{aligned}
<\bm x,\bm y>=\bm x\trans\bm y&=x_1x_2+y_1y_2\\
											 &=\|\bm x\|\|\bm y\|(\cos\theta\cos\varphi+\sin\theta\sin\varphi)\\
											 &=\|\bm x\|\|\bm y\|\cos(\theta-\varphi)
\end{aligned}
\]
Since $|\cos(\theta-\varphi)|$ never exceeds 1, the cosine formula gives a great inequality:
\begin{theorem}[Cauchy Schwarz Inequality]
\[
\inp{\bm x}{\bm y}\le\|\bm x\|\|\bm y\|
\]
holds for two vectors $\bm x$ and $\bm y$.
\end{theorem}
\begin{proof}
Firstly, we want to find optimizer $t^{*}$ such that 
\[
\min\|\bm x-t\bm y\|^2=\|\bm x-t^{*}\bm y\|^2.
\]
Note that
\[\begin{aligned}
\|\bm x-t\bm y\|^2&=\inp{\bm x-t\bm y}{\bm x-t\bm y}=\inp{\bm x}{\bm x}+\inp{-t\bm y}{\bm x}+\inp{\bm x}{-t\bm y}+\inp{-t\bm y}{-t\bm y}\\
&=\|\bm x\|^2-t\inp{\bm y}{\bm x}-t\inp{\bm x}{\bm y}+t^2\|\bm y\|^2\\
&=\|\bm x\|^2-2t\inp{\bm x}{\bm y}+t^2\|\bm y\|^2
\end{aligned}
\]
Hence the minimizer $t*$ must satisfy
\[
\Delta=0\implies
t^{*}=\frac{\inp{\bm x}{\bm y}}{\|\bm y\|^2}
\]
Hence we have 
\[\begin{aligned}
\|\bm x-t\bm y\|^2_{\min}=\|\bm x-t^*\bm y\|^2&=
\|\bm x\|^2-\frac{\inp{\bm x}{\bm y}^2}{\|\bm y\|^2}\\
&=\frac{\|\bm x\|^2\|\bm y\|^2-\inp{\bm x}{\bm y}^2}{\|\bm y\|^2}\ge 0\\
\qquad\qquad\implies
\|\bm x\|^2\|\bm y\|^2\ge\inp{\bm x}{\bm y}^2
\end{aligned}
\]
Or equivalently,
\[
|\inp{\bm x}{\bm y}|\le\|\bm x\|\|\bm y\|.
\]
\end{proof}
\begin{remark}
\paragraph{Cauchy-Schwarz inequality also holds for functions}
If we consider functions $f,g$ as vectors, then
\[
\left[
\int_0^1f(t)g(t)dt
\right]\le
\int_0^1f^2dt\int_0^1g^2dt
\]
\paragraph{The normalization of inner product is bounded by 1}
Since $|\inp{\bm x}{\bm y}|\le\|\bm x\|\|\bm y\|$, we have 
\[
-1\le\frac{\inp{\bm x}{\bm y}}{\|\bm x\|\|\bm y\|}\le 1
\]
If we define $\frac{\inp{\bm x}{\bm y}}{\|\bm x\|\|\bm y\|}:=\cos\theta$, then $\inp{\bm x}{\bm y}=\|\bm x\|\|\bm y\|\cos\theta$, the angle $\theta$ is said to be the intersection angle between $\bm x$ and $\bm y$.
\end{remark}
Cauchy-Schwarz equality holds for \emph{Hilbert space}, which will be discussed in other courses.

\subsubsection{Orthogonal for space}
After defining inner product, we can discuss the orthogonality for space:
\begin{definition}[Orthogonal subspaces]
Two subspaces $\bm U$ and $\bm V$ of a vector space are \emph{orthogonal} if every vector $\bm u$ in $\bm U$ is \textit{perpendicular} to every vector $\bm v$ in $\bm V$:
\[
\begin{array}{ll}
\mbox{\emph{Orthogonal subspaces}}
&
\inp{\bm u}{\bm v}=0\mbox{ for all $\bm u$ in $\bm U$ and all $\bm v$ in $\bm V$.}
\end{array}
\]
\end{definition}