\chapter{Week13}
\section{Wednesday}\index{week7_Thursday_lecture}
The trust region problem aims to sovle
\begin{equation}
\begin{array}{ll}
\min&\frac{1}{2}\bm x\trans\bm A\bm x-\bm b\trans\bm x\\
&\bm x\trans\bm x=1
\end{array}
\end{equation}
Consider the scale of $\bm b$. Consider the power method. Step-length w.r.t. the scale of $\|\bm A\|/\|\bm b\|$.

\subsection{Approximate Gradient Projection}
The constraint optimization problem aims to solve
\begin{equation}
\begin{array}{ll}
\min&f(x)\\
&x\in X
\end{array}
\end{equation}
The stationarity satisfies
\[
x^*=\Proj_X\left[x^*-\alpha\nabla f(x^*)\right]
\]
Thus we obtain an algorithm
\[
x^{r+1}=
\Proj_X\left[x^r-\alpha^r\nabla f(x^r)+e^r\right]
\]
where $e^r=\kappa_1\|x^{r+1} - x^r\|$.

\subsection{Conic Programming}
\begin{equation}
\begin{array}{ll}
\min&\inp{\bm c}{\bm X}\\
&\bm A\bm X=\bm b\\
&\bm X\in\mathcal{K}
\end{array}
\end{equation}
where $\mathcal{K}$ is a closed, convex cone.
\begin{remark}
Note that the final project can be solved by conic programming.
\end{remark}
\begin{definition}[Cone]
In $\mathbb{R}^n$, $\mathcal{K}$ is a \emph{cone} if $\forall x\in\mathcal{K}$, we have $\alpha x\in\mathcal{K}$, for $\forall\alpha\ge0$
\end{definition}
For example, $\mathbb{R}_+^n=\{\bm x\in\mathbb{R}^n\mid\bm x\ge0\}$. In this case, CP = LP.

SC = 2nd-order cone, or ice-cream cone.
\[
S_2^{1+n}=\{\bm x\in\mathbb{R}^{n_1}\mid\|x(1:n)\|\le x_{n+1}\}
\]
In $\mathbb{R}^2$, it looks like an ice-cream.

SDP cone:
\[
S_+^n=\{X\in\mathbb{R}^{n\times n}\mid X=X\trans\succeq0\}
\]

The dual cone of $\mathcal{K}$ is
\[
\mathcal{K}^*=\{\bm x\mid \inp{\bm x}{\bm y}\ge0,\forall y\in\mathcal{K}\}
\]

The Lagranian function is
\[
L(x,y)=\inp{C}{X}-\inp{y}{Ax-b}
\]
The dual problem aims to solve
\[
Q(\bm y)=\inf_{\bm x\in\mathcal{K}}L(x,y)
=
\inf_{\bm x\in\mathcal{K}}\inp{C-A^*y}{x}+\inp{b}{y}
\]
where $A^*$ is the adjoint of $\bm A$.
























