
\chapter{Week1}

\section{Monday}\index{Monday_lecture}
\subsection{Introduction to Optimizaiton}
The usual optimization formulation is given by:
\[\begin{array}{ll}
\min f(\bm x),
&
\mbox{where }f:\mathbb{R}^n\mapsto\mathbb{R}
\\
\mbox{such that}
&
\bm x\in X\subseteq\mathbb{R}^n
\end{array}
\]

One example of the set $X$ is given by:
\[
X=\left\{\bm x\in\mathbb{R}^n\middle|
\begin{array}{l}
C_i(\bm x) = \bm0, i=1,2,\dots,m\le n
 \\
 h_i(\bm x)\ge\bm0, i=1,2,\dots,p
\end{array}\right\}
\]

Linear programming can be easily solved, but Integer linear programming is much harder. The equivalent LP formulation is given by:
\[
\begin{array}{ll}
\min
&
\bm c\trans\bm x
\\
\mbox{s.t.}
&
\bm{Ax}=\bm b
\\
&
\bm c\le\bm{Bx}\le \bm c'
\end{array}
\]



\section{Wednesday}

\subsection{Reviewing for Linear Algebra}
Questions:
\begin{itemize}
\item
What is the necessary and sufficient condition for the linear system $\bm A\bm x=\bm b$ to have a solution $\bm x$?

Answer: $\bm b\in\mathcal{C}(\bm A)$.
\item
For $\bm A\in\mathbb{S}^n$, what is the necessary and sufficient condition for $\bm A\succeq0$?

Answer: $\bm x\trans\bm A\bm x\ge0$ for $\forall x\in\mathbb{R}^n$; or $\lambda_i(\bm A)\ge0$ for all $i$.
\end{itemize}
\subsection{Reviewing for Calculus}
For function $f:\mathbb{R}^n\mapsto\mathbb{R}$:
\begin{itemize}
\item
We use notation $f\in\mathcal{C}^n$ to denote $f$ is \emph{continuously differentiable to $n$th order}. This course will basically deal with such functions.
\item
We use notation $\nabla f(x)$ to denote the \emph{Gradient} of $f$ at $x$; and $\nabla^2 f(x)$ denotes the second order derivative of $f$ at $x$. Note that $\nabla^2 f(x)\in\mathbb{S}^n$ for $f\in\mathcal{C}^1$.
\item
We use notation $\mathbb{S}^n$ to denote the set of all symmetric $n\times n$ matrices, i.e., 
\[
\mathbb{S}^n=\{\bm X\in\mathbb{R}^{n\times n}\mid \bm X\trans=\bm X\}
\]
Moreover, $\mathbb{S}^n_+$ denotes the set of all symmetric $n\times n$ matrices with all eigenvalues non-negative:
\[
\mathbb{S}^n_+=\{\bm X\in\mathbb{R}^{n\times n}\mid \bm X\trans=\bm X\succeq0\}
\]
\end{itemize}
\subsection{Introduction to Optimization}
The usual optimization formulation is given by:
\[\begin{array}{ll}
\min f(\bm x),
&
\mbox{where }f:\mathbb{R}^n\mapsto\mathbb{R}
\\
\mbox{such that}
&
\bm x\in X\subseteq\mathbb{R}^n
\end{array}
\]
\begin{itemize}
\item
The simplest case for the constraint is $X=\mathbb{R}^n$, which leads to \emph{unconstrainted} optimization problem.
\item
Or $X=P$ is a \emph{polyhedron}, i.e., the boundaries for the region are all lines.
\end{itemize}
\begin{definition}[Constraint Regions]
In space $\mathbb{R}^n$, 
\begin{itemize}
\item
the hyper-plane is defined as:
\[
\left\{
\bm x\middle |\bm a\trans\bm x=\bm\beta
\right\}
\]
with constants $\bm a\in\mathbb{R}^n$ and $\bm\beta\in\mathbb{R}$
\item
the half-space is defined as
\[
\left\{
\bm x\middle |\bm a\trans\bm x\le\bm\beta
\right\}
\]
\item
the polyhedron is defined as the \emph{intersection} of a \emph{finite} number of hyperplanes or half-spaces
\end{itemize}
\end{definition}
Next, we give the definition for the basic optimization problem:
\begin{definition}[Linear Programming]
The Linear Programming is given by:
\[\begin{array}{ll}
\min& \bm c\trans\bm x,
\\
\mbox{such that}
&
\bm x\in P (\mbox{polyhedron})
\end{array}
\]
Or it can be reformulated as:
\[\begin{array}{ll}
\min& \bm c\trans\bm x,
\\
\mbox{such that}
&
\bm A_I\bm x\le\bm b_I\\
&
\bm A_E\bm x=\bm b_E\in\mathbb{R}^m,\quad m<n.
\end{array}
\]
\end{definition}

\begin{definition}[Optimality]
$\bm x^*$ is said to be :
\begin{itemize}
\item
the \emph{local minimum} of $f(\bm x)$ if there exists small $\epsilon$ such that
\[
\begin{array}{ll}
f(\bm x^*)\le f(\bm x),
&
\forall \bm x\in\mathcal{B}(\bm x^*,\epsilon)\bigcap X:=\{\bm x\mid \|\bm x-\bm x^*\|\le\epsilon\}\bigcap X
\end{array}
\]
\item
the \emph{global minimum} if
\[
\begin{array}{ll}
f(\bm x^*)\le f(\bm x),
&
\forall \bm x\in X
\end{array}
\]
\end{itemize}
\end{definition}
\begin{remark}
Unless specified, when we want to minimize a non-convex function, it usually means we only find its \emph{local minimum}. This is because usually the local minimum is good enough.
\end{remark}

The optimization task is essentially find $\bm x^*$ such that
\[
\bm x^*=\arg\min_{\bm x\in X}f(\bm x)\in\mathbb{R}^n.
\]

philosophy (optimization sufficient and necessity). 
philosophy of relaxation (convex nulls)

The Optimality conditions are the \emph{most important} theoretical tools for optimization.
\begin{theorem}[Optimality condition]The optimality condition contains
\begin{enumerate}
\item
Necessary Condition (exclude non-optimal points):
\[
\mbox{$n=1$ special case:}\left\{
\begin{aligned}
\mbox{1st order: }f'(x)=0\\
\mbox{2nd order: }f''(x)\ge0
\end{aligned}
\right.\implies
\left\{
\begin{aligned}
\mbox{1st order: }\nabla f(x)=0\\
\mbox{2nd order: }\nabla f^2(x)\succeq0
\end{aligned}
\right.
\]
\item
Sufficient Condition (may identify optimal solutions)
\[
\mbox{$n=1$ special case:}\left\{
\begin{aligned}
\mbox{1st order: }f'(x)=0\\
\mbox{2nd order: }f''(x)>0
\end{aligned}
\right.\implies
\left\{
\begin{aligned}
\mbox{1st order: }\nabla f(x)=0\\
\mbox{2nd order: }\nabla f^2(x)\succ0
\end{aligned}
\right.
\]
\end{enumerate}
\end{theorem}
\begin{proof}
The $n=1$ special case can imply the general case for optimality condition. For multivariate $f$, we set $\bm x=\bm x^*+td$ with $t$ to be the stepsize and $d$ to be the direction.  For fixed $t$ and $d$, we define $h(t) = f(\bm x)=f(\bm x^*+td)$. It follows that
\[
h'(t)=\nabla\trans f(\bm x^*+td)d
\]
We find $h'(0)=\nabla\trans f(\bm x^*)d$ for $\forall$ d, which implies $\nabla f(\bm x^*)=0$.
\end{proof}

Note that there is a gap between necessary and sufficient conditions, which puts us in an embarrassing position. However, the convex condition can save us:
\begin{theorem}
If $f$ is convex in $\mathcal{C}^1$, then $\nabla f(\bm x)=0$ is the \emph{necessary} and \emph{sufficient} condition.
\end{theorem}














