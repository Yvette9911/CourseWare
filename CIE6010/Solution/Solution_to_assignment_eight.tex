\subsection{Solution to Assignment Eight}
\begin{enumerate}
\item
\begin{proof}[Solution.] We factorize $\bm A\in\mathbb{R}^{n\times n}$ into:
\[
\bm A=\bm U\bm\Sigma\bm V\trans
\]
where $\bm U$ is a $n\times n$ \emph{orthogonal} matrix,
$\bm\Sigma$ is a $n\times n$ \textit{diagonal} matrix,
$\bm V$ is a $n\times n$ \emph{orthogonal} matrix.\\
Thus we write $\bm A\bm A\trans$ and $\bm A\trans\bm A$ as:
\begin{align*}
\bm A\bm A\trans&=\bm U\bm\Sigma\bm V\trans\bm V\bm \Sigma\trans\bm U\trans=\bm U\bm\Sigma\bm \Sigma\trans\bm U\trans=\bm U\bm\Sigma^2\bm U\trans.\qquad\text{Since $\bm V\trans\bm V=\bm I$ due to orthonormality.}\\
\bm A\trans\bm A&=\bm V\bm \Sigma\trans\bm U\trans\bm U\bm\Sigma\bm V\trans=\bm V\bm \Sigma\trans\bm\Sigma\bm V\trans=\bm V\bm\Sigma^2\bm V\trans.\qquad\text{Since $\bm U\trans\bm U=\bm I$ due to orthonormality.}
\end{align*}
If we set $\bm S=(\bm V\trans)^{-1}\bm U\trans=\bm V\bm U\trans$, then the inverse is given by $\bm S^{-1}=(\bm U\trans)^{-1}\bm V^{-1}=\bm U\bm V\trans$.\\
Hnece there exists invertible $\bm S=\bm V\bm U\trans$ such that
\[
\begin{aligned}
\bm S^{-1}(\bm A\trans\bm A)\bm S
&=\bm U\bm V\trans(\bm A\trans\bm A)\bm V\bm U\trans\\
&=\bm U\bm V\trans\bm V\bm\Sigma^2\bm V\trans\bm V\bm U\trans\\
&=\bm U\bm\Sigma^2\bm U\trans=\bm A\bm A\trans
\end{aligned}
\]
Hence $\bm A\trans\bm A$ is similiar to $\bm A\bm A\trans$, i.e. $\bm A\trans\bm A$ and $\bm A\bm A\trans$ are \emph{similar}.
\end{proof}
%q2
\item
Let $\bm A$ be $m\times n$ ($m\ge n$) matrix of rank $n$ with singular value decomposition $\bm U\Sigma\bm V\trans$. Let $\Sigma^+$ denote the $n\times m$ matrix
\[
\begin{pmatrix}
\frac{1}{\sigma_1}&&&0&\dots&0\\
&\ddots&&\vdots&\ddots&\vdots\\
&&\frac{1}{\sigma_n}&0&\dots&0\\
\end{pmatrix}
\]
And we define $\bm A^+=\bm V\Sigma^+\bm U\trans$
\begin{enumerate}
\item
Show that
\[
\bm A\bm A^+=\begin{bmatrix}
\bm I_n&\bm0\\\bm0&\bm0
\end{bmatrix}\qquad\text{and}\qquad
\bm A^+\bm A=\bm I_n.
\]
(Note that $\bm A^+$ is called the \emph{pseudo-inverse} of $\bm A$.)
\item
Show that $\hat{\bm x}=\bm A^+\bm b$ satisfies the normal equation $\bm A\trans\bm A\bm x=\bm A\trans\bm b.$
\end{enumerate}
\begin{proof}[Solution.]
\begin{enumerate}
\item
We write $\Sigma^+$ into block matrix:
\[
\Sigma^+=\begin{bmatrix}
\bm\Sigma^{-1}&\bm0_{n\times(m-n)}
\end{bmatrix}
\]
where $\bm\Sigma^{-1}:=\diag(\frac{1}{\sigma_1},\dots,\frac{1}{\sigma_n})$.\\
Hence $\bm\Sigma\Sigma^+=\begin{bmatrix}
\bm\Sigma\bm\Sigma^{-1}&\bm0_{m\times(m-n)}\end{bmatrix}=
\begin{bmatrix}
\bm I_{n}&\bm0_{n\times(m-n)}\\\bm0_{m-n}&\bm0_{(m-n)\times(m-n)}
\end{bmatrix}
.$\\
Thus we derive
\begin{align*}
\bm A\bm A^+&=\bm U\bm\Sigma\bm V\trans\bm V\bm\Sigma^+\bm U\trans\\
&=\bm U\bm\Sigma\bm\Sigma^+\bm U\trans=
\bm U\begin{bmatrix}
\bm I_{n}&\bm0_{n\times(m-n)}\\\bm0_{m-n}&\bm0_{(m-n)\times(m-n)}
\end{bmatrix}\bm U\trans
\end{align*}
We write $\bm U$ as block matrix:
\[
\bm U=\begin{bmatrix}
\bm U_1&\bm U_2
\end{bmatrix}
\]
where $\bm U_1$ is $m\times n$ matrix, $\bm U_2$ is $m\times (m-n)$ matrix.\\
Hence we derive
\[
\begin{aligned}
\bm A\bm A^+&=\bm U\begin{bmatrix}
\bm I_{n}&\bm0_{n\times(m-n)}\\\bm0_{m-n}&\bm0_{(m-n)\times(m-n)}
\end{bmatrix}\bm U\trans\\
&=\begin{bmatrix}
\bm U_1&\bm U_2
\end{bmatrix}\begin{bmatrix}
\bm I_{n}&\bm0_{n\times(m-n)}\\\bm0_{m-n}&\bm0_{(m-n)\times(m-n)}
\end{bmatrix}\begin{bmatrix}
\bm U_1\trans\\\bm U_2\trans
\end{bmatrix}\\
&=\begin{bmatrix}
\bm U_1\bm I_{n}\bm U_1\trans&\bm0_{n\times(m-n)}\\\bm0_{m-n}&\bm0_{(m-n)\times(m-n)}
\end{bmatrix}\\
&=\begin{bmatrix}
\bm I_{n}&\bm0_{n\times(m-n)}\\\bm0_{m-n}&\bm0_{(m-n)\times(m-n)}
\end{bmatrix}\qquad\text{due to the orthogonality of $\bm U$.}
\end{aligned}
\]
Moreover, $\bm A^+\bm A=\bm V\bm\Sigma^+\bm U\trans\bm U\bm\Sigma\bm V\trans
=\bm V\bm\Sigma^+\bm\Sigma\bm V\trans.$\\
You can verify by yourself that $\Sigma^+\bm\Sigma=\bm I$.\\
Hence $\bm A^+\bm A=\bm V\bm V\trans=\bm I_{n}.$
\item
We only need to show $\bm A\trans\bm A\bm A^+\bm b=\bm A\trans\bm b$.\\
Since $\rank(\bm A)=n$, the columns of $\bm A$ are ind. Hence $\bm A\trans\bm A$ is invertible.
\begin{itemize}
\item
Firstly, we show $(\bm A\trans\bm A)^{-1}\bm A\trans$ is the right inverse of $\bm A$:
\[
\begin{aligned}
\bm A(\bm A\trans\bm A)^{-1}\bm A\trans&=
\bm U\bm\Sigma\bm V\trans(\bm V\bm\Sigma\bm U\trans\bm U\bm\Sigma\bm V\trans)^{-1}
\bm V\bm\Sigma\bm U\trans\\
&=\bm U\bm\Sigma\bm V\trans(\bm V\bm\Sigma^2\bm V\trans)^{-1}\bm V\bm\Sigma\bm U\trans
=\bm U\bm\Sigma\bm V\trans\bm V\bm\Sigma^{-2}\bm V\trans\bm V\bm\Sigma\bm U\trans\\
&=\bm U\bm\Sigma\bm\Sigma^{-2}\bm\Sigma\bm U\trans\\
&=\bm I
\end{aligned}
\]
\item
Since we also obtain $\bm A^+\bm A=\bm I$, we derive 
\[
\bm A^+=\bm A^+\bm I=\bm A^+\bm A\left[(\bm A\trans\bm A)^{-1}\bm A\trans\right]=\bm I\left[(\bm A\trans\bm A)^{-1}\bm A\trans\right]=\left[(\bm A\trans\bm A)^{-1}\bm A\trans\right]
\]
\end{itemize}
Thus we have $\bm A\trans\bm A\bm A^+=\bm A\trans\implies\bm A\trans\bm A\bm A^+\bm b=\bm A\trans\bm b.$
\end{enumerate}
\end{proof}






%q3
\item
\begin{proof}
\begin{enumerate}
\item
\begin{align*}
\|\bm A\|_{\bm F}^2&=\trace(\bm A\trans\bm A)\\
&=\trace\left[\sum_{i=1}^{n}\sigma_i\bm v_i\bm u_i\trans\times\sum_{i=1}^{n}\sigma_i\bm u_i\bm v_i\trans\right]\\
&=\trace\left(\sum_{i=1}^{n}\sigma_i^2\bm v_i(\bm u_i\trans\bm u_i)\bm v_i\trans+\sum_{i\ne j}\sigma_i\sigma_j\bm v_i(\bm u_i\trans\bm u_j)\bm v_j\trans\right)\\
&=\trace\left(\sum_{i=1}^{n}\sigma_i^2\bm v_i\bm v_i\trans+\bm0\right)\qquad
\text{due to orthogonality for $\bm u_i$'s and $\bm v_i$'s.}\\
&=\sum_{i=1}^n\sigma_i^2\trace\left(
\bm v_i\bm v_i\trans
\right)
\end{align*}
Suppose $\bm v_i=\begin{bmatrix}
v_{1i}&v_{2i}&\dots&v_{ni}
\end{bmatrix}\trans$, then due to the orthonormality of $\bm v_i$, we obtain
\[
\trace\left(
\bm v_i\bm v_i\trans
\right)=\sum_{j=1}^{n}v_{ji}^2=1.
\]
Hence $\|\bm A\|_{\bm F}^2=\sum_{i=1}^n\sigma_i^2\trace\left(
\bm v_i\bm v_i\trans
\right)=\sum_{i=1}^n\sigma_i^2.$
\item
\begin{itemize}
\item
When $k<n$, it's obvious that 
\[
\bm A_k=\sigma_1\bm u_1\bm v_1\trans+\dots+\sigma_k\bm u_k\bm v_k\trans.
\]
Hence \[\bm A-\bm A_k=\sigma_{k+1}\bm u_{k+1}\bm v_{k+1}\trans+\dots+\sigma_{n}\bm u_{n}\bm v_{n}\trans.\]
And 
\begin{align*}
\|\bm A-\bm A_k\|_{\bm F}^{2}&=\trace\left(
\sum_{i=k+1}^{n}\sigma_i\bm v_i\bm u_i\trans\times\sum_{i=k+1}^{n}\sigma_i\bm u_i\bm v_i\trans
\right)
\end{align*}
Similarly, we obtain
\[
\|\bm A-\bm A_k\|_{\bm F}^{2}=\sum_{i=k+1}^{n}\sigma_i^2.
\]
\item
Otherwise, $\bm A_k=\bm A$, thus $\|\bm A-\bm A_k\|_{\bm F}^{2}=0.$
\end{itemize}
\end{enumerate}
\end{proof}
\item
\begin{proof}
We only need to show that $\max_{\bm x,\bm y}\|\bm x\trans\bm A\bm y\|^2=\sigma_1^2:$
\begin{itemize}
\item
we find
\begin{align*}
\|\bm x\trans\bm A\bm y\|^2&=\|\inp{\bm x}{\bm{Ay}}\|^2\le\|\bm x\|^2\cdot\|\bm{Ay}\|^2\\
&=\|\bm{Ay}\|^2
\end{align*}
The equality holds if and only if $\bm x=\bm{Ay}$.
\end{itemize}
Thus
\[
\max_{\bm x,\bm y}\|\bm x\trans\bm A\bm y\|^2=\max_{\bm y}\|\bm{Ay}\|^2=\max_{\bm y}\bm y\trans(\bm A\trans\bm A\bm y).
\]
We only need to show $\max_{\bm y}\bm y\trans(\bm A\trans\bm A\bm y)=\sigma_1^2$:
\begin{itemize}
\item
Since $\bm A\trans\bm A$ is real symmetric, there exists $n$ orthogonal eigenvectors of $\bm A\trans\bm A$. Moreover, we can divide these eigenvectors by their length to get $n$ \emph{orthonormal} eigenvectors $\bm p_1,\bm p_2,\dots,\bm p_n$ associated with eigenvalues $\lambda_1,\lambda_2,\dots,\lambda_n$ respectively.\\
Without loss of generality, we set $\lambda_1=\max_{i}\lambda_i$ for $i=1,\dots,n.$\\
Since they span $\mathbb{R}^{n}$, we can express arbitrary $\bm y$ as linear combination of $\bm p_1,\bm p_2,\dots,\bm p_n$:
\[
\bm y=\alpha_1\bm p_1+\alpha_2\bm p_2+\dots+\alpha_n\bm p_n.
\]
Moreover, the product $\bm y\trans\bm y$ is
\begin{align*}
\bm y\trans\bm y&=\|\bm y\|^2=1\\
&=\sum_{i=1}^{n}\sum_{j=1}^{n}\alpha_i\alpha_j\bm p_i\bm p_j\\
&=\sum_{i=1}^{n}\alpha_i^2=1.
\end{align*}
\item
Moreover, the product $\bm A\trans\bm A\bm y$ is given by:
\begin{align*}
\bm A\trans\bm A\bm y&=\bm A\trans\bm A(\alpha_1\bm p_1+\alpha_2\bm p_2+\dots+\alpha_n\bm p_n)\\
&=\alpha_1(\bm A\trans\bm A\bm p_1)+\alpha_2(\bm A\trans\bm A\bm p_2)+\dots+\alpha_n(\bm A\trans\bm A\bm p_n)\\
&=\alpha_1\lambda_1\bm p_1+\alpha_2\lambda_2\bm p_2+\dots+\alpha_n\lambda_n\bm p_n
\end{align*}
Hence the product $\bm y\trans(\bm A\trans\bm A\bm y)$ is given by:
\begin{align*}
\bm y\trans(\bm A\trans\bm A\bm y)&=
\bm y\trans(\alpha_1\lambda_1\bm p_1+\alpha_2\lambda_2\bm p_2+\dots+\alpha_n\lambda_n\bm p_n)\\
&=\left(
\sum_{i=1}^{n}\alpha_i\bm p_i\trans
\right)
\left(
\sum_{j=1}^{n}\alpha_j\lambda_j\bm p_j
\right)
=\sum_{i=1}^{n}\sum_{j=1}^{n}\alpha_i\alpha_j\lambda_j\bm p_i\trans\bm p_j\\
&=\sum_{i=1}^{n}\alpha_i^2\lambda_i\\
&\le\lambda_1\sum_{i=1}^{n}\alpha_i^2=\lambda_1.
\end{align*}
The equality is satisfied when $\bm y=\bm p_1.$ Hence $\max_{\bm y}\bm y\trans(\bm A\trans\bm A\bm y)=\lambda_1$.\\
Since $\lambda_1=\sigma_1^2$, we derive $\max_{\bm y}\bm y\trans(\bm A\trans\bm A\bm y)=\sigma^2_1$.
\end{itemize}
\end{proof}
\item
\begin{proof}
\begin{itemize}
\item
We do the eigendecomposition for $\bm A$:
\[
\bm A=\bm U\bm\Sigma\bm U\trans
\]
where $\bm U$ is a $n\times n$ \emph{orthogonal} matrix such that columns are eigenvectors of $\bm A^2$.\\
$\bm\Sigma=\diag(\lambda_1,\dots,\lambda_n)$ is a $n\times n$ \textit{diagonal} matrix, and $(\lambda_1,\dots,\lambda_n)$ are eigenvalues of $\bm A^2$.\\
And then we define $\sqrt{\bm\Sigma}:=\diag(\sqrt{\lambda_1},\dots,\sqrt{\lambda_n})$. Obviously, we have $\bm\Sigma=\sqrt{\bm\Sigma}\sqrt{\bm\Sigma}.$\\
Hence we could factorize $\bm A$ into
\[
(\bm U\sqrt{\bm\Sigma})(\bm U\sqrt{\bm\Sigma})\trans
=\bm U\sqrt{\bm\Sigma}\sqrt{\bm\Sigma}\trans\bm U
=\bm U\bm\Sigma\bm U=\bm A.
\]
Thus we define $\bm Q:=\bm U\sqrt{\bm\Sigma}$, which means we can factorize $\bm A$ into $\bm A=\bm Q\bm Q\trans$.
\item
Then we show the columns of $\bm Q$ are mutually orthogonal:\\
Suppose $\bm U=\begin{bmatrix}
\bm u_1&\dots&\bm u_n
\end{bmatrix}$, and $\{\bm u_1,\dots,\bm u_n\}$ is orthonormal basis.
\[
\bm Q=\bm U\sqrt{\bm\Sigma}
=\begin{bmatrix}
\bm u_1&\dots&\bm u_n
\end{bmatrix}\begin{pmatrix}
\sqrt{\lambda_1}&&\\&\ddots&\\&&\sqrt{\lambda_n}
\end{pmatrix}
=\begin{bmatrix}
\sqrt{\lambda_1}\bm u_1&\sqrt{\lambda_2}\bm u_2&\dots&\sqrt{\lambda_n}\bm u_n
\end{bmatrix}
\]
Since $\{\bm u_1,\dots,\bm u_n\}$ is orthonormal basis, we obtain:
\[
\bm u_i\bm u_j=0\text{ for }i\ne j.
\implies
(\sqrt{\lambda_i}\bm u_i)(\sqrt{\lambda_j}\bm u_j)=0\text{ for }i\ne j.
\]
which means columns of $\bm Q$ are mutually orthogonal.
\end{itemize}	
\end{proof}
\end{enumerate}