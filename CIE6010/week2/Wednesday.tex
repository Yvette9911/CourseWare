
%\chapter{Week2}

\section{Wednesday}\index{week2_Wednesday_lecture}
This lectue will study the concept for convexity.

\begin{definition}[Convex]
The subset $\mathcal{C}\subseteq\mathbb{R}^n$ is convex if
\[
x,y\in\mathcal{C}\implies
\{\lambda x + (1-\lambda)y\mid \alpha\in[0,1]\}\subset\mathcal{C},
\]
i.e., the line segment between arbitrarily two elements lines in $\mathcal{C}$
\end{definition}
\begin{remark}
Intersections of convex sets are convex. Empty set is assumed to be convex.
\end{remark}
\begin{definition}[Convex]\label{Def:2:3}
The function $f:\mathbb{R}^n\mapsto\mathbb{R}$ is convex if $\mbox{dom }f$ is convex and
\[
f(\lambda x+ (1-\lambda)y)\le\lambda f(x) + (1-\lambda) f(y)
\]
for $\forall x,y\in\mbox{dom }f$ and $\forall \lambda\in[0,1]$, i.e., the function evaluated in the line segment is lower than secant line between $x$ and $y$ ($f$ lies below secant line).
\end{definition}
\begin{remark}
$f$ is convex iff $-f$ is concave. (The concave definition simply changes the inequality direction in Def.(\ref{Def:2:3}))

The affine is both convex and concave.

Note that the convex depends on the domain of the function.
\end{remark}
For a second order differentiable function, we have a much easier way to determine its convexity.
\begin{theorem}\label{The:2:1}
If $f\in\mathcal{C}^1$, then the followings are equivalent:
\begin{enumerate}
\item
$f$ is convex
\item
$f(\bm y)\ge f(\bm x)+\nabla\trans f(\bm x)(\bm y - \bm x)$ for $\forall x,y\in\mbox{dom }f$, i.e., $f$ lines above the tangent line.
\end{enumerate}
\end{theorem}
\begin{proof}
From the definition for convexity,
\[
f(y) - f(x)\ge\frac{f(\lambda x + (1-\lambda) y) - f(x)}{1 - \lambda}
\]
Letting $\lambda\to1$, the RHS becomes a direction derivative:
\[
f(y) - f(x)\ge\nabla\trans f(\bm x)(\bm y-\bm x)
\]
The reverse direction of the proof is exercise.
\end{proof}
\begin{theorem}
If $f\in\mathcal{C}^2$, then the followings are equivalent:
\begin{enumerate}
\item
$f$ is convex
\item
$\nabla^2f(\bm x)\succeq0$ for $\forall x\in\mbox{dom }f$.
\end{enumerate}
\end{theorem}
\begin{proof}
We rewrite $f(\bm y)$ by applying Taylor expansion:
\[
f(\bm y) = f(\bm x)+\nabla\trans f(\bm x)(\bm y-\bm x)+\frac{1}{2}(\bm y-\bm x)\trans\nabla^2 f(\bm x+t(\bm y-\bm x))(\bm y-\bm x),\quad\mbox{for some }
t\in[0,1]
\]
Hence, if $f$ is convex, from Theorem(\ref{The:2:1}), we derive
\[
(\bm y-\bm x)\trans\nabla^2 f(\bm x+t(\bm y-\bm x))(\bm y-\bm x)\ge0\implies \frac{(\bm y-\bm x)\trans}{\|\bm y-\bm x\|}\nabla^2 f(\bm x+t(\bm y-\bm x))\frac{(\bm y-\bm x)}{\|\bm y-\bm x\|}\ge0
\]
Letting $\bm y\to\bm x$, we derive
\[
\bm d\trans\nabla^2 f(\bm x)\bm d\ge0
\]
for $d:=\frac{(\bm y-\bm x)}{\|\bm y-\bm x\|}$, which implies $\nabla^2 f(\bm x)\succeq0$.

The reverse direction is exercise.
\end{proof}
\begin{definition}[Epigraph]
The Epigraph of $f$ is given by:
\[
\mbox{Epi}(f):=\left\{
(x,t)\in\mathbb{R}^{n}\times\mathbb{R}\mid
x\in\mbox{dom }f,
t\ge f(x)
\right\}\subseteq\mathbb{R}^{n+1}
\]
\end{definition}
\begin{theorem}
$f$ is convex iff $\mbox{Epi}(f)$ is convex.
\end{theorem}
\begin{proof}
For any $(x,t),(y,s)\in \mbox{Epi}(f)$, it suffices to show
\[
(\lambda x+(1-\lambda)y, \lambda t+(1-\lambda)s)\in \mbox{Epi}(f)
\]
Note that
\begin{align*}
f(\lambda x+(1-\lambda)y)&\le \lambda f(x)+ (1-\lambda)f(y)\\
&\le \lambda t + (1-\lambda)s
\end{align*}
The reverse direction is exercise.
\end{proof}

\begin{definition}[Strict Convex]
The function $f:\mathbb{R}^n\mapsto\mathbb{R}$ is strict convex if $\mbox{dom }f$ is convex and
\[
f(\lambda x+ (1-\lambda)y)<\lambda f(x) + (1-\lambda) f(y)
\]
for $\forall x\ne y, x,y\in\mbox{dom }f$ and $\forall \lambda\in(0,1)$
\end{definition}
\begin{remark}
Strict convex implies the uniquesness of minimum
\end{remark}
However, for function $f(x)=\frac{1}{x}$, the curvature becomes more and more flat. We want to exclude such kind of functions.
\begin{definition}
The function $f:\mathbb{R}^n\mapsto\mathbb{R}$ is said to be strongly convex if $\mbox{dom }f$ is convex and $\exists\alpha>0$ such that $f(\bm x) - \alpha\bm x\trans\bm x$ is convex; or equivalently,
\[
f(\bm y)\ge f(\bm x) + \nabla\trans f(\bm x)(\bm y-\bm x) + \frac{\alpha}{2}\|\bm y-\bm x\|^2
\]
\end{definition}
The strongly convexity places a quadratic lower bound in the curvature of the function, i.e., the function must rise up at least as fast as a quadratic function.

Without convexity properties we cannot do so many theoretical analysis. However, most functions are not convex. 

Many nice properties require convexity, the most important one proerty is given below:
\begin{theorem}
If $f$ is convex in $\mathcal{C}^1$, then $\nabla f(\bm x)=0$ is the \emph{necessary} and \emph{sufficient} condition for global minimum.
\end{theorem}
Note that convex function does not have a local minimum that is not global minimum.
\begin{proof}
If $f\in\mathcal{C}^1$ is convex, then 
\[
f(\bm y)\ge f(\bm x)+\nabla\trans f(\bm x)(\bm y-\bm x)
\]

If $\nabla f(\bm x)=0$, then $f(\bm y)\ge f(\bm x)$ for $\forall y$.

If $\nabla f(y)\ge f(x)$ for $\forall y$, then $\nabla f(\bm x)=\bm0$, since otherwise we can construct $\bm y$ to derive a contradiction.
\end{proof}

In practice, we cannot solve all convex optimization problems. So we need to study the structure of every problem we have faced carefully. 







