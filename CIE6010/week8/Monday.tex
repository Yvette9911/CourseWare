
\chapter{Week8}

\section{Monday}\index{week7_Tuesday_lecture}
\paragraph{Comments for P-C method in A5}
The Newton's method gives the iterative formula
\[
x^{k+1} = x^k - [F'(x^k)]^{-1}F(x^k)
\]
Moreover, the two-step Newton's method gives:
\begin{equation}\label{Eq:8:1}
x^{k+2} = x^k - [F'(x^k)]^{-1}F(x^k)-[F'(x^{k+1})]^{-1}F(x^{k+1}),
\end{equation}
with convergence order $O(\|x^k-x^*\|^4)$.

However, computing Hessian matrix inverse twice in one iteration is expansive. The predictor-corrector method just change $F'(x^{k+1})$ as $F'(x^{k})$ in (\ref{Eq:8:1}):
\begin{align*}
x^{k+2} &= x^k - [F'(x^k)]^{-1}F(x^k)-[F'(x^{k})]^{-1}F(x^{k+1})\\
&=x^k - [F'(x^k)]^{-1}[F(x^k)+F(x^{k+1})],
\end{align*}
or equivalently,
\begin{align*}
x^{k+1} &= x^k - [F'(x^k)]^{-1}[F(x^k)+F(x^{k+1/2})],
\end{align*}
which is also called Composite Newton, with convergence rate $O(\|x^{k}-x^*\|^{3})$.
\paragraph{Comments for Incremental Gradient Method}
The reason why the disminishing step size performs poorer than the constant step size remains to be found.

\subsection{Constraint optimization}
%Given the problem
%\[
%\begin{array}{ll}
%\min&\quad f(x)\\
%\mbox{such that}&x\in X,
%\end{array}
%\]
%where $X$ is convex.
%\paragraph{Feasible Direction method}
%Given $\bar{x^r}\in X$, the new update is
%\[
%x^{r+1}=x^r+\alpha_r(\bar{x^r} - x^r),\qquad
%\alpha_r\in(0,1]
%\]
%\paragraph{Gradient projection method}
%\[
%\bar{x^r} = [x^r - s_r\nabla f(x^r)]^+
%\]
%\paragraph{Conditional Gradient}
%\[
%\bar{x^r}=\arg\min_{x\in X}\inp{\nabla f(x^r)}{(x-x^r)}
%\]
%\begin{example}
%Given the problem
%\[
%\begin{array}{ll}
%\min&x_1+x_2\\
%\mbox{such that}&x_1^2+x_2^2\le1
%\end{array}
%\]
%the gradient projection is as follows:
%\[
%x^0=\begin{pmatrix}
%0\\1
%\end{pmatrix}\implies \nabla f(x)=\begin{pmatrix}
%1\\1
%\end{pmatrix}\implies
%\bar{x^0}=\left[\begin{pmatrix}
%0\\1
%\end{pmatrix}-\frac{1}{2}\begin{pmatrix}
%1\\1
%\end{pmatrix}\right]^+=\frac{1}{2}\begin{pmatrix}
%-1\\1
%\end{pmatrix}
%\]
%It follows that
%\[
%x^1=\begin{pmatrix}
%0\\1
%\end{pmatrix}-\alpha_1(\bar{x^0}-x^0)
%\]
%Lagrange multiplier
%
%\end{example}
\begin{theorem}[Lagrange necesary optimality condition]
Given the problem
\begin{equation}\label{Eq:8:1}
\begin{array}{ll}
\min&f(\bm x)\\
\mbox{such that}&h_i(\bm x)=0,\quad i=1,\dots,m
\end{array}
\end{equation}
with $f:\mathbb{R}^n\mapsto\mathbb{R}$ and $h_i:\mathbb{R}^n\mapsto\mathbb{R}$ are continuously differentiable functions.Let $\bm x^*$ be a regular point, i.e., $\nabla h_i(\bm x^*)$ are linearly independent, $\lambda^*$ is the corresponding multiplier, then
\begin{itemize}
\item
The first order necessary optimality condition is 
\[
\nabla_x L(\bm x^*,\lambda^*)=0,\Longleftrightarrow
\nabla f(\bm x^*)+\sum_{i=1}^m\lambda_i^*\nabla h_i(\bm x^*)=0,
\]
where $\lambda^*$ is uniquely determined.
\item
The second order necessary optimality condition is (pre-assume $f,g\in\mathcal{C}^2$):
\[
\bm y\trans\left[\nabla_{xx}^2L(\bm x^*,\lambda^*)\right]\bm y\ge0,\quad
\forall\bm y\mbox{ s.t. }\inp{\nabla h(\bm x^*)}{\bm y}=0,
\]
or equivalently,
\[
\bm y\trans\left[\nabla^2f(\bm x^*)+\sum_{i=1}^m\lambda_i^*\nabla^2 h_i(\bm x^*)\right]\bm y\ge0,\quad
\forall\bm y\mbox{ s.t. }\inp{\nabla h(\bm x^*)}{\bm y}=0,
\]
i.e., the Hessian matrix $L_{xx}(\bm x^*,\lambda^*)$ is PSD over the null space of Jacobian matrix of $h(\bm x^*)$.
\end{itemize}

\end{theorem}

\begin{theorem}[Lagrange sufficient optimality condition]
Given the problem (\ref{Eq:8:1}), the second order sufficent optimality conditon is
\[
\left\{
\begin{aligned}
\nabla_{\bm x}L(\bm x^*,\lambda^*)&=0,\qquad
\nabla_{\lambda}L(\bm x^*,\lambda^*)=0\\
\bm y\trans\nabla^2_{\bm x\bm x}L(\bm x^*,\lambda^*)\bm y&>0,\qquad\forall\bm y\ne\bm0\mbox{ with }\inp{\nabla h(\bm x^*)}{\bm y}=\bm0
\end{aligned}
\right.
\]

\end{theorem}
\subsection{Inequality Constraint Problem}

Given the problem
\begin{equation}\label{Eq:8:2}
\begin{array}{ll}
\min&f(\bm x)\\
\mbox{such that}&h(\bm x)=0\\
&g(\bm x)\le0,
\end{array}
\end{equation}
with $f:\mathbb{R}^n\mapsto\mathbb{R},h:\mathbb{R}^n\mapsto\mathbb{R}^m,m<n;g:\mathbb{R}^n\mapsto\mathbb{R}^p$, we set a general set
\[
X=\{x\mid h(\bm x)=0,g(\bm x)\le0\}
\]
We set the Lagrange function
\[
L(\bm x,\lambda,\mu)=f(\bm x)+\lambda\trans h(\bm x)+\mu\trans g(\bm x)
\]
Define the dual function
\[
Q(\lambda,\mu)=\inf_XL(\bm x,\lambda,\mu)\le 
f(\bm x)
\]


\section{Monday Tutorial: Review for CIE6010}
\paragraph{Matrix Calculus}
\begin{itemize}
\item
\[
\frac{\partial \bm a\trans\bm x}{\partial \bm x}=\bm a
\]
\[
\frac{\partial\bm a\trans\bm A\bm x}{\partial\bm x}=\bm A\trans\bm a
\]
\item
\[
\frac{\partial\bm x\trans\bm x}{\partial \bm x}=2\bm x
\]
\item
\[
\frac{\partial{\bm{Ax}}}{\partial\bm x}=\bm A\trans
\]
\item
\[
\frac{\partial f\trans g}{\partial\bm x}=\frac{\partial f}{\bm x}g+\frac{\partial g}{\bm x}f
\]
\item
\[
\frac{\partial x\trans\bm{Ax}}{\partial\bm x}=(\bm A+\bm A\trans)\bm x
\]
\item
\[
\frac{\partial\|\bm y-\bm{Ax}\|_2^2}{\partial\bm x}=2\frac{\partial{\bm y-\bm{Ax}}}{\partial\bm x}(\bm y-\bm{Ax})=-2\bm A\trans(\bm y-\bm{Ax})
\]
\item
\begin{proposition}[Chain Rule]
Let $h(x)=g(f(x))$ with $g:\mathbb{R}^m\mapsto\mathbb{R}^n$ and $f:\mathbb{R}^k\mapsto\mathbb{R}^m$, then we have
\begin{equation}
\nabla h(x)=\nabla f(x)\nabla g(f(x))\label{Eq:8:4}
\end{equation}
\end{proposition}
\[
\nabla_{\bm X}\|\bm{AX}-\bm B\|_F^2=
\]
\begin{proof}
Firstly we compute $\nabla_{\bm X}\|\bm X\|_F^2$:
\[
\|\bm X\|_F^2=\sum_{i,j}x_{ij}^2\implies
\nabla_{\bm X}\|\bm X\|_F^2=\left[\frac{\partial\sum_{i,j} x_{ij}^2}{\partial\bm X}\right]
\]
Note that
\[
\left[\frac{\partial\sum_{i,j} x_{ij}^2}{\partial\bm X}\right]_{ij}=
\left[\frac{\partial\sum_{i,j} x_{ij}^2}{\partial x_{ij}}\right]=2x_{ij},
\]
which follows that $\nabla_{\bm X}\|\bm X\|_F^2=2\bm X$, and therefore
\[
\nabla_{\bm X}\|\underbrace{\bm{AX}-\bm B}_{\bm Y}\|_F^2=
\nabla_{\bm X}(\bm{AX}-\bm B)\nabla_{\bm Y}\|\bm Y\|_F^2=2\bm A\trans\bm Y
\]
\end{proof}
\end{itemize}
\paragraph{Necessary and Sufficient Optimality condition}
\begin{proposition}[Unconstraint optimality condition]
Suppose $f:X\mapsto\mathbb{R}$ is a second order differentiable function defined on domain $X$
\begin{itemize}
\item
Necessary condition for local minimum $x^*$:
\[
\left\{
\begin{aligned}
\nabla f(x^*)&=0\\
\nabla^2f(x^*)&\succeq0
\end{aligned}
\right.
\]
\item
Sufficient condition for local minimum $x^*$:
\[
\left\{
\begin{aligned}
\nabla f(x^*)&=0\\
\nabla^2f(x^*)&\succ0
\end{aligned}
\right.
\]
\end{itemize}
\end{proposition}
\begin{corollary}
For $x^*$ satisfying the sufficient condition, there exists $\gamma,\varepsilon>0$ such that
\[
f(x)\ge f(x^*)+\frac{\gamma}{2}\|x-x^*\|^2,\qquad
\forall x\mbox{ such that }\|x-x^*\|<\varepsilon
\]
\end{corollary}
\begin{proof}
For any direction $d$, we have
\begin{align*}
f(x^*+d)-f(x^*)&=\inp{\nabla f(x^*)}{d}+\frac{1}{2}d\trans\nabla^2f(x^*)d+o(\|d\|_2^2)\\
&\ge\frac{1}{2}\lambda_{\min}\|d\|^2+o(\|d\|_2^2)\\
&=\left[\frac{1}{2}\lambda_{\min}+o(1)\right]\|d\|_2^2
\end{align*}
\end{proof}
\begin{proposition}[Constraint but convex optimality condition]
Consider the convex optimization problem
\begin{equation}\label{Eq:8:5}
\begin{array}{ll}
\min&f_0(x)\\
&x\in X\mbox{ for convex set $X$},
\end{array}
\end{equation}
$x$ is optimal if and only if
\begin{equation}\label{Eq:8:6}
\begin{array}{ll}
\inp{\nabla f_0(x)}{(y-x)}\ge0,
&
\forall y\in X
\end{array}
\end{equation}
\end{proposition}
\begin{proof}
For the reverse direction, consider the Taylor expansion and the inequality below:
\[
\begin{array}{ll}
f_0(y)\ge f_0(x)+\inp{\nabla f_0(x)}{(y-x)},
&
\forall y\in X
\end{array}
\]  
For the forward direction, assume $x$ is optimal but (\ref{Eq:8:6}) does not hold, i.e., for some $y\in X$, we have
\[
\inp{\nabla f_0(x)}{(y-x)}<0.
\]
Consider the point $z(t)=ty+(1-t)x$ with the derivative
\[
\frac{\diff}{\diff t}f_0(z(t))|_{t=0}=\inp{\nabla f_0(x)}{(y-x)}<0,
\]
i.e., for some $t>0$, we have $f_0(z(t))<f_0(x)$, which is a contradiction.
\end{proof}
\begin{proposition}[KKT conditions]
Consider the standard optimization problem
\begin{equation}
\begin{array}{ll}
\min&f_0(x)\\
\mbox{such that}&f_i(x)\le0,\quad i=1,\dots,m\\
&h_i(x)=0,\quad i=1,\dots,p
\end{array}
\end{equation}
with the dual problem
\begin{equation}
\begin{array}{ll}
\max&g(\lambda,\gamma)\\
&\lambda_i\ge0,\quad i=1,\dots,m
\end{array}
\end{equation}
\begin{enumerate}
\item
The necessary condition for primal and dual optimal points $x^*$, $(\lambda,\gamma^*)$ is
\begin{itemize}
\item
Primal Feasibility: $f_i(x)\le0,\quad i=1,\dots,m$; $h_i(x)=0,\quad i=1,\dots,p$
\item
Dual Feasibility: $\lambda\ge0$
\item
Complementary Slacknes: $\lambda_if_i(x^*)=0,\quad i=1,\dots,m$
\item
Stationarity of Lagrange: $\nabla_xL(x^*,\lambda^*,\gamma^*)=0$
\end{itemize}
\item
When the primal problem is convex, the KKT conditions above are both necessary and sufficient optimality conditions.
\end{enumerate}
\end{proposition}
\paragraph{Convex functions and convex sets}
How to verify the convexity?
\begin{enumerate}
\item
By definition:
\begin{proposition}
The functions $f_1,f_2$ are convex implies that $f:=\max\{f_1,f_2\}$ is convex.
\end{proposition}
\begin{proof}
For any $\lambda\in(0,1)$ and $x,y\in\mbox{dom }f$, we have
\begin{align*}
f(\lambda x+(1-\lambda)y)&=\max\{f_1(\lambda x+(1-\lambda)y),f_2(\lambda x+(1-\lambda)y)\}\\
&\le\max\{\lambda f_1(x)+(1-\lambda)f_1(y),\lambda f_2(x)+(1-\lambda)f_2(y)\}\\
&\le\lambda\max\{f_1(x),f_2(x)\}+(1-\lambda)\max\{f_1(y),f_2(y)\}\\
&=\lambda f(x)+(1-\lambda)f(y)
\end{align*}
\end{proof}
\item
By properties:
\begin{proposition}[First Order Conditions]
For differentiable function $f$, $f$ is convex if and only if for all $x,y\in\mbox{dom }f$,
\[
f(y)\ge f(x)+\inp{\nabla f(x)}{(y-x)}
\]
\end{proposition}
\begin{proposition}[Second Order Conditions]
For twice differentiable function $f$, $f$ is convex if and only if for all $x\in\mbox{dom }f$,
\[
\nabla^2f(x)\succeq0
\]
\end{proposition}
\end{enumerate}
\paragraph{Algorithms}
For algorithms, make clear of those things below:
\begin{itemize}
\item
How it works?
\item
What kind of problems does it intend for?
\item
What is the convergence rate?
\end{itemize}
\paragraph{Primal and Dual}
\begin{example}
Derive the dual for the optimization problem
\begin{equation}
\begin{array}{ll}
\min&\|\bm{Ax}-\bm b\|_\infty.
\end{array}
\end{equation}
\begin{enumerate}
\item
First make some transformations. We set $t:=\|\bm{Ax}-\bm b\|_\infty$, and thus suffices to solve
\begin{equation}
\begin{array}{ll}
\min&t\\
&\bm {Ax}-\bm b\le t\\
&-\bm{Ax}+\bm b\le t
\end{array}
\end{equation}
\item
Thus the Lagrange function is defined as:
\[
L(\bm x,t,u,v)=t+\begin{bmatrix}
\bm u\trans&\bm v\trans
\end{bmatrix}\left(
\begin{bmatrix}
\bm A&-\bm 1\\-\bm A&-\bm 1
\end{bmatrix}\begin{bmatrix}
\bm x\\ t
\end{bmatrix}-\begin{bmatrix}
\bm b\\-\bm b
\end{bmatrix}
\right)
\]
and $\inf_{\bm x,t}L(\bm x,t,u,v)=-\begin{bmatrix}
\bm u\trans&\bm v\trans
\end{bmatrix}\begin{pmatrix}
\bm b\\-\bm b
\end{pmatrix}$, which implies the dual problem
\[
\begin{array}{ll}
\max&-\begin{bmatrix}
\bm u\trans&\bm v\trans
\end{bmatrix}\begin{pmatrix}
\bm b\\-\bm b
\end{pmatrix}\\
&\bm u\trans\bm A-\bm v\trans\bm A=0\\
&-\bm u\trans\bm 1-\bm v\trans\bm 1=-1
\end{array}
\]

\end{enumerate}
\end{example}
\begin{theorem}[Weak Duality]
For any optimization problem with dual optimal value $q^*$ and primal optimal value $f^*$, we have
\[
q^*\le f^*
\]
\end{theorem}
\begin{theorem}[Strong Duality]
For convex optimization problem, we have
\begin{enumerate}
\item
Both primal and dual has optimal feasible solution
\item
One is infeasible implies the other is unbounded
\item
If both has optimal feasible solution, then $q^*=f^*$.
\end{enumerate}
\end{theorem}
\paragraph{Convergence Analysis}
For (not necessarily quadratic) optimization problem $f(\bm x)=\frac{1}{2}\bm x\trans\bm Q\bm x$, consider the steepest descent
\[
\bm x^{k+1}=\bm x^k-\alpha^k\nabla f(\bm x^k)=(\bm I-\alpha^k\bm Q)\bm x^k,
\]
which implies
\begin{align*}
\|\bm x^{k+1}\|^2&=(\bm x^k)\trans (\bm I-\alpha^k\bm Q)^2\bm x^r\le \lambda_1[(\bm I-\alpha^k\bm Q)^2]\|\bm x^{r}\|^2\\
&=\|\bm x^{k}\|^2\cdot\max\{(1-\alpha^km)^2,(1-\alpha^kM)^2\},
\end{align*}
which implies
\[
\frac{\|\bm x^{k+1}\|}{\|\bm x^k\|}\le\max\{|1-\alpha^km|,|1-\alpha^kM|\}
\]
Choosing $\alpha^k=\frac{2}{M+m}$ s.t. $\max\{|1-\alpha^km|,|1-\alpha^kM|\}$ is maximized, we have
\[
\frac{\|\bm x^{k+1}\|}{\|\bm x^k\|}\le\frac{M-m}{M+m}.
\]
































