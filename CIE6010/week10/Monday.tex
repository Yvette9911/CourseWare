
\chapter{Week10}
\section{Monday}\index{week2_Tuesday_lecture}
\paragraph{Announcement}
No assignment in this week, so you may take a break. However, in next week new assignments and projects will be updated, which requires you to apply penalty algorithms.

\begin{theorem}[Farka's Lemma]
Let $\bm a_1,\dots,\bm a_r\in\mathbb{R}^n$, and 
\[
\bm A=\begin{pmatrix}
\bm a_1\trans\\
\bm a_2\trans\\
\vdots\\
\bm a_r\trans
\end{pmatrix},
\]
then for any $\bm c\in\mathbb{R}^n$,
\begin{subequations}
\begin{equation}
\bm c\trans\bm y\le0,\quad\forall\bm y\mbox{ such that }\bm A\trans\bm y\le0,
\end{equation}
if and only if
\begin{equation}
\bm c=\bm{Au},\forall \bm u\ge0,\bm u\in\mathbb{R}^r
\end{equation}
\end{subequations}
\end{theorem}
\begin{remark}
The interpretation is that the vector $\bm c$ has more than 90 degrees angle with all vectors $\bm a_i$ in the polar cone, if and only if $\bm c$ is in the polar cone.
\end{remark}
\begin{proof}
To show the converse, we have
\[
\bm c\trans\bm y=\bm u\trans\bm A\trans\bm y,
\]
with $\bm u\ge0,\bm A\trans\bm y\le0$, and thereofre $\bm c\trans\bm y\le0$.
\end{proof}
\begin{proof}


\end{proof}
\paragraph{Convex Program}
\[
\begin{array}{ll}
\min&f(x)\\
\mbox{such that}&\bm{Ax}=\bm b\\
&g(\bm x)\le0
\end{array}
\]
with $f,g$ to be convex. The Lagrangian function is given by:
\[
L(\bm x,\bm\lambda,\bm\mu)=f(\bm x)+\bm\lambda(\bm{Ax}-\bm b)+\bm\mu\trans g(\bm x),
\]
with $\bm\mu\ge0$. This function is convex in $\bm x$. Therefore the dual function is given by:
\[
Q(\bm\lambda,\bm\mu)=\inf_{\bm x,\bm\mu\ge0}L(\bm x,\bm\lambda,\bm\mu)
\]
\begin{proposition}[Weak Deality]
\[
Q(\bm\lambda,\bm\mu)\le f(\bm x)
\]
for dual feasible $\bm\lambda,\bm\mu$ and primal feasible $\bm x$.
\end{proposition}
We are curious on the tightest lower bound on LHS, thus maximizing the dual function to obtain the dual program:
\[
\begin{array}{ll}
\max&Q(\bm\lambda,\bm\mu)\\
\mbox{such that}&\bm\mu\ge0
\end{array}
\]
\begin{proposition}[Strong Duality]
For convex programming, we have
\[
d^*=p^*,
\]
with $d^*,p^*$ to be the optimal value from dual and primal problems, respectively.
\end{proposition}
QC: one of the followings is satisfied
\begin{itemize}
\item
$g_i(\bm x)$ are linear
\item
$\bm{Ax}=\bm b, g(\bm x)\le0$
\item
Regularity
\end{itemize}
Under QC, the primal and dual could attain optimality togethoer iff
\begin{itemize}
\item
$\bm{Ax}=\bm b,g(\bm x)\le0$
\item
$\bm\mu\ge0$
\item
$\bm\mu\circ g(\bm x)=\bm0$
\end{itemize}
We have derive the dual formula for linear programming, but how about the quadratic programming?
\begin{example}
\begin{equation}
\begin{array}{ll}
p^*=\min&\frac{1}{2}\bm x\trans\bm Q\bm x+\bm c\trans\bm x,\quad\bm Q\succ0\\
\mbox{such that}&\bm{Ax}\le\bm b
\end{array}
\end{equation}
The Lagrangian function $L(\bm x,\bm\mu)=\frac{1}{2}\bm x\trans\bm Q\bm x+\bm c\trans\bm x+\bm\mu\trans(\bm{Ax}-\bm b)$, and therefore
\begin{equation}\label{Eq:10:3}
Q(\bm\mu)=\min_{\bm x,\bm\mu\ge0}L(\bm x,\bm\mu)
\end{equation}
The optimality condition implies that
\[
\nabla_{\bm x}L(\bm x,\bm\mu)=\bm{Qx}+\bm c+\bm A\trans\bm\mu=0\implies
\bm x=-\bm Q^{-1}(\bm c+\bm A\trans\bm\mu)
\]
Thus substituting optimal $\bm x$ into (\ref{Eq:10:3}), we derive
\begin{align*}
Q(\bm\mu)&=-\frac{1}{2}\bm\mu\trans\bm A\bm Q^{-1}\bm A\trans\bm\mu-\bm t\trans\bm\mu+\mbox{constant}
\end{align*}
Thus we derive the dual program:
\begin{equation}
\begin{array}{ll}
d^*=\min&\frac{1}{2}\bm\mu\trans\bm P\bm\mu+\bm t\trans\bm\mu,\\
\mbox{such that}&\bm\mu\ge0
\end{array}
\end{equation}
where $\bm P:=\bm A\bm Q^{-1}\bm A\trans$.
\end{example}
\subsection{Penalty Algorithms}
\paragraph{Logarithm Penalty}
Consider the inequality constraint problem
\[
\begin{array}{ll}
\min&f(\bm x)\\
&g_i(\bm x)\le0
\end{array}
\]
The Barrier problem is given by:
\[
\min f(x)-\mu\sum_{i}\log(-g_i(\bm x)),\quad\mu>0
\]
As $\mu\to0$, $x(\mu)$ converges to the optimal solution. We pick big $\mu$ at first and obtain a good initial guess, and then we continue to decrease $\mu$.
\paragraph{Quadratic Penalty}
For the constraint problem
\[
\begin{array}{ll}
\min&f(\bm x)\\
&h(\bm x)=0\\
&\bm x\in X
\end{array}
\]
The quadratic penalty algorithm aims to solve
\[
\begin{array}{ll}
\min&f(\bm x)+\lambda\trans h(\bm x)+\frac{c}{2}\|h(\bm x)\|_2^2\\
&\bm x\in X,
\end{array}
\]
where $\lambda$ is \emph{bounded}. Conversely, as $c\to\infty$, $\bm x(c)$ converges to the optimal solution. We pick small $c$ at first and obtain a good initial guess, and then we continus to increase $c$.
\begin{example}
\[
\begin{array}{ll}
\min&\frac{1}{2}(x_1^2+x_2^2)\\
&x_1=1
\end{array}
\]
The quadratic penalty function is 
\[
L_c(x)=\frac{1}{2}(x_1^2+x_2^2) + \lambda(x_1-1)+\frac{c}{2}(x_1-1)^2
\]
and therefore
\[
\nabla_{\bm x}L_c(\bm x)=\begin{pmatrix}
x_1\\x_2
\end{pmatrix}+\lambda\begin{pmatrix}
1\\0
\end{pmatrix}+c\begin{pmatrix}
x_1-1\\0
\end{pmatrix}=\begin{pmatrix}
0\\0
\end{pmatrix},
\]
which follows that
\[
\begin{array}{ll}
x_1(\lambda,c)=\frac{c-\lambda}{c+1},
&
x_2(\lambda,c)=0
\end{array}
\]
We can apply two algorithm to converge to optimal solution:
\begin{enumerate}
\item
1. 
Quadratic Pendlty Method:
As $c\to\infty$ with $\lambda$ bounded, we derive $x_1(\lambda,c)\to1$.
\item[2.]
Lagrangian Multiplier Method: 
We set $\nabla L(\bm x,\lambda)=0$ to obtain an appropriate $\lambda^*=-1$. As $\lambda\to\lambda^*$, we obtain $x_1(\lambda,c)\to1$ for $c>1$ (the key for this kind of algorithm is to choose big $c$).
\end{enumerate}
\end{example}
Such an algorithm can also be applied for the non-convex problem, e.g.,
\[
\begin{array}{ll}
\min&\frac{1}{2}(-x_1^2+x_2^2)\\
&x_1=1
\end{array}
\]













