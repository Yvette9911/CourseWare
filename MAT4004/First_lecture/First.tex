\chapter{Definition}

\begin{definition}[Graph]
A graph $G=(V,E)$ consists of a non-empty finite set $V$ of elements called \emph{vertices};
and a finite family of \emph{unordered} pairs of vertices called \emph{edges}.
\begin{enumerate}
\item
The edge $e=\{v,w\}$ is said to join the vertices $v$ and $w$
\item
The vertex $v$ is \emph{adjacent} to the vertex $w$ if there exists an edge $\{v,w\}$
\item
The edge $e$ is also said to be \emph{incident} to $v$.
\end{enumerate}
\end{definition}

\begin{definition}[Simple]
Multiple (parallel) edges are the edges joining the same pair of vertices;
Loops are the edges of the form $\{v,v\}$;
A \emph{loopless} graph with \emph{no} multiple edges is called a \emph{simple} graph.
\end{definition}

\begin{definition}[Isomorphism]
Two graphs $G_1$ and $G_2$ are \emph{isomorphic} if
there is a one-to-one correspondence between the vertices of $G_1$ and the vertices of $G_2$
such that
the number of edges joining any two vertices of $G_1$ equals the number of edges joining the corresponding pair of vertices in $G_2$.
\end{definition}

\begin{definition}[Adjacent $\&$ Incident]
Two vertices are \emph{adjacent} if there is an edge joining them;
the edges are \emph{incident} to the edge;
two edges are \emph{adjacent} if they share a common vertex.
\end{definition}

\begin{definition}
The \emph{degree} of a vertex $v$, say $\text{deg}(v)$, is the number of edges incident to $v$.
\end{definition}

In particular, a loop contributes $2$ to the $\text{deg}(v)$;
a vertex with degree $0$ is an \emph{isolated} vertex;
a vertex of degree $1$ is an \emph{end-vertex}.

\begin{definition}[Degree sequence]
The \emph{degree sequence} of a graph is the sequence of degrees of its vertices, 
written in \emph{non-decreasing} order. 
\end{definition}

\begin{theorem}
In any graph, the sum of all the vertex-degrees is an \emph{even} number.
\end{theorem}
\begin{corollary}
In any graph, the number of vertices with \emph{odd} degree is \emph{even}.
\end{corollary}

\begin{definition}
A graph $H$ is a \emph{subgraph} of a graph $G=(V,E)$,
if each of its vertices belong to $V(G)$,
and each of its edges belongs to $E(G)$.
\end{definition}

\begin{definition}
The subgraphs can be obtained by the following operations:
\begin{enumerate}
\item
$G-e$: removing edge $e$
\item
$G-F$: removing the set of edges $F$
\item
$G-v$: removing vertex $v$ and all its incident edges
\item
$G-S$: removing the vertices in the set $S$ and all edges incident to any vertex in $S$
\item
$G\setminus e$: \emph{contracting} edge $e$. (question: identifying endpoints of $e$ and delete the possible parallel edges)
\end{enumerate}

\end{definition}


\begin{definition}[Complement]
If $G$ is a simple graph, its complement, denoted as $\bar{G}$, has the same vertex set,
while two vertices are adjacent in $\bar{G}$ if and only if thery are not adjacent in $G$.
\end{definition}

\begin{definition}[Null $\&$ Complete]
A \emph{null} graph is the one where the edge set is empty;
a simple graph where each distinct pair of vertices are adjacent is a \emph{complete} graph;
a complete graph on $n$ vertices is denoted as $K_n$.
\end{definition}

\begin{definition}[Walk $\&$ Path $\&$ Cycle]
\begin{enumerate}
\item
A walk consists of a sequence of edges, one following after another
\item
A walk in which no vertex appears \emph{more than once} is called a path
\item
A walk in which no vertex appears \emph{more than once}, except for begining and end vertices which coincide, is called a \emph{cycle}.
\end{enumerate}
\end{definition}

\begin{definition}[Bipartite]
If the vertex set of a graph consists of the union of two \emph{disjoint} sets $A$ and $B$ such that each edge of $G$ joints a vertex in $A$ and a vertex in $B$, then $G$ is a \emph{bipartite} graph.
Moreover, if each vertex in $A$ is joined to each vertex in $B$ by an edge, it is a \emph{complete bipartite graph}, denoted as $K_{|A|,|B|}$.
\end{definition}

\begin{definition}[Directed Graph]
The graph $G=(N,A)$ is a \emph{directed graph} (or \emph{digraph}) if $V$ is a finite set of nodes and $A$ is a finite family of \emph{directed edges} (\emph{arcs}).
Each arc in $A$, denoted as $(v,w)$, is an ordered pair of nodes.

The diagraph $D$ is simple if the arcs are all distinct, and there are no loops.
\end{definition}
\begin{definition}[Underlying Graph]
The underlying graph $G = (N,E)$ of a directed graph $G = (N,A)$ has the same node set; 
every edge $e = \{v,w\}$ of $E$ corresponds to an arc $(v,w)\in A$.

Question: does underlying graph mean the \emph{undirected} graph?
\end{definition}

\begin{definition}[Out $\&$ In-degree]
Recall that the edge $(v,w)$ is \emph{incident} from $v$ and \emph{incident} to $w$.
The \emph{out-degree} of vertex $v$ is the number of edges \emph{incident from} $v$;
the \emph{in-degree} of vertex $v$ is the number of edges \emph{incident to} $v$.
\end{definition}
\begin{theorem}
In any digraph, the sum of all the \emph{in-degrees} is equal to the sum of all the \emph{out-degrees}.
\end{theorem}
\begin{proof}
Question
Each edge $(v,w)$ contribute 1 to the in-degree and 1 to the out-degree.
\end{proof}

\begin{definition}[Adjacency matrix]
The \emph{adjacency matrix} for an \emph{undirected graph} $G=(V,E)$ is a $|V|\times|V|$ square matrix where the element $(i,j)$ is the number of edges joining the vertices $i$ and $j$.
\end{definition}
\begin{definition}[Incidence matrix]
The \emph{incidence matrix} for an \emph{undirected graph} $G=(V,E)$ is a $|V|\times|E|$ matrix,
where the element $(i,j)$ is $1$ if the vertex $i$ is incident to edge $j$.
\end{definition}

\begin{definition}[Node-Arc Incidence matrix]
The \emph{node-arc incidence matrix} for a directed graph $G=(V,E)$ is a $|V|\times|E|$ matrix where the element $(i,j)$ is $1$ if edge $j$ is incident from vertex $i$,
and is $-1$ if edge $j$ is incident to $i$.
\end{definition}




















