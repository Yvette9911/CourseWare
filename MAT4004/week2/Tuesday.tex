
\chapter{Week2}
\section{Tuesday}\index{week2_Tuesday_lecture}
\subsection{Optimal order quantity}
The optimal $y^*$ is the smallest $y$ such that
\[
F(y)\ge\frac{c_p-c_v}{c_p-c_s}
\]
\paragraph{Discrete Case Optimization} Given the expected profit function $h(y)$, we want to find the optimal $y$. Note that $y$ cannot be optimal if $h(y+1) - h(y)>0$.
\[
\begin{aligned}
h(y+1) - h(y)&=\mathbb{E}\mbox{profit}(y+1,D) - \mbox{profit}(y,D)\\
&=\mathbb{E}\mbox{revenue}(y+1,D) - \mbox{revenue}(y,D) - c_v
\end{aligned}
\]
If $D\ge y+1$, there will be no leftover iterms in both systems:
\[
\mbox{revenue}(y+1,D) - \mbox{revenue}(y,D) = c_p
\]
If $D\le y$,
\[
\mbox{revenue}(y+1,D) - \mbox{revenue}(y,D) = c_s
\]
It follows that
\[
\begin{aligned}
h(y+1) - h(y) &=c_p\mathbb{P}[D\ge y+1] + c_s\mathbb{P}[D\le y] - c_v\\
&=c_p-c_v-(c_p-c_s)\mathbb{P}[D\le y]
\end{aligned}
\]
Hence, $h(y+1)>h(y)$ iff
\[
\mathbb{P}[D\le y]<\frac{c_p - c_v}{c_p - c_s}
\]
In this case $y$ cannot be optimal. Hence, $y^*$ is the smallest $y$ s.t.
\[
\mathbb{P}[D\le y]\ge\frac{c_p - c_v}{c_p - c_s}
\]
\paragraph{Holding Cost for $h=.1$} If each leftover item cost $h$, we have
\[
\mathbb{E}\mbox{Profit}(q,D)
=\mathbb{E}\min(q,D)c_p - c_vq - h\mathbb{E}(q-D)^+
\]
In addiction, if given a fixed cost of order $c_f$, we derive
\[
\mathbb{E}\mbox{Profit}(q,D)
=\mathbb{E}\min(q,D)c_p - c_vq - h\mathbb{E}(q-D)^+ - c_f
\]
\paragraph{Confidence Interval}
The expected profit in next three months is a summation of $90$ i.i.d. RVs:
\[
R(25) = P_1(25)+\cdots+P_{90}(25)
\]
Using central limit theorem, this RV is approximatly normal. It follows that
\[
R(25)\sim\mathcal{N}(11812,90\sigma^2)
\]
where $\sigma^2 = Var(R(25))$. Hence,
\[
\mathbb{P}
\left(
\left|\frac{R(25) - 11812}{\sqrt{90}\sigma}\right|<1.96
\right)
= 0.95
\]
With $95\%$ level of confidence, $R(25)$ is between $11812 + 1.96\cdot\sqrt{90}\sigma$ and $11812 - 1.96\cdot\sqrt{90}\sigma$
\subsection{Non-perishable Products}
\begin{enumerate}
\item
For $X_n\ge s$, do not order anything
\item
Otherwise, order enough to bring the inventory level to $S$ at the beginning of the next period.
\end{enumerate}
For example, $(s,S) = (20,30)$, determine the probability $\mathbb{P}\{X_10=10\mid X_9=10\}$ and $\mathbb{P}\{X_10=10\mid X_9=20\}$:
\[
\mathbb{P}\{X_{10}=10\mid X_9={10}\} = \mathbb{P}(D_{10}=20)=\frac{1}{8}
\]
\[
\mathbb{P}\{X_{10}=10\mid X_9=20\}
=
\mathbb{P}(D_{10} = 10) = \frac{1}{4}.
\]
Actually, we can create a matrix to describe these conditional probabilities:
\[
S_{ij} = \mathbb{P}\{X_{10} = j \mid X_9 = i\}
\]
Knowing current state, past states are irrelevant to predict future states. 









